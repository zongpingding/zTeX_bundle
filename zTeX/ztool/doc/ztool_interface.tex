\InputIfFileExists{ztex_doc-cfg.tex}{}{}
\documentclass[
  hyper, lang=cn, 
  class=l3dox, 
]{../../zlatex/code/ztex}
\usepackage[shell-escape, file-io, box, zdraw]{ztool}
\ztexloadlib{alias, thm}
\ztexset{
  toc={
    column  = 2,
    title   = 总目录,
    stretch = 1.3
  }
}
\zcolorset{ link=purple } 
\geometry{left=2in, right=1in}
\usepackage{zhnumber}
\usepackage{minted}
\usepackage{multicol}
\usepackage{hologo}
\usepackage[bottom]{footmisc}


% ==> l3doc patches
\AtBeginDocument{
  % \DeleteShortVerb \"
  \DeleteShortVerb \|
}
% key-value env for typesetting key-value options
\ExplSyntaxOn
\cs_new_protected:Npn \__codedoc_function_ztex:nnw #1#2
  {
    \__codedoc_function_typeset_start:
    \__codedoc_function_init:
    \tl_set:Nn \l__codedoc_macro_argument_tl {#2}
    \keys_set:nn { l3doc/function } {#1}
    \__codedoc_names_get_seq_ztex:nN {#2} \l__codedoc_names_seq
    \__codedoc_names_parse:
    \__codedoc_function_typeset:
    \__codedoc_function_reset:
    \__codedoc_function_descr_start:w
  }
\cs_new_protected:Npn \__codedoc_names_get_seq_ztex:nN #1#2
  {
    \bool_if:NTF \l__codedoc_names_verb_bool
      {
        \seq_clear:N #2
        \seq_put_right:No #2 { {#1} }
      }
      {
        \tl_set:Nn \l__codedoc_tmpa_tl {#1}
        \tl_remove_all:Ne \l__codedoc_tmpa_tl
          { \exp_not:N \obeyedline \c_percent_str }
        \tl_remove_all:Ne \l__codedoc_tmpa_tl
          { \exp_not:N \obeyedline }
        \__kernel_tl_set:Nx \l__codedoc_tmpa_tl { \l__codedoc_tmpa_tl }
        \tl_remove_all:Ne \l__codedoc_tmpa_tl
          { \iow_char:N \^^M \c_percent_str }
        \tl_remove_all:Ne \l__codedoc_tmpa_tl { \tl_to_str:n { ^ ^ A } }
        \tl_remove_all:Ne \l__codedoc_tmpa_tl { \iow_char:N \^^I }
        \tl_remove_all:Ne \l__codedoc_tmpa_tl { \iow_char:N \^^M }
        \__codedoc_detect_internals:N \l__codedoc_tmpa_tl
        \__codedoc_replace_at_at:N \l__codedoc_tmpa_tl
        \tl_set:Ne \l__codedoc_tmpa_tl 
          { 
            \clist_map_function:NN \l__codedoc_tmpa_tl 
              \__ztex_add_parent_key:n
          }
        \exp_args:NNe \seq_set_from_clist:Nn #2
          { \l__codedoc_tmpa_tl }
      }
  }
\cs_set:Npn \__ztex_add_parent_key:n #1 
  {
    \textcolor{gray}{\l__codedoc_parent_key_ztex_tl/}
    \tl_trim_spaces:n {#1},
  }
\DeclareDocumentEnvironment { keyval } { O{} +m }
  {
    \__codedoc_function_ztex:nnw {#1} {#2}
  }{ 
    \__codedoc_function_end: 
  }
%% catcode hack ref: https://tex.stackexchange.com/a/392770/294585
\cctab_const:Nn \g__ztex_keyval_cctab 
  {
    \cctab_select:N \c_document_cctab
    \char_set_catcode_active:n {124} % '|'
    \char_set_catcode_active:n {62}  % '>'
    \char_set_catcode_letter:n {95}  % '_'
  }
\group_begin:
  \catcode`|=\active
  \catcode`>=\active
  \cs_gset:Nn \__ztex_bar_active: 
    { \def|{\textup{\string|}} }
  \cs_gset:Nn \__ztex_underscore_active: 
    { \def>{\dotfill} }
\group_end:
\DeclareDocumentEnvironment { syntax } { }
  { 
    \cctab_begin:N \g__ztex_keyval_cctab
    \__ztex_bar_active:
    \__ztex_underscore_active:
    \__codedoc_syntax:w 
  }{
    \__codedoc_syntax_end:
    \cctab_end: 
    \ignorespacesafterend
  }
% default value
\gdef\nval{不可设置值}
\newlength{\dvalWidth}
\newlength{\dvalwidth}
\setlength{\dvalWidth}{2.25em}
\setlength{\dvalwidth}{2em}
\gdef\dval{\@ifstar\@@dval\@dval}
\def\@dval#1{初始值:\hb@xt@\dvalWidth{\hfill\textcolor{blue}{\ztool_scale_to_wd:nn {\dvalwidth}{#1}}}}
\def\@@dval#1{初始值:\textcolor{blue}{#1}}
\gdef\vsp{\usefont{OT1}{cmtt}{m}{n}\asciispace}
\ExplSyntaxOff


% ==> source example env
\fvset{gobble=0}
\newcounter{DocExample}
\setmonofont{Latin Modern Mono}
  [
    BoldFont=*,
    ItalicFont=* Slanted,
    BoldItalicFont=* Slanted,
    BoldFeatures={FakeBold=2},
    BoldItalicFeatures={FakeBold=2},
  ]
% I. use 'texhigh' to highlight code
% \usepackage[most]{tcolorbox}
\usepackage{texhigh}
\texhighloadstyle{texhigh}
\SetKeys[texhigh/high]{config-file=texhigh.cfg}
\SetKeys[texhigh/high]{use-ctab=latex3, lexer-catcode*={0}{}{@=11}}
\SetKeys[texhigh/high]{before-text+={\catcode`\ =10 \catcode`\"=12 }}
\tcbset{listing engine=texhigh}%, listing file=\jobname.listing}
\DeclareTCBListing{DocExample}{!s!O{//}}{
  enhanced, 
  breakable,
  % frame hidden, arc=2pt,
  enhanced jigsaw,
  opacityback=0, 
  sharp corners, 
  colframe=black, boxrule=.4pt,
  left=.5mm, right=1mm,
  top=0mm, bottom=0mm, 
  \IfBooleanTF{#1} 
    {listing and text}
    {listing only},
  overlay unbroken and first = {
    \node[anchor=north east, outer sep=0pt, text=red] 
      at (frame.north east) { \stepcounter{DocExample}\textbf{例\ \theDocExample}};
  }
}

% II. use 'minted' to highlight code
\definecolor{bg}{rgb}{0.95,0.95,0.95}
\setminted{
  bgcolor=bg,
  breaklines=true, 
  tabsize=2,
  breakanywhere=true,
  breaksymbolright=$\swarrow$,
  breakanywheresymbolpre=,
  breaksymbolleft=,
}
% \usepackage[most]{tcolorbox}
% \tcbuselibrary{listings, minted, breakable, skins}
% \tcbuselibrary{minted}
% \DeclareTCBListing{DocExample}{!s!O{//}}{
%   enhanced, 
%   breakable,
%   % frame hidden, arc=2pt,
%   enhanced jigsaw,
%   opacityback=0,
%   sharp corners, 
%   colframe=black, boxrule=.4pt,
%   left=.5mm, right=1mm,
%   top=0mm, bottom=0mm, 
%   \IfBooleanTF{#1} 
%     {listing and text}
%     {listing only},
%   minted language=tex, 
%   minted options = {
%     autogobble,
%     escapeinside=#2,
%     bgcolor=,
%     fontsize=\small,
%     % style=manni
%   },
%   overlay unbroken and first = {
%     \node[anchor=north east, outer sep=0pt, text=red] 
%       at (frame.north east) { \stepcounter{DocExample}\textbf{例\ \theDocExample}};
%   }
% }


% ==> print ztex source
\ExplSyntaxOn
\clist_new:N \l__ztool_doc_source_clist
\clist_clear:N \l__ztool_doc_source_clist
\cs_set:Npn \__ztex_doc_source:nn #1#2 
  {
    \clist_map_inline:nn {#2}
      {
        \clist_put_right:Nn \l__ztool_doc_source_clist 
          {
            \texhighfile{../code/#1/ztool.#1.##1.tex}
          }
      }
  }
\newcommand{\inputZtoolSource}[2][module]
  {
    \__ztex_doc_source:nn {#1}{#2}
    \clist_use:Nn \l__ztool_doc_source_clist   
      { \newpage }
    \clist_clear:N \l__ztool_doc_source_clist
  }
\ExplSyntaxOff
\newcommand{\ztexDocPrintSource}{%
  \THlinenumtrue
  \ztexslideTF{}{
    \section{zTool 源码}
    \pagestyle{empty}
    \zpagemask*[anchor=mr, position={(\zpw, .5\zph)}]{{\sffamily\color{gray}\scalebox{5}{\thepage}}}
    \renewcommand{\theFancyVerbLine}{\sffamily
      \textcolor{gray}{\small\oldstylenums{\arabic{FancyVerbLine}}}}
    % \setminted{ bgcolor=, linenos=true, numbers=both, texcomments, escapeinside=«» }
    \texhighfile{../../ztool/code/ztool.sty}
    \clearpage
    \inputZtoolSource[library]{shell-escape, file-io, box, zdraw}
  }
}


% ==> aux commands
\newcommand{\block}[1]{{\color{#1}\rule{1em}{1em}}}
\newcommand{\Footnote}[1]{\stepcounter{footnote}\footnote[\thefootnote]{#1}}
\NewDocumentCommand{\zdefault}{sm}{%
  \IfBooleanTF{#1}%
    {\textcolor{red}{\textbf{#2}}}%
    {\textcolor{red}{:\textbf{#2}}}%
}
\ExplSyntaxOn
\newcommand{\zarg}[1]{\texttt{\{}\cmd{#1}\texttt{\}}}
\newcommand{\zkey}[1]{
  \clist_clear:N \l_tmpa_clist
  \clist_map_inline:nn {#1}{
    \clist_put_right:Nn \l_tmpa_clist {\meta{##1}}
  }
  \clist_use:Nn \l_tmpa_clist {,~}
}
\ExplSyntaxOff


% todo list
% REF: https://tex.stackexchange.com/q/247681/294585
% syntax:
%     *: wont fix
%     [<arg>]: done
%     [<blank>]: undone
\usepackage{enumitem}
\newlist{todolist}{itemize}{2}
\setlist[todolist]{label=\checkmark}
\usepackage{amssymb}
\newcommand{\done}{\rlap{\raisebox{0.3ex}{\hspace{0.4ex}\tiny \ding{52}}}$\square$}
\newcommand{\undone}{$\square$}
\newcommand{\wontfix}{\rlap{\raisebox{0.3ex}{\hspace{0.4ex}\scriptsize \ding{56}}}$\square$}



\title{\texorpdfstring{zTool{} 接口文档}{zTool 接口文档}}
\author{Eureka}
\date{\today}
\begin{document}
% cover
\ExplSyntaxOn
\zpagemask[position={(0pt, .25\zph)}, anchor=l]{
  \rlap{\color{gray!25}\rule{\zpw}{6em}}
  \hcoffin_set:Nn \l_tmpa_coffin {\sffamily\Large\color{black!75} 由于本人时间有限, 目前此宏包的开发暂停.}
  \coffin_typeset:Nnnnn \l_tmpa_coffin {hc}{vc}{.5\zpw}{3em}
}
\ExplSyntaxOff
\newgeometry{hmargin=1cm, vmargin=1.5in}
  \maketitle
\restoregeometry

% contents
\newgeometry{hmargin=3cm, vmargin=1in}
  \ztexslideTF{
    \thispagestyle{empty}
    \tableofcontents
  }{
    \thispagestyle{empty}
    \vspace*{-3em}
    \tableofcontents
    \clearpage
  }
\restoregeometry

\section{基本介绍}
\ztex{} 宏集已独立实现了一个 \pkg{ztool} 宏包, 此宏包中包含原来已被废弃的 \pkg{l3sys-shell} 中的所有命令.
除此之外, \pkg{ztool} 提供了 box 操作, 文件 IO 以及基本图形绘制相关的函数. 在 \pkg{ztool} 的协助下,\ztex{} 能够
避免或减少命令行 \cmd{-shell-escape} 参数或其它相关宏包的调用(如 \pkg{robust-externalize} 宏包).

本宏包在 Github 上的地址如下:

\begin{center}
  \fbox{https://github.com/zongpingding/zTeX\_bundle}
\end{center}

该仓库中包含本宏集的源码与用户手册; 当前宏集的稳定版本于半年之前发布, 最新的开发版请切换到 ``dev'' 分支; 
本手册适用于当前最新的开发版.

\clearpage
\section{宏包选项}
\pkg{ztool} 分为了 ``\texttt{shell-escape, file-io, box, zdraw}'' 四个库, 每一个库之间互不影响, 均可单独加载.
默认不加载任意的 \pkg{ztool} 库.


\begin{keyval}[added=2025-05-22, parent=ztool]{shell-escape, file-io, box, zdraw}
  \begin{syntax}
    shell-escape = \meta{\textbf{false}|true}>\dval{false}
    file-io      = \meta{\textbf{false}|true}>\dval{false}
    box          = \meta{\textbf{false}|true}>\dval{false}
    zdraw        = \meta{\textbf{false}|true}>\dval{false}
  \end{syntax}
  这四个选项为 \pkg{ztool} 宏包的选项, 可以在加载 \pkg{ztool} 宏包时使用, 一个基本的使用样例如下,
  该示例加载了 \pkg{ztool} 的 \pkg{shell-escape} 库和 \pkg{box} 库:
\end{keyval}
\begin{DocExample}
\usepackage[shell-escape, box=true]{ztool}
\end{DocExample}


\begin{function}[added=2025-05-22]{\ztoolloadlib}
  \begin{syntax}
    \cs{ztoolloadlib} \marg{library}
  \end{syntax}
  此命令用于加载 \pkg{ztool} 库, \meta{library} 为库的名称, 可选值有: ``\texttt{shell-escape, file-io, box, zdraw}''.
\end{function}
一个基本的使用样例如下, 该示例加载了 \pkg{ztool} 的 \pkg{shell-escape} 库和 \pkg{box} 库:
\begin{DocExample}
\ztoolloadlib{shell-escape, box}
\end{DocExample}



\clearpage
\section{l3sys-shell}
本部分主要介绍 \pkg{ztool} 中实现的原始 \pkg{l3sys-shell} 宏包中的命令. 所以使用本部分的命令时需在编译
\LaTeX{} 文档时启用 \cmd{-shell-escape} 参数, 否则此系列命令将不会执行任何操作.

\vskip2em
\noindent\textcolor{red}{\sffamily WARNING: 请谨慎使用此部分的命令, 部分不当操作可能会导致一些无法挽救的后果.}

\begin{function}[updated=2024-12-05]{\ztool_shell_escape:n, \ztool_shell_escape:e}
  \begin{syntax}
    \cs{ztool_shell_escape:n} \marg{command}
  \end{syntax}
  当 \cmd{-shell-escape} 参数启用时, 此命令会在 shell 中执行 \meta{command}, 如果 \cmd{-shell-escape} 参数未启用,
  此命令将不会执行任何操作.
\end{function}


\begin{function}[updated=2024-12-05]{\ztool_shell_mkdir:n, \ztool_shell_mkdir:e}
  \begin{syntax}
    \cs{ztool_shell_mkdir:n} \marg{dir}
  \end{syntax}
  当 \cmd{-shell-escape} 参数启用时, 此命令会创建一个目录 \meta{dir}, 如果 \cmd{-shell-escape} 参数未启用,
  此命令将不会执行任何操作.
\end{function}


\begin{function}[updated=2024-12-05]{\ztool_shell_cp:nn, \ztool_shell_cp:ee, \ztool_shell_cp:ne, \ztool_shell_cp:en}
  \begin{syntax}
    \cs{ztool_shell_cp:nn} \marg{source}\marg{target}
  \end{syntax}
  当 \cmd{-shell-escape} 参数启用时, 此命令将把文件 \meta{source} 复制为文件 \meta{target}, 如果 \cmd{-shell-escape} 参数未启用,
  此命令将不会执行任何操作.
\end{function}


\begin{function}[updated=2024-12-05]{\ztool_shell_mv:nn, \ztool_shell_mv:ee, \ztool_shell_mv:ne, \ztool_shell_mv:en}
  \begin{syntax}
    \cs{ztool_shell_mv:nn} \marg{source}\marg{target}
  \end{syntax}
  当 \cmd{-shell-escape} 参数启用时, 此命令将把文件 \meta{source} 移动到目录 \meta{target}, 如果 \cmd{-shell-escape} 参数未启用,
  此命令将不会执行任何操作.
\end{function}

\begin{function}[updated=2024-12-05]{\ztool_shell_rm:n, \ztool_shell_rm:e}
  \begin{syntax}
    \cs{ztool_shell_rm:n} \marg{file}
  \end{syntax}
  当 \cmd{-shell-escape} 参数启用时, 此命令将删除文件 \meta{file}, 如果 \cmd{-shell-escape} 参数未启用,
  此命令将不会执行任何操作.
\end{function}

\begin{function}[updated=2024-12-05]{\ztool_shell_rmdir:n, \ztool_shell_rmdir:e}
  \begin{syntax}
    \cs{ztool_shell_rmdir:n} \marg{dir}
  \end{syntax}
  当 \cmd{-shell-escape} 参数启用时, 此命令将删除目录 \meta{dir}, 如果 \cmd{-shell-escape} 参数未启用,
  此命令将不会执行任何操作.
\end{function}

\begin{function}[updated=2024-12-05]{\ztool_get_shell_pwd:N, \ztool_get_shell_pwd:c}
  \begin{syntax}
    \cs{ztool_get_shell_pwd:N} \meta{tl}
  \end{syntax}
  当 \cmd{-shell-escape} 参数启用时, 此命令将返回当前的工作目录, 并将其存放在 \meta{tl} 中, 如果 \cmd{-shell-escape} 参数未启用,
  此命令将不会执行任何操作.
\end{function}


\begin{function}[updated=2024-12-05]{\ztool_shell_split_ls:nN}
  \begin{syntax}
    \cs{ztool_shell_split_ls:nN} \marg{dir}\meta{tl}
  \end{syntax}
  当 \cmd{-shell-escape} 参数启用时, 此命令将返回目录 \meta{dir} 下的所有文件名, 并将其存放在 \meta{tl} 中, 如果 \cmd{-shell-escape} 参数未启用,
  此命令将不会执行任何操作.
\end{function}


\clearpage
\section{File IO}
本部分主要介绍 \pkg{ztool} 中实现的文件 IO 操作, 包括: 读取文件, 写入文件, 追加文件等操作. 本部分的系列命令均不需要
启用 \cmd{-shell-escape} 参数.


\begin{function}[updated=2024-12-05]{\ztool_file_new:nn}
  \begin{syntax}
    \cs{ztool_file_new:nn} \marg{bool}\marg{file}
  \end{syntax}
  此命令用于创建一个名为 \meta{file} 的新文件, 如果 \meta{file} 不存在, 则会创建一个名为 \meta{file} 的新文件.
  若文件已存在,那么当 \meta{bool} 为 \cs{c_true_bool} 时,\textcolor{red}{\sffamily 会覆盖原文件},否则不会进行任何操作.
\end{function}


\begin{function}[updated=2024-12-05]{
  \ztool_read_file_as_seq:nnN, \ztool_read_file_as_seq:neN, 
  \ztool_read_file_as_seq:nnc, \ztool_read_file_as_seq:nec}
  \begin{syntax}
    \cs{ztool_read_file_as_seq:nnN} \marg{bool}\marg{file}\meta{seq}
  \end{syntax}
  此命令用于读取文件 \meta{file} 的内容, 并将其存放在 \meta{seq} 中, 如果 \meta{file} 不存在, 则 \meta{seq} 会被置为空.
  \meta{bool} 用于控制是否保留行尾的空格, 可选值有:\cs{c_true_bool}, \cs{c_false_bool}; 如果 \meta{bool} 
  为 \cs{c_true_bool}, 则保留行尾的空格, 否则不保留. \textbf{注意}: \meta{seq} 中的内容对应的 catcode 不变, 且此命令仅在当前组生效.
\end{function}


\begin{function}[updated=2025-01-05]{\ztool_gread_file_as_seq:nnN, \ztool_gread_file_as_seq:neN, \ztool_gread_file_as_seq:nnc, \ztool_gread_file_as_seq:nec}
  \begin{syntax}
    \cs{ztool_read_file_as_seq:nnN} \marg{bool}\marg{file}\meta{seq}
  \end{syntax}
  此命令用于读取文件 \meta{file} 的内容, 并将其存放在 \meta{seq} 中, 如果 \meta{file} 不存在, 则 \meta{seq} 会被置为空.
  \meta{bool} 用于控制是否保留行尾的空格, 可选值有:\cs{c_true_bool}, \cs{c_false_bool}; 如果 \meta{bool} 
  为 \cs{c_true_bool}, 则保留行尾的空格, 否则不保留. \textbf{注意}: \meta{seq} 中的内容对应的 catcode 不变, 且此命令仅在当前组生效.
\end{function}


\begin{function}[added=2025-05-27]{
  \ztool_write_seq_to_file:nNn, 
  \ztool_write_seq_to_file:nNe, 
  \ztool_write_seq_to_file:nNV}
  \begin{syntax}
    \cs{ztool_write_seq_to_file:nNn} \marg{bool}\meta{seq}\marg{file}
  \end{syntax}
  此命令用于将 \meta{seq} 按行写入到文件 \meta{file} 中, 如果 \meta{file} 不存在, 则会创建一个名为 \meta{file} 的
  新文件; 若 \meta{file} 已经存在, 则可以使用 \meta{bool} 控制当前的写入模式: \meta{bool} 为 \cs{c_true_bool} 时, 
  \textcolor{red}{\sffamily 覆盖写入}; \meta{bool} 为 \cs{c_false_bool} 时, 追加写入; 如果 \meta{seq} 为空, 则不会进行任何操作.
\end{function}


\begin{function}[updated=2025-01-05]{\ztool_append_to_file:nn, \ztool_append_to_file:no, \ztool_append_to_file:nf, \ztool_append_to_file:ee}
  \begin{syntax}
    \cs{ztool_append_to_file:nn} \marg{file}\marg{content}
  \end{syntax}
  此命令用于将 \meta{content} 追加到文件 \meta{file} 中, 如果 \meta{file} 不存在, 则会创建一个名为 \meta{file} 的
  新文件, 并将 \meta{content} 写入其中.
\end{function}


\begin{function}[updated=2025-01-05]{\ztool_replace_file_line:nnn, \ztool_replace_file_line:enn, \ztool_replace_file_line:ene, \ztool_replace_file_line:eee}
  \begin{syntax}
    \cs{ztool_replace_file_line:nnn} \marg{file}\marg{line}\marg{content}
  \end{syntax}
  此命令用于将文件 \meta{file} 中的第 \meta{line} 行替换为 \meta{content}, 如果 \meta{file} 不存在, 则不会进行任何操作.
\end{function}


\begin{function}[updated=2025-01-05]{\ztool_insert_to_file:nnn, \ztool_insert_to_file:nen, \ztool_insert_to_file:nfn, \ztool_insert_to_file:een}
  \begin{syntax}
    \cs{ztool_insert_to_file:nnn} \marg{file}\marg{line}\marg{content}
  \end{syntax}
  此命令用于将 \meta{content} 插入到文件 \meta{file} 的第 \meta{line} 行之前, 如果 \meta{file} 不存在, 则不会进行任何操作.
\end{function}


下面一个示例展示了如何使用 \pkg{ztool} 中的几个文件 IO 操作命令:
\begin{DocExample}*
\ExplSyntaxOn
\ztool_file_new:nn {\c_true_bool}{testIO.txt}
\seq_new:N \l_ztool_tmp_seq \seq_clear:N \l_ztool_tmp_seq
\ztool_append_to_file:nn {testIO.txt} {|APPEND-CONTENT|}
\ztool_insert_to_file:nnn {testIO.txt} {1} {|INSERT-~-CONTENT|}
\ztool_append_to_file:nn {testIO.txt} {|APPEND-CONTENT-II|}
\ztool_replace_file_line:nnn {testIO.txt} {3} {|REPLACE-CONTENT|}
\ztool_gread_file_as_seq:nnN {\c_false_bool} {testIO.txt} \l_ztool_tmp_seq
\seq_use:Nn \l_ztool_tmp_seq {\par}
\ExplSyntaxOff
\inputminted{text}{testIO.txt}
\end{DocExample}


\clearpage
\section{盒子操作}
本部分介绍 \pkg{ztool} 中实现的 Box 操作, 包括 box 的测量以及 box 的简单变换.


\begin{function}[updated=2024-12-05]{\ztool_get_ht:Nn, \ztool_get_ht:Ne, \ztool_get_ht:ce}
  \begin{syntax}
    \cs{ztool_get_ht:Nn} \meta{dim}\marg{content}
  \end{syntax}
  此命令用于将 \meta{content} 的高度保存在 \meta{dim} 这一寄存器中.
\end{function}

\begin{function}[updated=2024-12-05]{\ztool_get_ht_plus_dp:Nn, \ztool_get_ht_plus_dp:Ne, \ztool_get_ht_plus_dp:ce}
  \begin{syntax}
    \cs{ztool_get_ht:Nn} \meta{dim}\marg{content}
  \end{syntax}
  此命令用于将 \meta{content} 的高度和深度的和保存在 \meta{dim} 这一寄存器中.
\end{function}

\begin{function}[updated=2024-12-05]{\ztool_get_wd:Nn, \ztool_get_wd:Ne, \ztool_get_wd:ce}
  \begin{syntax}
    \cs{ztool_get_wd:Nn} \meta{dim}\marg{content}
  \end{syntax}
  此命令用于将 \meta{content} 的宽度保存在 \meta{dim} 这一寄存器中.
\end{function}

\begin{function}[updated=2024-12-05]{\ztool_get_dp:Nn, \ztool_get_dp:Ne, \ztool_get_dp:ce}
  \begin{syntax}
    \cs{ztool_get_dp:Nn} \meta{dim}\marg{content}
  \end{syntax}
  此命令用于将 \meta{content} 的深度保存在 \meta{dim} 这一寄存器中.
\end{function}


\begin{function}[updated=2024-12-05]{\ztool_gget_ht:Nn, \ztool_gget_ht:Ne, \ztool_gget_ht:ce}
  \begin{syntax}
    \cs{ztool_gget_ht:Nn} \meta{dim}\marg{content}
  \end{syntax}
  此命令用于将 \meta{content} 的高度保存在 \meta{dim} 这一寄存器中, 并且此操作是全局的.
\end{function}

\begin{function}[updated=2024-12-05]{\ztool_gget_wd:Nn, \ztool_gget_wd:Ne, \ztool_gget_wd:ce}
  \begin{syntax}
    \cs{ztool_gget_wd:Nn} \meta{dim}\marg{content}
  \end{syntax}
  此命令用于将 \meta{content} 的宽度保存在 \meta{dim} 这一寄存器中, 并且此操作是全局的.
\end{function}

\begin{function}[updated=2024-12-05]{\ztool_gget_dp:Nn, \ztool_gget_dp:Ne, \ztool_gget_dp:ce}
  \begin{syntax}
    \cs{ztool_gget_dp:nn} \meta{dim}\marg{content}
  \end{syntax}
  此命令用于将 \meta{content} 的深度保存在 \meta{dim} 这一寄存器中, 并且此操作是全局的.
\end{function}


\begin{function}[updated=2024-12-05]{\ztool_set_to_wd:nn, \ztool_set_to_wd:en, \ztool_set_to_wd:ne}
  \begin{syntax}
    \cs{ztool_set_to_wd:nn} \marg{dim}\marg{content}
  \end{syntax}
  此命令用于将 \meta{content} 的宽度调整为 \meta{dim}, 然后排版出来.
\end{function}

\begin{function}[updated=2024-12-05]{\ztool_set_to_ht:nn, \ztool_set_to_ht:en, \ztool_set_to_ht:ne}
  \begin{syntax}
    \cs{ztool_set_to_ht:nn} \marg{dim}\marg{content}
  \end{syntax}
  此命令用于将 \meta{content} 的高度调整为 \meta{dim}, 然后排版出来.
\end{function}


\begin{function}[updated=2025-04-29]{
  \ztool_autoset_to_wd_and_ht:nnn, \ztool_autoset_to_wd_and_ht:nne, 
  \ztool_autoset_to_wd_and_ht:een, \ztool_autoset_to_wd_and_ht:eee}
  \begin{syntax}
    \cs{ztool_autoset_to_wd_and_ht:nn} \marg{width}\marg{height}\marg{content}
  \end{syntax}
  此命令用于将 \meta{content} 的宽度调整为 $\min(\text{\texttt{\meta{width}}, \texttt{\meta{height}}})$,
  然后排版出来.
\end{function}


\begin{function}[added=2025-04-29]{
  \ztool_rotate:nn, \ztool_rotate:en, 
  \ztool_rotate:ne, \ztool_rotate:ee}
  \begin{syntax}
    \cs{ztool_rotate:nn} \marg{angle}\marg{content}
  \end{syntax}
  此命令用于将 \meta{content} 旋转 \meta{angle} 度, 然后排版出来.
\end{function}



\begin{function}[added=2025-04-29]{
  \ztool_scale_to_wd:nn, \ztool_scale_to_wd:en, 
  \ztool_scale_to_wd:ne, \ztool_scale_to_wd:ee}
  \begin{syntax}
    \cs{ztool_scale_to_wd:nn} \marg{dim}\marg{content}
  \end{syntax}
  此命令用于将 \meta{content} 的宽度调整为 \meta{dim}, 但是不对盒子的高度做任何的调整, 然后排版出来.
\end{function}


\begin{function}[added=2025-04-29]{
  \ztool_scale_to_ht:nn, \ztool_scale_to_ht:en, 
  \ztool_scale_to_ht:ne, \ztool_scale_to_ht:ee}
  \begin{syntax}
    \cs{ztool_scale_to_ht:nn} \marg{dim}\marg{content}
  \end{syntax}
  此命令用于将 \meta{content} 的高度 $+$ 深度整体调整为 \meta{dim}, 但是不对盒子的宽度做任何的调整, 然后排版出来.
\end{function}



\begin{function}[added=2025-04-29]{
  \ztool_scale_to_wd_and_ht:nnn, \ztool_scale_to_wd_and_ht:nno, 
  \ztool_scale_to_wd_and_ht:nne, \ztool_scale_to_wd_and_ht:eee}
  \begin{syntax}
    \cs{ztool_scale_to_wd_and_ht:nnn} \marg{width}\marg{height}\marg{content}
  \end{syntax}
  此命令用于将 \meta{content} 的宽度调整为 \meta{width}, 高度 $+$ 深度整体调整为 \meta{height},
  然后排版出来.
\end{function}


% #1:cmd, #2:width, #3:object, #4:align format(left, right, scatter, center)
\begin{function}[updated=2025-05-13]{
  \ztool_box_item_align:Nnnn, \ztool_box_item_align:cnnn,
  \ztool_box_item_align:Nnno, \ztool_box_item_align:cnno, 
  \ztool_box_item_align:Nnen, \ztool_box_item_align:Nnee }
  \begin{syntax}
    \cs{ztool_box_item_align:Nnnn} \meta{cmd}\marg{width}\marg{content}\marg{align}
  \end{syntax}
  此命令用于将 \meta{content} 的宽度调整为 \meta{width}, 然后排版出来, \meta{align} 用于控制对齐方式, 可选值有:
  \texttt{left, center, right, scatter}. \meta{cmd} 为一个命令,其接受一个参数,它将应用到 \meta{content} 
  的每一个 Token 上. \textbf{注意}: \meta{content} 中的空格会被忽略, 如果需要空格,请使用 ``\texttt{\char92\vsp}'' 
  或 ``\;\,\texttt{\~}'' 替代.
\end{function}


\begin{function}[added=2025-05-12]{\ztool_fp_to_rad:n}
  \begin{syntax}
    \cs{ztool_fp_to_rad:n} \marg{angle}
  \end{syntax}
  此命令用于将 \meta{angle} 从弧度制转换为角度制.
\end{function}



\begin{function}[added=2025-05-12]{
  \ztool_affine_transformation:Nnnnn,
  \ztool_affine_transformation:Neeee,
  \ztool_affine_transformation:cnnnn,
  \ztool_affine_transformation:ceeee}
  \begin{syntax}
    \cs{ztool_affine_transformation:Nnnnn} \meta{coffin}\marg{a}\marg{b}\marg{c}\marg{d}
  \end{syntax}
  此命令用于对 \meta{coffin} 进行任意的仿射变换(线性变换), 具体的使用方法可以参见前述的 \cmd{ztoolboxaffine} 命令; 
  上述参数对应的仿射变换矩阵 $\Lambda$ 为
  \[\Lambda = \begin{bmatrix}
    a & c \\
    b & d
  \end{bmatrix}.\]
\end{function}



{\sffamily\color{red} 关于上述函数 \cmd{\ztool_affine_transformation:Nnnnn} 的一些技术细节}: 给定任意一个仿射
变换 $\Lambda$, 不妨设
\[
  \Lambda = \begin{bmatrix}
    A_{11} & A_{12} \\
    A_{21} & A_{22}
  \end{bmatrix}.
\]
我们可以做如下的分解(与 SVD 分解类似), 令 $m=2x$, 则有:
\begin{align}\label{eq:affine-matrix}
  \Lambda 
  & = \begin{bmatrix} A_{11} & A_{12} \ \\ A_{21} & A_{22} \end{bmatrix} 
    = \begin{bmatrix} \cos \theta & -\sin \theta \\ \sin \theta & \cos \theta \end{bmatrix} 
    \begin{bmatrix} 1 & m \\ 0 & 1 \end{bmatrix} 
    \begin{bmatrix} s_x & 0 \\ 0 & s_y \end{bmatrix} \notag \\
  & = \begin{bmatrix} \cos \theta & -\sin \theta \\ \sin \theta & \cos \theta \end{bmatrix}
    \begin{bmatrix}\cos \phi&-\sin\phi\\\sin\phi&\cos\phi\end{bmatrix}
    \begin{bmatrix}S_x&0\\0&S_y\end{bmatrix}
    \begin{bmatrix}\cos \omega&-\sin\omega\\\sin\omega&\cos\omega\end{bmatrix}
    \begin{bmatrix} s_x & 0 \\ 0 & s_y \end{bmatrix}.
\end{align}

我们给出如下的记号:
\begin{itemize}
  \item $\mathbf{T}_1(\theta)$: 旋转矩阵, 绕原点逆时针旋转 $\theta$ 角;
  \item $\mathbf{T}_2(x)$: 缩放矩阵, 把 $x$ 轴方向的所有向量变为原来的 $x$ 倍;
  \item $\mathbf{T}_3(y)$: 缩放矩阵, 把 $y$ 轴方向的所有向量变为原来的 $y$ 倍;
\end{itemize}

那么我们可以认为 $\{\mathbf{T}_1(\theta), \mathbf{T}_2(x), \mathbf{T}_3(y)\}$ 就是 $A_{2\times 2}$ 的基.
所以我们可以把上面的 \cref{eq:affine-matrix} 写成如下表达式:
\begin{align}\label{eq:affine-matrix-res}
  \Lambda = \mathbf{T}_1(\theta) \cdot \mathbf{T}_1(\phi)
  \cdot \mathbf{T}_2(S_x)\cdot \mathbf{T}_3(S_y)
  \cdot \mathbf{T}_1(\omega)\cdot \mathbf{T}_2(s_x)\cdot \mathbf{T}_3(s_y).
\end{align}

根据矩阵乘法的结果, 我们可以知道上述的 $m, s_x, S_x, \phi$ 等参数如下:
\begin{align*}
  s_x = \sqrt{A_{11}^2 + A_{21}^2},\qquad \theta = \arctan\left( \frac{A_{21}}{A_{11}} \right).
\end{align*}
$s_y$ 和 $m$ 的求解结果如下:
\begin{align*}
  ms_y = A_{12}\cos\theta + A_{22}\sin\theta,\qquad
  s_y = \left\{\begin{aligned}
    & \frac{m s_y\cos\theta - A_{12}}{\sin\theta} && \text{ 如果 } \sin\theta \neq 0, \\
    & \frac{A_{22} - m s_y\sin\theta}{\cos\theta} && \text{ 如果 } \sin\theta = 0;
  \end{aligned}\right. 
\end{align*}
那么此时很容易知道 $m = ms_y / s_y$. 对 shear matrix 的分解结果如下:
\begin{align*}
  & S_x = \sqrt{\frac{m^2}4+1} - \frac{m}2, \qquad
    S_y = \sqrt{\frac{m^2}4+1} + \frac{m}2, \\
  & \phi = -\frac{\pi}{4} - \frac{1}{2}\arctan(\frac{m}2), \qquad
    \omega = \frac{\pi}{4} - \frac{1}{2}\arctan(\frac{m}2).
\end{align*}

最后我们只需要从右到左将这一系列的变换应用到 \meta{box} 上即可. 从上面也可以看出, 命令 \cmd{\ztool_affine_transformation:Nnnnn} 
仅依赖于 \LaTeX3 中的 \cs{coffin_scale:Nnn} 和 \cs{coffin_rotate:Nn} 两个函数. 命令 \cmd{\ztool_affine_transformation:Nnnnn} 实现
过程中相关的参考链接如下:
\begin{itemize}
  \item \texttt{https://math.stackexchange.com/a/3521141/1235323};
  \item \texttt{https://math.stackexchange.com/a/281087/1235323}.
\end{itemize}

\begin{leftbar}%
\noindent 如果原 \TeX{} 引擎提供了 shear transformation 相关的 primitive, 那么上述
对 shear matrix 的分解就是不必要的. 部分的引擎中原始提供了仿射变换矩阵这一 primitive, 
比如 \hologo{pdfTeX} 中的 \cmd{\pdfsetmatrix} 命令.
\end{leftbar}



下面的示例展示了如何使用这一章节中的几个 Box 操作命令:
\begin{DocExample}*
\ExplSyntaxOn
\setlength{\fboxsep}{0pt}
% get dim of content
\dotfill\par
\dim_new:N \l_ztool_tmp_H_dim
\dim_new:N \l_ztool_tmp_W_dim
\ztool_get_ht:Nn \l_ztool_tmp_H_dim {Hello,~world!}
\ztool_get_wd:Nn \l_ztool_tmp_W_dim {Hello,~world!}
\dim_use:N \l_ztool_tmp_H_dim \quad \dim_use:N \l_ztool_tmp_W_dim\par

% set content to dim
\dotfill\par
Hello,~world|
\ztool_set_to_ht:nn {.5cm} {Hello,~world}|
\ztool_set_to_wd:nn {40pt} {Hello,~world}\par

% scale one dimension
\dotfill\par
\ztool_scale_to_wd:nn {2em}{\fbox{AA}}\par
\ztool_scale_to_wd:nn {2em}{\fbox{AAA}}\par
\ztool_scale_to_wd:nn {2em}{\fbox{AAAAA}}\par
\ztool_scale_to_ht:nn {2.5em}{\fbox{\vbox{\hbox{A}}}}\quad
\ztool_scale_to_ht:nn {2.5em}{\fbox{\vbox{\hbox{A}\hbox{A}}}}\quad
\ztool_scale_to_ht:nn {2.5em}{\fbox{\vbox{\hbox{A}\hbox{A}\hbox{A}}}}\quad
\ztool_scale_to_ht:nn {2.5em}{\fbox{\vbox{\hbox{A}\hbox{A}\hbox{A}\hbox{A}}}}\par

% box item align
\dotfill\par
\def\boxItemCmd#1{\textcolor{blue}{|#1|}}
\underline{
  \ztool_box_item_align:Nnnn \boxItemCmd{15em}{{Tom}{Amy}{Jennery}}{scatter}
}\par
\underline{
  \ztool_box_item_align:Nnnn \boxItemCmd{15em}{{Tom} {Amy}\ {Jennery}}{center}
}\par 

% affine transform
\dotfill\par
\hcoffin_set:Nn \l_tmpa_coffin {\rule{2em}{2em}}
\coffin_typeset:Nnnnn \l_tmpa_coffin {l}{b}{0pt}{0pt}
\ztool_affine_transformation:Nnnnn \l_tmpa_coffin {1}{0}{.5}{1}
\coffin_typeset:Nnnnn \l_tmpa_coffin {l}{b}{0pt}{0pt}
\ExplSyntaxOff
\end{DocExample}



\clearpage
\section{zdraw}
这部分主要包含一些图像绘制命令, 这系列的命令并不依赖于 \pkg{tikz} 宏包, 它们的主要依赖项如下: 
\begin{itemize}
  \item \hologo{LaTeX2e} 内置 \env{picture} 环境;
  \item \pkg{pict2e} : \hologo{LaTeX2e} 内置 \env{picture} 环境的增强版, 提供了更好的绘图功能;
  \item \pkg{bxeepic}: 可以用于提供 dash line 支持, 目前还未引入该宏包.
\end{itemize}


\begin{function}[added=2025-05-13]{zpic}
  \begin{syntax}
    \cs{begin}\{zpic\}\oarg{key-value} \meta{draw commands} \cs{end}\{zpic\}
  \end{syntax}
  此环境基于 \hologo{LaTeX2e} 内置 \env{picture} 环境定义, 
\end{function}

\begin{keyval}[parent=ztool/draw/picture]{unit, width, height, xoffset, yoffset, opacity-color}
  \begin{syntax}
    unit    = \meta{长度}>\dval{1cm}
    width   = \meta{浮点数}>\dval{0}
    height  = \meta{浮点数}>\dval{0}
    xoffset = \meta{浮点数}>\dval{0}
    yoffset = \meta{浮点数}>\dval{0}
    opacity-color = \meta{颜色}>\dval{white}
  \end{syntax}
  上述的 \meta{opacity-color} 选项用于设置当前 \env{zpic} 环境中的``透明''色彩, 也就是和当前文档默认背景色相同的色彩; 
  所以可能会出现 \meta{opacity-color} 覆盖到其它 object 上的情况.
\end{keyval}


\begin{function}[added=2025-05-13]{\put}
  \begin{syntax}
    \cs{put} \parg{x, y} \marg{content}
  \end{syntax}
  此命令即为 \hologo{LaTeX2e} 内置 \env{picture} 环境中的 \cs{put} 命令.
  \textbf{注意}: 此命令需要在 \env{picture} 或 \env{zpic} 环境中使用.
\end{function}


\begin{function}[added=2025-05-13]{\zline}
  \begin{syntax}
    \cs{zline} \oarg{key-value}\parg{coor-1}\parg{coor-2}
  \end{syntax}
  此命令用于绘制一条从 \meta{coor-1} 到 \meta{coor-2} 的线段, \meta{key-value} 用于设置线条的属性,
  可用选项请参见后续的 \meta{parent=ztool/draw/picture/line}.
\end{function}


\begin{keyval}[parent=ztool/../line]{draw, width, dash}
  \begin{syntax}
    draw  = \meta{颜色}>\dval{black}
    width = \meta{长度}>\dval{.4pt}
    dash  = \meta{true|\textbf{false}}>\dval{false}
  \end{syntax}
  上述 \meta{width} 用于设置线条的宽度, \meta{draw} 用于设置线条的颜色, \meta{dash} 用于设置线条是否为虚线.
  \textbf{注意}: 目前 \meta{dash} 选项还未适配, 处于不可用的状态.
\end{keyval}


\begin{function}[added=2025-05-13]{\zvector}
  \begin{syntax}
    \cs{zvector} \oarg{key-value}\parg{coor-1}\parg{coor-2}
  \end{syntax}
  此命令用于绘制向量, 该向量的起点为 \meta{coor-1}, 终点为 \meta{coor-2};
  \meta{key-value} 用于设置该向量的外观属性, 其继承自 \meta{parent=ztool/draw/picture/line}, 
  其余的可用选项请参见后续 \meta{parent=ztool/draw/picture/line/vector}.  
\end{function}


\begin{keyval}[parent=ztool/../vector]{>}
  \begin{syntax}
    \texttt{\char62} = \meta{\textbf{latex}|pst}>\dval{latex}
  \end{syntax}
  此选项用于控制箭头的样式, 默认为 \LaTeX{} 样式, 即 \cs{ltxarrows}; 
  \meta{pst}, 即 PsTricks, 对应于 \cs{pstarrows} 命令.
\end{keyval}



\begin{function}[added=2025-05-13]{\zdraw}
  \begin{syntax}
    \cs{zdraw} \oarg{key-value}\parg{coor-1}...\parg{coor-n};
  \end{syntax}
  此命令将绘制一条从点 \meta{coor-1} 到点 \meta{coor-n} 的折线段, \meta{key-value} 继承自 \meta{parent=ztool/draw/picture/line}, 
  可以用于设置线条的属性, 额外可用的选项请参见后续的 \meta{parent=ztool/draw/picture/zdraw}.\par 
  \textbf{注意}: 此命令末尾的 ``;'' 是不能省略的, 否则会报错. 
\end{function}


\begin{keyval}[parent=ztool/../zdraw]{vector, cycle, fill, shift}
  \begin{syntax}
    vector = \meta{\textbf{false}|true}>\dval{false}
    cycle  = \meta{\textbf{false}|true}>\dval{false}
    fill   = \meta{\textbf{false}|true|颜色}>\dval{false}
    shift  = \marg{浮点数, 浮点数}>\dval{\{0, 0\}}
  \end{syntax}
  当 \meta{fill} 设置为 \texttt{true} 时, \meta{cycle} 会自动设置为 \texttt{true}; 
  \meta{vector} 用于设置是否将每一个子线段替换为向量. \meta{shift} 分别表示 $x$ 和 $y$ 方向的偏移量. 
  \textbf{注意}: \meta{shift} 选项中的 \texttt{\{\}} 不能省略.
\end{keyval}


\begin{function}[added=2025-05-13]{\zarc}
  \begin{syntax}
    \cs{zarc}\oarg{key-value}\parg{浮点数, 浮点数}
  \end{syntax}
  此命令用于绘制一个圆弧, \texttt{\parg{浮点数, 浮点数}} 为其圆心, 默认绘制 $\frac{1}{4}$ 圆弧; 
  \meta{key-value} 继承自 \meta{parent=ztool/draw/picture/line}, 
  可以用于设置线条的属性, 额外可用的选项请参见后续的 \meta{parent=ztool/draw/picture/zarc}.
\end{function}


\begin{keyval}[parent=ztool/../zarc]{radius, start, end, fill}
  \begin{syntax}
    radius = \meta{浮点数}>\dval{.5}
    start  = \meta{浮点数}>\dval{0}
    end    = \meta{浮点数}>\dval{90}
    fill   = \meta{\textbf{false}|true|颜色}>\dval{false}
  \end{syntax}
  \meta{start} 按照逆时针旋转到角度 \meta{end} 结束; \meta{radius} 为圆弧的半径;
  \meta{fill} 用于设置圆弧的填充颜色.
\end{keyval}


\begin{function}[added=2025-05-13]{\zcircle}
  \begin{syntax}
    \cs{zcircle}\oarg{key-value}\parg{浮点数, 浮点数}
  \end{syntax}
  此命令基于上述的 \cs{zarc} 命令, 默认情况下将以 \texttt{\parg{浮点数, 浮点数}} 为圆心绘制一个完整的圆;
  \meta{key-value} 和上述的 \cs{zrac} 命令中的 \meta{key-value} 选项相同,
\end{function}


\begin{function}[added=2025-05-13]{\zrectangle}
  \begin{syntax}
    \cs{zrectangle}\oarg{key-value}\parg{coor-1}\parg{coor-2}
  \end{syntax}
  此命令用于绘制矩形, \texttt{\parg{coor-1}} 和 \texttt{\parg{coor-2}} 为矩形对角线的两个端点坐标;
  \meta{key-value} 继承自 \meta{parent=ztool/draw/picture/line}, 其余的 \meta{key-value} 
  请参见后续 \meta{parent=ztool/draw/picture/zrectangle}.
\end{function}


\begin{keyval}[parent=ztool/../zrectangle]{arc, fill}
  \begin{syntax}
    arc  = \meta{浮点数}>\dval{0}
    fill = \meta{\textbf{false}|true|颜色}>\dval{false}
  \end{syntax}
  \meta{fill} 用于设置矩形的填充颜色, \meta{arc} 用于设置矩形圆角对应的半径.
\end{keyval}


\newpage
下面给出一些绘图示例, 方便读者理解上述绘图命令的基本使用方法:
\begin{DocExample}*
\mbox{}\vskip2em
\begin{zpic}[unit=2em]
  \zdraw[fill, cycle] (0, 0)(1, 0)(1, 1)(0, 1);
  \zdraw[cycle, shift={2, 0}] (0, 0)(1, 0)(1, 1)(0, 1);
  \zdraw[fill, shift={4, 0}] (0, 0)(1, 0)(1, 1)(0, 1);
  \zdraw[draw=red, width=1pt, shift={6, 0}] (0, 0)(1, 0)(1, 1)(0, 1);
  \zdraw[vector, shift={8, 0}] (0, 0)(1, 0)(1, 1)(0, 1);
  \zdraw[vector, cycle, shift={10, 0}] (0, 0)(1, 0)(1, 1)(0, 1);
  \zdraw[vector, fill, shift={12, 0}] (0, 0)(1, 0)(1, 1)(0, 1);
  \zdraw[vector, cycle, fill, shift={14, 0}] (0, 0)(1, 0)(1, 1)(0, 1);
\end{zpic}

\vskip2cm
\begin{zpic}[unit=2cm, xoffset=2]
  % 1. rectangle
  \zrectangle[arc=.1, fill=gray!20](0, 0)(2, 1)
  \zrectangle[draw=green, width=1pt](.5, .25)(1.5, .75)
  % 2. line / vecter
  \zline[width=3pt, draw=red](0, .5)(2, .5)
  \zvector[>=pst](0, 0)(1, 1)
  \zvector[draw=blue, width=2pt](1, 1)(2, 0)
  % 3. arc / circle
  \zarc[draw=blue, end=45](0, 0) % fill=<empty>
  \zarc[draw=blue, width=2pt, end=15, fill=, draw=red](0, 0)
  \zcircle[radius=.25, fill, draw=purple](1, .5)
  \zcircle[radius=.25, fill=orange, draw=none](1.5, 1)
  \zcircle[radius=.25, fill=red, draw=](2, .5)
\end{zpic}
\end{DocExample}


\clearpage
\section{TODO}
\pkg{ztool} 在将来也许会有改动, 这里列出部分将来可能会完善的功能%
(\undone{} -- 未完成; \done{} -- 已完成; \wontfix{} -- 不考虑该功能):


\let\olditem\item
\RenewDocumentCommand{\item}{so}
  {
    \IfValueTF{#2}
      {\color{black}\def\checkmark{\IfBooleanTF{#1}{\wontfix}{\done}}% 
        \olditem\IfBooleanTF{#1}{(#2)}{\IfValueT{#2}{#2-}已完成:}\color{gray}}
      {\color{black}\def\checkmark{\IfBooleanTF{#1}{\wontfix}{\undone}}%
        \olditem}
  }
\begin{todolist}
  \item 重新实现 \pkg{xsimverb} 宏包中的 \cs{xsim_file_write_start:nn} 和 \cs{xsim_file_write_stop:} 命令, 使其和 \pkg{ztool} 宏包适配.
  \item[2025-05-22] 修复 \cs{ztool_append_to_file:nn} 文件首行空行的问题.
\end{todolist}



% ----------------------------------------------------------------------
%                              Implement
% ----------------------------------------------------------------------
\cleardoublepage
\newgeometry{left=1in, top=0pt, right=.9in, bottom=0pt}
\ztexDocPrintSource


\newgeometry{left=1in, top=.75in, right=.9in, bottom=.75in}
\renewcommand\indexname{索引}
\PrintIndex
\end{document}