\documentclass[aspectratio=169]{beamer}
\usepackage[
  % lang=cn,
  minted,
  % serif=math,
  block={split, rounded},
]{../code/zslide}
\usepackage{lipsum}
\usepackage{setspace}

\title{Beamer Template}
\author{Eureka}
\date{\today}
% \logo{\includegraphics[width=4em]{./Arch.png}} 
\begin{document}
\section{FIRST}
\subsection{FIRST-I}
  \begin{frame}%[plain]
    \titlepage
  \end{frame}
  
  \begin{frame}{Table of Contents}
    \tableofcontents
  \end{frame}

  \section{sec 1}
  \subsection{subsec 1}
  \begin{frame}
    \begin{block}{BLOCK TITLE}
      This is a simple block %你好
      \begin{align}
        \sum_{i=1}^{+\infty}{\int_{0}^{i}-\frac{1}{t}\mathrm{d}t} = \frac{\pi^2}{6}
      \end{align}
    \end{block}
  \end{frame} 
  \begin{frame}
    \begin{multicols}{2}
      \begin{itemize}
        \item hello
        \begin{itemize}
          \item hello 
          \begin{itemize}
            \item Hello 
            \item world
          \end{itemize}
          \item world
        \end{itemize} 
        \item world
      \end{itemize}

      \begin{enumerate}
        \item hello
        \begin{enumerate}
          \item hello 
          \begin{enumerate}
            \item Hello 
            \item world
          \end{enumerate}
          \item world
        \end{enumerate} 
        \item world
      \end{enumerate}
    \end{multicols}
  \end{frame}
  \subsection{subsec 2}
  \begin{frame}\setstretch{1.5}
    \begin{lemma}[Goldbach's conjecture]
      %\upshape\ %
      dass jede Zahl, welche aus zweyen numeris primis zusammengesetzt ist, ein aggregatum so vieler 
      numerorum primorum sey, 
      \begin{align}
          & \sum_{q\leq x}\frac{q}{\varphi(q)}\sum_{x_{q}}^{*}\Big|\sum_{n=M+1}^{M+N}a_{n}\chi_{q}(n)\Big|^{2}\le (X^{2}+\pi N)\sum_{n=M+1}^{M+N}\Big|u_{n}\Big|^{2} \\
          % & \sum_{D<q\leq Q}\frac{1}{\varphi(q)}\sum_{x_{q}}^{*}\Big|\sum_{n=M+1}^{M+N}a_{n}\chi_{q}(n)\Big|^{2}\ll \Big(Q+\frac{N}{D}\Big)\sum_{n=M+1}^{M+N}|a_{n}|^{2}
      \end{align}
      dass jede Zahl, welche aus zweyen numeris primis zusammengesetzt ist, ein aggregatum so vieler numerorum primorum sey,
    \end{lemma}
  \end{frame} 
  \begin{frame}
    \begin{block}{Title}
      \lipsum[1]
    \end{block}
  \end{frame} 

  \section{sec 2}
  \subsection{subsec 3}
  \subsection{subsec 4}

  \begin{frame}[containsverbatim]
  Add `containsverbatim' to framed option, then you can use `minted' package.
  \begin{minted}{c}
  #include <stdio.h>
  int main()
  {
    printf("Hello World");
    return 0;
  }
  \end{minted}
  \end{frame} 
\end{document}