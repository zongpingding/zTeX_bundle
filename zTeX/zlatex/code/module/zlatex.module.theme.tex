\ProvidesExplFile{zlatex.module.theme.tex}{2024/12/17}{1.0.0}{theme~module~for~zlatex}


%%%%%     color module for zlatex thm and structure    %%%%%
% ==> color setup
% dynamic color setup
\regex_new:N \l__zlatex_color_mode_regex
\regex_set:Nn \l__zlatex_color_mode_regex {(\cB..{1,}\cE.){2}}
\cs_new:Npn \__zlatex_color_set:n #1 {
  \regex_match:NnTF \l__zlatex_color_mode_regex {#1}{
    \definecolor{zlatex@color@\l_keys_key_str}#1
  }{
    \colorlet{zlatex@color@\l_keys_key_str}{#1}
  }
  \tl_set:ce 
    {l__zlatex_\l_keys_key_str _color_tl}
    {zlatex@color@\l_keys_key_str}
}

% all colors
% How to use the clist in "thm" module ???
\definecolor{zlatex@color@royalred}{RGB}{157, 16, 45}
\definecolor{zlatex@color@axiom}{HTML}{000000}
\definecolor{zlatex@color@definition}{HTML}{bdc3c7}
\definecolor{zlatex@color@theorem}{HTML}{27ae60}
\definecolor{zlatex@color@lemma}{HTML}{2980b9}
\definecolor{zlatex@color@corollary}{HTML}{8e44ad}
\definecolor{zlatex@color@proposition}{HTML}{f39c12}
\definecolor{zlatex@color@remark}{HTML}{c92a2a}


% ==> structure theme
\zlatex_keys_define:nn {color}{
  chapter         .tl_set:N     =  \l__zlatex_chapter_color_tl,
  chapter         .initial:n    =  { zlatex@color@royalred },  
  chapter         .code:n       =  { \__zlatex_color_set:n {#1} },
  chapter-rule    .tl_set:N     =  \l__zlatex_chapter_rule_color_tl,
  chapter-rule    .initial:n    =  { black },
  chapter-rule    .code:n       =  { \__zlatex_color_set:n {#1} },
}


% ==> index and ref theme
\zlatex_keys_define:nn {color}{
  link            .tl_set:N     =  \l__zlatex_link_color_tl,
  link            .initial:n    =  { purple },
  link            .code:n       =  { \__zlatex_color_set:n {#1} },
  cite            .tl_set:N     =  \l__zlatex_cite_color_tl,
  cite            .initial:n    =  { blue },
  cite            .code:n       =  { \__zlatex_color_set:n {#1} },
  url             .tl_set:N     =  \l__zlatex_url_color_tl,
  url             .initial:n    =  { zlatex@color@royalred },
  url             .code:n       =  { \__zlatex_color_set:n {#1} },
}


% ==> thm env themecolor
\zlatex_keys_define:nn {color}{
  % theorem-like envs (numbered)
  axiom           .tl_set:N     =  \l__zlatex_axiom_color_tl,
  axiom           .initial:n    =  { zlatex@color@axiom },
  axiom           .code:n       =  { \__zlatex_color_set:n {#1} },
  definition      .tl_set:N     =  \l__zlatex_definition_color_tl,
  definition      .initial:n    =  { zlatex@color@definition },
  definition      .code:n       =  { \__zlatex_color_set:n {#1} },
  theorem         .tl_set:N     =  \l__zlatex_theorem_color_tl,
  theorem         .initial:n    =  { zlatex@color@theorem },
  theorem         .code:n       =  { \__zlatex_color_set:n {#1} },
  lemma           .tl_set:N     =  \l__zlatex_lemma_color_tl,
  lemma           .initial:n    =  { zlatex@color@lemma },
  lemma           .code:n       =  { \__zlatex_color_set:n {#1} },
  corollary       .tl_set:N     =  \l__zlatex_corollary_color_tl,
  corollary       .initial:n    =  { zlatex@color@corollary },
  corollary       .code:n       =  { \__zlatex_color_set:n {#1} },
  proposition     .tl_set:N     =  \l__zlatex_proposition_color_tl,
  proposition     .initial:n    =  { zlatex@color@proposition },
  proposition     .code:n       =  { \__zlatex_color_set:n {#1} },
  remark          .tl_set:N     =  \l__zlatex_remark_color_tl,
  remark          .initial:n    =  { zlatex@color@remark },
  remark          .code:n       =  { \__zlatex_color_set:n {#1} },
  % proof-like envs (unnumbered)
  proof           .tl_set:N     =  \l__zlatex_proof_color_tl,
  proof           .initial:n    =  { black },
  proof           .code:n       =  { \__zlatex_color_set:n {#1} },
  exercise        .tl_set:N     =  \l__zlatex_exercise_color_tl,
  exercise        .initial:n    =  { black },
  exercise        .code:n       =  { \__zlatex_color_set:n {#1} },
  example         .tl_set:N     =  \l__zlatex_example_color_tl,
  example         .initial:n    =  { black },
  example         .code:n       =  { \__zlatex_color_set:n {#1} },
  solution        .tl_set:N     =  \l__zlatex_solution_color_tl,
  solution        .initial:n    =  { black },
  solution        .code:n       =  { \__zlatex_color_set:n {#1} },
}


% ==> unknown color key
\zlatex_keys_define:nn {color}{
  unknown         .code:n       =  { 
    \zlatex_metakey_msg_warning:nn {color}
      {link, cite, url, chapter, chapter-rule, axiom, definition, 
      theorem, lemma, corollary, proposition, remark} 
    }
}


% ==> init color theme
\DeclareHookRule{env/document/before}
  {zlatex-themecolor-setup-user}{>}{zlatex-thmptheorem-setup-inner}
\DeclareHookRule{env/document/before}
  {zlatex-themecolor-setup-user}{>}{zlatex-thmpproof-setup-inner}
\NewDocumentCommand{\zlatexColorSetup}{m}
  {\zlatex_label_hook_preamble_last:nn {zlatex-themecolor-setup-user}
    {
      \zlatex_keys_set:nn {color}{#1}
      \bool_if:NT \g__zlatex_hyperref_bool {
        \hypersetup {
          colorlinks = true,
          urlcolor   = \tl_use:N \l__zlatex_url_color_tl,
          linkcolor  = \tl_use:N \l__zlatex_link_color_tl,
          citecolor  = \tl_use:N \l__zlatex_cite_color_tl,
        }
      }
    }
  }
\@onlypreamble\zlatexColorSetup
\zlatexColorSetup{link=purple, cite=blue, url=zlatex@color@royalred}