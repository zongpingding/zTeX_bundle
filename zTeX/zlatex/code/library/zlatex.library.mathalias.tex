\ProvidesExplFile{zlatex.library.mathalias.tex}{2025/04/20}{1.0.1}{mathalias~library~for~zlatex}


%%%%%     mathalias for zlatex     %%%%%
% In case of duplicated macro name, save the original one 
% in your preamble, for example the '\FF' in package 'ascii', 
%   1. put '\let\asciiFF\FF' in your preamble to store it 
%   2. or load 'mathalias' library after 'ascii' package
\bool_gset_true:N \g__zlatex_math_alias_bool
\RequirePackage{amssymb, mathrsfs}
\RequirePackage{mathtools}
\let\oldS\S    
\let\olddiv\div
\let\oldhom\hom


% ==> Alias switch on/off
\bool_new:N \g__zlatex_math_alias_switch_bool
\bool_gset_false:N \g__zlatex_math_alias_switch_bool
\NewDocumentCommand{\zlatexMathAliasOn}{o}
  {
    \group_begin:
    \bool_gset_true:N \g__zlatex_math_alias_switch_bool
  }
\NewDocumentCommand{\zlatexMathAliasOff}{o}
  {
    \bool_gset_false:N \g__zlatex_math_alias_switch_bool
    \group_end:
  }
\NewDocumentCommand{\zlatexMathAliasError}{}
  {
    \zlatex_msg_set:nn {math-alias-cmd}{
      Math~alias~related~commands~only~available~ 
      between~'\zlatexMathAliasOn'~and~'\zlatexMathAliasOff'~
      or~in~the~environment~'zlatexMathAlias'
    }
    \zlatex_msg_error:n {math-alias-cmd}
  }
\NewDocumentEnvironment{zlatexMathAlias}{}
  { \group_begin:
    \bool_gset_true:N \g__zlatex_math_alias_switch_bool
  }{
    \bool_gset_false:N \g__zlatex_math_alias_switch_bool
    \group_end:
  }


% ==> mathalias commands setup interface
\clist_new:N \g__zlatex_mathalias_user_clist
\clist_new:N \g__zlatex_mathalias_internal_clist
\clist_gclear:N \g__zlatex_mathalias_user_clist
\clist_gclear:N \g__zlatex_mathalias_internal_clist
\cs_new:Npn \zlatex_mathalias_set:nn #1#2
  {% #1:the users' interface; #2: the internal interface
    \clist_push:Nn \g__zlatex_mathalias_user_clist {#1}
    \clist_push:Nn \g__zlatex_mathalias_internal_clist {#2}
    \seq_set_from_clist:Nn \l_tmpa_seq {#1}
    \seq_set_from_clist:Nn \l_tmpb_seq {#2}
    \seq_map_pairwise_function:NNN 
      \l_tmpa_seq \l_tmpb_seq 
      \__zlatex_math_alias_set:nn
  }
\cs_set:Npn \__zlatex_math_alias_set:nn #1#2 
  {
    % \typeout{--->\string #1:\string #2}
    \cs_set:Npn #1 
      {
        \bool_if:NTF \g__zlatex_math_alias_switch_bool
          { #2 }
          { \zlatexMathAliasError }
      }
  }


% ==> Math Font 
\newcommand{\z@R}[1]{\ensuremath{\mathrm{#1}}}
\newcommand{\z@K}[1]{\ensuremath{\mathfrak{#1}}}
\newcommand{\z@C}[1]{\ensuremath{\mathcal{#1}}}
\newcommand{\z@B}[1]{\ensuremath{\mathbb{#1}}}
\newcommand{\z@S}[1]{\ensuremath{\mathscr{#1}}}
\newcommand{\z@F}[1]{\ensuremath{\boldsymbol{#1}}}
\newcommand{\z@FF}[1]{\ensuremath{\mathbf{#1}}}
\zlatex_mathalias_set:nn
  { \R,   \K,   \C,   \B,   \S,   \F,   \FF   }
  { \z@R, \z@K, \z@C, \z@B, \z@S, \z@F, \z@FF }


% ==> Math Arrow 
% simple arrow
\prop_new:N \g_zlatex_math_simple_arrow_prop
\prop_set_from_keyval:Nn \g_zlatex_math_simple_arrow_prop 
  { % 1.double:long;  2.capital:double line; 
    % 3.neg:negation; 4.No '\nlongleftarrow', '\nLongleftarrow' etc.
    ma   = \mapsto,
    mma  = \longmapsto,
    % left arrow
    la   = \leftarrow,
    La   = \Leftarrow,
    nla  = \nleftarrow,
    Nla  = \nLeftarrow,
    lla  = \longleftarrow,
    Lla  = \Longleftarrow,
    % right arrow
    ra   = \rightarrow,
    Ra   = \Rightarrow,
    nra  = \nrightarrow,
    Nra  = \nRightarrow,
    rra  = \longrightarrow,
    Rra  = \Longrightarrow,
    % bidirectional arrow
    da   = \leftrightarrow,
    Da   = \Leftrightarrow,
    nda  = \nleftrightarrow,
    Nda  = \nLeftrightarrow,
    dda  = \longleftrightarrow,
    Dda  = \Longleftrightarrow,
  }
\prop_map_inline:Nn \g_zlatex_math_simple_arrow_prop 
  {
    \cs_new:cpn {z@#1}{#2}
  }
\zlatex_mathalias_set:nn 
  { \ma,  \mma, \la, \La, \nla, \Nla, 
    \lla, \Lla, \ra, \Ra, \nra, \Nra, 
    \rra, \Rra, \da, \Da, \nda, \Nda, 
    \dda, \Dda }
  { \z@ma,  \z@mma, \z@la, \z@La, \z@nla, \z@Nla,
    \z@lla, \z@Lla, \z@ra, \z@Ra, \z@nra, \z@Nra,
    \z@rra, \z@Rra, \z@da, \z@Da, \z@nda, \z@Nda, 
    \z@dda, \z@Dda }
% extend text arrow
\cs_new:Npn \ext_arrow_set:nn #1#2 
  { \exp_args:Nee \NewDocumentCommand{\use:c {z@#1}}{sO{}D(){}}
      {
        \IfBooleanTF{##1}
          {#2[\text{##3}]{\text{##2}}}
          {#2[##3]{##2}}
      }
  }
\keyval_parse:NNn \use_none:n \ext_arrow_set:nn 
  {
    xla  = \xleftarrow,
    Xla  = \xLeftarrow,
    xxla = \xLongleftarrow,
    xra  = \xrightarrow,
    Xra  = \xRightarrow,
    xxra = \xLongrightarrow,
    hla  = \xhookleftarrow,
    hra  = \xhookrightarrow,
  }
\zlatex_mathalias_set:nn 
  { \xla,   \Xla,   \xxla,   \xra,   \Xra,   \xxra,   \hla,   \hra }
  { \z@xla, \z@Xla, \z@xxla, \z@xra, \z@Xra, \z@xxra, \z@hla, \z@hra }


% ==> Math Operator and symbols
% REF: 1. https://en.wikipedia.org/wiki/List_of_mathematical_abbreviations
%      2. https://tex.stackexchange.com/a/289946/294585
\newcommand{\z@A}{\ensuremath{\forall}}
\newcommand{\z@E}{\ensuremath{\exists}}
\newcommand{\z@ns}{\ensuremath{\varnothing}}
\newcommand{\z@se}{\ensuremath{\backsimeq}}
\newcommand{\z@sse}{\ensuremath{\cong}}
\newcommand{\z@CC}{\ensuremath{\mathbb{C}}}
\newcommand{\z@RR}{\ensuremath{\mathbb{R}}}
\newcommand{\z@ZZ}{\ensuremath{\mathbb{Z}}}
\newcommand{\z@NN}{\ensuremath{\mathbb{N}}}
\newcommand{\z@dd}{\ensuremath{\mathchoice{\:}{\mspace{1.5mu}}{}{}\mathrm{d}}}
\zlatex_mathalias_set:nn 
  { \A, \E, \ns, \se, \sse, \CC, \RR, \ZZ, \NN, \dd }
  { \z@A, \z@E, \z@ns, \z@se, \z@sse, \z@CC, \z@RR, \z@ZZ, \z@NN, \z@dd }
% math operator alias setup
\cs_set_protected:Npn \zlatex_op_name_set:nn #1#2 
  { \exp_args:Nee \DeclareMathOperator{\use:c {z@#1}}{\exp_not:n {#2}} }
\prop_set_from_keyval:Nn \g_zlatex_math_op_prop 
  {
    alt   = alt,
    rot   = rot,
    div   = div,
    curl  = curl,
    grad  = grad,
    id    = Id,
    im    = Im,
    ker   = Ker,
    cok   = Cok,
    hom   = Hom,
    sign  = sign,
    trace = trace,
  }
\prop_map_inline:Nn \g_zlatex_math_op_prop 
  {
    \zlatex_op_name_set:nn {#1}
      { \prop_item:Nn \g_zlatex_math_op_prop {#1} }
  }
\zlatex_mathalias_set:nn 
  { \alt, \rot, \div, \curl, \grad, \id, 
    \im,  \ker, \cok, \hom,  \sign, \trace }
  { \z@alt, \z@rot, \z@div, \z@curl, \z@grad, \z@id, 
    \z@im,  \z@ker, \z@cok, \z@hom,  \z@sign, \z@trace }
% additional math spacing spec
\tl_const:Nn \c_zlatex_math_ops_tl { \cdot \wedge \times \oplus \otimes }
\clist_map_inline:nn 
  { alt, rot, div, curl, grad, id, im, ker, cok, hom, sign, trace }
  { \cs_set_protected_nopar:cpn {#1}{\use:c {z@#1} \peek_after:Nw \zlatex_op_check:} }
\cs_new_protected:Nn \zlatex_op_check: { 
  \tl_map_inline:Nn \c_zlatex_math_ops_tl {
    \token_if_eq_meaning:NNT \l_peek_token ##1 { \tl_map_break:n {{\!}} }
  }
}
\NewDocumentCommand\zlatexMathAliasOpSet{m}
  {
    \prop_put_from_keyval:Nn \g_zlatex_math_op_prop {#1}
  }
\@onlypreamble\zlatexMathAliasOpSet


% ==> pyhsics package commands (\qty, ...) implementation ???
% \qty(#1) --> \left(#1\right)
\NewDocumentCommand{\z@ab}{d()d[]d\{\}}
  { 
    \IfValueT{#1}{ \left(#1\right) }
    \IfValueT{#2}{ \left[#2\right] }
    \IfValueT{#3}{ \left\{#3\right\} }
  }
\zlatex_mathalias_set:nn { \zab }{ \z@ab }