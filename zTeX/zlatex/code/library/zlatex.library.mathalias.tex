\ProvidesExplFile{zlatex.library.mathalias.tex}{2024/10/24}{1.0.0}{mathalias~library~for~zlatex}


%%%%%     mathalias for zlatex     %%%%%
% alias command
\cs_new_protected:Npn \zlatex_math_alias:nn #1#2 {
  \cs_new:cpn {#1} {#2}
}

% alias setup
\zlatex_hook_preamble_last:n {
  \RequirePackage{amssymb, mathrsfs}
  \RequirePackage{mathtools}
  \let\oldS\S
  \let\S\undefined       
  \let\olddiv\div
  \let\oldhom\hom


  % ==> Math Font 
  \newcommand{\F}[1]{\ensuremath{\boldsymbol{#1}}}
  \newcommand{\R}[1]{\ensuremath{\mathrm{#1}}}
  \newcommand{\K}[1]{\ensuremath{\mathfrak{#1}}}
  \newcommand{\C}[1]{\ensuremath{\mathcal{#1}}}
  \newcommand{\B}[1]{\ensuremath{\mathbb{#1}}}
  \newcommand{\S}[1]{\ensuremath{\mathscr{#1}}}
  \@ifpackageloaded{ascii}
    {\let\asciiFF\FF\renewcommand{\FF}[1]{\ensuremath{\mathbf{#1}}}}
    {\newcommand{\FF}[1]{\ensuremath{\mathbf{#1}}}}
  

  % ==> Math Arrow 
  \newcommand{\ma}{\ensuremath{\longmapsto}}
  \newcommand{\la}{\ensuremath{\leftarrow}}
  \newcommand{\lla}{\ensuremath{\longleftarrow}}
  \newcommand{\ra}{\ensuremath{\rightarrow}}
  \newcommand{\rra}{\ensuremath{\longrightarrow}}
  \newcommand{\da}{\ensuremath{\longleftrightarrow}}
  \newcommand{\dda}{\ensuremath{\Longleftrightarrow}}
  \NewDocumentCommand{\xl}{O{}D(){}}{\ensuremath{\xlongleftarrow[#2]{#1}}}
  \NewDocumentCommand{\xxl}{O{}D(){}}{\ensuremath{\xLongleftarrow[#2]{#1}}}
  \NewDocumentCommand{\xr}{O{}D(){}}{\ensuremath{\xlongrightarrow[#2]{#1}}}
  \NewDocumentCommand{\xxr}{O{}D(){}}{\ensuremath{\xLongrightarrow[#2]{#1}}}
  

  % ==> Math Operator and symbols
  % REF: 1. https://en.wikipedia.org/wiki/List_of_mathematical_abbreviations
  %      2. https://tex.stackexchange.com/a/289946/294585
  \newcommand{\A}{\ensuremath{\forall}}
  \newcommand{\E}{\ensuremath{\exists}}
  \newcommand{\ns}{\ensuremath{\varnothing}}
  \newcommand{\se}{\ensuremath{\backsimeq}}
  \newcommand{\sse}{\ensuremath{\cong}}
  \newcommand{\CC}{\ensuremath{\mathrm{C}}}
  \newcommand{\RR}{\ensuremath{\mathbb{R}}}
  \newcommand{\dd}{\ensuremath{\mathchoice{\:}{\mspace{1.5mu}}{}{}\mathrm{d}}}
  % math operator alias setup
  \cs_set_protected:Npn \zlatex_op_name_set:nn #1#2 
    { \exp_args:Nee \DeclareMathOperator{\use:c {z#1}}{#2} }
  \keyval_parse:NNn \use_none:n \zlatex_op_name_set:nn {
    alt = alt,
    rot = rot,
    div = div,
    curl = curl,
    grad = grad,
    id = Id,
    im = Im,
    ker = Ker,
    cok = Cok,
    hom = Hom,
    sign = sign,
    trace = trace,
  }
  \tl_const:Nn \c_zlatex_math_ops_tl { \cdot \wedge \times \oplus \otimes }
  \clist_map_inline:nn 
    { trace, alt, rot, div, curl, grad, id, im, ker, cok, hom, sign }
    {\cs_set_protected_nopar:cpn {#1}{\use:c {z#1} \peek_after:Nw \zlatex_op_check:}}
  \cs_new_protected:Nn \zlatex_op_check: { 
    \tl_map_inline:Nn \c_zlatex_math_ops_tl {
      \token_if_eq_meaning:NNT \l_peek_token ##1 {\tl_map_break:n {{\!}}}
    }
  }


  % ==> pyhsics package commands (\qty, ...) implementation ???
}