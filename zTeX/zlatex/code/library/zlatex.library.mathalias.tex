\ProvidesExplFile{zlatex.library.mathalias.tex}{2024/12/17}{1.0.0}{mathalias~library~for~zlatex}


%%%%%     mathalias for zlatex     %%%%%
\RequirePackage{amssymb, mathrsfs}
\RequirePackage{mathtools}
\let\oldS\S
\let\S\undefined       
\let\olddiv\div
\let\oldhom\hom


% ==> Math Font 
\cs_new_protected:Npn \zlatex_math_alias:nn #1#2 
  { \cs_new:cpn {#1}{#2} }
\newcommand{\F}[1]{\ensuremath{\boldsymbol{#1}}}
\newcommand{\R}[1]{\ensuremath{\mathrm{#1}}}
\newcommand{\K}[1]{\ensuremath{\mathfrak{#1}}}
\newcommand{\C}[1]{\ensuremath{\mathcal{#1}}}
\newcommand{\B}[1]{\ensuremath{\mathbb{#1}}}
\newcommand{\S}[1]{\ensuremath{\mathscr{#1}}}
\@ifpackageloaded{ascii}
  {\let\asciiFF\FF\renewcommand{\FF}[1]{\ensuremath{\mathbf{#1}}}}
  {\newcommand{\FF}[1]{\ensuremath{\mathbf{#1}}}}


% ==> Math Arrow 
% simple arrow
\prop_new:N \g_zlatex_math_simple_arrow_prop
\prop_set_from_keyval:Nn \g_zlatex_math_simple_arrow_prop 
  {% 1.double:long;  2.capital:double line; 
    % 3.neg:negation; 4.No '\nlongleftarrow', '\nLongleftarrow' etc.
    ma   = \mapsto,
    mma  = \longmapsto,
    % left arrow
    la   = \leftarrow,
    La   = \Leftarrow,
    nla  = \nleftarrow,
    Nla  = \nLeftarrow,
    lla  = \longleftarrow,
    Lla  = \Longleftarrow,
    % right arrow
    ra   = \rightarrow,
    Ra   = \Rightarrow,
    nra  = \nrightarrow,
    Nra  = \nRightarrow,
    rra  = \longrightarrow,
    Rra  = \Longrightarrow,
    % bidirectional arrow
    da   = \leftrightarrow,
    Da   = \Leftrightarrow,
    nda  = \nleftrightarrow,
    Nda  = \nLeftrightarrow,
    dda  = \longleftrightarrow,
    Dda  = \Longleftrightarrow,
  }
\prop_map_inline:Nn \g_zlatex_math_simple_arrow_prop 
  {
    \zlatex_math_alias:nn {#1}{#2}
  }
% extend text arrow
\cs_new:Npn \ext_arrow_set:nn #1#2 
  { \exp_args:Nee \NewDocumentCommand{\use:c {#1}}{sO{}D(){}}
      {
        \IfBooleanTF{##1}
          {#2[\text{##3}]{\text{##2}}}
          {#2[##3]{##2}}
      }
  }
\keyval_parse:NNn \use_none:n \ext_arrow_set:nn 
  {
    xla  = \xleftarrow,
    Xla  = \xLeftarrow,
    xxla = \xLongleftarrow,
    xra  = \xrightarrow,
    Xra  = \xRightarrow,
    xxra = \xLongrightarrow,
    hla  = \xhookleftarrow,
    hra  = \xhookrightarrow,
  }


% ==> Math Operator and symbols
% REF: 1. https://en.wikipedia.org/wiki/List_of_mathematical_abbreviations
%      2. https://tex.stackexchange.com/a/289946/294585
\newcommand{\A}{\ensuremath{\forall}}
\newcommand{\E}{\ensuremath{\exists}}
\newcommand{\ns}{\ensuremath{\varnothing}}
\newcommand{\se}{\ensuremath{\backsimeq}}
\newcommand{\sse}{\ensuremath{\cong}}
\newcommand{\CC}{\ensuremath{\mathbb{C}}}
\newcommand{\RR}{\ensuremath{\mathbb{R}}}
\newcommand{\ZZ}{\ensuremath{\mathbb{Z}}}
\newcommand{\NN}{\ensuremath{\mathbb{N}}}
\newcommand{\dd}{\ensuremath{\mathchoice{\:}{\mspace{1.5mu}}{}{}\mathrm{d}}}
% math operator alias setup
\cs_set_protected:Npn \zlatex_op_name_set:nn #1#2 
  { \exp_args:Nee \DeclareMathOperator{\use:c {z#1}}{\exp_not:n {#2}} }
\prop_set_from_keyval:Nn \g_zlatex_math_op_prop 
  {
    alt   = alt,
    rot   = rot,
    div   = div,
    curl  = curl,
    grad  = grad,
    id    = Id,
    im    = Im,
    ker   = Ker,
    cok   = Cok,
    hom   = Hom,
    sign  = sign,
    trace = trace,
  }
\prop_map_inline:Nn \g_zlatex_math_op_prop 
  {
    \zlatex_op_name_set:nn {#1}{\prop_item:Nn \g_zlatex_math_op_prop {#1}}
  }
\tl_const:Nn \c_zlatex_math_ops_tl { \cdot \wedge \times \oplus \otimes }
\clist_map_inline:nn 
  { trace, alt, rot, div, curl, grad, id, im, ker, cok, hom, sign }
  {\cs_set_protected_nopar:cpn {#1}{\use:c {z#1} \peek_after:Nw \zlatex_op_check:}}
\cs_new_protected:Nn \zlatex_op_check: { 
  \tl_map_inline:Nn \c_zlatex_math_ops_tl {
    \token_if_eq_meaning:NNT \l_peek_token ##1 {\tl_map_break:n {{\!}}}
  }
}
\NewDocumentCommand\zlatexMathAliasOpSet{m}
  {
    \prop_set_from_keyval:Nn \g_zlatex_math_op_prop {#1}
  }
\@onlypreamble\zlatexMathAliasOpSet


% ==> pyhsics package commands (\qty, ...) implementation ???
% \qty(#1) --> \left(#1\right)
\NewDocumentCommand{\zab}{d()d[]d\{\}}
  { 
    \IfValueT{#1}{ \left(#1\right) }
    \IfValueT{#2}{ \left[#2\right] }
    \IfValueT{#3}{ \left\{#3\right\} }
  }