\ProvidesExplFile{zlatex.library.theme.tex}{2025/04/22}{1.0.1}{theme~library~for~zlatex}


%%%%%     theme library for zlatex thm and structure    %%%%%
\bool_gset_true:N \g__zlatex_theme_lib_load_bool
%% ==> preamble
\RequirePackage[many]{tcolorbox}
\RequirePackage{adjustbox}
\RequirePackage{tikz}
\RequirePackage{etoolbox}
\patchcmd{\pgfutil@InputIfFileExists}{\input #1}{
  \@pushfilename
  \xdef\@currname{#1}
  \input #1
  \@popfilename
}{}{}
\usetikzlibrary{fadings, calc}
\RequirePackage{pifont}



%% ==> thm icon interface
\prop_new:N \g__zlatex_thm_icon_prop
\prop_gclear:N \g__zlatex_thm_icon_prop
\cs_new_protected:Npn \__zlatex_thm_icon_set:n #1 
  {
    \prop_gput_from_keyval:cn {g__zlatex_thm_icon_prop}{#1}  
  }
\cs_new_protected:Npn \__zlatex_thm_icon_use:n #1
  {% #1: thm env type name
    \prop_item:cn {g__zlatex_thm_icon_prop}{#1}
  }
\cs_generate_variant:Nn \__zlatex_thm_icon_use:n {o, e}
\NewDocumentCommand{\zlatexThmIconSetup}{m}
  {
    \__zlatex_thm_icon_set:n {#1}
  }
\NewDocumentCommand{\zlatexThmIconUse}{m}
  {
    \__zlatex_thm_icon_use:n {#1}
  }
\NewDocumentCommand{\zlatexThmIconClear}{}
  {
    \prop_gclear:N \g__zlatex_thm_icon_prop
  }
\@onlypreamble\zlatexThmIconSetup



%% ==> thm additional theme
\zlatexThmStyleNew{
  % theme shadow: copy from an old book
  shadow = {
    begin = {
      \begin{tcolorbox}[
        enhanced~ jigsaw, breakable,
        top=1.5pt,  bottom=1.5pt,
        left=3pt,   right=3pt,
        boxrule=0pt, sharp~corners,
        drop~fuzzy~shadow,
        colback={\thm@temp@color!10}, 
        borderline~west={3pt}{0pt}{\thm@temp@color}
      ]
    },
    end = {\end{tcolorbox}},
    option = {
      \__zlatex_thm_title_inline:n {T}
      \__zlatex_thm_tcolorbox_warning:
    }
  },
  % theme paris from: An internet sketch book
  paris = {
    begin = {
      \begin{tcolorbox}[
        enhanced,   breakable,
        top=1.5pt,  bottom=1.5pt,
        left=3pt,   right=3pt,
        boxrule=0pt,    sharp~corners,
        colback=gray!5, drop~fuzzy~shadow,
        overlay~unbroken={
          \draw[\thm@temp@color, line~width=0.2pt] (frame.north~west)--(frame.north~east);
          \draw[\thm@temp@color, line~width=3pt] ([yshift=1.5pt]frame.north~west) -- +(2.5cm, 0);
          \node[anchor=south~east, outer~sep=0pt, text=\thm@temp@color] 
            at (\linewidth-width, 1.5pt) { \__zlatex_thm_icon_use:o {\thm@temp@name} };
        },
        overlay~first={
          \draw[\thm@temp@color, line~width=0.2pt] (frame.north~west)--(frame.north~east);
          \draw[\thm@temp@color, line~width=3pt] ([yshift=1.5pt]frame.north~west) -- +(2.5cm, 0);
        },
        overlay~last={
          \node[anchor=south~east, outer~sep=0pt, text=\thm@temp@color] 
            at (\linewidth-width, 1.5pt) { \__zlatex_thm_icon_use:o {\thm@temp@name} };
        },
      ]  
    },
    end = {\end{tcolorbox}},
    option = {
      \__zlatex_thm_title_inline:n {T}
      \__zlatex_thm_tcolorbox_warning:
    },
    preamble = {
      \__zlatex_thm_icon_set:n 
        {
          axiom       = \ding{118},
          definition  = \ding{168}, 
          theorem     = \(\heartsuit\), 
          lemma       = \ding{68},
          corollary   = \ding{168}, 
          proposition = \(\spadesuit\), 
          remark      = \ding{102}  ,
          proof       = , 
          exercise    = ,
          example     = ,
          solution    = , 
          problem     = ,
        }
    }
  },
  % elegant theme from: ElegantLaTeX Project
  elegant = {
    begin = {
      \begin{tcolorbox}[
        enhanced,   breakable,
        top=8pt,    bottom=1.5pt,
        left=3pt,   right=3pt,
        arc=3pt,    boxrule=0.5pt,
        before~upper*={\setlength{\parindent}{1em}},
        fontupper=\rmfamily,   fonttitle=\bfseries,
        lower~separated=false, separator~sign={.},
        attach~boxed~title~to~top~left={yshift=-0.11in, xshift=0.15in},
        boxed~title~style={boxrule=0pt, colframe=white, arc=0pt, outer~arc=0pt},
        title=\zlatexThmTitle*,
        coltitle = white,           colbacktitle = \thm@temp@color,
        colframe = \thm@temp@color, colback  = \thm@temp@color!5,
        overlay~unbroken~and~last = {
          \node[anchor=south~east, outer~sep=0pt, text=\thm@temp@color] 
            at (\linewidth-width, 1.5pt) { \__zlatex_thm_icon_use:o {\thm@temp@name} };
        },
      ]
    },
    end = {\end{tcolorbox}},
    option = {
      \__zlatex_thm_title_inline:n {F}
      \__zlatex_thm_tcolorbox_warning:
    },
    preamble = {
      % color
      \zlatex_keys_set:nn {color}{
        axiom       = {HTML}{2c3e50},
        definition  = {RGB}{0, 166, 82},
        theorem     = {RGB}{255, 134, 23},
        lemma       = {RGB}{255, 134, 23},
        corollary   = {RGB}{255, 134, 23},
        proposition = {RGB}{0, 173, 247},
      }
      % icon
      \__zlatex_thm_icon_set:n 
        {
          axiom       = \ding{118},
          definition  = \ding{168}, 
          theorem     = \(\heartsuit\), 
          lemma       = \ding{68},
          corollary   = \ding{168}, 
          proposition = \(\spadesuit\), 
          remark      = \ding{102}  ,
          proof       = , 
          exercise    = ,
          example     = ,
          solution    = , 
          problem     = ,
        }
    }
  },
  % obsidian theme from: obsidian plug 'Callouts'
  obsidian = {
    begin = {
      \begin{tcolorbox}[
        enhanced,   breakable,
        top=5pt,    bottom=8pt,
        left=10pt,  right=10pt,
        arc=3pt,    frame~hidden, 
        colback = \thm@temp@color!20,
      ] { \zlatexThmTitle* }\par
    },
    end = {\end{tcolorbox}},
    preamble = {
      % title format
      \zlatexThmTitleFormat*{
        \noindent\sffamily\bfseries\textcolor{\thm@temp@color}{
          \__zlatex_thm_icon_use:o {\thm@temp@name}\ \zlatexThmName :\zlatexThmNumber
          \zlatexThmNoteEmptyTF{}{\ }
          \zlatexThmNote{}{}\ \raise7pt\hbox{\rotatebox{-90}{\(\triangleright\)}}
        }
      }
      % icon
      \__zlatex_thm_icon_set:n 
        {
          axiom       = \ding{111},
          definition  = \ding{118}, 
          theorem     = \ding{169}, 
          lemma       = \ding{170},
          corollary   = \ding{168}, 
          proposition = \ding{125}, 
          remark      = \ding{46},
          proof       = , 
          exercise    = \ding{45},
          example     = ,
          solution    = \ding{45}, 
          problem     = ,
        }
    },
    option = {
      \__zlatex_thm_title_inline:n {F}
      \__zlatex_thm_tcolorbox_warning:
    }
  },
  % lapsis theme from: book 'Foundation Mathematics for the Physical Sciences'
  % NOTE: 'overlay' --> 'frame~code~app' ???
  lapsis = {
    begin = {
      \begin{tcolorbox}[
        enhanced,  breakable,
        top=1.5pt, bottom=1.5pt,
        left=2pt,  leftlower=-3pt,
        right=3pt, arc=0pt, frame~hidden,
        bicolor,   colback=\thm@temp@color!60,
        opacitybacklower=0,
        overlay~first = {
          \fill[color=\thm@temp@color!50, path~fading=east] 
            (frame.north~west)++(-\linewidth/2+width/2, 0pt) 
              rectangle 
            ($(frame.south~east)+(\linewidth/2-width/2, 0pt)$);
          \draw[color=\thm@temp@color, thick] 
            (frame.north~west)++(-\linewidth/2+width/2, 0pt)
              -- 
            ($(frame.north~east)+(\linewidth/2-width/2, 0pt)$);
        },
        overlay~last={
          \draw[color=\thm@temp@color, thick] 
            (frame.south~west)++(-\linewidth/2+width/2, 0pt)
              --  
            ($(frame.south~east)+(\linewidth/2-width/2, 0pt)$);
          \fill[color=\thm@temp@color!50, path~fading=east] 
            (frame.north~west)++(-\linewidth/2+width/2, 0pt) 
              rectangle 
            ($(frame.south~east)+(\linewidth/2-width/2, 0pt)$);
          \node[anchor=south~east, outer~sep=0pt, text=\thm@temp@color] 
            at (\linewidth-width, 0) { \__zlatex_thm_icon_use:o {\thm@temp@name} };
        },
        overlay~unbroken={
          \fill[color=\thm@temp@color!50, path~fading=east] 
            (frame.north~west)++(-\linewidth/2+width/2, 0pt) 
              rectangle 
            ($(frame.south~east)+(\linewidth/2-width/2, 0pt)$);
          \draw[color=\thm@temp@color, thick] 
            (frame.north~west)++(-\linewidth/2+width/2, 0pt) 
              --  
            ($(frame.north~east)+(\linewidth/2-width/2, 0pt)$);
          \draw[color=\thm@temp@color, thick] 
            (frame.south~west)++(-\linewidth/2+width/2, 0pt)
              --  
            ($(frame.south~east)+(\linewidth/2-width/2, 0pt)$);
          \node[anchor=south~east, outer~sep=0pt, text=\thm@temp@color] 
            at (\linewidth-width, 1.5pt) { \__zlatex_thm_icon_use:o {\thm@temp@name} };
        },
      ]\zlatex@llapnote{\zlatexThmTitle*}
    },
    end = {\end{tcolorbox}},
    option = {
      \__zlatex_thm_title_inline:n {F}
      \__zlatex_thm_tcolorbox_warning:
    },
    preamble = {
      % title foramt
      \DeclareMathSymbol{\blacktriangleleft}{\mathrel}{AMSa}{"4A}
      \zlatexThmTitleFormat*{\bfseries
        \zlatexThmName\ \zlatexThmNumber
        \zlatexThmNoteEmptyTF{}{\\}
        \zlatexThmNote{}{}
      }
      \newcommand{\zlatex@llapnote}[1]{
        \mbox{}\llap{
        \adjustbox{set~height=0pt, set~depth=0pt}{
          \parbox[t]{2.85cm}{\raggedleft #1}}\hspace*{.75em}}
      }
      % icon
      \__zlatex_thm_icon_set:n 
        {
          axiom       = \ding{111},
          definition  = \ding{118}, 
          theorem     = \ding{169}, 
          lemma       = \ding{170},
          corollary   = \ding{168}, 
          proposition = \ding{125}, 
          remark      = \ding{46},
          proof       = , 
          exercise    = \ding{45},
          example     = ,
          solution    = \ding{45}, 
          problem     = ,
        }
    }
  },
}
