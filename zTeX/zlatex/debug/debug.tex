% ==> zlatex thm icon interface test
\InputIfFileExists{zlatex-cfg.tex}{}{}
\documentclass{../code/zlatex}
\zlatexThmStyle{paris}
\zlatexloadlibrary{theme}
\usepackage{lipsum}
% \geometry{a1paper, margin=20em}

\begin{document}
\section{FIRST}
\lipsum[1]

\begin{remark}[REMARK]
  \lipsum[2][1-5]
  \begin{align}
    a^2 + b^2 = c^2
  \end{align}
  \lipsum[2][6-8]
  \begin{align*}
    a^2 + b^2 = c^2
  \end{align*}
\end{remark}

\lipsum[2-3]
\begin{proposition}[PROP]
  \lipsum[2][1-5]
  \begin{align}
    a^2 + b^2 = c^2
  \end{align}
  \lipsum[2][6-8]
  \begin{align*}
    a^2 + b^2 = c^2
  \end{align*}
\end{proposition}
\end{document}


% ==> fixed:wrong bookmark target for contents
\InputIfFileExists{zlatex-cfg.tex}{}{}
% \DocumentMetadata{} % this line + lualatex ==> cause this bug.
% \documentclass{l3doc} % works
\documentclass[hyper, class=l3doc]{../code/zlatex}


\begin{document}
\tableofcontents
\newpage

\section{A}
\subsection{A-1}
\subsection{A-2}

\newpage
\section{B}
\subsection{B-1}
\subsection{B-2}
\end{document}


% ==> add CUS struct module to zlatex
% ==> THANKS TO:https://github.com/Sophanatprime/cus
\InputIfFileExists{zlatex-cfg.tex}{}{}
% \documentclass{ctexart}
% % \usepackage{cus}
% % \usepackage{ctex}
% \def\text{Lorem ipsum dolor sit amet, consectetuer adipiscing elit. Ut purus elit, vestibulum
% ut, placerat ac, adipiscing vitae, felis. Curabitur dictum gravida mauris. Nam arcu
% libero, nonummy eget, consectetuer id, vulputate a, magna. Donec vehicula augue
% eu neque.}
% % \setuptitle[section]{
% %   runin=true,
% % }
% \ctexset{
%   section/runin = true,
%   section/aftertitle={}, % important
% }
% \usepackage{xtemplate}

% \begin{document}
% \section{FIRST}
% \text
% \end{document}

\documentclass[
  cus,
  lang=cn,
  class=book,
  hyper
]{../code/zlatex}
\usepackage{newtxtext}
\CUSLoadLibrary{box}
\usepackage{lipsum}
\usepackage{tabto}
\usepackage[hshift=0mm,vshift=0mm]{fgruler}
\setuptitle[section]{
  runin=true,
  aftertitle={},
  % name={XX, YY},
  % format+=\raggedright,
}
% \setuptitle[chapter]{numbering=false, pagestyle=}     % default latex style toc
\setuptitle[chapter]{numbering=true, pagestyle=plain} % custom style toc
\def\text{Lorem ipsum dolor sit amet, consectetuer adipiscing elit. Ut purus elit, vestibulum
ut, placerat ac, adipiscing vitae, felis. Curabitur dictum gravida mauris. Nam arcu
libero, nonummy eget, consectetuer id, vulputate a, magna. Donec vehicula augue
eu neque.}


\enablecombinedlist
\begin{document}
\frontmatter
\def\contentsname{\textsc{Contents}}
% \tableofcontents 
% \multicolplaincombinedlist[ragged,outer-sep=0pt]{\contentsname}{toc}


%%%%%% STYLE I
\makeatletter
\templatetoc[
  * = { ignore=true }, show = {chapter,section},
  section = {
    code.before=\S\sbox{\@tempboxa}{\tmcblthetitle}\needhspace{\wd\@tempboxa},
    code.after=\quad,
    code.leader=---, code.page=\tmcblthepage,
    space.left=0pt, space.right=0pt, space.hang=0pt
  },
  chapter = { 
    code.before=\par, space.before=1pt plus 1pt,
    code.leader=\tabto{5cm}, % 使得页码移动到距页面左侧 5 厘米处
    code.after=\hfill\zkern\par, % 加上 \hfill\kern0pt 使得不会出现 underfull 警告
    code.name=\ifx\tmcblthename\empty
      \makebox[1em]{\rule{1ex}{1ex}}\quad
    \else
      \tmcbl@name@
    \fi
  },
]
\makeatother


%%%%%% STYLE II
% \centerline{\huge\textbf{\contentsname}}
% \vskip2em
% \startmulticolumns[ragged,outer-sep=0pt,column-sep=2em]
% \colorlet{tocgreen}{green!65!black}
% \hypersetup{hidelinks}
% \newcommand{\tochyperpage}{\toclink{\tocthepage}}
% \makeatletter
% \tocsetstyle {chapter}
%   {}
%   {\noindent}
%   {\fparbox{\linewidth}[padding={0pt,\fboxsep},
%       border-color=tocgreen, background-color=tocgreen]
%     {\bfseries\large \raggedright \color{white}%
%       \hangindent4\ccwd \hangafter1 % 更推荐使用 list 环境或 description 环境 
%       \strut \tocifnamed{\tocthename\unskip\quad}{}\tocthetitle
%       \breakablefiller[space]\tochyperpage \strut\par }\par }
%   {\smallskip}
%   {}
% \tocsetstyle {section}
%   {\smallskip
%     \begin{list}{}{\leftmargin3\ccwd \labelsep\z@ \rightmargin 2em 
%       \itemindent-\ccwd \listparindent\itemindent
%       \topsep\z@ \partopsep\z@ \itemsep\z@ \parsep\z@ \parskip\z@}}
%   {\item \begingroup\color{tocgreen}\bfseries}
%   {\tocifnamed{\tocthename\unskip\quad}{}\tocthetitle \breakablefiller[space]%
%     \rlap{\makebox[2em][r]{\tochyperpage\;}}\par }
%   {\endgroup}
%   {\end{list}}
% \tocsetstyle {subsection}
%   {}
%   {\begingroup\color{black}\bfseries}
%   {\tocifnamed{\tocthename\unskip\quad}{}\tocthetitle \breakablefiller[dotted]%
%     \rlap{\makebox[2em][r]{\tochyperpage\;}}\par }
%   {\endgroup}
%   {}
% \makeatother 
% \specifiedtoc
% \stopmulticolumns


%%%%%% STYLE III
% \makeatletter
% \ekeysdeclarecmd\fixedwidthtext{smm}{\leavevmode@ifvmode
%   \setbox\z@\hbox{{#3}}%
%   \ifdim\dimeval{#2}>\wd\z@ 
%     \hbox to\dimeval{#2}{\IfBooleanTF{#1}{\spreadtext*{#2}{#3}}{#3\hfill}}%
%   \else
%     \resizebox{\dimeval{#2}}{\height}{#3}%
%   \fi}
% \definecolor{toccol1}{HTML}{006DAA}
% \definecolor{toccol2}{HTML}{C4D4E3}
% \newcommand*{\zhphantom}{\vphantom{好hig}}

% \tocsetstyle{chapter}{}
%   {}
%   {\begingroup\noindent \bfseries\large \fboxrule\z@ 
%     \fcolorbox{toccol1}{toccol1}{\zhphantom\color{white}%
%       \tocifnamed{\fixedwidthtext*{4\ccwd}{\tocthename\unskip}}
%         {\fixedwidthtext*{4\ccwd}{\tocthetitle}}}%
%     \toclinkbox{\fcolorbox{toccol2}{toccol2}{\zhphantom
%       \fixedwidthtext{\linewidth-4\ccwd-4\fboxsep-\@pnumwidth}
%         {\tocifnamed{\tocthetitle}{}}%
%       \makebox[\@pnumwidth][r]{\tocthepage}}}
%     \endgroup\par \medskip}
%   {\bigskip}{}
% \tocsetstyle{section}{}{}
%   {\@dottedtocline{\tocthelevel}{1.5em}{2.3em}{\tocthename\enskip\tocthetitle}
%     {\hss\toclink{\tocthepage}\hspace{\fboxsep}}}
%   {}{}
% \tocsetstyle{subsection}{}{}
%   {\@dottedtocline{\tocthelevel}{3.8em}{3.2em}{\tocthename\enskip\tocthetitle}
%     {\hss\toclink{\tocthepage}\hspace{\fboxsep}}}
%   {}{}
% \renewcommand*{\@pnumwidth}{1.3em}
% \makeatother
% \startmulticolumns[ragged,outer-sep=0pt,column-sep=1.5em]
% \specifiedtoc 
% \stopmulticolumns
\newpage% ----------> START NEW PAGE
\mainmatter
\pagestyle{fancy}


\zlatexPageMask{\rule{3em}{3em}}
\chapter{CHAPTER FIRST}
\section{FIRST}% \thispagestyle{empty}
\CusTeX{} \text

\subsection{sss-1}
\begin{align}
  a^2 + b^2 = c^2
\end{align}
\lipsum[1][6-10]

\subsection{sss-2}
\noindent\llap{|}\dashfiller [.5ex]\par
\lipsum[2]

\section{SECOND}
\text
\subsection{sss-3}
\text
\subsection{sss-4} 

\section{THIRD}
\text
\subsection{sss-5}
\text
\subsection{sss-6} 

\section{FOURTH}
\text
\subsection{sss-7}
\text
\subsection{sss-8} 

\chapter{CHAPTER SECON}
\section{FIFTH}
\text
\fvarbox[c]{3cm}[border-color={}]{可以分段的\par \verb|\lfbox|}

\subsection{sss-9}
\text
\subsection{sss-10} 

\section{SIXTH}
\text
\subsection{sss-11}
\text
\subsection{sss-12} 

\chapter{CHAPTER THIRD}
\section{SEVENTH}
\text
\subsection{sss-13}
\text
\subsection{sss-14} 

\section{EIGHTH}
\text
\subsection{sss-15}
\text
\subsection{sss-16} 

\end{document}


% ==> Test zref-clever and cleveref
\InputIfFileExists{zlatex-cfg.tex}{}{}
\documentclass[hyper, lang=cn]{../code/zlatex}
% \creflabelformat{lemma}{#2(#1)#3}
% \documentclass{article}
% \usepackage{hyperref}
% \usepackage[nameinlink]{cleveref}
% \creflabelformat{lemma}{#2[#1)#3}
% \crefname{lemma}{lemmaXX}{lemmas}
% \newtheorem{lemma}{Lemma}[section]
% \newcounter{lemma}
% \newenvironment{lemma}{\par\refstepcounter{lemma}\textbf{\thelemma}:}{}
% NOTE: without '\refstepcounter', cleveref labelformat will be lost.
% \zlatexThmCnt{share}


\begin{document}
\section{First}
Reference a single lemma: \cref{definition:cc}, \cref{axiom:aa}; Or 
multi-labels, two labels: \zcref{lemma:bb,lemma:cc}, Three labels: \cref{axiom:aa,lemma:cc,theorem:dd}.

\begin{axiom}\label{axiom:aa}
  This a axiom:
  \[a^2 + b^2 = c^2\]
\end{axiom}

\begin{lemma}\label{lemma:bb}
  This a lemma:
  \[a^2 + b^2 = c^2\]
\end{lemma}

Can NOT Reference a section using \verb|\cref{sec:test}| in cleveref package after loading \verb|\usepackage[headings]{ctex}|, 
for that the patches for headings in C\TeX{} are conflict to cleveref.

But you can ref a section item using zref-clever: \cref{sec:test}.

\begin{lemma}\label{lemma:cc}
  This a lemma II:
  \[a^2 + b^2 = c^2\]
\end{lemma}

\begin{definition}\label{definition:cc}
  This a definition III:
  \[a^2 + b^2 = c^2\]
\end{definition}

\section{SECOND}\label{sec:test}
\begin{theorem}\label{theorem:dd}
  This a theorem IV:
  \[a^2 + b^2 = c^2\]
\end{theorem}
\end{document}



% ==> cleverref bug: see:https://github.com/CTeX-org/ctex-kit/issues/725
% I shall move to 'zref-clever' instead !!!
\documentclass{article}
\usepackage{zref-clever}
\zcsetup{nameinlink, lang=EEE}
\usepackage[colorlinks,linkcolor=blue]{hyperref}
\zcLanguageSetup{english}{
  type = part,
    name-sg = XXXX,
  type = chapter,
    name-sg = chapterXXX,
  type = section,
    Name-sg = Section,
    name-sg = sectionXXXX,
    Name-pl = Sections,
    name-pl = sections,
  type = figure,
    Name-sg = FigureXXX,
}
% \zcsetup{

% }
\zcDeclareLanguageAlias{EEE}{english}
\newtheorem{lemma}{Lemma}[section]

\begin{document}
% \chapter{XXX}\label{chap:aaa}
\section{Section 1}\label{sec:section-1}
A Test figure as follows:
\begin{figure}[!htb]
\centerline{\rule{5em}{5em}}
\caption{Figure 1}
\label{fig:figure-1}
\end{figure}


% Chapter \zcref{chap:aaa} 
A reference to \zcref[S]{sec:section-1} \zcref{sec:section-1}. \zcref[S]{fig:figure-1} shows some
interesting information.
A page reference can be done with either \zcpageref{sec:section-1} or with
\zcref[page]{sec:section-1}.
A reference can also be made to multiple labels, as in \zcref{sec:section-1,fig:figure-1}.


\section{Second}
\begin{lemma}\label{lemma:aa}
  This is a lemma:
  \[a^2 + b^2 = c^2\]
\end{lemma}

Reference a lemma: \zcref{lemma:aa}.
\end{document}



% \InputIfFileExists{zlatex-cfg.tex}{}{}
% \documentclass[lang=cn]{../code/zlatex}

% \documentclass{article}
% \usepackage{ctex}
% ! Extra \endcsname.
% <recently read> \endcsname                          
% l.12 Hello world:\cref{sec:second}
\documentclass{ctexart}
\usepackage{hyperref}
\usepackage{cleveref}


\begin{document}
\section{First}
Hello world:\cref{sec:second}

\section{Second}\label{sec:second}
It me. 
\end{document}


% ==> make mathalias command only valid in a group.
\InputIfFileExists{zlatex-cfg.tex}{}{}
\documentclass[]{../code/zlatex}
% \usepackage{ascii}
\zlatexloadlibrary{mathalias}
\zlatexMathAliasOpSet{rot=Rot}
\makeatletter
% \let\asciiFF\FF
% \let\FF\z@FF
% \newcommand\internal@cmda[1]{CMD-A:#1}
% \def\MathAliasOn{\let\cmdA\internal@cmda}
% \def\MathAliasOff{\def\cmdA{UNDEFINED}}

\begin{document}
\meaning\B

% The FF is:\FF

% Hello world: $\B{R}$

\zlatexMathAliasOn
Hello world I: $\B{R}, \S{A}, \FF{B}$.

\begin{align}
  f:A\ra B \mid x\ma y\in \RR \\
  g:\im(f)\xra[\ZZ](\NN) \rot(h)
\end{align}

\meaning\FF\par
\meaning\xra
\zlatexMathAliasOff

% then the FF is:\FF

\begin{zlatexMathAlias}
  Hello world II: $\B{R}, \K{A}$
  \begin{align}
    f:A\ra B \mid x\ma y\in \RR \\
    g:\im(f)\xra[\ZZ](\NN) \rot(h)
  \end{align}
\end{zlatexMathAlias}

\ExplSyntaxOn
\clist_map_inline:Nn \g__zlatex_mathalias_user_clist {\string #1\par}
\vskip3em
\hrule

\clist_map_inline:Nn \g__zlatex_mathalias_internal_clist {\string #1\par}
\ExplSyntaxOff



%%%%% EXP3 seq TEST %%%%%
% \ExplSyntaxOn
% \let\B\mathbb 
% \prop_gput_from_keyval:Nn \l_tmpa_prop {\B=\mathbb}
% \prop_map_inline:Nn \l_tmpa_prop
%   {
%     \typeout{--->#1:#2}
%     % \exp_after:wN \cs_set:Npn \cs:w #1\cs_end: {#2} %B=\mathbb, it works !
%     % \exp_after:wN \cs_set:Npn \exp_after:wN \exp_not:N #1 {#2} % error:\inaccessible
%     % \cs_set:Npn #1 {#2} %error: \inaccessible 
%     % \exp_after:wN \xdef #1 {\exp_not:N #2}
%     % Analysis: Maybe this is the limitation of '\prop_map_inline:Nn', 
%     %           the 'key' can not be a macro ?
%   }

% use 2 seqs instead:
% \seq_set_from_clist:Nn \l_tmpa_seq {\B, \F}
% \seq_set_from_clist:Nn \l_tmpb_seq {\mathbb, \mathbf}
% \seq_map_indexed_inline:Nn \l_tmpa_seq {
%   \cs_set:Npe #2 {\seq_item:Nn \l_tmpb_seq {#1}}
% }
% \cs_set:Npn \set_alias:nn #1#2 {
%   \cs_set:Npn #1 {#2}
% }
% \seq_map_pairwise_function:NNN 
%   \l_tmpa_seq \l_tmpb_seq 
%   \set_alias:nn
% \meaning\B\par
% \meaning\F
% \ExplSyntaxOff
%%%%% EXP3 seq TEST %%%%%
\end{document}





% ==> Test proof interface
\InputIfFileExists{zlatex-cfg.tex}{}{}
\documentclass[]{../code/zlatex}

\author{Our own fastidiousness requires us to point out that the familiar adjective}
\title{Eureka}
\date{\today}
\begin{document}
\maketitle

Hello world

\end{document}




% ==> Test proof interface
\InputIfFileExists{zlatex-cfg.tex}{}{}
\documentclass[
  lang=cn,
  % fancy,
  class=book,
  % mathSpec={font=euler},
  % layout={slide, aspect=16|9},
]{../code/zlatex}
\makeatletter
\zlatexThmTitleFormat*{\color{\thm@temp@color}\bfseries【\zlatexThmName\;\zlatexThmNumber】\ }
\zlatexThmTitleFormat*[proof]{\color{\thm@proof@temp@color}\kaishu\underline{\zlatexThmName}.\;\,}
\makeatother
\zlatexThmCnt{share=true}


\title{TEST}
\author{Eureka}
\date{\today}
\begin{document}
\maketitle
\chapter{集合与映射}
\section{集合}
\begin{definition}
  设 $A, B$ 是两个集合, 如果 $A$ 中的每一个元素都是 $B$ 中的元素, 则称 $A$ 是 $B$ 的\textbf{子集}, 记作 $A \subseteq B$.
  此时也称 $B$ 包含 $A$, 记作 $B \supseteq A$. 若 $A \subseteq B$ 且 $B \subseteq A$, 则称 $A$ 与 $B$ \textbf{相等}, 记作 $A = B$.
\end{definition}

\begin{theorem}
\end{theorem}

\begin{lemma}
  $(a, b) = (c, d)$ 当且仅当 $a = c$ 且 $b = d$.
\end{lemma}
\begin{proof}
  充分性是显然的. 下证必要性. 如果 $(a, b) = (c, d)$, 那么:
  \begin{align}
    \{\{a\}, \{a, b\}\} =\{\{c\}, \{c, d\}\}
  \end{align} 
  \begin{enumerate}
    \item 若 $a = b$, 那么 $\{a\} = \{a, b\}$, 从而 $\{a\} = \{c\}$, 也就是 $a = c$.
    \item 若 $a \neq b$, 那么 $\{a\} \neq \{a, b\}$, 从而 $\{a\} = \{c, d\}$, 也就是 $a = c$.
  \end{enumerate}
\end{proof}
\end{document}





% ==> spaces in the very beginning of a thm/proof env
\InputIfFileExists{zlatex-cfg.tex}{}{}
\documentclass{../code/zlatex}



\begin{document}
Hello world 

\begin{theorem}[THM-1]
                      A theorem 1.
\end{theorem}

\begin{theorem}[THM-2]%
  A theorem 2.
\end{theorem}

\begin{proof}
  AA This is a proof 1.
\end{proof}

\begin{proof}%
              A This is a proof 2.
\end{proof}
\end{document}




% ==> bug of \zlatexThmTitleBefore
\InputIfFileExists{zlatex-cfg.tex}{}{}
\documentclass{../code/zlatex}
\zlatexThmTitleBefore[proof]{|PRF-LIKE|}


\begin{document}
Hello world

\begin{solution}
  This is solution I.
\end{solution}
\end{document}


% ==> thm detailed hooks
% if there is no indicate to a zlatex thm command or variable, 
%    then this command is related to theorem-like env NOT proof-like env.
\InputIfFileExists{zlatex-cfg.tex}{}{}
\documentclass[lang=cn]{../code/zlatex}
\zlatexThmProofHook*[solution]{
  before=\noindent\textbf{\color{red}BEFORE},
  begin=\textbf{\color{red}BEGIN},
  end=\textbf{\color{red}END},
  after=\textbf{\color{red}AFTER},
}
\zlatexThmColorSetup{proof=blue!50, solution=green!50}
% \zlatexThmTitleBefore{THM}
% \zlatexThmTitleBefore[proof]{PRF}
% \zlatexThmTitleBefore[proof]{|PRF-LIKE|}
% \zlatexThmBefore{}
% \zlatexThmBefore[proof]{\par}
\makeatletter
\zlatexThmTitleFormat*{\bfseries\color{\thm@temp@color}\zlatexThmName|\zlatexThmNote{\{}{\}}|\zlatexThmNumber\ }
\zlatexThmTitleFormat*[proof]{\color{\thm@proof@temp@color}\bfseries|:\scshape\zlatexThmName:|\ }
\makeatother

\zlatexThmNameSet{cn}{
  theorem=新定理,
  proof=证
}
\zlatexThmLang{cn}

\zlatexThmCreate{Zaxiom, Ztheorem=Thm|{HTML}{a0d911}, Zproposition=Prop|blue}
\zlatexThmCreate[proof]{Zproof, Zexample=EXAMPLE|red, Zsolution=Solution|}
\let\OzlatexThmCreate\zlatexThmCreate
\let\OzlatexThmTitleBefore\zlatexThmTitleBefore

\begin{document}
\let\zlatexThmCreate\OzlatexThmCreate
\let\zlatexThmTitleBefore\OzlatexThmTitleBefore
Hello world

\section{new thm}
\begin{Zproof}[zlatexThmCreate-1]
This is an Zproof zlatexThmCreate-1.
\end{Zproof}

\begin{Zexample}[zlatexThmCreate-2]
This is an Zexample zlatexThmCreate-2.
\end{Zexample}

\begin{Ztheorem}[zlatexThmCreate-3]
This is a Ztheorem zlatexThmCreate-3
\end{Ztheorem}


\section{theorem like}
Inline item:
\begin{theorem}
A is a theorem.%
\end{theorem}%
\begin{proposition}
  This is proposition I.
\end{proposition}
\begin{proposition}[PROP]
  A is proposition II.
\end{proposition}

\begin{remark}
  this is a remark.
\end{remark}


\section{proof like}
\begin{proof}
  This is a proof.
\end{proof}

\begin{solution}
  This is solution I.
\end{solution}

\zlatexThmTitleBefore[proof]{|PRF-LIKE|}
\begin{solution}
  This is solution II.
\end{solution}
\zlatexThmTitleBefore[proof]{\noindent}

\begin{problem}
  This is a problem.
\end{problem}

\section{peek problem}
\ExplSyntaxOn
\peek_remove_spaces:n {~|A~is~B~|}
\tl_trim_spaces:n {~|A~is~B~|}
\ExplSyntaxOff
\end{document}







% ==> thm hooks re-define
\InputIfFileExists{zlatex-cfg.tex}{}{}
\documentclass{../code/zlatex}
% \usepackage{lipsum}
% \newif\ifShowSolution
% \ShowSolutiontrue     %显示答案
% \ShowSolutionfalse  %隐藏答案
% \newenvironment{showsolu}{\ifShowSolution}{\fi} % does NOT work
% \NewDocumentEnvironment{ShowSolu}{+b}{\ifShowSolution#1\fi}{} % works
% \newcommand{\ShowSolu}[1]{\ifShowSolution#1\fi}   % works
% \AddToHook{env/solution/before}{\ifShowSolution}
% \AddToHook{env/solution/after}{\fi}
% \zlatexThmProofHook*{
%   before=BEFORE,
%   begin=BEGIN,
%   end=END,
%   after=AFTER,
%   before=\ifShowSolution,
%   after=\fi
% }
% \usepackage{version} % does NOT work
% \excludeversion{solution}
\usepackage{environ}
\AtBeginDocument{\RenewEnviron{solution}{}} % works



\begin{document}
Hello world\vskip5em

\begin{solution}
  This is a solution:
  \begin{align}
    a^2 + b^2 = c^2
  \end{align}
\end{solution}

% \begin{showsolu}
%   \textbf{This is a solution}
% \end{showsolu}\vskip3em                % do NOT work
% \ShowSolu{This is a solution}\vskip3em % works
% \begin{ShowSolu}
%   This is a solution
%   \begin{align}
%     a^2 + b^2 = c^2
%   \end{align}
% \end{ShowSolu}\vskip3em                % works


% \ifShowSolution
% \lipsum[1-2]
% \fi                                    % works                 
\end{document}



% ==> tcolorbox icon add
\documentclass{article}
\usepackage[many]{tcolorbox}
\usepackage{pifont}
\usepackage{lipsum}
\usepackage{graphicx}
\newcounter{test}


\begin{document}
Hello world.

\begin{tcolorbox}[
  enhanced,
  overlay unbroken and first = {
    \node[anchor = north east, outer sep=0pt, text=red] 
      at (frame.north east) { \stepcounter{test}\thetest:\ding{73} };
  }
]
\lipsum[1-2]
\end{tcolorbox}

Hello world. \ding{73}

% \vskip5em
\begin{tcolorbox}[
  enhanced,breakable,
  overlay unbroken and first = {
    \node[anchor = north east, outer sep=0pt, text=red] 
      at (frame.north east) { \stepcounter{test}\thetest:Example 7.12 };
  }
]
\lipsum[1-2]
\end{tcolorbox}
\end{document}



% ==> bug of thm toc add function: \__zlatex_thm_toc_add:eeoe
\InputIfFileExists{zlatex-cfg.tex}{}{}
\documentclass{../code/zlatex}
\usepackage{pifont}
\let\OzlatexThmLang\zlatexThmLang
\zlatexThmTitleFormat{\bfseries\color{red}\zlatexThmNumber|\ :|\zlatexThmNote{[}{]}:(\zlatexThmName)}


\begin{document}
\let\zlatexThmLang\OzlatexThmLang
\zlatexThmToc[1.25]

% \begin{DocExample}*
\begin{theorem}[AAA-1-$\alpha + \beta$]
  This is a theorem AAA-1.
\end{theorem}
\zlatexThmLang{fr}
\begin{theorem}[BBB-1-\ding{56}]
  This is a france theorem BBB-1.
\end{theorem}
\zlatexThmLang{en}
\begin{theorem}[CCC-1]
  This is a english theorem CCC-1.
\end{theorem}
% \end{DocExample}


% ==> BUG: \bfseries can NOT be used in 'e'-type
% \ExplSyntaxOn
% \tl_set:Ne \l_tmpa_tl {\bfseries AA} % log:! Incomplete \iffalse; all text was ignored after line 29.
% \tl_use:N \l_tmpa_tl
% \ExplSyntaxOff
\end{document}





% ==> bug of \zlatexThmToc and \zlatexPartialToc bug in 'slide' mode
\documentclass{article}
\usepackage{lipsum, titletoc}
\usepackage{pifont}
\titlecontents{subsection}
  [3.8em] % ie, 1.5em (chapter) + 2.3em
  {}
  {\contentslabel{2.3em}}
  {\hspace*{-2.3em}}
  {\titlerule*[1pc]{.}\contentspage}


\begin{document}
\input{TOC.txt}

% \contentsline {subsection}{\numberline {2.1}\ding {74}\;2.1:[DEFINITION]:({Définition})}{6}{definition.2.1}%
% \contentsline {subsection}{\numberline {2.1}\ding {98}\;2.1:[REMARK]:({Remarque})}{9}{remark.2.1}%


% \section{First Chapter}
% \lipsum[1]
% akhdsghkjk dshk  gasjhdg gajfg:
% \startcontents[sections]
% \printcontents[sections]{p}{2}{}


% \subsection{AAA}

% \stopcontents[sections]
% \section{Second Chapter}
% \subsection{First subsection}
% \section{Third Chapter}
% \subsection{Second subsection}
% \subsubsection{First subsubsection}
\end{document}

\InputIfFileExists{zlatex-cfg.tex}{}{}
\documentclass[
    lang=cn,
    fancy,
    toc={stretch=1.5},
    class=book,
    % font={config=true},
    layout={slide, aspect=16|9}
]{../code/zlatex}
\usepackage{pifont}
\usepackage{lipsum, zhlipsum}

%%%%%% zlatex config %%%%%%%
\zlatexThmStyle{lapsis} % must bofore the loading of 'theme' library
\zlatexloadlibrary{mathalias, theme}
\zlatexThmColorSetup{remark=red!75, definition=orange!75}
\newcounter{PRB}
\zlatexThmCreate{proof}{problem={PRB\ \thePRB}|red}
\zlatexPageMask[anchor=tl, position={(0pt, \zph)}]{\parbox{\paperwidth}{\color{gray}\itshape\huge\lipsum[1-4]}}
% \zlatexThmTocSymbol{
% definition=\ding{74}\;,
% remark=\ding{98}\;,
% }
\zlatexColorSetup{
definition=blue,
}
\zlatexThmLang{fr}
\zlatexThmTitleFormat*{\color{green}\zlatexThmNumber|\ :|\zlatexThmNote{[}{]}:(\zlatexThmName)}
%%%% slide setup
\zslideSetup[doc]{
bg-color=yellow!20,
text-color=red
}
\zslideSetup[toc/leftmargin]{
  chapter=2em,
  section=8em,
}
%%%%%%%%%%%%%%%%%%%%%%%%%%%    


\title{\zLaTeX{} Online}
\author{Eureka}
\date{\today}
\begin{document}
\maketitle
\frontmatter
\tableofcontents
\newpage
\mainmatter

\part{这是一个 Part}
\chapter{欢迎使用 \zLaTeX{}}
\section{定理列表}
\subsection{AAA}
\noindent 本章节涉及到的所有定理如下所示:
\zlatexThmToc[1.25]
% \input{TOC.txt}

% \contentsline {subsection}{\numberline {2.1}\ding {74}\;2.1:[DEFINITION]:({Définition})}{6}{definition.2.1}%
% \contentsline {subsection}{\numberline {2.1}\ding {98}\;2.1:[REMARK]:({Remarque})}{9}{remark.2.1}%


% \vskip5em
% \contentsline {subsection}{\numberline {2.1}\ding {74}\;2.1:[DEFINITION]:({Définition})}{6}{definition.2.1}%
% \contentsline {subsection}{\numberline {2.1}\ding {98}\;2.1:[REMARK]:({Remarque})}{9}{remark.2.1}%

% \contentsline {subsection}{\ding {74}\;{Définition}\ 2.1\ (DEFINITION)}{6}{definition.2.1}%
% \contentsline {subsection}{\ding {98}\;{Remarque}\ 2.1\ (REMARK)}{9}{remark.2.1}%
% \contentsfinish 

% BUG: uncomment the following 2 lines will cause error:
% log: ! LaTeX Error: Something's wrong--perhaps a missing \item.
% \lipsum[1][1-4]
% \zlatexPartialToc


\section{FIRST}
\lipsum[1] 
\begin{definition}[DEFINITION]
As any dedicated reader can clearly see, the Ideal of practical
reason is a representation of, as far as I know, the things in themselves;
\begin{align}
\underset{}{\mathbf{v} \bigotimes \mathbf{w}}
& = \sum_{i=1}^3\left(a_{i1}u^iv^1+a_{i2}u^iv^2+a_{i3}u^iv^3\right) \\
& = \int x \dd x = \frac12 x^2 + \R{C}
\end{align}
As any dedicated reader can clearly see, the Ideal of practical
reason is a representation of, as far as I know, the things in themselves;%
\end{definition}
\lipsum[2]

下面是一个测试所用的 problem 环境:

\begin{problem}
这是一个问题环境, 一些含无意义的废话:
\begin{align}
  \int x \dd x = \frac12 x^2 + \R{C}
\end{align}
\end{problem}

\begin{proof}
这是一个问题环境, 一些含无意义的废话:
\begin{align}
    \int x \dd x = \frac12 x^2 + \R{C}
\end{align}
这是一个问题环境, 一些含无意义的废话.
\end{proof}

% BUG: uncomment the following 2 lines will cause error:
% log: ! LaTeX Error: Something's wrong--perhaps a missing \item.
% \subsection{BUG-SUBSECTION} 
% 你好, \zLaTeX{} 

\lipsum[1][1-5]\footnote{这是一个脚注-1}\footnote{这是一个脚注-2}\par
\zhlipsum[1]
\begin{remark}[REMARK]
As any dedicated reader can clearly see, the Ideal of practical
reason is a representation of, as far as I know, the things in themselves;
\begin{align}
\underset{}{\mathbf{v} \bigotimes \mathbf{w}}
& = \sum_{i=1}^3\left(a_{i1}u^iv^1+a_{i2}u^iv^2+a_{i3}u^iv^3\right) \\
& = \int x \dd x = \frac12 x^2 + \R{C}
\end{align}
As any dedicated reader can clearly see, the Ideal of practical
reason is a representation of, as far as I know, the things in themselves;%
\end{remark}
\zhlipsum[2]
\end{document}


\InputIfFileExists{zlatex-cfg.tex}{}{}
\documentclass[lang=cn,layout={slide}]{../code/zlatex}


\begin{document}
Hello world
\end{document}






% ==> Test zLaTeX TeXPage Online
\InputIfFileExists{zlatex-cfg.tex}{}{}
\documentclass[lang=cn]{../code/zlatex}
\zlatexThmStyle{lapsis}
\zlatexloadlibrary{mathalias, theme}
\zlatexThmColorSetup{remark=orange!75, definition=green!75}
\newcounter{PRB}
\zlatexThmCreate{proof}{problem={PRB\ \thePRB}|red}
\usepackage{lipsum}
\usepackage{zhlipsum}

\usepackage{adjustbox}
% \makeatletter
% \newcommand{\zlatex@llapnote}[1]{
%   \mbox{}\llap{
%   \adjustbox{set height=0pt, set depth=0pt}{
%     \parbox[t]{2.85cm}{\raggedleft #1}}\hspace*{.75em}}
% }
% \makeatother

\title{\zLaTeX{} Online}
\author{Eureka}
\date{\today}
\begin{document}
\maketitle


\AAA
\section{定理列表}
\noindent 本章节涉及到的所有定理如下所示:
\zlatexThmToc[1.25]

\section{FIRST}
\lipsum[1] 
\begin{definition}[DEFINITION]
As any dedicated reader can clearly see, the Ideal of practical
reason is a representation of, as far as I know, the things in themselves;
\begin{align}
\underset{}{\mathbf{v} \bigotimes \mathbf{w}}
& = \sum_{i=1}^3\left(a_{i1}u^iv^1+a_{i2}u^iv^2+a_{i3}u^iv^3\right) \\
& = \int x \dd x = \frac12 x^2 + \R{C}
\end{align}
As any dedicated reader can clearly see, the Ideal of practical
reason is a representation of, as far as I know, the things in themselves;%
\end{definition}
\lipsum[2]

下面是一个测试所用的 problem 环境:

\begin{problem}
这是一个问题环境, 一些含无意义的废话:
\begin{align}
    \int x \dd x = \frac12 x^2 + \R{C}
\end{align}
\end{problem}

\subsection{FIRST-I}
你好, \zLaTeX{} 

\zhlipsum[1]
\begin{remark}[REMARK]
As any dedicated reader can clearly see, the Ideal of practical
reason is a representation of, as far as I know, the things in themselves;
\begin{align}
\underset{}{\mathbf{v} \bigotimes \mathbf{w}}
& = \sum_{i=1}^3\left(a_{i1}u^iv^1+a_{i2}u^iv^2+a_{i3}u^iv^3\right) \\
& = \int x \dd x = \frac12 x^2 + \R{C}
\end{align}
As any dedicated reader can clearly see, the Ideal of practical
reason is a representation of, as far as I know, the things in themselves;%
\end{remark}
\zhlipsum[2]
\end{document}




% ==> Test thm style warning msg
\InputIfFileExists{zlatex-cfg.tex}{}{}
\documentclass[
  % hyper,
]{../code/zlatex}
\zlatexloadlibrary{mathalias}
% \zlatexMathAliasOpSet{alt=ALT, im=IM}

\zlatexThmStyle{background}
\zlatexThmCreate{theorem}{Ztheorem=NEW-THM|orange}
% \zlatexThmColorSetup{theorem=blue!50} % EXPETC: works ---> pass
\zlatexThmColorSetup{Ztheorem=blue!50}  % EXPETC: works ---> pass
% \zlatexThmColorSetup{link=blue!50}    % EXPETC: error ---> pass
\begin{document}

\zlatexThmToc[1.25]
\vskip4em

HELLO 

\begin{Ztheorem}[AAA]
  This is a theorem AAA.
\end{Ztheorem}


\ExplSyntaxOn
\keys_define:nn {colorTest}{
  keyA    .tl_set:N     =  \l__zlatex_keyA_color_tl,
  keyA    .code:n       =  { \__zlatex_color_set:n {#1} },
}
\keys_set:nn {colorTest}{keyA={RGB}{55, 183, 98}}
\textcolor{zlatex@color@keyA}{This~is~a~test.}
\ExplSyntaxOff


\begin{lemma}[LEMMA]
  This is a lemma.

  \begin{align}
    \alt, \im
  \end{align}

  % \zlatexMathAliasOpSet{alt=ALT, im=IM}
  \begin{align}
    \alt, \im
  \end{align}
\end{lemma}

\newpage
\zlatexThmTocAdd[section]{name=NEW-SEC}
\zlatexThmTocStop
\begin{proposition}[PROP]
  This is a proposition.
\end{proposition}


\ExplSyntaxOn
% \ztool_read_file_as_seq:nnN {\c_true_bool}{aaa.txt}\l_tmpa_seq
% \seq_show:N \l_tmpa_seq

% \ztool_gread_file_as_seq:nnN {\c_true_bool}{texput.log}\g_tmpa_seq
% \seq_show:N \g_tmpa_seq

% \ztool_file_new:nn {\c_false_bool}{bbb.txt}
% \ztool_file_new:nn {\c_true_bool}{bbb.txt}
% \ztool_append_to_file:nn {bbb.txt}{BBB2-content}
% \ztool_replace_file_line:nnn {bbxxb.txt}{2}{BBB1-content}
\ztool_insert_to_file:nnn {bbb.txt}{2}{NEW-content}
\ExplSyntaxOff

\end{document}




% ==> Make Metropolis slide theme
\InputIfFileExists{zlatex-cfg.tex}{}{}
\documentclass[
  hyper,
  layout={slide, aspect=16|9, theme=AnnArborSpruce},
]{../code/zlatex}
\zslideThemeUse[
  UL={bg=white, text=}, 
  UR={bg=white, text=},
  BL={bg=white, text=},
  BC={bg=white, text=},
  BR={bg=white, text=},  
]{AnnArborSpruce}
\ExplSyntaxOn
\dim_gset:Nn \g_zslide_status_bar_sec_H_dim {2.2em}
\dim_gset:Nn \g_zslide_status_bar_sec_B_dim {-2.2em}
\dim_gset:Nn \g_zslide_status_info_sec_B_dim {-1.55em}
\dim_gset:Nn \g_zslide_status_info_head_B_dim {-2.5em}
% navigator
\AddToHook{shipout/foreground}{
  \_zslide_status_info:nnnn {head}{0}{1}{
    \zslideNavigateBall[
      \RPITEM[red]
    ][
      \RPITEM
    ]
  }
}
\newcommand\ITEM[1][gray]{\hspace*{-4pt}\pgfornamenthan[color=#1, scale=0.06]{39}}
\newcommand\RPITEM[1][gray]{\prg_replicate:nn {10}{\ITEM[#1]}}
\ExplSyntaxOff
\usepackage{pgfornament-han}


\title{zSlide Metropolis Theme}
\author{Eureka}
\date{\today}
\begin{document}
\maketitle


\section{SECTION AAA}
AAA

\zslideNavigateBall[\ITEM[red]][\ITEM]

\newpage
% \section{SECTION BBB}
BBB 
\newpage
% \section{SECTION CCC}
CCC 
\newpage
% \section{SECTION DDD}
DDD 

\end{document}




% ==> Test zlatex working condition
\InputIfFileExists{zlatex-cfg.tex}{}{}
\documentclass[
  % hyper,
  layout={slide, aspect=16|9, theme=AnnArborSpruce},
]{../code/zlatex}
\zslideThemeUse[
  UR={text=\zLaTeX{} Slide:\ \zslideNavigateBall},
]{AnnArborSpruce}
\zslideLogo[width=4em, exclude={0, 2, 4}]{example-image-duck}

% BEGIN TEST INTERFACE
\ExplSyntaxOn
\AddToHook{shipout/background}{
  \_zslide_status_bar:nnnn {BC}{(.5\paperwidth, -.5\paperheight)}{.2}{80pt}
}
\AddToHook{shipout/foreground}{
  \_zslide_status_info:nnnn {foot}{.66}{.75}{AAA\zslideNavigateBall[\textcolor{red}{XX}][\textcolor{gray}{XX}]}
}
\ExplSyntaxOff
% END TEST INTERFACE

\title{TEST}  
\author{Eureka}
\date{\today}
\begin{document}
\maketitle


\section{FIRST SEC}
Hello world, 

\newpage

aaa 

\newpage

bbb

\section{SECOND SEC}
this is me.

\section{THIRD SEC}
HAHA.

\end{document}



% ==> l3doc function env
\keys_define:nn { l3doc/function }
{
  TF .value_forbidden:n = true ,
  TF .code:n =
    {
      \bool_set_true:N \l__codedoc_macro_TF_bool
    } ,
  EXP .value_forbidden:n = true ,
  EXP .code:n =
    {
      \bool_set_true:N \l__codedoc_macro_EXP_bool
      \bool_set_false:N \l__codedoc_macro_rEXP_bool
    } ,
  rEXP .value_forbidden:n = true ,
  rEXP .code:n =
    {
      \bool_set_false:N \l__codedoc_macro_EXP_bool
      \bool_set_true:N \l__codedoc_macro_rEXP_bool
    } ,
  pTF .value_forbidden:n = true ,
  pTF .code:n =
    {
      \bool_set_true:N \l__codedoc_macro_pTF_bool
      \bool_set_true:N \l__codedoc_macro_TF_bool
      \bool_set_true:N \l__codedoc_macro_EXP_bool
      \bool_set_false:N \l__codedoc_macro_rEXP_bool
    } ,
  noTF .value_forbidden:n = true ,
  noTF .code:n =
    {
      \bool_set_true:N \l__codedoc_macro_noTF_bool
      \bool_set_true:N \l__codedoc_macro_TF_bool
    } ,
  added .code:n = { \__codedoc_date_set_past:Nn \l__codedoc_date_added_tl {#1} },
  updated .code:n = { \__codedoc_date_set_past:Nn \l__codedoc_date_updated_tl {#1} } ,
  deprecated .bool_set:N = \l__codedoc_macro_deprecated_bool ,
  no-user-doc .bool_set:N = \l__codedoc_macro_nodoc_bool ,
  tested .code:n = { } ,
  label .code:n =
    {
      \clist_set:Nn \l__codedoc_function_label_clist {#1}
      \bool_set_true:N \l__codedoc_no_label_bool
    } ,
  verb .value_forbidden:n = true ,
  verb .bool_set:N = \l__codedoc_names_verb_bool ,
  module .tl_set:N = \l__codedoc_override_module_tl ,
}


\documentclass{l3doc}
\usepackage{lipsum} % to generate some text



\begin{document}
\lipsum


\begin{function}{\macro}
this is a macro.
\end{function}

\begin{environment}{env}
this is an env.
\end{environment}

\begin{function}{opt}
\begin{syntax}
    opt = \meta{a}
\end{syntax}
this is an opt.
\end{function}

\PrintIndex
\end{document}



% ==> transparent bug
\InputIfFileExists{zlatex-cfg.tex}{}{}
\documentclass{../code/zlatex}
% \usepackage{tikz}
% \usepackage{transparent}

\begin{document}
% \zlatexPageMask[anchor=tr, position={(\zpw, \zph)}]{
%   \begin{tikzpicture}
%     \node[opacity=0.5, inner sep=0pt] at (current page.center) 
%       {\includegraphics{Epmmy.jpg}};
%   \end{tikzpicture}
% }

% \zlatexPageMask{\transparent{.1}\includegraphics{Epmmy.jpg}}
% Hello world

\begin{theorem}[AAA]
  This is a theorem AAA.
\end{theorem}

\zlatexThmStyleNew{
  aaa={begin=, end=, option=\zlatexThmTitleSwitch},
}
% \zlatexThmTitleSwitch
\zlatexThmStyle{aaa}

\begin{theorem}[BBB]
  This is a theorem BBB.
\end{theorem}

\end{document}


% \DocumentMetadata{}
% \documentclass{article}
\documentclass{l3doc}

\usepackage{eso-pic,graphicx,transparent,lipsum}

\begin{document}
\lipsum[1-3]

\AddToShipoutPictureFG*{%
  \AtPageLowerLeft{%
    \transparent{.6}\includegraphics[width=\paperwidth,height=\paperheight]{back.jpeg}%
  }%
}

\lipsum[4-6]

\end{document}




% ==> cleverref bug: fixed after updating
\documentclass{book}
\usepackage[UTF8, heading]{ctex}
\usepackage{cleveref}
% \InputIfFileExists{zlatex-cfg.tex}{}{}
% \documentclass[class=book, lang=cn]{../code/zlatex}

\begin{document}
\section{A}
Hello world\label{aaa}

Hello \cref{aaa}
\end{document}



% ==> l3doc cover
\InputIfFileExists{zlatex-cfg.tex}{}{}
\documentclass[lang=cn]{../code/zlatex}


\title{z\LaTeX{} 用户手册}
\author{Eureka}
\date{\today}
\begin{document}
\maketitle
\end{document}




% ==> custom thm style
\InputIfFileExists{zlatex-cfg.tex}{}{}
\documentclass{../code/zlatex}
\usepackage{lipsum}


\begin{document}
\ExplSyntaxOn
\zlatexThmStyleNew {
  test = {
    begin  = \noindent\hspace*{1em}{\zlatexThmTitle}, 
    end    = , 
    option = \__zlatex_thm_title_inline:n {F}
  }
}
\ExplSyntaxOff
\zlatexThmStyle{test}

\section{First}
This is a simple theorem style in z\LaTeX{}:
\begin{theorem}[AAA]
  This is a simple definition. \lipsum[1][1-2]
  \begin{align}
    a^2 + b^2 = c^2
  \end{align}
  \lipsum[1][6-8]
\end{theorem}

\end{document}



\InputIfFileExists{zlatex-cfg.tex}{}{}
\documentclass[
  fancy, 
  % hyper,
  mathSpec={envStyle=elegant}, 
  % layout={slide, aspect=16|9}
]{../code/zlatex}
\usepackage{lipsum}
% \zlatexThmStyle{lapsis}
% \zlatexSetup{mathSpec={envStyle=shadow}}
\zlatexloadlibrary{theme}
\zlatexThmCreate{theorem}{Ztheorem=NEW-THM|{HTML}{8e22ad}, Zproposition} 
% \zlatexThmTitleFormat{\bfseries\zlatexThmName\ \zlatexThmNumber\zlatexThmNote{(}{)}}
% \zlatexThmBefore{\quad}

\def\boomen{As any dedicated reader can clearly see, the Ideal of practical
reason is a representation of, as far as I know, the things in themselves; 
\begin{align}
\underset{}{\mathbf{v} \bigotimes \mathbf{w}} 
  & = \underset{}{\mathbf{v} \otimes \mathbf{w}}
      = \sum_{i=1}^3\sum_{j=1}^3a_{ij}u^iv^j \\[-.75em]
  & = \sum_{i=1}^3\left(a_{i1}u^iv^1+a_{i2}u^iv^2+a_{i3}u^iv^3\right) 
  \end{align}  
As any dedicated reader can clearly see, the Ideal of practical
reason is a representation of, as far as I know, the things in themselves;%
}
% \zlatexThmLang{fr}
\zlatexThmCnt{share, parent=section}
\zlatexThmTocPrefix{\ding{118}\;}
\zlatexColorSetup{
  remark=red!50,
  axiom = gray!40,
  definition  = {RGB}{219,48,122},
  theorem = teal!20
} % OK
\ExplSyntaxOn
\newcommand{\Example}{
\clist_map_inline:nn 
  {axiom, definition, theorem, lemma, corollary, proposition, remark}{
    \lipsum[1][1-2]
    \begin{##1}[\MakeUppercase{##1}]
      \boomen
      % folowing contents for 'lapsis' style
      % \tcblower
      % \boomen
    \end{##1}
    \lipsum[2][1-3]
  }
}
\newcommand{\ExampleII}{
\clist_map_inline:nn 
  {plain, leftbar, background, fancy, shadow, paris, elegant, obsidian}{
    \zlatexThmStyle{##1}
    % \Example  % NOTE: --> lapsis faild for slide note
    % \lipsum[1][1-2]
    \begin{remark}[STYLE:##1]
      \boomen
    \end{remark}
    \begin{proposition}[STYLE:##1]
      \boomen
    \end{proposition}
    \lipsum[2][1-2]
  }
}
\ExplSyntaxOff


\title{DEBUG}
\author{Eureka}
\date{\today}
\begin{document}
% \section{List of Theorems}
% \zlatexThmToc

\section{Example Env}
% \Example

\ExampleII
\end{document}





% new thm them 'lapsis'
\InputIfFileExists{zlatex-cfg.tex}{}{}
\documentclass{../code/zlatex}
\zlatexloadlibrary{theme}
\zlatexThmStyle{lapsis}
\usepackage{lipsum}
\def\AAA{\lipsum[1][1-3]
\begin{align}
  \int u\;\mathrm{d}v = uv - \int v \;\mathrm{d}u
\end{align}
\lipsum[1][1-3]}
\def\BBB{\lipsum[1][3-5]
\begin{align}
  {\frac {\mathrm{d}}{\mathrm{d}x}}{\Big (}u(x)v(x){\Big )}=v(x){\frac {\mathrm{d}}{\mathrm{d}x}}
    \left(u(x)\right)+u(x){\frac {\mathrm{d}}{\mathrm{d}x}}\left(v(x)\right)
\end{align}
\lipsum[1][1-3] }

\begin{document}
\section{Lapsis thm Theme}
\lipsum[1-2]
\begin{theorem}[TEST THM]
  \AAA 
  \tcblower
  \BBB
\end{theorem}

\lipsum[3]
\begin{definition}
\AAA
\end{definition}
\lipsum[4-7]

\begin{proposition}[PROP NEW]
  \tcblower
  \BBB
\end{proposition}
\end{document}



% ==> thm backgroud to paperwidth
\InputIfFileExists{zlatex-cfg.tex}{}{}
\documentclass{../code/zlatex}
\usepackage[many]{tcolorbox}
\tcbuselibrary{skins}
\usepackage{lipsum}
\usepackage{xcolor}
\usepackage{tikz}
\usepackage{adjustbox}
\DeclareMathSymbol{\blacktriangleleft}{\mathrel}{AMSa}{"4A}
\usetikzlibrary{fadings, calc}
\parindent0pt
\definecolor{shadecolor}{HTML}{bfbfbf}
\newenvironment{back}{
  \def\FrameCommand{\colorbox{shadecolor}}
  \MakeFramed{\advance\hsize-\width \FrameRestore}
}{\endMakeFramed }
\newcommand{\llapnote}[1]{
  \mbox{}\llap{%
  \adjustbox{set height=0pt, set depth=0pt}{%
    \parbox[t]{2.85cm}{\raggedleft #1}}\hspace*{.75em}}%
}


\begin{document}
% \begin{tcolorbox}[width=\textwidth, arc=0pt]
\lipsum[1]
% \end{tcolorbox}

\zlatexPageMask[
  anchor=tl,
  % layer=foreground,
  position={(0pt, .669\zph)}
]{\tikz\fill[color=gray!50, path fading=east] (0, 0) rectangle +(\zpw, 20em);}
% \begin{tcolorbox}[
%   standard jigsaw,
%   width=\paperwidth, 
%   arc=0pt, opacityback=0,
% ]
%   \lipsum[1]
% \end{tcolorbox}

\begin{back}%
% \adjustbox{set height=0pt, set depth=0pt}{%
%   \tikz[remember picture, overlay, baseline]%baseline=(current bounding box.north), 
%     \fill[color=red!50, path fading=east] (-3.5cm, 0pt) rectangle +(\zpw, -20em);
% }
\llapnote{\sffamily\textbf{Example} 1.1\\Integrate by Part}%
% \llapnote{\sffamily\textbf{Example}\\ 1.1|Integrate by Part\\HELLO WORLD}%
\lipsum[1][1-3]
\begin{align}
  \int u\;\mathrm{d}v = uv - \int v \;\mathrm{d}u
\end{align}
\lipsum[1][1-3]
\end{back}
\lipsum[1][3-5]
\begin{align}
  {\frac {\mathrm{d}}{\mathrm{d}x}}{\Big (}u(x)v(x){\Big )}=v(x){\frac {\mathrm{d}}{\mathrm{d}x}}
    \left(u(x)\right)+u(x){\frac {\mathrm{d}}{\mathrm{d}x}}\left(v(x)\right)
\end{align}
\lipsum[1][1-3]\hfill$\mathcolor{gray}{\blacktriangleleft}$


\newpage
\section{Tcolor box}
Hello world 
\vspace*{6em}

\begin{tcolorbox}[
  enhanced,  breakable,
  top=1.5pt, bottom=1.5pt,
  left=2pt,  leftlower=-3pt,
  right=3pt, arc=0pt, frame hidden,
  bicolor, colback=shadecolor,
  opacitybacklower=0,
  frame code app={
    \draw[color=gray, thick] 
      (frame.north west)++(-5cm, -1pt)--($(frame.north east)+(5cm, 0pt)$);
    \draw[color=gray, thick] 
      (frame.south west)++(-5cm, -1pt)--($(frame.south east)+(5cm, 0pt)$);
    \fill[color=gray!50, path fading=east] 
      (frame.north west)++(-5cm, -1pt) rectangle ($(frame.south east)+(5cm, 0pt)$);
  },
]\llapnote{\sffamily\textbf{Example} 1.1\\Integrate by Part}%
\lipsum[1][1-3]
\begin{align}
  \int u\;\mathrm{d}v = uv - \int v \;\mathrm{d}u
\end{align}
\lipsum[1][1-3]
\tcblower
\lipsum[1][3-5]
\begin{align}
  {\frac {\mathrm{d}}{\mathrm{d}x}}{\Big (}u(x)v(x){\Big )}=v(x){\frac {\mathrm{d}}{\mathrm{d}x}}
    \left(u(x)\right)+u(x){\frac {\mathrm{d}}{\mathrm{d}x}}\left(v(x)\right)
\end{align}
\lipsum[1][1-3]\hfill$\mathcolor{gray}{\blacktriangleleft}$
\end{tcolorbox}


\newpage
\section{NEWSEC}
\subsection{parbox}
Hello world\parbox[t]{6em}{aaaaaa bbbbb cccc}


% Hello world \tikz[anchor=north, baseline=(A.base)]\fill[color=red!50, path fading=east] (0, 0)++(-3cm, 0pt) rectangle +(5em, 8em);

\subsection{tikz object}
before text 
\tikz[baseline={([yshift={-\ht\strutbox}]a.north)},outer sep=0pt,inner sep=0pt] \node[align=center] (a) {\strut line1\\line2\\line3\\line4};
text between
\tikz[baseline=(a.south),outer sep=0pt,inner sep=0pt] \node[align=center] (a) {\strut line1\\line2\\line3\\line4};
text after


\subsection{tcb upper/lower}
\newtcolorbox{BBB}{breakable,
enhanced,  breakable,
  bicolor,
  colback=red!10!white,
  colbacklower=blue!5!white,
  title={Example}}
\begin{BBB}
  \lipsum[1]
  \tcblower
  \lipsum[2]
\end{BBB}
\end{document}




% ==> test obsidian thm style 
\InputIfFileExists{zlatex-cfg.tex}{}{}
\documentclass[layout={slide, aspect=16|9}]{../code/zlatex}
\usepackage[many]{tcolorbox}
\zlatexloadlibrary{theme}
\zlatexThmStyle{obsidian}
\usepackage{lipsum}

\title{test OBSIDIAN THM}
\author{Eureka}
\date{\today}
\begin{document}
\maketitle
\section{FISRT}
\lipsum[1] 
\begin{theorem}[THM]
  \lipsum[1][1-5] 
  \[
    \sum_{i=1}^{+\infty}{\int_{0}^{i}-\frac{1}{t}\mathrm{d}t} = \frac{\pi^2}{6}
  \]
\end{theorem}

\lipsum[2] 
\begin{proposition}[PROP]
  \lipsum[1][1-5] 
  \[
    \sum_{i=1}^{+\infty}{\int_{0}^{i}-\frac{1}{t}\mathrm{d}t} = \frac{\pi^2}{6}
  \]
\end{proposition}

\lipsum[3] 
\begin{definition}
  \lipsum[1][1-5] 
  \[
    \sum_{i=1}^{+\infty}{\int_{0}^{i}-\frac{1}{t}\mathrm{d}t} = \frac{\pi^2}{6}
  \]
\end{definition}
\end{document}


% ==> '+=' keys feature
\InputIfFileExists{zlatex-cfg.tex}{}{}
\documentclass{../code/zlatex}
\parindent=0pt

\begin{document}
\ExplSyntaxOn
\keys_define:ne {test}{
  % simple key
  keyA    .tl_set:N = \exp_not:c {l__test_keyA_tl},
  keyA+   .code:n   = \tl_put_right:Nn \exp_not:c {l__test_keyA_tl} {#1},
  keyA~+  .code:n   = \tl_put_right:Nn \exp_not:c {l__test_keyA_tl} {#1},
  % zlatex internal
  keyB    .meta:nn  = { text / keyB }{#1},
  \__zlatex_plus_key_aux:nnn {l__test_keyB_tl}{keyB}{keyB-x},
  \__zlatex_plus_key_aux:nnn {l__test_keyB_tl}{keyB}{keyB-y},
}
\NewDocumentCommand{\testPlusKeyA}{m}{
  \keys_set:nn {test}{#1}
  \tl_use:N \l__test_keyA_tl\par
}
\NewDocumentCommand{\testPlusKeyB}{m}{
  \keys_set:nn {test}{#1}
  \tl_use:N \l__test_keyB_tl\par
}

\ExplSyntaxOff
\section{Simple}
\testPlusKeyA{keyA = AA}
\testPlusKeyA{keyA+ = aa} 
\testPlusKeyA{keyA += xx} 
\testPlusKeyA{keyA=zz}

\section{Internal}
\testPlusKeyB{keyB/keyB-x = BB}
\testPlusKeyB{keyB/keyB-x+ = bb}
\testPlusKeyB{keyB/keyB-x += xx}
\testPlusKeyB{keyB/keyB-x = zz}

\dotfill\par
\testPlusKeyB{keyB/keyB-y = BBYY}
\testPlusKeyB{keyB/keyB-y+ = bbyy}
\end{document}





% ==> mathalias library/pagemask test
\InputIfFileExists{zlatex-cfg.tex}{}{}
\documentclass{../code/zlatex}
\zlatexloadlibrary{mathalias}
\zlatexThmStyle{fancy}
\usepackage{tikz}
\usetikzlibrary{fadings}
\usepackage{lipsum}


\begin{document}
\zlatexPageMask[
  anchor=bl,
  position={(0pt, -14em+\zph)}
]{\includegraphics[width=\zpw]{back.jpeg}}
\zlatexPageMask[
  % label=FIRST, % EXPECT: Class zlatex Warning: Only star version of \zlatexPageMask is label-allowed.
  anchor=bl,
  position={(0pt, -14em+\zph)}
]{%
  \tikz\fill[color=gray, path fading=north] (0, 0) rectangle +(\zpw, 5em);
}
\thispagestyle{empty}
{
  \def\thesection{\textcolor{white}{\arabic{section}}}
  \section{\textcolor{white}{Symbols}}
}\vspace*{3em}
\subsection{simple}
\lipsum[1]
\begin{align}
  & \la \\
  & \rra \\
  & \Nra \\
  & \dda \\
  & \xra*[Up](down)\\
  & \xla(\beta) \\
  & \xxla[2](1)
\end{align}

\subsection{Complex}
Some Examples: $H^1(\Omega)\hra L^p(\Omega)$, And further more: 
\begin{align}
  H^1(\Omega)\hra[E:t-1 = \gamma] L^p(\Omega)
\end{align}

Some symbols: \CC, \RR, \ZZ, \NN

\begin{align}
  & \zab \alpha + \beta \text{ V.S. } \alpha + \beta \\
  & \zab(\frac12) \text{ V.S. } (\frac12) \\
  & \zab[\frac12] \text{ V.S. } [\frac12] \\
  & \zab\{\frac12\} \text{ V.S. } \{\frac12\}
\end{align}

\newpage
\zlatexPageMask*[anchor=tc,label=TEST,layer=foreground]{%
  % \tikz\fill[color=blue, path fading=north] (0, 0) rectangle +(\zpw, 5em);
  \tikz\fill[semitransparent, teal] (-10em,-10em) rectangle +(10em,10em); 
}
\section{Text}
\lipsum[1]

\begin{definition}[Test THM]
  A smooth vector field \(\mathbf{v}\) is defined in a domain \(M\) if to each point \(x\) there is 
  assigned a vector \(\mathbf{v}\left( x\right)  \in  {T}_{x}M\) attached at that point and depending 
  smoothly on the point \(x\) (if a system of \(m\) coordinates is chosen, the field is defined by 
  its \(m\) components, which are smooth functions of \(m\) variables). The vector \(\mathbf{v}\left( x\right)\) 
  is called the value of the field \(\mathbf{v}\) at the point \(x\) .
\end{definition}

\lipsum[1]

\begin{definition}[Continuous Mapping]
  A smooth vector field \(\mathbf{v}\) is defined in a domain \(M\) if to each point \(x\) there is 
  assigned a vector \(\mathbf{v}\left( x\right)  \in  {T}_{x}M\) attached at that point and depending 
  smoothly on the point \(x\) (if a system of \(m\) coordinates is chosen, the field is defined by 
  its \(m\) components, which are smooth functions of \(m\) variables). The vector \(\mathbf{v}\left( x\right)\) 
  is called the value of the field \(\mathbf{v}\) at the point \(x\) .
\end{definition}

\lipsum[3]

\begin{lemma}[Test THM]
  A smooth vector field \(\mathbf{v}\) is defined in a domain \(M\) if to each point \(x\) there is 
  assigned a vector \(\mathbf{v}\left( x\right)  \in  {T}_{x}M\) attached at that point and depending 
  smoothly on the point \(x\) (if a system of \(m\) coordinates is chosen, the field is defined by 
  its \(m\) components, which are smooth functions of \(m\) variables). The vector \(\mathbf{v}\left( x\right)\) 
  is called the value of the field \(\mathbf{v}\) at the point \(x\) .
\end{lemma}

\lipsum[4]
\zlatexPageMaskRemove{foreground}{TEST}
\end{document}


% ==> Fixed Bugs: 
% 1. THM note expansion bug; 
% 2. hyper init faild; 
% 3. toc line stretch 
%    --> fixed bug: line-stretch failed when no \tableofcontents before
\InputIfFileExists{zlatex-cfg.tex}{}{}
\documentclass[hyper]{../code/zlatex}
\zlatexloadlibrary{mathalias}
\zlatexColorSetup{link=teal}
\zlatexSetup{
  toc={
    stretch=2, 
    title-vspace=-5em, 
    title=\centerline{\bfseries\Large \textsc{Contents}}
  }
}
\usepackage{lipsum}
\zlatexThmBefore{\par\noindent\dotfill\par}
\zlatexThmTocPrefix{\P\;}
% \zlatexThmTocSymbolClear
\zlatexThmTocSymbol{
  lemma=$\star$\;,
  definition=$\diamond$\;,
  theorem=$\clubsuit$\;,
}
\ExplSyntaxOn
\newcommand{\addthmsec}[4]{
  \__zlatex_thm_toc_add:nnnn {#1}{#2}{#3}{#4}
}
\ExplSyntaxOff


\begin{document}
\thispagestyle{empty}
\tableofcontents
\vskip5em

\lipsum[1]
\section{List of Theorems}
Lorem ipsum dolor sit amet, consectetuer adipiscing elit. Ut purus elit, vestibulum
ut, placerat ac, adipiscing vitae, felis.

\zlatexThmToc[3]

\section{Accent}
\addthmsec{section}{BLANK}{\thepage}{section.1}
\lipsum[2]
\begin{lemma}[N\"ormal Text]
  This is a simple lemma.
  \begin{align}
    a^2 + b^2 = c^2
  \end{align}
\end{lemma}

\lipsum[2]
\begin{proposition}[Normal Text]
  This is a simple lemma.
  \begin{align}
    a^2 + b^2 = c^2
  \end{align}
\end{proposition}


\section{Macros/Equation}
\zlatexThmTocAdd[name=NEW-SEC]{section}
\begin{theorem}[$\mathbb{Z}[\sqrt{2}]$]%
  This is a simple theorem.
  \begin{align}
    a^2 + b^2 = c^2
  \end{align}
\end{theorem}


\begin{definition}
  This is a simple definition.
  \begin{align}
    a^2 + b^2 = c^2
  \end{align}
\end{definition}

\ExplSyntaxOn
% \cs_show:N \sqrt

\cs_set:Npn \__test_sqrt:n #1 
  { #1--$Z[\sqrt {2}]$ }

\__test_sqrt:n {AA}

\the\baselineskip
\ExplSyntaxOff


\section{AAA}
\subsection{AAA-1}
\subsection{AAA-2}
\subsection{AAA-3}

\section{BBB}
\subsection{BBB-1}
\subsection{BBB-2}
\subsection{BBB-3}
\end{document}


% ==> thmnote update
\InputIfFileExists{zlatex-cfg.tex}{}{}
\documentclass{../code/zlatex}


\begin{document}
\section{First}
This is a simple definition style in z\LaTeX{}:

\begin{definition}
  This is a simple definition. 
  \begin{align}
    a^2 + b^2 = c^2
  \end{align}
\end{definition}

A simple theorem environment follows the above definition as below:
\begin{theorem}[THM-NOTE]
  This is a simple theorem.
  \begin{align}
    a^2 + b^2 = c^2
  \end{align}
\end{theorem}


\section{SET to HEIGHT}
\ExplSyntaxOn
\ztool_set_to_ht:nn {2.5pt}{XXX}
\ExplSyntaxOff


\section{THM LIST}
Test List of Theorems:

\zlatexThmToc
\end{document}





% ==> slide mode test
\InputIfFileExists{zlatex-cfg.tex}{}{}
\documentclass[
  hyper,
  % class=book,
  % lang=cn,
  layout={slide, aspect=16|9, theme=AnnArborSpruce},
]{../code/zlatex}
\zslideThemeUse[
  UR={text=\zslideIfPageTF{=1}{}{\zslideDefaultUR:\ \zslideNavigateBall}},
]{AnnArborSpruce}
\makeatletter
\def\seeTargets{%
  \par\noindent\dotfill\\
  \string\@currentHref\ = \meaning\@currentHref\par
  \string\@currentHpage\ = \meaning\@currentHpage\par
  \string\Hy@currentbookmarklevel\ = \meaning\Hy@currentbookmarklevel\par\vskip3em
}
\makeatother
\zlatexPageMask*[layer=foreground]{AAA}
\zlatexPageMask[layer=foreground]{BBB}
% \zslideLogo[position={(40pt, 10pt)}, exclude={0, 2}]{Epmmy.jpg}
\zslideLogo[width=3em, exclude={0, 1}]{Epmmy.jpg}


\title{Debug}
\author{Eureka}
\date{\today}
\begin{document}
\maketitle
\tableofcontents

\section{First Section}
\subsection{Test}
BULLET TEST: \textbullet

\ExplSyntaxOn
m\ztool_set_to_wd:nn {1em}{mmmm}\par 

\rule{1em}{3pt}
\ExplSyntaxOff

\zslideIfPageTF{<2}{PAGE-II}{NOT PAGE-II}

\newpage
\subsection[Short First Subsection Name]{First Subsection Name}
  \begin{itemize}
    \item
    An item
    \item
    Another item
    \begin{itemize}
      \item Something
      \begin{itemize}
        \item Something else
      \end{itemize}
    \end{itemize}
    \begin{enumerate}
      \item Thing A
      \item Thing B
      \item Thing C
    \end{enumerate}
  \end{itemize}
\end{document}




% ===> Re-write titlesec package
\InputIfFileExists{zlatex-cfg.tex}{}{}
\documentclass[
  % hyper,
  class=book,
]{../code/zlatex}
\makeatletter
% --> LaTeX 2e interface
\newcounter{sector}
\newcommand{\sectormark}[1]{\markright{#1}}
\newcommand{\sector}{\@startsection{sector}{1}{0pt}%
  {-3.5ex plus -1ex minus -.2ex}
  {2.3ex plus.2ex}
  {\large\sffamily}% start normal sec syntax: "*[]{}"
}
\makeatother

% --> zlatex interface
% \zlatexTitleNew{page}{part}



% \zlatexTitleStyle{section}
% {
%   shape=display,
%   % align=left,
%   label=X-\thesection-Y.,
%   sep=10pt,
%   format=\color{blue},
%   labelformat=\bfseries,
%   titleformat=\sffamily\color{red},
%   titlecmd=\testcmd,
%   before=\raggedleft\titlerule[1pt]\vskip2em|,
%   after=\vskip3em,
% }
% \ExplSyntaxOn
% \group_begin:
% \__zlatex_title_format_copy:nnnnnnn 
%   {\section}{hang}{\sffamily}{\thesection~ XX}{0pt}{}{}
% \group_end:
% \ExplSyntaxOff


\begin{document}
% \part{PART}
\chapter{CHAPTER}
\section{AAA}
The prior examples redefined existing sectional unit title 
commands. This defines a new one, illustrating the needed 
counter and macros to display that counter.


\sector{BBB}
The prior examples redefined existing sectional unit title 
commands. This defines a new one, illustrating the needed 
counter and macros to display that counter.



\begin{align}
  \substack{\smash{A^T}\\\scriptscriptstyle n\text{by}m}
  \substack{\smash{C}\\\scriptscriptstyle m\text{by}m}
  \substack{\smash{A}\\\scriptscriptstyle m\text{by}n}
  = 1
\end{align}
\end{document}





% ==> learn titlesec
\documentclass{article}
\usepackage[hmargin=2in]{geometry}
\usepackage[raggedleft, sf]{titlesec}
\titlelabel{\thetitle.\;}
\usepackage{xcolor}

% \newcounter{part}
\titleclass{\part}{straight}
\titleformat{\part}{\bfseries}{\thepart.}{10pt}{}

% 1. \part is usually implemented in a non-standard way, it 
%    remains untouched by the simple settings and should be 
%    changed with the help of the “Advanced Interface.”


\begin{document}
\part{First Part}
\section{FIRST}
The prior examples redefined existing sectional unit title 
commands. This defines a new one, illustrating the needed 
counter and macros to display that counter.


\section{SECOND}
The prior examples {redefined existing} sectional unit title.
commands. This defines a new one, illustrating the needed 
counter and macros to display that counter. commands This 
defines a new one, illustrating the needed counter and macros 
to display that counter illustrating the needed defines a new one, 
illustrating the needed counter and macros to display that counter.

\ExplSyntaxOn
\cs_new:Nn \_test:nn {
  #1 == #2
}
\_test:nn {AA}{BB}
\ExplSyntaxOff

\end{document}




% ===> page info
\InputIfFileExists{zlatex-cfg.tex}{}{}
\documentclass[
  hyper
]{../code/zlatex}
\usepackage[many]{tcolorbox}
\makeatletter
\zlatexPageMask[
  anchor=tr, 
  position={(\zpw, \zph)}
]{\color{orange}\rule{\zpw}{10pt}}
\colorlet{colexam}{red!75!black}
\zlatexThmCnt{share, parent=subsection}

\begin{document}
HELLO WORLD \rule{10pt}{10pt}

\section{TEXT}
\zlatexPageMask{AAA}
\zlatexPageMask{\includegraphics{Epmmy.jpg}}
\zlatexPageMask[layer=foreground, position={(10em, 10em)}]{\parbox{3em}{BBB-1\\BBB-2}}
\zlatexPageMask*[position={(.25\zpw, .25\zph)}]{\ifnum\thepage=2 CCC2\else CCC1\fi}
\zlatexPageMask*[position={(0pt, 0pt)}, anchor=lb]{XXX-\thepage}
\zlatexPageMask[position={(\zpw, \zph)}, anchor=tr]{YYY}

\newpage
\zlatexPageMask[
  anchor=br, 
  position={(\zpw, 0pt)}
]{\color{purple}\rule{15pt}{\zph}}


\section{GRAPH}
\subsection{AAA}
% ==> style definitions
\tcbset{
  styleA/.style={
    arc=0mm, 
    bottomtitle=0.5mm,
    boxrule=0mm,
    colbacktitle=black!10!white, 
    coltitle=black, 
    fonttitle=\bfseries, 
    left=2.5mm,
    leftrule=1mm,
    right=3.5mm,
    title={\zlatexThmTitle},
    toptitle=0.75mm, 
  },
  styleB/.style={
    empty,
    frame engine=path,
    colframe=yellow!10,
    sharp corners,
    title={\zlatexThmTitle},
    attach boxed title to top left={yshift*=-\tcboxedtitleheight},
    boxed title style={size=minimal, top=4pt, left=4pt},
    coltitle=colexam,fonttitle=\large\bfseries\sffamily,
    drop fuzzy shadow,
    coltitle=black,
    borderline west={3pt}{-3pt}{teal!50},
    attach boxed title to top left={xshift=-3mm, yshift*=-\tcboxedtitleheight/2},
    boxed title style={right=3pt, bottom=3pt, overlay={
      \draw[draw=teal!70, fill=teal!70, line join=round]
        (frame.south west) -- (frame.north west) -- (frame.north east) --
        (frame.south east) -- ++(-2pt, 0) -- ++(-2pt, -4pt) --
        ++(-2pt, 4pt) -- cycle;
    }},
    overlay unbroken={
      \scoped \shade[left color=teal!10!black, right color=teal]
        ([yshift=-0.2pt]title.south west) -- ([xshift=-1.5pt, yshift=-0.2pt]%
        title.south-|frame.west)-- ++(0, -6pt) -- cycle;
    },
  },
  styleC/.style={
    enhanced, 
    breakable,
    colback=white,
    colframe=blue!30!black,
    attach boxed title to top left={yshift*=-\tcboxedtitleheight}, 
    title={\zlatexThmTitle},
    boxed title size=title,
    boxed title style={%
        sharp corners, 
        rounded corners=northwest, 
        colback=tcbcolframe, 
        boxrule=0pt,
    },
    underlay boxed title={%
        \path[fill=tcbcolframe] (title.south west)--(title.south east) 
            to[out=0, in=180] ([xshift=5mm]title.east)--
            (title.center-|frame.east)
            [rounded corners=\kvtcb@arc] |- 
            (frame.north) -| cycle; 
    },
  }
}
\zlatexThmStyleNew{
  % ---->
  styleA={
    begin={\begin{tcolorbox}[styleA]},
    end={\end{tcolorbox}}
  },
  % ---->
  styleB={
    begin={\begin{tcolorbox}[styleB]},
    end={\end{tcolorbox}}
  },
  % ---->
  styleC={
    begin={\begin{tcolorbox}[styleC]},
    end={\end{tcolorbox}}
  }
}
% ===> test text
\def\bench{Mathematics is a field of study that discovers and organizes methods, 
theories and theorems that are developed and proved for the needs of empirical 
sciences and mathematics itself. There are many areas of mathematics, 
\begin{align}
  \mathrm{d}^2x + \mathrm{d}^y2 = \mathrm{d}^2z
\end{align}
which include number theory (the study of numbers), algebra (the study of formulas and related
structures), geometry (the study of shapes and spaces that contain them), analysis 
(the study of continuous changes), and set theory (presently used as a 
foundation for all mathematics).}

% \zlatexThmStyle{styleA}
\zlatexThmTitleFormat*{%
  \bfseries\color{orange}%
  \zlatexThmNote\ \zlatexThmNumber\ \zlatexThmName
}
\begin{definition}[DEF]
\bench
\end{definition}


% \zlatexThmStyle{styleB}
\vskip8em
\begin{theorem}[THM]
\bench
\end{theorem}


\newpage
% \zlatexThmStyle{styleC}
\begin{axiom}[AXIOM]
\bench
\end{axiom}


\newpage
\zlatexThmToc
\end{document}






% ===> keytheorems ref
\InputIfFileExists{zlatex-cfg.tex}{}{}
\documentclass{article}
\usepackage{keytheorems}
\newkeytheorem{definition}
\usepackage[colorlinks]{hyperref}

\begin{document}
\stepcounter{page}
\begin{definition}[DEF]\label{def:a}
Hello world, this is a definition. $a^2 + b^2 = \mathrm{d}^2z$
\end{definition}


\listofkeytheorems


\newpage
\listofkeytheorems

\@writefile{thlist}{\KeyThmsSavedTheorem{definition}{1}{definition.1}{2}{}{note={DEF}}{}}
\KeyThmsSavedTheorem {definition}{1}{}{1}{}{note={DEF}}{}

\cs_new_protected:Npn \__keythms_thm_addcontentsdata:nnnn #1#2#3#4
  { % #1 = theorem name, #2 = stored counters, #3 = keys, #4 = body
    \keythms_listof_chaptervspacehack:
    \iow_shipout:Ne \@auxout
      {
        \token_to_str:N \@writefile { thlist }
          {
            \token_to_str:N \KeyThmsSavedTheorem{ #1 }
              { \@currentlabel }
              { \@currentHref }
              { \thepage }
              { #2 }
              { \exp_not:n { #3 } } % do we want any expansion here, perhaps
              { \exp_not:n { #4 } } % with \text_expand:n ?
          }
      }
  }
\end{document}




% ==> add list of theorem interface
\InputIfFileExists{zlatex-cfg.tex}{}{}
\documentclass[hyper]{../code/zlatex}
\makeatletter
\usepackage[many]{tcolorbox}

\zlatexThmTocLevel{subsection}
\zlatexThmStyle{shadow}
\begin{document}
\zlatexThmToc

\vskip6em
\subsection{AAA}
\begin{definition}[DEF]\label{def:aaa}
  Hello world, this is a definition. $a^2 + b^2 = \mathrm{d}^2z$
\end{definition}


\newpage
\subsection{BBB}
\begin{theorem}[THM]
  Hello world, this is a theorem. $a^2 + b^2 = \mathrm{d}^2z$
\end{theorem}


\newpage
\zlatexThmToc[vspace=15pt, title=THM LIST]
\end{document}


\InputIfFileExists{zlatex-cfg.tex}{}{}
\documentclass[hyper]{../code/zlatex}
\parindent=0pt \makeatletter
\ExplSyntaxOn
% \NewHook{zlatex/thm/thlist}
\def\@starttoc#1{%
  \begingroup
  \@input{\jobname.#1}
  % \@fileswfalse
  % \UseHook{zlatex/thm/thlist}
  % \legacy_if_set_false:n { @filesw }
  \typeout{------->\checksave}
  \if@filesw
    \typeout{----->enter~stream}
    \expandafter\newwrite\csname tf@#1\endcsname
    \immediate\openout \csname tf@#1\endcsname \jobname.#1\relax
  \fi
  \@nobreakfalse
  \endgroup
}


% optional commands
\def\seeTargets{%
  \par\noindent\dotfill\\
  \string\@currentHref\ = \meaning\@currentHref\par
  \string\@currentHpage\ = \meaning\@currentHpage\par
  \string\Hy@currentbookmarklevel\ = \meaning\Hy@currentbookmarklevel\par\vskip3em
}
\def\checksave{
  \if@filesw SAVED \else NOT SAVED \fi
}
\ExplSyntaxOff

\begin{document}
\checksave
\vskip3em

\@starttoc{thlist}
% \@@input debug.thlist
\newpage

\section{FIRST}
\seeTargets

\begin{definition}[DEF]\label{def:aa}
Hello world, this is a definition. $a^2 + b^2 = \mathrm{d}^2z$
\end{definition}
\addcontentsline{thlist}{subsection}{DEF-ITEM}
\seeTargets

\section{THM-LIST}
\meaning\@@input

% \checksave
\begingroup
\@fileswfalse
\@starttoc{thlist}
% \input{\jobname.thlist}
\@@input debug.thlist
\endgroup
% \contentsline {subsection}{DEF-ITEM}{2}{definition.1.1}%


\end{document}







% ==> New thm style interface
\InputIfFileExists{zlatex-cfg.tex}{}{}
\documentclass[
  fancy, 
  hyper, 
  % layout={slide, aspect=16|9}
]{../code/zlatex}
\usepackage{pifont}
\zlatexSetup{mathSpec={envStyle=shadow}}
\zlatexThmCreate{theorem}{Ztheorem=NEW-THM|{HTML}{8e22ad}, Zproposition} 


\def\boomen{As any dedicated reader can clearly see, the Ideal of practical
reason is a representation of, as far as I know, the things in themselves; 
\begin{align}
\underset{}{\mathbf{v} \bigotimes \mathbf{w}} 
  & = \underset{}{\mathbf{v} \otimes \mathbf{w}}
      = \sum_{i=1}^3\sum_{j=1}^3a_{ij}u^iv^j \\[-.75em]
  & = \sum_{i=1}^3\left(a_{i1}u^iv^1+a_{i2}u^iv^2+a_{i3}u^iv^3\right) 
  \end{align}  
}
\zlatexThmLang{fr}
\zlatexThmCnt{share, parent=section}
\zlatexThmTocPrefix{\ding{118}\;}
\zlatexColorSetup{
  definition  = {RGB}{219,48,122},
  theorem = teal!20
} % OK

\title{DEBUG}
\author{Eureka}
\date{\today}
\begin{document}
% BUG: stretch not working in thm toc.
% \maketitle
% \zlatexThmToc[stretch=20]
% \zlatexThmToc[
%   title={\textsc{List of Theorems}},
%   stretch=1.5, 
%   title-vspace=-10pt, 
%   after-vspace=20pt
% ]


\section{NEWLY}
\begin{Ztheorem}[ZTHM]
\boomen
\end{Ztheorem}

\zlatexThmColorSetup{
  Zproposition={HTML}{c0eb75},
}
\zlatexThmStyle{leftbar}
\begin{Zproposition}[ZPROP]
\boomen
\end{Zproposition}


\section{DEBUG}
\subsection{AAA}
% \zlatexColorSetup{
%   definition  = {RGB}{219,48,122},
% } % NO effect
\zlatexThmStyle{fancy}
\begin{definition}[DEF]\label{def:1}
\boomen
\end{definition}
\zlatexThmStyleNew{
  aaa={begin=\par\begin{center}, end=\end{center}, option=\zlatexThmTitleSwitch},
  bbb={begin=BBB\begin{center}, end=\end{center}\kaaklfg\ahjksdh\par},
}
% \ExplSyntaxOn
% \cs_meaning:N \g__zlatex_thm_style_aaa_begin_tl
% \cs_meaning:N \g__zlatex_thm_style_aaa_end_tl

% \cs_meaning:N \g__zlatex_thm_style_bbb_begin_tl
% \cs_meaning:N \g__zlatex_thm_style_bbb_end_tl
% \ExplSyntaxOff


\zlatexThmStyle{background}
% \ShowHook{zlatex/thm/titleformat}
\zlatexThmTitleFormat{\bfseries\zlatexThmNumber\ \zlatexThmName}
% \ShowHook{zlatex/thm/titleformat}
\begin{theorem}[THM]\label{thm:2}
  \boomen
\end{theorem}
% \ShowHook{zlatex/thm/titleformat}

\subsection{CCC}
\zlatexThmStyle{elegant}
% \ShowHook{zlatex/thm/titleformat}
\begin{proposition}[PROP]\label{prop:5}
  \boomen
\end{proposition}
% \ShowHook{zlatex/thm/titleformat}

\zlatexThmStyle{aaa}
\section{BBB}
\zlatexThmColorSetup{
  definition={HTML}{22b8cf},
  proof=orange
}
\zlatexThmHook{before=\color{blue}\begin{framed}, after=\end{framed}}
\begin{definition}[DEF-II]\label{def:3}
\boomen
\end{definition}

\zlatexThmStyle{paris}
\begin{definition}[DEF-III]\label{def:4}
  \boomen
\end{definition}


\zlatexThmStyle{leftbar}
\begin{theorem}[REMARK]\label{rem:1}
  As any dedicated reader can clearly see, the Ideal of practical
  reason is a representation of, as far as I know, the things in themselves;
\boomen
As any dedicated reader can clearly see, the Ideal of practical
reason is a representation of, as far as I know, the things in themselves;
\end{theorem}


\section{PROOF-LIKE}
\begin{proof}
Hello this is a simple proof env test.
\end{proof}


\cref{def:1,prop:5} | \Cref{thm:2} | \cref{def:3,rem:1}
\begin{axiom}[AXIOM]
  \boomen
\end{axiom}
\end{document}











% Marker in zslide bug fixed
\InputIfFileExists{zlatex-cfg.tex}{}{}
\documentclass[
  hyper,
  fancy,
  layout={slide, aspect=16|9, theme=AnnArborBeaver}
]{../code/zlatex}
\zlatexSetup{mathSpec={envStyle=fancy}}
\zlatexThmCreate{theorem}{Ztheorem=THM|{HTML}{8e44ad}, Zproposition} 
% \zslideThemeUse[
%   UR={text=\zslideNavigateBall},
% ]{AnnArborBeaver}
\NewMarkClass{test}
\usepackage{lipsum}
% \usepackage{minted}
\def\boomen{As any dedicated reader can clearly see, the Ideal of practical
reason is a representation of, as far as I know, the things in themselves; 
\begin{align}
\underset{}{\mathbf{v} \bigotimes \mathbf{w}} 
  & = \underset{}{\mathbf{v} \otimes \mathbf{w}}
      = \sum_{i=1}^3\sum_{j=1}^3a_{ij}u^iv^j \\[-.75em]
  & = \sum_{i=1}^3\left(a_{i1}u^iv^1+a_{i2}u^iv^2+a_{i3}u^iv^3\right) 
  \end{align}  
}


\ExplSyntaxOn\makeatletter
\makeatother\ExplSyntaxOff

\title{ltMarks Test}
\author{Eureka}
\date{\today}
\begin{document}
\maketitle

% blank page
% \newpage

\section{HELLO}
\InsertMark{test}{AAA-1}
\subsection{WORLD}
\lipsum[1][1-5]
\InsertMark{test}{AAA-2}
\vspace*{8em}

\subsection{SPACE}
\lipsum[1][6-15]

\subsection{ROOM}
\lipsum[1][13-17]
\vspace*{6em}

\subsection{OCEAN}
\lipsum[2][1-6]
\vspace*{10em}

\subsection{HEAVEN}
\lipsum[2][1-16]

% \newpage
% aa


\section{DEBUG}
\subsection{AAA}
\lipsum[1][1-5]
\InsertMark{test}{AAA-2}
\vspace*{8em}

\subsection{BBB}
\lipsum[1][6-15]


\section{FIRST}
\begin{definition}[DEF]
  Hello world: $a^2+b^2=c^2$.
\end{definition}

\zlatexThmStyle{elegant}
\begin{theorem}[theorem Test]
  Hello world: $a^2+b^2=c^2$.
\end{theorem}


\section{SECOND}
% \zlatexColorSetup{
%   % Ztheorem={HTML}{8e44ad},
%   Ztheorem=red 
% }
\begin{Ztheorem}[Ztheorem Test]
  Hello world: $a^2+b^2=c^2$.
\end{Ztheorem}

\begin{Zproposition}
  Hello world: $a^2+b^2=c^2$.
\end{Zproposition}


\newpage
% \zlatexColorSetup{theorem=teal}
\section{Elegant Style Math Env}
\begin{axiom}[prime number]
  \boomen
\end{axiom}

\zlatexThmStyle{elegant}
\begin{definition}[prime number]
  \boomen
\end{definition}

\zlatexThmStyle{leftbar}
\begin{theorem}[prime number]
  \boomen
\end{theorem}

\zlatexThmTitleFormat{\textcolor{red}{\thmname{#1}:\thmnote{[#3]-}\thmnumber{#2}}}
\begin{lemma}[prime number]
  \boomen
\end{lemma}

\zlatexThmStyle{plain}
\begin{corollary}[prime number]
  \boomen
\end{corollary}

\newpage
\zlatexThmStyle{fancy}
\zlatexThmHook{begin=BEGIN, end=END, after=AFTER}
% \zlatexThmHook{begin=\begin{center}, end=\end{center}}
% \zlatexThmHook{before=\begin{framed}, after=\end{framed}}
\begin{proposition}
  \boomen
\end{proposition}


\zlatexThmStyleNew{before=\color{blue}\begin{framed}, after=\end{framed}}
\begin{definition}
  \boomen
\end{definition}

% \section{Final}
% Other trial as below:
% \begin{minted}{latex}
% \zslideThemeUse[
%     % UR={text=\FirstMark{2e-right-nonempty}} % failed
%     % UR={text=\FirstMark{test}}              % successed
%     % UR={text=\FirstMark{zslide-right}}      % successed
%     UR={text=\FirstMark{zslide-right}\, --\, \LastMark{zslide-right}} % successed
% ]{AnnArborDefault}
% % hyper link test
% \ExplSyntaxOn\makeatletter
% \int_step_inline:nnn {1}{4}{
%   % \hyper@link{link}{page.#1}{(#1)~|~}
%   \hyper@link{link}{zslide@HELLO.#1}{(HELLO.#1)~|~}
% }
% \makeatother\ExplSyntaxOff
% % navigate test
% \ExplSyntaxOn\makeatletter
% \cs_set:Npn \zslide@hyper@navigate:nn #1#2 {
%   % #1: total frame
%   % #2: current frame
%   \int_step_inline:nnn {1}{#1}{
%     \int_compare:nNnTF {#2} = {##1}
%       {\hyper@link{link}{zslide@HELLO.##1}{\textbullet}}
%       {\hyper@link{link}{zslide@HELLO.##1}{$\circ$}}
%   }
% }
% \zslide@hyper@navigate:nn {4}{2}
% \makeatother\ExplSyntaxOff
% \end{minted}

% \section{graphics}
% Auto scale graphics:

% \includegraphics[width=.2\linewidth]{Epmmy.jpg}
\end{document}









% ".color_set:nn" in l3keys
\documentclass{article}
\usepackage{xcolor}


\begin{document}
\ExplSyntaxOn\makeatletter
\cs_set:Npn \__temp__:nn #1#2 {\typeout{#1:#2}}
\cs_set:Npn \__temp__:n #1 {\typeout{#1}}
\keyval_parse:nnn 
  {\__temp__:n}
  {\__temp__:nn }
  {begin=AAA, end=BBB}


\def\ctext#1{\par\noindent\textcolor{\tl_use:c {l_test_#1_tl}}{Sample-TEXT}}


\cs_new:Npn \__zlatex_color_name_selector:w #1#2#3#4\q_stop 
  {#2}
\cs_new:Npn \__zlatex_color_init:nn #1#2 {
  \definecolor{zlatex@color@#1}{#1}{#2}\q_stop
  zlatex@color@#1
}

\tl_new:N \zlatex@color@keyC
\tl_set:Nn \zlatex@color@keyC {blue}
\keys_define:nn {test}{
  keyA  .tl_set:N   = \l_test_keyA_tl,
  keyA  .initial:n  = {red},
  keyB  .tl_set:N   = \l_test_keyB_tl,
  % keyB  .initial:n  = {blue},
  keyB  .initial:e  = {\__zlatex_color_name_selector:w \definecolor{orange}{HTML}{5f3dc4}\q_stop},
}
\ctext{keyA}
\ctext{keyB}
% \keys_set:nn {test}{keyA=black}
% \tl_show:N \l_test_keyB_tl

\par\noindent
\__zlatex_color_name_selector:w \definecolor{zlatex@color@keyC}{HTML}{5f3dc4}\q_stop

\makeatother\ExplSyntaxOff
\end{document}






\InputIfFileExists{zlatex-cfg.tex}{}{}
\documentclass[fancy, ]{../code/zlatex}
\zlatexSetup{mathSpec={envStyle=shadow}}
\zlatexThmCreate{theorem}{Ztheorem=THM|{HTML}{8e44ad}, Zproposition} 
% \zlatexColorSetup{
%   definition=blue,
%   theorem={HTML}{007175},
%   Ztheorem={RGB}{219,48,122}
% }
% \newsavebox{\zlatexBox}
% \zlatexThmStyleNew*{begin=\color{red}\begin{framed}, end=\end{framed}}

\def\boomen{As any dedicated reader can clearly see, the Ideal of practical
reason is a representation of, as far as I know, the things in themselves; 
\begin{align}
\underset{}{\mathbf{v} \bigotimes \mathbf{w}} 
  & = \underset{}{\mathbf{v} \otimes \mathbf{w}}
      = \sum_{i=1}^3\sum_{j=1}^3a_{ij}u^iv^j \\[-.75em]
  & = \sum_{i=1}^3\left(a_{i1}u^iv^1+a_{i2}u^iv^2+a_{i3}u^iv^3\right) 
  \end{align}  
}


\begin{document}
\section{FIRST}
\begin{definition}[DEF]
  Hello world: $a^2+b^2=c^2$.
\end{definition}

\zlatexThmStyle{elegant}
\begin{theorem}[theorem Test]
  Hello world: $a^2+b^2=c^2$.
\end{theorem}


\section{SECOND}
% \zlatexColorSetup{
%   % Ztheorem={HTML}{8e44ad},
%   Ztheorem=red 
% }
\begin{Ztheorem}[Ztheorem Test]
  Hello world: $a^2+b^2=c^2$.
\end{Ztheorem}

\begin{Zproposition}
  Hello world: $a^2+b^2=c^2$.
\end{Zproposition}


\newpage
% \zlatexColorSetup{theorem=teal}
\section{Elegant Style Math Env}
\begin{axiom}[prime number]
  \boomen
\end{axiom}

% \zlatexThmStyle{HOOK}
\begin{definition}[prime number]
  \boomen
\end{definition}

\zlatexThmStyle{leftbar}
\begin{theorem}[prime number]
  \boomen
\end{theorem}

\zlatexThmTitleFormat{\textcolor{red}{\thmname{#1}:\thmnote{[#3]-}\thmnumber{#2}}}
\begin{lemma}[prime number]
  \boomen
\end{lemma}

\zlatexThmStyle{plain}
\begin{corollary}[prime number]
  \boomen
\end{corollary}

\newpage
\zlatexThmStyle{fancy}
\zlatexThmHook{begin=BEGIN, end=END, after=AFTER}
% \zlatexThmHook{begin=\begin{center}, end=\end{center}}
% \zlatexThmHook{before=\begin{framed}, after=\end{framed}}
\begin{proposition}
  \boomen
\end{proposition}


\zlatexThmStyleNew{before=\color{blue}\begin{framed}, after=\end{framed}}
\begin{definition}
  \boomen
\end{definition}
\end{document}














\documentclass{article}
\usepackage{xcolor}

\begin{document}
\ExplSyntaxOn\makeatletter
\regex_new:N \l__color_mode_regex
\regex_set:Nn \l__color_mode_regex {(\cB..{1,}\cE.){2}}
\cs_new:Npn \test__color_set:n #1 {
  \regex_match:NnTF \l__color_mode_regex {#1}{
    \definecolor{Test@\l_keys_key_str}#1
    \tl_set:ce {l_test_\l_keys_key_str _tl}{Test@\l_keys_key_str}
  }{
    \@ifundefined{\string\color@#1}{
      \msg_new:nnn {color} {undefined} {--->~Color~`#1'~undefined}
      \msg_error:nn {color} {undefined}
    }{
      \tl_set:ce {l_test_\l_keys_key_str _tl}{#1}
    }
  }
}
\keys_define:nn {test}{
  keyA  .tl_set:N  = \l_test_keyA_tl,
  keyA  .initial:n = {green},
  keyA  .code:n    = {\test__color_set:n {#1}},
  keyB  .tl_set:N  = \l_test_keyB_tl,
  keyB  .initial:n = {black},
  keyB  .code:n    = {\test__color_set:n {#1}},
}

% EXAMPLES
\textcolor{\l_test_keyA_tl}{TEXT-keyA}\par
\textcolor{\l_test_keyB_tl}{TEXT-keyB}\par

\keys_set:nn {test}{keyA={HTML}{0f45f3}}
\textcolor{\l_test_keyA_tl}{TEXT-keyA}\par

\keys_set:nn {test}{keyA={HTML}{8975f3}}
\textcolor{\l_test_keyA_tl}{TEXT-keyA}\par

\keys_set:nn {test}{keyB=orange}
\textcolor{\l_test_keyB_tl}{TEXT-keyB}\par

\keys_set:nn {test}{keyB=orangee}
\textcolor{\l_test_keyB_tl}{TEXT-keyA}\par
\makeatother\ExplSyntaxOff
\end{document}






% hyperref test
\documentclass{article}
\usepackage[T1]{fontenc}
\usepackage{amsthm}
\newcounter{theorem}
\newenvironment{theorem}{\begin{flushleft}\textbf{Theorem}\;\refstepcounter{theorem}\thetheorem\enspace}{\end{flushleft}}
\newtheorem{definition}{Definition}
\makeatletter
\setlength{\parindent}{0pt}
\def\hyper@nopatch@thm{}
\usepackage{lipsum}
\usepackage{xcolor}
\usepackage{hyperref}
% \let\Hy@theorem@refstepcounter\stepcounter
\hypersetup{
  bookmarksnumbered,
  colorlinks = true,
  linkcolor = red,
  urlcolor = blue,
}
\def\seeTargets{%
  \par\noindent\dotfill\\
  \string\@currentHref\ = \meaning\@currentHref\par
  \string\@currentHpage\ = \meaning\@currentHpage\par
  \string\Hy@currentbookmarklevel\ = \meaning\Hy@currentbookmarklevel\par\vskip3em
}

\begin{document}
\tableofcontents

% \@ifundefined{hyper@nopatch@thm}{UN-DEF}{DEF}
% \meaning\@thm


\newpage
\section{Test 1}\label{test1}
\subsection{sss b}
Hello world I. 

AAAA=BBBB
% \MakeLinkTarget{ztargetA}
\hyper@anchor{ztargetB}
\seeTargets

\newpage
\section{thm link}
\begin{theorem}\label{thm1}
This is a theorem: $a^2+b^2=c^2$.
\end{theorem}

\begin{definition}\label{def1}
This is a definition: $a^2+b^2=c^2$.
\end{definition}

\seeTargets


\newpage
\section*{Test 2}
Hello world II, the star version section target is:\par
\seeTargets

\MakeLinkTarget[newType]{}
\seeTargets


\section{see targets}
\subsection{sss aaa}
\subsection{sss bbb}
\seeTargets


\section{Manually change target}
\subsection{theHCOUNTER}
Original current target is:\par
\seeTargets

Then i change current target manually:\par 
\def\theHsection{1000}
\seeTargets

\textbf{Can NOT change the target manually by \string\theHCOUNTER.}

\subsection{MakeLinkTarget-star}
The command \string\MakeLinkTarget* can change \string\@currentHref{} manually:\par
\MakeLinkTarget*{newManualType} \label{checkTarget}
\seeTargets

% \vskip3em
% \meaning\ref

% \vskip2em
% \ExplSyntaxOn
% \cs_meaning:N \__cmd_start:nNNnnn
% \cs_meaning:N \__cmd_start_aux:NNnnnn
% \cs_meaning:N \__cmd_run_code:
% \ExplSyntaxOff


\newpage
\section{book mark}
% \meaning\pdfbookmark\par
% \meaning\new@pdflink\par

AAA:\pdfbookmark[1]{ManuallyBookMark-I}{targetA}
\seeTargets

BBB: \currentpdfbookmark{ManuallyBookMark-II}{targetB}
\seeTargets

CCC: \subpdfbookmark{ManuallyBookMark-III}{targetC}
\seeTargets

DDD:(Even bookmark level not step, the number in dstination show step) \belowpdfbookmark{ManuallyBookMark-IV}{targetD}
\seeTargets

EEE: \hyper@anchorstart{target.X}\hyper@anchorend
\seeTargets


\newpage
1. \string\ref{} commad Test:
\begin{itemize}
  \item Hello world III:\ref{test1}
  \item Hello THM: \ref{thm1}
  \item Hello DEF: \ref{def1}
\end{itemize}

\vskip3em
2. \string\hyperlink{} commad Test:
\begin{itemize}
  \item Link Target Manually: % \LinkTargetOn ztargetA \LinkTargetOff
  \item \hyper@link {url}{ztargetB}{link text}; 
        \hyper@linkstart{link}{ztargetB} long long link text \hyper@linkend
\end{itemize}

\vskip3em
3. \string\MakeLinkTarget{} commad Test:
\begin{itemize}
  \item \hyper@link {link}{section*.2}{link text};
  \item \hyper@link {link}{newType*.3}{link text};
\end{itemize} 

\vskip3em
4. bookmark related target link:
\begin{itemize}
  \item AAA: \hyper@link {link}{targetA.1}{link text 1};
  \item BBB: \hyper@link {link}{targetB.1}{link text 2};
  \item CCC: \hyper@link {link}{targetC.2}{link text 3};
  \item DDD: \hyper@link {link}{targetD.3}{link text 4};
  \item EEE: \hyper@link {link}{target.X}{link text 5};
\end{itemize} 
\end{document}





\InputIfFileExists{zlatex-cfg.tex}{}{}
\documentclass[
  fancy,
  % hyper,
  % lang=cn,
  % class=book,
  % font={config},
  % layout={margin},
  % layout={slide, theme=AnnArborSeahorse, aspect=16|9},
  % classOption={12pt}
]{../code/zlatex}
\zlatexThmLang{fr}
\zlatexloadlibrary{mathalias}
\zlatexSetup{mathSpec={envStyle=fancy}}
% \zlatexColorSetup{theorem=teal}
\zlatexThmCnt{share, parent=section}
\zlatexThmCreate{theorem}{Zaxiom, Ztheorem=Thm|green, Zproposition=Prop|orange} %Ztheorem=Thm|{HTML}{0f45f3}
\zlatexThmCreate{proof}{Zproof, Zexample=Example|red, Zsolution=Solution|}


\def\boomen{As any dedicated reader can clearly see, the Ideal of practical
reason is a representation of, as far as I know, the things in themselves; 
\begin{align}
\underset{}{\mathbf{v} \bigotimes \mathbf{w}} 
  & = \underset{}{\mathbf{v} \otimes \mathbf{w}}
      = \sum_{i=1}^3\sum_{j=1}^3a_{ij}u^iv^j \\[-.75em]
  & = \sum_{i=1}^3\left(a_{i1}u^iv^1+a_{i2}u^iv^2+a_{i3}u^iv^3\right) 
  \end{align}  
}

% \setlength{\fboxsep}{0pt}
\title{z\LaTeX{} Benchmark Test}
\author{Eureka}
\date{\today}
\begin{document}
\maketitle
% \chapter{Hello}
% \section{Py Fijw}
\section{FIRST}
Hello world; \meaning\FF


\begin{theorem}[Pythagorean theorem]\label{Pythagorean theorem}
This is a theorem:
  \begin{align}
    \R{d}^2x + \R{d}^2y = \R{d}^2z\\
    dx = 2
  \end{align}  
\end{theorem}

\begin{proof}\label{proof-1}
This is a proof: $a^2 + b^2 = c^2$
\end{proof}


\section{Internal Math Env}
\begin{zlatexZtheorem}[internal Env]
  Hello Internal Ztheorem
\end{zlatexZtheorem}


\section{New Math Env}
\begin{Zaxiom}
Hello Zaxiom
\end{Zaxiom}

\begin{Ztheorem}[New Thm Users' ENV] % ---> inside thm source
\boomen

\boomen
\end{Ztheorem}

\zlatexThmStyle{elegant}
\begin{Zproposition}[New Prop Users' ENV]\label{zprop-1}
\boomen

\boomen
\end{Zproposition}



\newpage\zlatexColorSetup{theorem=teal}
\section{Elegant Style Math Env}
\begin{axiom}[prime number]
  \boomen
\end{axiom}

\begin{definition}[prime number]
  \boomen
\end{definition}

\begin{theorem}[prime number]
  \boomen
\end{theorem}

\zlatexThmTitleFormat{\thmname{#1}:\thmnote{[#3]-}\thmnumber{#2}}
\begin{lemma}[prime number]
  \boomen
\end{lemma}

\zlatexThmStyle{plain}
\begin{corollary}[prime number]
  \boomen
\end{corollary}

\begin{proposition}
  \boomen
\end{proposition}


\section{New Proof Env}
\begin{Zproof}
Hello Zproof
\end{Zproof}

\begin{Zexample}
Hello Zexample
\end{Zexample}

\begin{Zsolution}\label{zsolu-1}
Hello Zsolution
\end{Zsolution}



% \href{http://www.google.com}{Google}

% \newlength{\myl}
% \settowidth\myl{\hbox{\Large\bfseries Eureka-Eureka-Eureka-Eureka-Eureka-}}
% \the\myl % 289.98325pt


% \ExplSyntaxOn
% \__zlatex_math_env_color_set:w thm|red\q_stop
% <==>
% \exp_last_unbraced:Ne \__zlatex_math_env_color_set:w {thm|red}\q_stop
% \tl_use:c {l__zlatex_thm_color_tl}
% \ExplSyntaxOff

\newpage
Hello world:

\begin{itemize}
  \item \cref{Pythagorean theorem}
  \item \cref{proof-1}
  \item \cref{zprop-1}
  \item \cref{zsolu-1}
\end{itemize}


% \includegraphics[width=.5\paperwidth]{example-image-a}
\end{document}

















\documentclass[
  % fancy,
  % hyper,
  % lang=cn,
  % class=book,
  % classOption={12pt}
  % font={config},
  % layout={margin},
  % layout={slide, theme=AnnArborSeahorse, aspect=16|9},
]{../code/zlatex}
\zlatexloadlibrary{mathalias}
% \zlatexThmCnt{share, parent=section}


\begin{document}
Hello world: $\int x \dd x$.
\begin{align}
  & \int x \dd x = \frac{1}{2}x^2 + \CC \\
  & e^{\int x \dd x} = e^{\frac{1}{2}x^2 + \CC} \\
  & \K{A} = \S{A} = \C{A} = \B{A} \\
  & A\pi+\sum = \F{A\pi+\sum} 
\end{align}


\begin{align}
  & \grad \cdot F = \div F \\
  & \grad F = \div F \\
  & \olddiv F \sim \oldhom F \\
  & \cong = \backsimeq
\end{align}


\begin{align}
  & \A a \xl b \xl[\alpha]((\psi)) c \\
  & a \ma b \xxl(\alpha) c \\
  & \E a \xr(\Theta = \Gamma + \zeta) b \xr[\alpha] c \\
  & a \da b \xxr{[}{[}\alpha{]}{]}(\beta - \gamma = \xi) c 
\end{align}
\end{document}







\documentclass{article}
\usepackage{amsmath, amsthm}


% preamble
\ExplSyntaxOn
\NewHook{zlatex/thm/titleformat}
\newtheoremstyle{zlatexMathEnv}
  {2pt}{2pt}{}
  {0pt}{\bfseries}{}
  {.25em}
  {
    \cs_set:Npn \ThmName { \thmname{#1} }
    \cs_set:Npn \ThmNumber { \thmnumber{#2} }
    \cs_set:Npn \ThmNote { \thmnote{#3} }
    % \tl_if_empty:nTF {#3}
    %   { \ThmName \ThmNumber }
    %   {  }
    \UseHook{zlatex/thm/titleformat}
  }
\theoremstyle{zlatexMathEnv}

\newcommand{\zlatexThmTitleFormat}[1]{
  \IfHookEmptyTF{zlatex/thm/titleformat}{}
    {\RemoveFromHook{zlatex/thm/titleformat}[once]}
  \AddToHook{zlatex/thm/titleformat}[once]{#1}
}
\ExplSyntaxOff


% main document
\begin{document}
\newtheorem{aaa}{AAA}
\zlatexThmTitleFormat{\ThmName. \ThmNumber [\ThmNote)}
\begin{aaa}[SECOND]
  ENV-CONTENT-SECOND
\end{aaa}
\end{document}









\documentclass{article}
\usepackage{amsmath, amsthm}


\begin{document}
\ExplSyntaxOn
\NewHookWithArguments{zlatex/thm/titleformat}{3}
\newtheoremstyle{zlatexMathEnv}
  {2pt}{2pt}{}
  {0pt}{\bfseries}{}
  {.25em}
  {
    \cs_set:Npn \ThmName { \thmname{#1} }
    \cs_set:Npn \ThmNumber { \thmnumber{#2} }
    \cs_set:Npn \ThmNote { \thmnote{#3} }
    % { \ThmName \ThmNumber \ThmNote }
    \UseHookWithArguments{zlatex/thm/titleformat}{3}{#1}{#2}{#3}
  }
\theoremstyle{zlatexMathEnv}

% \tl_put_right:Nn \l_text_expand_exclude_tl {\thmnote \thmnumber \thmname}
% \cs_generate_variant:Nn \text_expand:n {e}
% \newcommand{\foo}[1]{
%   \tl_set:Nn \l_tmpa_tl {\text_expand:n {#1}}
%   \tl_show:N \l_tmpa_tl
% }
% \foo{\thmname{##1} \thmnumber{#2} \thmnote{[#3]}}


% ===>
% \cs_new_protected:Nn \keythms_thmstyle_thmname:n { \thmname{#1} }
% \cs_new_protected:Nn \keythms_thmstyle_thmnumber:n { \thmnumber{#1} }
% \cs_new_protected:Nn \keythms_thmstyle_thmnote:n { \thmnote{#1} }

% \tl_put_right:Nn \l_text_expand_exclude_tl { \thmnote \thmnumber \thmname }
% % ^ allows \thmnote, etc. to work in headstyle; hope no bad side effects!
% % user is in charge of spacing with \NAME and \NUMBER (thmtools compat...)
% \cs_set:Npn \NAME { \keythms_thmstyle_thmname:n { ##1 } }
% \cs_set:Npn \NUMBER
%   {
%     \keythms_thmstyle_thmnumber:n
%       {
%         \group_begin:
%         ##2
%         \group_end:
%       }
%   }
% \cs_set:Npn \NOTE
%   {
%     \keythms_thmstyle_thmnote:n {##3}
%   }

% \text_expand:n { \keythms_thmstyle_headcmd:nnn{##1}{##2}{##3} }
% \cs_set:Nn \keythms_thmstyle_headcmd:nnn
% {
%   \makebox[0pt][r]{
%     \keythms_thmstyle_thmnumber:n {
%         ##2
%       }
%     }
%   \NAME\NOTE
% }
% \foo{\NAME \NUMBER \NOTE}
% \foo{\thmname{##1} \thmnumber{##2} \thmnote{[##3]}}



\newcommand{\zlatexThmTitleFormat}[1]{
  \IfHookEmptyTF{zlatex/thm/titleformat}{}
    {\RemoveFromHook{zlatex/thm/titleformat}[once]}
  \AddToHookWithArguments{zlatex/thm/titleformat}[once]{#1}
}
\ExplSyntaxOff
\newtheorem{aaa}{AAA}

% \zlatexThmTitleFormat{\thmname{#1}: \thmnumber{#2} \thmnote{[#3]}}
\begin{aaa}[FIRST]
  ENV-CONTENT-FIRST
\end{aaa}

% \zlatexThmTitleFormat{\thmname{#1}. \thmnumber{#2} \thmnote{(#3)}}
\zlatexThmTitleFormat{\ThmName. \ThmNumber [\ThmNote)}
\begin{aaa}[SECOND]
  ENV-CONTENT-SECOND
\end{aaa}

\begin{aaa}
  ENV-CONTENT-THIRD
\end{aaa}
\end{document}








\documentclass[
  layout={
    slide, 
    aspect=16|9, 
    % theme=AnnArborAlbatross
  },
  % mathSpec={
  %   font=hello,
  %   world=haha
  % },
  % lang=cn,
  packageOption={
    % geometry={margin=3in},
    % fontspec={quiet},
    % fontenc={T1}
  }
]{../code/zlatex}
% \geometry{margin=3in}%{geometry}
% \documentclass[
%   fancy,
%   hyper,
%   % class=hello,
%   classOption={12pt},
%   layout={margin, slide, aspect=16|9, theme=AnnArborAlbatrosss},
%   % font=config,
%   toc={titlee=CONTENTS},
%   mathSpec={
%     font=eulerr, 
%     counter=world
%   },
%   font=lm,
%   layout=openAll,
%   bib_index={Notload},
%   % lang=cn,
%   hello=world,
% ]{../code/zlatex}
% \usepackage{pifont}
% \usepackage{anyfontsize}
\zlatexSetup{mathSpec={envStyle=fancy, font=mathpazo}}
% \zslideSetup{
%   world=haha,
%   toc={hello=hi}
% }



% \documentclass{article}


\title{Debug}
\author{Eureka}
\date{\today}
\begin{document}
\maketitle
% \tableofcontents
% \newpage


Hello world
\begin{theorem}
This is Hello world:

\begin{align}
  \sum_{i=1}^{+\infty}{\int_{0}^{i}-\frac{1}{t}\mathrm{d}t} = \frac{\pi^2}{6}
\end{align}
\end{theorem}


% \newpage 
% \section{Page 2}
% \subsection{sss}
% world

% \newpage
% \section{Math Formula}
% \begin{align}
%   \sum_{i=1}^{+\infty}{\int_{0}^{i}-\frac{1}{t}\mathrm{d}t} = \frac{\pi^2}{6}
% \end{align}


\ExplSyntaxOn
\ExplSyntaxOff

\end{document}
