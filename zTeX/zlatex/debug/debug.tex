% ==>  expandable tl replace
\InputIfFileExists{zlatex-cfg.tex}{}{}
\documentclass[
  % layout=margin
]{../code/ztex}
\usepackage[T1]{fontenc}
\parindent0pt


\begin{document}
\ExplSyntaxOn
% \cs_new:Npn \ztex_tl_pattern_range:nn #1#2
%   {
%     \exp_args:Ne \int_step_tokens:nn { \tl_count:n {#1}-\tl_count:n {#2}+1 }
%       {
%         \__ztex_tl_pattern_range:nnn { #1 }{ #2 }
%       };
%   }
% \cs_new:Npn \__ztex_tl_pattern_range:nnn #1#2#3
%   {
%     \exp_args:Ne \ztex_tl_if_eq:nnTF 
%       { \tl_range:nnn {#1}{#3}{#3+\tl_count:n {#2}-1} }{ #2 }
%       { ;#3, \int_eval:n {#3+\tl_count:n {#2}-1} }
%       { }
%   } 
% \cs_generate_variant:Nn \tl_range:nnn {nee, nne, nen}
% \cs_new:Npn \__ztex_gen_token_in_range:nnnn #1#2#3#4
%   {
%     \int_case:nnF {#4}
%       {
%         {1}{
%           \tl_range:nne {#1}{1}
%             {
%               \clist_item:en { \sclist_item:nn {#2}{#4} }{1} - 1
%             }
%         }
%         {\sclist_count:n {#2}}{
%           \tl_range:nen {#1}
%             {
%               \clist_item:en { \sclist_item:nn {#2}{#4} }{2} + 1
%             }{-1}
%         }
%       }{ #3
%         \int_compare:nNnTF 
%           {\clist_item:en { \sclist_item:nn {#2}{#4-1} }{2} + 1}
%           =
%           {\clist_item:en { \sclist_item:nn {#2}{#4} }{1}}
%         {}{
%           \tl_range:nee {#1}
%             {
%               \clist_item:en { \sclist_item:nn {#2}{#4-1} }{2} + 1
%             }{
%               \clist_item:en { \sclist_item:nn {#2}{#4} }{1} - 1
%             }
%         }
%       }
%   }


xxxxabc123def123123fgh123xxx123asdwzzz\par

\ztex_tl_pattern_range:nn 
  {xxxxabc123def123123fgh123xxx123asdwzzz}{123}\par % ';8,10;14,16;17,19;23,25;29,31;'

% syntax is the same as \tl_replace_<>:Nnn 
% ⟨tl var⟩ {⟨old tokens⟩} {⟨new tokens⟩}
\cs_new:Npn \ztex_tl_replace:nnn #1#2#3 
  {
    \int_step_tokens:nn 
      {
        \sclist_count:e {\ztex_tl_pattern_range:nn {#1}{#2}}
      }{
        \exp_args:Nee \__ztex_gen_token_in_range:nnnn {#1}
          {
            \ztex_tl_pattern_range:nn {#1}{#2}
          }{ #3 }
      }
  }

% \cs_new:Npn \ztex_tl_replace_once:nnn #1#2#3 
%   {
%     \exp_args:Nee \__ztex_gen_token_in_range:nnnn {#1}
%       {
%         \sclist_item:en { \ztex_tl_pattern_range:nn {#1}{#2} }
%           { 1 }
%       }{ #3 }{ 1 }
%   }
\ztex_tl_replace:nnn {xxxxabc123def123123fgh123xxx123asdwzzz}{123}{|XXX|}\par

\ztex_tl_replace_once:nnn {xxxxabc123def123123fgh123xxx123asdwzzz}{123}{|XXX|}\par
% \__ztex_gen_token_in_range:nnnn {xxxxabc123def123123fgh123xxx123asdwzzz}{8,10;}{|XXX|}{1}


% \edef\TTTa{\ztex_tl_replace_once:nnn {xxxxabc123def123123fgh123xxx123asdwzzz}{123}{|XXX|}}
% \show\TTTa % ---> ''
% \edef\TTTb{\ztex_tl_replace:nnn {xxxxabc123def123123fgh123xxx123asdwzzz}{123}{|XXX|}}
% \show\TTTb % ---> 'xxxxabc|XXX|def|XXX||XXX|fghasdwzzz'


% \exp_args:NNf \def\TTTc{\ztex_tl_replace:nnn {xxxxabc123def123123fgh123xxx123asdwzzz}{123}{|XXX|}} % does NOT works
\exp_args:NNe \def\TTTc{\ztex_tl_replace:nnn {xxxxabc123def123123fgh123xxx123asdwzzz}{123}{|XXX|}}
\show\TTTc
\ExplSyntaxOff
\end{document}





% ===> expandable '\tl_if_eq:nnTF' and '\tl_if_in:nnTF'
\InputIfFileExists{zlatex-cfg.tex}{}{}
\documentclass[
  % layout=margin
]{../code/ztex}
\ExplSyntaxOn
\NewDocumentCommand{\tlifeq}{mmmm}
  {
    \ztex_tl_if_eq:nnTF {#1}{#2}{#3}{#4}
  }


% \prg_new_conditional:Npnn \ztex_tl_if_eq:nn #1#2 {p, T, F, TF}
%   {
%     \exp_args:Ne \int_compare:nTF {\tl_count:n {#1} = \tl_count:n {#2}}
%       {
%         \exp_args:Ne \int_compare:nTF {
%           \exp_not:N \int_from_bin:n { \__ztex_tl_if_eq_aux:nn {#1}{#2} } 
%           = 
%           \exp_not:N \int_from_bin:n { \prg_replicate:nn {\tl_count:n {#1}}{1} }
%         }{ \prg_return_true: }{ \prg_return_false: }
%       }{ \prg_return_false: }
%   }
% \cs_new:Npn \__ztex_tl_if_eq_aux:nn #1#2
%   {
%     \exp_args:Ne \int_compare:nTF {\tl_count:n {#1} = \tl_count:n {#2}}
%       {
%         \int_step_tokens:nn {\tl_count:n {#1}}
%           {
%             \__ztex_tl_if_eq_aux_iii:nnnnn {#1}{#2} % '{ab}' will raise error
%               { 1 } { 0 } 
%           }
%       }{ 0 }
%   }
% \prg_new_conditional:Npnn \__ztex_tl_if_eq_aux_ii:nnn #1#2#3 {T, F, TF}
%   {
%     \exp_args:Nee \__ztex_token_if_eq:nnTF 
%       {\tl_item:nn {#1}{#3}}{\tl_item:nn {#2}{#3}}
%       { \prg_return_true:  }
%       { \prg_return_false: }
%   }
% \cs_new:Npn \__ztex_tl_if_eq_aux_iii:nnnnn #1#2#3#4#5
%   {
%     \__ztex_tl_if_eq_aux_ii:nnnTF {#1}{#2}{#5}{#3}{#4}
%   }
\ExplSyntaxOff



\begin{document}
\tlifeq{a}{a}{EQ}{NOT~EQ},   % EQ 
\tlifeq{a}{b}{EQ}{NOT~EQ},   % NOT EQ
\tlifeq{aa}{aa}{EQ}{NOT~EQ}, % EQ
\tlifeq{aa}{ab}{EQ}{NOT~EQ}.\par % NOT EQ


\ExplSyntaxOn
\edef\TTTa{\ztex_tl_if_eq:nnTF {abcdefg}{abcdefh}{EQ}{NOT~EQ}}
% \show\TTTa  % NOT EQ
\detokenize\expandafter{\expanded{\TTTa}},~

\edef\TTTb{\ztex_tl_if_eq:nnTF {ab\c_colon_str cd}{ab:cd}{EQ}{NOT~EQ}}
% \show\TTTb % NOT EQ
\detokenize\expandafter{\expanded{\TTTb}},~

\str_set:Nn \l_tmpa_str {:}
\edef\TTTc{\ztex_tl_if_eq:nnTF {ab\c_colon_str cd}{ab\l_tmpa_str cd}{EQ}{NOT~EQ}}
% \show\TTTc % EQ
\detokenize\expandafter{\expanded{\TTTc}}.\par
\ExplSyntaxOff


\tlifeq{a{a}}{aa}{EQ}{NOT~EQ},   % EQ
\tlifeq{aaa}{a{aa}}{EQ}{NOT~EQ}, % NOT EQ
\tlifeq{aaa}{aaa}{EQ}{NOT~EQ}.\par   % EQ 
\end{document}








% ==> test minus order in \pdv
% NOTE: many bugs when excute symbolic calculation
\InputIfFileExists{zlatex-cfg.tex}{}{}
\documentclass[
  % layout=margin
]{../code/ztex}
\ztexloadlib{alias}

\begin{document}
\zaliasOn
Hello world.

\begin{align}
  1 + 1 & = \dv{\varphi, x, y, z_e, \tau}
            [-2, ++3, -4.125, x, +z] \\
  1 + 1 & = \dv{\varphi, x, y, z_e, \tau}
        [--2, ++3, -4.125, x] \\
  1 + 1 & = \dv{\varphi, x, y, z_e, \tau}
          [{--}2, ++3, -4.125, x, \textsf{z--z}]
\end{align}
\zaliasOff
\end{document}





% ==> tabularray compatiblity test
\InputIfFileExists{zlatex-cfg.tex}{}{}
\documentclass[
  % layout=margin
]{../code/ztex}
\ztexloadlib{alias}
\parindent0pt
\setlength{\fboxsep}{0pt}
\ExplSyntaxOn
\edef\MatDataA{\zalias_matrix_from_list:n {1, 2.00, , 4, ; , 6, 7.00, 9, 10 ; , 12, 13.00, , }}
\edef\MatDataB{\zalias_diag_mat_data:nnnn {\c_false_bool}{?}{*}{1.00, , 2, 3, , 5}}
\edef\MatDataC{\zalias_diag_mat_data:nnnn {\c_true_bool}{@}{*}{1.00, , 2, 3, , 5}}
\edef\MatDataD{\zalias_jmat_data:nn {textstyle}{f, g; x, y, z}}
\protected\def\cmdA#1#2{g^{#1#2}}
\edef\MatDataF{\zalias_xmat_data:nn {\cmdA}{3, 4}}
\ExplSyntaxOff
\usepackage{tabularray}



\begin{document}
% \edef\TTT{\MatData}
% \show\TTT

\SetTblrOuter{expand={\TTT, \MatDataA, \MatDataB, \MatDataC, \MatDataD, \MatDataF}}
\hskip6em
\begin{tblr}
  {
    rowspec = {
      |[2pt,green7]Q|[teal7]Q|[green7]Q|[2pt, green6]
      Q|[green5]Q|[green4]Q|[green3]Q|[3pt,teal7]
    }
  }
  \MatDataA
\end{tblr}

Anti-Diagnoal = 
\begin{tblr}{ hlines, vlines }
  \MatDataB
\end{tblr}

Diagnoal = 
\begin{tblr}{ hlines, vlines }
  \MatDataC
\end{tblr}

jmat =
\begin{tblr}{ hlines, vlines, cells={mode=math} }
  \MatDataD
\end{tblr}

xmat =
\begin{tblr}{ hlines, vlines, cells={mode=math} }
  \MatDataF
\end{tblr}

\zaliasOn
Hello world: $\dd x/\dd y$ = $\mathrm{d}x / \mathrm{d}y$.

\begin{align}
  \dd x/\dd y = \mathrm{d}x / \mathrm{d}y
\end{align}
\zaliasOff
\end{document}







% ==> create matrix-relatex alias 
\InputIfFileExists{zlatex-cfg.tex}{}{}
\documentclass[
  % layout=margin
]{../code/ztex}
\ztexloadlib{alias}
\parindent0pt
\setlength{\fboxsep}{0pt}
\ExplSyntaxOn
\ExplSyntaxOff


\begin{document}
\zaliasOn
\section{TEST}
Hello world:

\begin{align}
  \begin{Bmatrix}
    1 & 2 & 3 \\
    4 & 5 &  \\
    6 & 7 & 8 %\\ % this '\\' is optional
  \end{Bmatrix} & 
  % \begin{Vmatrix}
  %   \mat { 1, , 3;   4, 5, ; , 7, 8 }
  % \end{Vmatrix}
  % \text{mat-1}  = \mat  { 1, , 3;   4, 5, ; , 7, 8 } \qquad
  %   & \text{mat-2} = \begin{Vmatrix}\mat{1, , 3; 4, 5, ; , 7, 8 }\end{Vmatrix} \\
  % \text{pmat} = \pmat { 1, , 3;   4, 5, ; , 7, 8 } \qquad
  %   & \text{bmat}   = \bmat { 1, , 3;   4, 5, ; , 7, 8 } \\
  % \text{Bmat} = \Bmat { 1, , 3;   4, 5, ; , 7, 8 } \qquad
  %   & \text{vmat}   = \vmat { 1, , 3;   4, 5, ; , 7, 8 } \\
  % \text{Vmat-1} = \Vmat { 1, , 3;   40.102, 55, ; , 7, 8 } \qquad
  %   & \text{Vmat-2}   = \Vmat { 1, , 3;   \textsf{xxx}, \mathbb{XX}, ; , 7, 8 }
\end{align}


% \begin{align}
%   \text{mat}  = & \mat  { 1, , 3;   4, 5, ; , 7, 8 }\\
%   \text{pmat} = & \pmat { 1, , 3;   4, 5, ; , 7, 8 }\\
%   \text{bmat} = & \bmat { 1, , 3;   4, 5, ; , 7, 8 }\\
%   \text{Bmat} = & \Bmat { 1, , 3;   4, 5, ; , 7, 8 }\\
%   \text{vmat} = & \vmat { 1, , 3;   4, 5, ; , 7, 8 }\\
%   \text{Vmat-1} = & \Vmat { 1, , 3;   40.102, 55, ; , 7, 8 } \\
%   \text{Vmat-2} = & \Vmat { 1, , 3;   \textsf{xxx}, \mathbb{XX}, ; , 7, 8 }
% \end{align}



% \begin{align}
%   % \imat{x}{1, ,3}
%   % \mat{\imat{}{}} =
%   \mat{\zmat{5}} = 
%   \mat{\zmat[z]{5}} = 
%   \mat{\zmat[a]{5}} = 
%   \mat{\imat{x}{1, ,3}} =
%   \pmat{\admat{}{1, 2, , 4, 5}} =
%   \vmat{\imat{\cdot}{1,,,2}}
% \end{align}

% \begin{align}
%   \begin{bmatrix}
%     \frac {\partial f_1}{\partial x}&\frac {\partial f_1}{\partial y}\\
%     \frac {\partial f_2}{\partial x}&\frac {\partial f_2}{\partial y}\\
%   \end{bmatrix}
% \end{align}

\begin{align}
  \jmat{f_1, f_2; x, y} = 
  \jmat[c=displaystyle, b=V, s=2]{f, g, h; \textsf{x}, \mathbb{Y}, \F{z}} =
  \jmat[b=b]{f, g; x, y, z} =
\end{align}


% \hmat{f;x,y,z}
\begin{align}
  \pdv{,x}[2] = \pdv{,x,x}[1, 1] \\
  \pdv{,x}[2] = \pdv{h,x,y}[1, 1] 
\end{align}
% 'f' --> 'x': loop 'x', 'y', 'z';
% 'f' --> 'y': loop 'x', 'y', 'z';
% 'f' --> 'z': loop 'x', 'y', 'z';
\begin{align}
  \hmat{f;x,y}  = 
  \hmat[c=displaystyle, s=2.5]{;x,y,z, {w\textbf{w}}} =
  \hmat[b=v, s=1.5]{g;\textsf{x},\mathbb{K},z}
\end{align}


\protected\def\cmdA#1#2{g^{#1#2}}
\begin{align}
  \begin{bmatrix}
    \xmat{3, 4, \cmdA}
  \end{bmatrix} = 
  \begin{bmatrix}
    \gmat{v_1, v_2, v_3, v_4}
  \end{bmatrix}
\end{align}


\ExplSyntaxOn\makeatletter
% \zclistpatch {, 1, , 3, }
% \edef\TTTa{\zclistpatch {, 1, , 3, }} % \scan_stop: ,1,\scan_stop: ,3,\scan_stop:,
% \show\TTTa

% \edef\TTT{\__zalias_matrix_from_list:n { 1, , 3; 4, 5,; 6, 7, 8 }}
% \show\TTT
% \begin{align}
%   \begin{bmatrix}
%     % \__zalias_matrix_from_list:n { 1, , 3; 4, 5, ; 6, 7, 8 }
%     \mat { 1, , 3; 4, 5, ; 6, 7, 8 }
%   \end{bmatrix}
% \end{align}


% --> '\zcmd_clist_patch:n' fails when enter list '1, ,3'
% 1. turn '1, ,3' ---> '1, ,3,'  (add a trim ',')
% 2. add a scan mark and ',' to end the Recursion 
%    ---> '1, ,3  ,\s__clist_patch_stop,'
% \scan_new:N \s__clist_patch_stop
% \cs_new:Npn \__zzcmd_clist_patch:w #1, 
%   { 
%     \tl_if_empty:eTF { #1 } 
%       {
%         (FILLER)\__zzcmd_clist_patch:w
%       }{
%         \tl_if_eq:NNTF #1\s__clist_patch_stop
%           {[END]}
%           {
%             % \typeout{------->\#1=#1}
%             (#1)\__zzcmd_clist_patch:w
%           }
%       }
%   }
% % \__zcmd_clist_patch:w  1, ,3  ,\s__clist_patch_stop,\par % output is: '(1)(FILLER)(3)[END]'
% % \__zcmd_clist_patch:w ,1, ,3, ,\s__clist_patch_stop,\par
%%%% then just add ',\s__clist_patch_stop,' is enough 
% \cs_new:Npn \z_clist_patch:n #1 
%   { 
%     \__zzcmd_clist_patch:w #1 
%       ,\s__clist_patch_stop, 
%   }
% \z_clist_patch:n {  1, 2, 3}\par
% \z_clist_patch:n {, 1, 2, 3}\par
% \z_clist_patch:n {  1, 2, 3, }\par
% \z_clist_patch:n {, 1, 2, 3, }\par

% \edef\TTT{\z_clist_patch:n {, 1, 2, 3, }}
% \edef\TTTa{\zcmd_clist_patch:n {, 1, , 3, }}
% \show\TTTa
% \edef\TTTb{\zcmd_sclist_patch:n {; 1; ; 3; }}
% \show\TTTb


% ---> space check:
% \tl_if_empty:nTF {~}{\show\def}{\show\xdef} % log is:> \xdef=\xdef.
% \tl_if_blank:nTF {~}{\show\def}{\show\xdef} % log is:> \def=\def.
% \tl_if_blank:nTF {}{\show\def}{\show\xdef} % log is:> \def=\def.


% ---> \mat { 1, , 3;   400, 555, ; , 7, 8 } ERROR
% \zcmd_clist_patch:n {1, 2, , 3} % works
% \zcmd_clist_patch:n {1, 2, , 33} % ERROR
% \zcmd_clist_patch:n {11} % ERROR
% \__zcmd_clist_patch:w 11 ,\s__clist_patch_stop, % ERROR
% \tl_count:n {{11}} + \tl_count:n {\use:n{11}} + \tl_count:n {11}
% \__zcmd_clist_patch:w 11 ,\s__clist_patch_stop,
% \edef\TTT{\zcmd_clist_patch:n {1, 2, , 3,}}
% \zcmd_clist_patch:n {1, 2, , 3,}
% \edef\TTT{\zcmd_clist_patch:n {1, 2, , 33, \textsf{xxx}, \mathbb{ZZ},,}} 
% \show\TTTa  % 1,2,\scan_stop: ,33,\textsf {xxx},\mathbb {ZZ},\scan_stop: ,\scan_stop: ,
% \edef\TTTb{\zcmd_sclist_patch:n {1; 2; ; 33; \textsf{xxx}; \mathbb{ZZ};}}
% \show\TTTb  % 1;2;\scan_stop: ;33;\textsf {xxx};\mathbb {ZZ};\scan_stop: ;


% ---> general list patch
% \__zcmd_sclist_patch:nw {X} 111; \textbf{2}   ;\s__sclist_patch_stop;
% \edef\TTT{\zcmd_clist_patch:nn {\scan_stop:}{1, 222, ,\textbf{333}, , }}
% \edef\TTT{\zcmd_sclist_patch:nn {\scan_stop:}{1; 222; ;\textbf{333}; ; }}
% \show\TTT


% \cs_new:
% \typeout{-------->:\zclist_count:n {1, , 2, 3, }}
% \edef\TTT{\zcmd_clist_patch:nn {\scan_stop:}{1, , 2, 3, }} % 5
% \cs_new:Npn \__zalias_imat:nn #1#2 
%   {
%     % \def\TTT{#1} % 1,0,2,3,0,
%     % \show\def
%     \typeout{----->\#1=(#1), \#2=(#2)}
%     \prg_replicate:nn { #2-1 }{ 0 }
%       [\clist_item:nn { #1 }{#2}]
%     \prg_replicate:nn { \clist_count:n {#1} - #2 }{ 0 }
%     \par
%   }
% \exp_args:Ne \int_step_tokens:nn {\zclist_count:n {1, , 2, 3, }}
%   { 
%     \exp_args:Ne \__zalias_imat:nn {\zcmd_clist_patch:nn {0}{1, , 2, 3, }} 
%   }
% \show\TTT

% \edef\TTT{\z@imat{1, 2}}
% \edef\TTT{\imat{0}{1, , 3}}
% \imat{FILLER}{1, ,3}
% \show\TTT % [1]&0&0\\0&[0]&0\\0&0&[3]


% --> jmat and hmat test
% \__zalias_jmat_data:nn {frac}{f_1, f_2; x, y}
% \edef\TTT{\__zalias_jmat_data:nn {frac}{f_1, f_2; x, y}}
% \edef\TTT{\__zalias_jmat_data:n {f_1, f_2; x, y}}
% \show\TTT
% \frac {\partial f_1}{\partial x}&\frac {\partial f_1}{\partial y}\\
% \frac {\partial f_2}{\partial x}&\frac {\partial f_2}{\partial y}\\

% \edef\TTT{\__zalias_hmat_data:nn {}{f; x, y}}
% \show\TTT
% \z@pdv {f,x,x}[1,1]&\z@pdv {f,x,y}[1,1]\\
% \z@pdv {f,y,x}[1,1]&\z@pdv {f,y,y}[1,1]\\


% ---> \hmat bug
% \__zalias_hmat_data:nn {}{;x,y,z}
% \__zalias_hmat_data:nn {}{; x, y, z}
% \edef\TTT{\__zalias_hmat_data:nn {}{; x, y, z}}
% \show\TTT
% \[
% \begin{Vmatrix}
%   \z@pdv {~,x}[2]&\z@pdv {,x,y}[1,1]&\z@pdv {\mbox{},x,z}[1,1]\\
%   \z@pdv {\scan_stop: ,y,x}[1,1]&\z@pdv {\scan_stop: ,y}[2]&\z@pdv {\scan_stop: ,y,z}[1,1]\\
%   \z@pdv {\scan_stop: ,z,x}[1,1]&\z@pdv {\scan_stop: ,z,y}[1,1]&\z@pdv {\scan_stop: ,z}[2]\\
% \end{Vmatrix} \]
% TODO: '\z@pdv {\scan_stop: ,z,x}' will raise bug !!!
% \__zalias_hmat_data:nn {}{; \textsf{x}, y, z}


% ---> \xmat test
% \cs_set_protected:Npn \__cmdA:nn #1#2 {(#1)-[#2]}
% \edef\TTTa{\z@xmat{2, 3, \__cmdA:nn}}
% \show\TTTa
% \[
%   \begin{bmatrix}
%     \TTTa
%   \end{bmatrix}
% \]
\makeatother\ExplSyntaxOff
\zaliasOff
\end{document}








% ==> semicolon list implementation
\InputIfFileExists{zlatex-cfg.tex}{}{}
\documentclass[
  % layout=margin
]{../code/ztex}
\ztexloadlib{alias}
\setlength{\fboxsep}{0pt}
\usepackage{expl3}
\begin{document}
\section{TEST}


\ExplSyntaxOn\makeatletter
% \__ztex_token_if_eq:nnTF {12}{12}{}{}
% \edef\TTT{\sclist_item:nn {a; b; c; d}{-1}}
% \show\TTT

% \edef\XXX{\sclist_count:n {a; b; c;}}
% \show\XXX


% \sclist_set:Nn \l_tmpa_sclist {a; b; c; d; e}
% \sclist_show:N \l_tmpa_sclist
% \group_begin:
% \sclist_gset:Nn \l_tmpa_sclist {a; b; c; d;}
% \group_end:
% \sclist_show:N \l_tmpa_sclist

% \edef\AAA{\sclist_item:cn {l_tmpa_sclist}{-2}}
% \show\AAA

% % expandable test
\cs_set:Npn \__test_sclist:n #1 {(#1)|}
% \edef\YYY{
%   \sclist_map_function:NN \l_tmpa_sclist
%     \__test_sclist:n
% }
% \edef\YYY{
%   \sclist_map_function:nN {1;2;3;4}
%     \__test_sclist:n
% }
% \detokenize\expandafter{\expanded{Hello \YYY\space #}} % 'Hello (1)|(2)|(3)|(4)| ##'
% \show\YYY


% \edef\TTT{\zcmd_clist_patch:n {,1,2,}}
% \show\TTT
% \setlength{\fboxsep}{3pt}
% \def \clistA {,1,2,}
% \zclist_count:o { \clistA };
% \fbox{\zclist_item:on { \clistA }{2}}, \fbox{\zclist_item:on { \clistA }{-1}}; 
% \detokenize\expandafter{\expanded{\zclist_range:onn { \clistA }{1}{3}}}


% \sclist_new:N \l_tmpc_sclist 
% \sclist_set:Nn \l_tmpc_sclist {1;23;456;} 
% \cs_set:Npn \__test_sclist_map:nn #1#2 {[#1](#2)|}
% \edef\TTTa{
%   \sclist_map_tokens:nn {a;bc;def}
%     { \__test_sclist_map:nn {XX} }
% }
% \edef\TTTb{
%   \sclist_map_tokens:Nn \l_tmpc_sclist
%     { \__test_sclist_map:nn {YY} }
% }
% \show\TTTa
% \show\TTTb

% --> test sclist clear
% \sclist_new:N \l_tmpc_sclist 
% \sclist_show:c {l_tmpc_sclist}
% \sclist_set:Nn \l_tmpc_sclist {1;23;456;} 
% \sclist_show:c {l_tmpc_sclist}
% \sclist_clear:c {l_tmpc_sclist}
% \sclist_show:c {l_tmpc_sclist}
\makeatother\ExplSyntaxOff
\end{document}





% ==> create commands from physics package
\InputIfFileExists{zlatex-cfg.tex}{}{}
\documentclass[
  % layout=margin
]{../code/ztex}
\ztexloadlib{alias}
\setlength{\fboxsep}{0pt}
\NewDocumentCommand{\TTT}{m}{\mathsf{#1}}



\begin{document}
\section{TEST}
% \ExplSyntaxOn
% $\cs:w mathbb \cs_end:{a}$
% \ExplSyntaxOff

% \show\def
\zaliasOn
% \ExplSyntaxOn
% \seq_show:N \g__ztex_mathalias_internal_seq
% \ExplSyntaxOff
% % Hello world: $\dv*{\TTT{h}, \alpha}[2]$ is excellent.

% % Hello world: $\dv*{\FF{h}, \alpha}[2]$ is excellent.
% \show\F
% \show\R
% \begin{align}
%   \F{A} = \R{A}
% \end{align}
% \zaliasOff
% \end{document}
% \show\mathbf
% Hello world: $\dv*{\mathbf{h}, \alpha, \textit{zz}}[2, \F{x}]$ is excellent.
% Hello world: $\dv*{\mathbb{H}, \alpha, \textsf{x}}[\textrm{x}, \textbf{x}]$ is excellent.


\begin{align}
  % \dv{, x}\\
  \dv{, xx, y, \textsf{ww}}[zz, \mathbf{g}, \B{X}] & = 
  % \dv{, x}[\textsf{z}]\\
  % \dv{, x, y}[1, x+2]\\
  \dv{, x, y, z}[, +++\alpha+1, +\xi+3+, \eta+2]\\
  \dv{, x} & + \dv{, t}[2] = \dv*{f, \xi}\\
  1 + 1 & = \dv{\varphi, x, y, z_e, \tau}
                [2, 3, 4.125, x] \\
  1 + 1 & = \dv{, x, y, z}
                [1, \xi, \eta+2] \\
  1 + 1 & = \dv{, (x^1), (x^2), (x^3)}
                [1, 3, 1]
\end{align}


\begin{align}
  \pdv{, x} & + \pdv{, t}[2] = \pdv*{f, \xi}\\
  1 + 1 & = \pdv{\varphi, x, y, z, \tau}
                [2, 2, 2, 1] \\
  1 + 1 & = \pdv{, x, y, z}
                [1, \xi, \eta+2] \\
  1 + 1 & = \pdv{, (x^1), (x^2), (x^3)}
                [1, 3, 1]
\end{align}
% \zaliasOff
% \show\mathbf

\ExplSyntaxOn
% ---> clist patch
% \clist_item:nn {, 1, 2, 3}{1}\par
% \__zalias_item_first:w x,1, 2, 3\scan_stop:\par
% \clist_clist_tail:n {x, y, z, w}
% \seq_set_from_clist:Nn \l_tmpa_seq {, 2, 3, }
% \zcmd_clist_tail:n {,2,3,}
% \edef\TTT{2, 3, \scan_stop:}
% \show\TTT % --> 2, 3, \scan_stop:
% \zcmd_clist_patch:n {, 2, 3, }
% \exp_args:NNe \seq_set_from_clist:Nn \l_tmpa_seq {\zcmd_clist_patch:n {, 2, 3, }}
% \seq_show:N \l_tmpa_seq
% \fbox{\seq_item:Nn \l_tmpa_seq {1}}
% \fbox{\seq_item:Nn \l_tmpa_seq {-1}}

% ---> zclist empty item test
% \zclist_item:nn {xxx, yyy}{1} % --> works 
% \zclist_item:nn {xxx, yyy}{0} % --> works 
% \zclist_item:nn {xxx, yyy}{3} % --> works 
% \zclist_item:en {}{3}   % --> works
% \zclist_item:en {,1,}{2} % --> works
% \zclist_item:en {2}{2} % --> ERROR: ! Argument of \__zcmd_clist_head:w has an extra }.
% \zclist_item:nn {2}{2} % --> ERROR: ! Argument of \__zcmd_clist_head:w has an extra }.
% \zclist_item:nn {2}{1}

% ---> zclist count
% \zclist_count:n {, x}

% ---> clist range
% \edef\TTT{"\zclist_range:nnn {, 2, 3, 4, }{2}{4}"}
% \show\TTT % --> "2,3,4"

% --> extract all valid numbers using l3regex
% \tl_new:N \l_extract_tl
% \regex_set:Nn \l_extract_tl { -?(?:\d+\.\d*|\.\d+|\d+) }
% \seq_new:N \l_extract_seq
% \regex_extract_all:NnN \l_extract_tl 
%   {-8.0162$abc_+1.234$-23.9235} \l_extract_seq
% \seq_show:N \l_extract_seq
% >  {-8.0162}
% >  {1.234}
% >  {-23.9235}.

% ---> apply \exp_not:n to clist item
% \edef\TTT{\exp_not:n {\textsf{aaa}}, bbb, ccc,}
% \edef\TTT{\textsf{aaa}, bbb, ccc,}
% \edef\TTT{\zcmd_clist_patch:n {, aaa, $\alpha$, }}
% \edef\TTT{\__zcmd_make_robust:n {\textsf{1}, aaa, $\alpha$}}
% \show\TTT


% --> empty token test
% \ztex_index_token_if_eq:ennTF {}{1}{+}
%   {XXX}{YYY}


% ----> \zclist_item: bug 
% \zclist_item:nn {, xx, y, \textsf {ww}}{1} % ---> ERROR
% \zclist_item:nn {xx, y, \textsf {ww}}{1} % ---> ERROR
% \clist_item:en {\zcmd_clist_patch:n {xx, y, \textsf {ww}}}
%   { 1 } % ---> ERROR
% \zcmd_clist_patch:n {xx, y, \textsf{ww}} % ----> ERROR
% \clist_item:nn {xx, y, \textsf{ww}}{-1} % ----> works
% \zcmd_clist_patch:n {xx, y, \textsf{w}} % ----> works
% \zcmd_clist_patch:n {xx, y, \textsf{ww}} % ----> BUG: ! Missing number, treated as zero
% \zcmd_clist_tail:n {1, 2, 33} % ---> works
% \zcmd_clist_tail:n {1, 2, \textsf{a}} % ---> ERROR: ! Missing } inserted.
%%%%% BUG lies in '\tl_reverse:n': '\textbf{ab}' --> '{ba}\textbf'
% \zcmd_clist_tail:n {1, 2, \textsf{a}} % --> works
% \zcmd_clist_tail:n {1, 2, \textsf{aa}} 
% \zcmd_clist_tail:n {1, 2, \textsf{aa}, } 
% \zcmd_clist_tail:n {1, 2, \textsf{aa}, x} 

% \[
%   \dv{, xx, y, \textsf{ww}}[zz, \mathbf{g}]
% \]
% \zclist_item:nn {, xx, y, \textsf{ww}}{1}   % ----> ERROR:! Missing number, treated as zero
% \zcmd_clist_patch:n {, xx, y, \textsf{ww}}  % ----> ERROR:! Missing number, treated as zero
% \zcmd_clist_tail:n {, xx, y, \textsf{ww}}   % ----> works
% \zcmd_clist_head:n {, xx, y, \textsf{ww}}   % ----> works
% \zcmd_clist_head:n {f, xx, y, \textsf{ww}}     % ----> works
% \zclist_item:nn {, xx, y, \textsf{ww}}{1}
% \edef\TTT{\zcmd_clist_patch:n {f,xx, y, \textsf{ww},}}
% \show\TTT
\zaliasOff
% \clist_item:nn {xx, y, \textsf{ww}}{1}
\ExplSyntaxOff
\end{document}




% TODO: map multiple tokens in '\<>_map_function:nnN'
% \tl_map_function:nN {abcd}
%   \my_function:nn {arg} % ---> fails 
% But the following works:
% \cs_set:Npn \my_product:nn #1#2
%   { $#1 \times #2 = \int_eval:n { #1 * #2 }$, \quad }
% \int_step_tokens:nnnn { 1 } { 1 } { 4 } 
%   { \my_product:nn { 2 } }
% output:2 × 1 = 2, 2 × 2 = 4, 2 × 3 = 6, 2 × 4 = 8




\InputIfFileExists{zlatex-cfg.tex}{}{}
\documentclass[
  % layout=margin
]{../code/ztex}
\parindent0pt
\usepackage[T1]{fontenc}
\ExplSyntaxOn\makeatletter
\cs_set_eq:NN \tlEQNnTF \tl_if_eq:NnTF
\makeatother\ExplSyntaxOff

\begin{document}
\znewcmd\cmda
  {
    argA = Hello world,
    argB:str = Hello world,
    argC:int = 100,
    argD:clist = {item-1, item-2, item-3},
    arg-4:[fp] = {0.1, 0.22, 0.333},
  }{
    \argA = \expandafter\uppercase\expandafter{\argB} ??\par
    \tlEQNnTF \argB {Hello world}{Equal}{Not Equal};\par
    Integer = \intuse\argC;\par 
    Clist use = \clistuse\argD{-1};\par
    Token list = \cmdvar{arg-4}{3}
  }
\cmda{argC = 500}
\end{document}







% ==> update \znewcmd interface: '\znewcmd[arg-spec]{code}'
\InputIfFileExists{zlatex-cfg.tex}{}{}
\documentclass[
  % layout=margin
]{../code/ztex}
\parindent0pt
\usepackage[T1]{fontenc}
\ExplSyntaxOn\makeatletter
\cs_set_eq:NN \tlEQNnTF \tl_if_eq:NnTF

\def\ZCMDNEW#1
  {
    parent=#1;
    \peek_meaning:NTF {[}
      {\fullcmdTMP}
      {\shortcmd}
  }
\def\shortcmd#1{\#1=#1}
\def\fullcmd#1#2{(\#1=#1), (\#2=#2)}
\def\fullcmdTMP[#1]#2{\fullcmd {#1}{#2}}
\makeatother\ExplSyntaxOff



\begin{document}
\section{DEBUG}

\ZCMDNEW{short}{ARG}\par
\ZCMDNEW{long}[ARG-1]{ARG-2}



\end{document}











% Python-like command setup syntax
% DEF:
% \ztexdef\<cmd>{<arg-1>:str=<val-1>, <arg-2>:int=<val-2>}
%   {
%      \<arg-1> % --> similar to #1
%      \<arg-2> % --> similar to #2
%   }
% USE:
% \<cmd>{<arg-1>=hello world, <arg-2>=100}
\InputIfFileExists{zlatex-cfg.tex}{}{}
\documentclass[
  % layout=margin
]{../code/ztex}
\parindent0pt
\usepackage[T1]{fontenc}
\ExplSyntaxOn\makeatletter
\cs_set_eq:NN \tlEQNnTF \tl_if_eq:NnTF
\makeatother\ExplSyntaxOff


\begin{document}
\section{DEBUG section}
\znewcmd\CMDA{argA=argA-val, argB:str=argB-val, argC}
  { 
    \tlEQNnTF \argA {argA-val}{argA~EQUALS}{argA~not~EQUALS}\par
    \argA{} = \argB = \argC\par
  }
% \ExplSyntaxOn
% \ztex_cmd_create:nnn {CMDA}
%   {argA=argA-val, argB=argB-val}
%   {\argA{} = \argB}
% \keys_show:nn {ztex/cmd/CMDA}{argA}
% \keys_show:nn {ztex/cmd/CMDA}{argB}
% \keys_show:nn {ztex/cmd/CMDA}{argC}
% \ExplSyntaxOff
% \show\CMDA
\CMDA{argB=argB-val-new} % --> works
% \CMDA{argB=argB-val-new, argC} % --> works
% \CMDA{argB=argB-val-new, argC=ARG-C} % --> works
\dotfill\par


% TODO:
% 1. argX:[int]} --> use '\argX[1]' to get the first item, '\argX' will get the whole items
% 2. argX:clist  --> use '\argX[1]' to get the first item
\znewcmd\CMDB{
  argD:str=argD-val, 
  argE:tl=argE-val, 
  argF:fp=3.1415926,
  argG:int = 100,
  argH:dim = 12pt+1em,
  argI:clist = {AA, BB, CC},
  argA = "Group variable range Test"
}{
  \tlEQNnTF \argD {argD-val}{argD~EQUALS}{argD~not~EQUALS}\par
  \tlEQNnTF \argE {argE-val}{argE~EQUALS}{argE~not~EQUALS}\par
  \fpuse\argF=\fpuse{\argF}\par
  \string\argG=\intuse\argG\par
  \string\argH=\dimuse\argH\par
  \string\argI=\clistuse\argI{2}\par
  argument of \string\CMDA{} TEST:
  \string\argA=\argA\par
}
% \show\CMDB
\CMDB{argF=6.2830178, argG=200,}
% \CMDB{argF=aabb} % --> raise error
\dotfill\par

% \znewcmd\CMDA{}{}
\begingroup
\zsetcmd\CMDA{arg-1=aaa}{CODE=\cmdvar{arg-1}\par}
INNER: \CMDA{};
\endgroup
OUTER: \CMDA{}




\section{L3 TEST}
% ==> list syntax
\znewcmd\CMDD{argA:[int]={1, 2, 3, 4}, argB:[str]}
  { 
    CODE 1=(\argA{1}), (\argA{4})\par
    CODE 2=(\argB{1}), (\argB{-1})
  }
\CMDD{argA={5.55, 6, 7, 8}, argB={AAA, BBB, CCC}}



%%%%%%%     debug section   %%%%%%
\dotfill\par
\ExplSyntaxOn
% ==> expandable token if in check
% REF: https://tex.stackexchange.com/a/690186/294585
% \long\def\isinlist#1#2{
%   \immediateassignment\long\def\isinlistA##1#2##2\end/
%   {\if\relax\detokenize{##2}\relax \expandafter\unless\fi}
%   \expandafter\isinlistA#1\endlistsep#2\end/
% }
% \ztex_token_if_in:nNTF {abxc}x{IN}{NOT~IN}
% \tl_set:Ne \l_tmpa_tl {\ztex_tl_if_in:nnTF {abcdef}{abc}{IN}{NOT~IN}}
% \tl_show:N \l_tmpa_tl
% \tl_set:Ne \l_tmpb_tl {\ztex_colon_if_in:nTF {abxx==cd}{IN}{NOT~IN}}
% \tl_show:N \l_tmpb_tl


% ==> keys group limit
\def\XXX{
  \group_begin:
  \keys_define:nn {XXX}
    {
      xxx .tl_set:c  = {xxx},
      xxx .initial:n = {XXX}
    }
  \keys_set:nn {XXX}{xxx=xxx-new}
  \tl_new:N \xxxtl
  \tl_set:Nn \xxxtl {xxx-tl}
  \xxx
  \group_end:
}
\def\YYY{
  \group_begin:
  \keys_define:nn {YYY}
    {
      yyy .tl_set:c  = {yyy},
      yyy .initial:n = {YYY}
    }
  \keys_set:nn {YYY}{yyy=yyy-new}
  |\yyy|=|\xxx|
  \group_end:
}
(\XXX)|\xxxtl|(\YYY)

% ==> expandable of \ztexdef
\znewcmd\CMDC{arg-1=aaa}{CODE=\cmdvar{arg-1}}
\tl_set:Ne \l_tmpa_tl {\CMDC{}}\par
% \tl_show:N \l_tmpa_tl
% NOTE: all of the commands defined by `\ztexdef' is robust.



% <list> syntax for \znewcmd
\dotfill\par
% <type> = int for example:
\keys_define:nn {ZZZ}
  {
    % zzz .clist_set:c = {__zzz_clist},
    zzz .code:n      = {
      \cs_set:Npn \zzz ##1
        {
          \clist_item:en
            {
              \clist_map_function:nN {#1}
              % \clist_map_function:eN {\l_keys_value_tl}
                \__zcmd_list_arg_str:n 
            }{##1}
        }
    }
  }
\keys_set:nn {ZZZ}{zzz={zzz-1, zzz-2, zzz-3, zzz-4, zzz-5}}
LIST~ITEM~TEST:(\zzz{1}), (\zzz{1}), (\zzz{3}), (\zzz{5})
% \edef\TTT{\zzz{1}}
% \show\TTT % zzz-1 ---> OK!

% head and tail check in ztex
% \ztex_head_tail_if_eq:nnnTF {[abc]}{[}{]}{EQUAL}{NOT~EQUAL}
% \tl_if_eq:NNTF \tl_item:nn {xyz}z{EQUAL}{NOT~EQUAL}


% \edef\TTT{\__ztex_tl_if_eq:eeTF {x}{\tl_item:nn {xyz}{2}}{EQUAL}{NOT~EQUAL}}
% \edef\TTT{\ztex_head_tail_if_eq:nnnTF {xyz}{c}{z}{EQUAL}{NOT~EQUAL}} % works
% \show\TTT % NOT EQUAL. --> works

\edef\TTT{\ztex_token_strip_right:n {XabcZ}}
% \show\TTT % works
\ExplSyntaxOff
\end{document}







% ==> QED symbol position: --> BUG
\InputIfFileExists{zlatex-cfg.tex}{}{}
\documentclass[
  % layout=margin
]{../code/ztex}
\zthmtitleformat*[proof]{\textsc{\color{blue}\zthmname}:\;}
\usepackage{lipsum}
\usepackage[hshift=0mm,vshift=0mm]{fgruler}
% \usepackage{amsthm}
\ExplSyntaxOn\makeatletter
% \tex_lastxpos:D
% \tex_lastypos:D
% NOTE: you can use '\if@eqsw'(which lies in 'amsmath') to check if current equation env will be numbered !!!
\newcommand{\zqedhere}
  {
    \tex_savepos:D
    \property_record:nn {zqed@pos}{xpos, ypos}
    \tex_savepos:D
  }
\ztex_msg_set:nn {zthm@qed@pos}
  {
    Compile~twice~to~ensure~the~QED~symbol~is~positioned~correctly
  }
\newcommand{\zshowqed}
  {
    \__ztex_get_qed_coor:
    % \dim_show:N \l__ztex_qed_x_dim
    \dim_compare:nNnTF {\l__ztex_qed_y_dim} = {0pt}
      {
        \ztex_msg_warn:n {zthm@qed@pos}
      }{
        \zpagemask[
          anchor=br, 
          % position={(\dim_eval:n {\l__ztex_qed_x_dim+3em}, \l__ztex_qed_y_dim)}
          position={(\dim_eval:n {\Gm@lmargin+\textwidth}, \l__ztex_qed_y_dim)}
        ]{
          \qedsymbol
        }
      }
  }
\dim_new:N \l__ztex_qed_x_dim
\dim_new:N \l__ztex_qed_y_dim
\cs_set:Nn \__ztex_get_qed_coor:
  {
    \dim_set:Nn \l__ztex_qed_x_dim 
      {
        \fp_eval:n {
          \property_ref:nn {zqed@pos}{xpos}
            /65536
        } pt
      }
    \dim_set:Nn \l__ztex_qed_y_dim 
      {
        \fp_eval:n {
          \property_ref:nn {zqed@pos}{ypos}
            /65536
        } pt
      }
  }
\cs_set:Nn \__ztex_clear_qed_coor: 
  {
    \dim_set:Nn \l__ztex_qed_x_dim {0pt}
    \dim_set:Nn \l__ztex_qed_y_dim {0pt}
  }
\makeatother\ExplSyntaxOff



\begin{document}
\lipsum[1] 

\fpeval{\textwidth/28}

\begin{proof}
\lipsum[2][1-5]
\begin{align}
  1 + 1 = 3\\
  \alpha + \beta = \gamma\zqedhere
\end{align}
\lipsum[2][6-7]
\end{proof}
\zshowqed\par
\dotfill\vskip3em

\begin{proof}
\lipsum[2][1-5]
\begin{align}
  1 + 1 = 3\\
  \alpha + \beta = \gamma\zqedhere\notag
\end{align}
\end{proof}
\zshowqed\par % the post QED position will be overide by the following QED position
\dotfill\vskip3em


\begin{proof}
\lipsum[2][1-5]
\begin{align}
  1 + 1 = 3\\
  \alpha + \beta = \gamma
\end{align}
\end{proof}
\end{document}



% ==> ztex 'fixdif' implementation
\InputIfFileExists{zlatex-cfg.tex}{}{}
\documentclass[
  % layout=margin
]{../code/ztex}
\usepackage{amsmath}
\usepackage{ascii}
\ztexloadlib{alias}
\ExplSyntaxOn\makeatletter
% \char_set_mathcode:nn {"2F}{"413D}
% \DeclareRobustCommand\dd{\mathinner{\mathrm{d}}\fd@mu@p}
% \def\fd@mu@p{\mathchoice{\mskip-\thinmuskip}{\mskip-\thinmuskip}{}{}{}}
\def\aaa{AAA}
\def\aaabbb{AAABBB}
\cs_set_eq:Nc \xxx{aaa\cs_to_str:N\bbb}
% \show\xxx
\makeatother\ExplSyntaxOff


\begin{document}
Hello, \S ~~ \FF ~~\xxx
\def\grad{New Grad}
test existing math command: $\grad, \div$;


\zaliasOn[XXX]
Hello world: $\div f = \dd y/\dd x$;

Hello world: $a^{f(x)\dd x}$;


\begin{align}
  \int \FF{o(x)}\cdot a^{h(x)\dd x}\cdot\XXXhom(\S{F}(x))\XXXdiv g(x)\dd x\\
  \dd y/\dd x
\end{align}
\zaliasOff


test existing math command: $\grad, \div$


\begin{zalias}[ZZZ]
\begin{align}
  \int \FF{o(x)}\cdot a^{h(x)\dd x} f(x)\ZZZdiv g(x)\dd x\\
  \dd y/\dd x
\end{align}


\[ \alt, \im \]
\zaliasopset{alt=ALT, im=IM} % only preamble, just for test
\[ \alt, \im \]
\end{zalias}

\ExplSyntaxOn
\seq_show:N \g__ztex_mathalias_protected_seq
% >  {\S }
% >  {\FF }
% >  {\div }
% >  {\grad }
% >  {\ker }
% >  {\hom }.
\ExplSyntaxOff
\end{document}





% ==> ztex partial toc test
% \InputIfFileExists{zlatex-cfg.tex}{}{}
% \documentclass[
%   % class=l3doc,
%   class=book,
% ]{../code/ztex}
\documentclass{book}


\begin{document}
% \ztexptoc[1]
% \startcontents[sections]
% \printcontents[sections]{}{1}{}

% \dotfill\par
%%% book
\chapter{A-1}
% \ztexptoc[2]

\section{A-1-1}
\subsection{A-1-1-1}
\subsection{A-1-1-2}
\section{A-1-2}

\chapter{A-2}
\section{A-2-1}
\section{A-2-2}
\chapter{A-3}


%%% article
% \section{A-1}
% \subsection{A-1-1}
% \subsection{A-1-2}
% \subsubsection{A-1-2-1}
% \subsubsection{A-1-2-2}
% \subsection{A-1-3}

% \section{A-2}
% \section{A-3}
\end{document}




% ==> add dashed line feature
\InputIfFileExists{zlatex-cfg.tex}{}{}
\documentclass[
  % layout=margin
]{../code/ztex}
\usepackage{lipsum}
\setlength{\fboxsep}{0pt}


\begin{document}
\setlength{\unitlength}{2mm}
\begin{picture}(27,16)(-1,-1)
\path(-1,-1)(-1,15)(26,15)(26,-1)(-1,-1)
% \linethickness{2pt}
\drawline[-10](0,0)(10,7)(5,11)
(0,7)(10,0)(10,7)(0,7)(0,0)(10,0)
% \linethickness{3pt} \roundcap \beveljoin
% \polyline(15.1,0.1)(25,7)(20,11)
% (15,7)(25,0)(25,7)(15,7)(15,0)(25,0)
\end{picture}
\end{document}




\InputIfFileExists{zlatex-cfg.tex}{}{}
\documentclass[
  % layout=margin
]{../code/ztex}
\usepackage{lipsum}
\setlength{\fboxsep}{0pt}



\begin{document}
\section{Internal commands}
\vskip1cm
\def\poly#1#2{\put(0, 0){\polygon #1#2}}
\makeatletter\ExplSyntaxOn
\begin{zpic}[unit=2em]
  % \poly {}{(1, 2)(2, 3)(4, 5)(4, 4)}
  % \poly {*}{(1, 2)(2, 3)(4, 5)(4, 4)}
  % \put (0, 0) {\polygon (0, 0)(1, 1)}
  % \put (2, 0) {\__@@_pic_polygon:nn {}{(0, 0)(1, 1)}} % works
  % \put (2, 0) {\__@@_pic_polyline:n {(0, 0)(1, 1)}}   % works
  % \put (2, 0) {\__@@_pic_polyvector:n {(0, 0)(1, 1)}}   % works
  % \zdraw[shift={2, 0}] (0, 0)(1, 1); % works
  % \zdraw[shift={2, 0}, vector, cycle] (0, 0)(1, 1)(2, 1); % works
  % \zdraw[shift={4, 0}, vector, fill] (0, 0)(1, 1)(2, 1); % works
  % \zdraw[shift={6, 0}, vector, cycle, fill] (0, 0)(1, 1)(2, 1); % works
  % \zdraw[shift={4, 0}, cycle, fill] (0, 0)(1, 1)(2, 1); % works
  % \zdraw[fill, cycle] (0, 0)(1, 0)(1, 1)(0, 1);
\end{zpic}
\ExplSyntaxOff\makeatother


\section{zpic env}
\lipsum[1]

XXX\rule{2cm}{2cm}%
\begin{zpic}[unit=2cm, xoffset=1]%[unit=.5cm, height=2, width=1, xoffset=1, yoffset=1]
  % 1. rectangle
  \zrectangle[fill=gray!20, arc=.1](0, 0)(2, 1)
  \zrectangle[draw=red, width=1pt](.5, .25)(1.5, .75)
  % 2. line / vecter
  \zline[width=3pt, draw=red](0, .5)(2, .5)
  \zvector[>=pst](0, 0)(1, 1)
  \zvector[draw=purple, width=2pt](1, 1)(2, 0)
  \put (0, 0){\vector(2, 1){2}}
  % 3. arc / circle
  \zarc[draw=blue, end=45](0, 0) % fill=<empty>
  \zarc[draw=blue, width=2pt, end=15, fill=, draw=red](0, 0)
  \zcircle[radius=.25, fill, draw=blue](1, .5)
  \zcircle[radius=.25, fill=orange, draw=none](1.5, 1)
  \zcircle[radius=.25, fill=red, draw=](2, .5)
  \put (2, 0){\circle{.5}}
  % 4. ovaL
  \put(1, 0){\oval(1, .5)[lb]}
\end{zpic}


\section{zdraw}
XXX{\color{teal}\rule{4em}{4em}}
\begin{zpic}[unit=2em]
  \zdraw (0, 0)(1, 0)(1, 1)(0, 1);
  \zdraw[] (0, 0)(1, 0)(1, 1)(0, 1);
  \zdraw[cycle, shift={2, 0}]  (0, 0)(1, 0)(1, 1)(0, 1);
  \zdraw[fill, shift={4, 0}]     (0, 0)(1, 0)(1, 1)(0, 1);
  \zdraw[draw=red, width=1pt, shift={6, 0}] (0, 0)(1, 0)(1, 1)(0, 1);
  \zdraw[vector, shift={8, 0}] (0, 0)(1, 0)(1, 1)(0, 1);
  \zdraw[vector, cycle, shift={10, 0}] (0, 0)    (1, 0)(1, 1)(0, 1);
  \zdraw[vector, fill, shift={12, 0}]    (0, 0)(1, 0)(1, 1)  (0, 1);
  \zdraw[vector, cycle, fill, shift={14, 0}] (0, 0)  (1, 0)(1, 1)(0, 1);
\end{zpic}


% \mbox{}\vskip6em
% XXX\rlap{\rule{1cm}{1cm}}\zrectangle[yoffset=1, xoffset=1, arc=.25]

% \mbox{}\vskip6em
% XXX\rlap{\zsquare[unit=.5cm, fill=black]}\zrectangle[unit=.5cm, yoffset=1, xoffset=1, arc=.125]

% \vskip1em
% XXX\zsquare[unit=1cm, fill=teal, width=.5, arc=.1]


% \section{vector}
% \setlength{\unitlength}{1cm}
% \dotfill\par
% \vskip4em
% XXX%
% \begin{picture}(0, 0)
%   \zvector(0, 0) (1, 1)
%   \zvector (1, 1) (2, 0)
% \end{picture}


% \def\P@radius{1}
% \def\P@width{10}
% \def\P@height{6}
% \begin{picture}(8,4)
%   \put(0,0){\vector(1,0){8}}  % x axis
%   \put(0,0){\vector(0,1){4}}  % y axis
%   \put(2,0){\line(0,1){3}}       % left side
%   \put(4,0){\line(0,1){3.5}}     % right side
%   \qbezier(2,3)(2.5,2.9)(3,3.25)
%     \qbezier(3,3.25)(3.5,3.6)(4,3.5)
%   \thicklines                 % below here, lines are twice as thick
%   \put(2,3){\line(4,1){2}}
%   \put(4.5,2.5){\framebox{Trapezoidal Rule}}
% \end{picture}
\end{document}







% ===> l3keys bug patch
\documentclass{article}

\begin{document}
\ExplSyntaxOn
\cs_set_protected:Npn \__keys_initialise:n #1
  {
    \exp_after:wN \__keys_find_key_module:wNN
      \l_keys_path_str \s__keys_stop
      \l_keys_key_tl \l_keys_key_str
    \tl_set_eq:NN \l_keys_key_tl \l_keys_key_str
    \tl_set:Nn \l_keys_value_tl {#1}
    \cs_if_exist:cTF { \c__keys_code_root_str \l_keys_path_str }
      {
        \str_clear:N \l__keys_inherit_str
        \__keys_execute:nn \l_keys_path_str {#1}
      }
      {
        \cs_if_exist:cT
          { \c__keys_inherit_root_str \__keys_parent:o \l_keys_path_str }
          { \__keys_execute_inherit: }
      }
  }

\keys_define:nn { a } {
  A .tl_set:N = \l__A_tl,
  A .code:n = {PARENT~`a'\par}
}
\keys_define:nn { aa } {
  AA .tl_set:N = \l__AA_tl,
  AA .code:n = {PARENT~`aa'\par}
}
\keys_define:nn { } {
  b .inherit:n = {a, aa},
  % b .inherit:n = a,  % ---> works
  % c .inherit:n = aa, % ---> works
  % b .inherit:n = aa, % ---> error
}
\keys_define:nn { b } {
  B .tl_set:N = \l__B_tl,
  B .default:n = {DEFAULT}, % --> works
  B .initial:n = {INITIAL}, % --> works
}
%% DOES NOT work EXAMPLE BEGIN
% \keys_define:nn { } {
%     b .inherit:n = aa,
% }
%% DOES NOT work EXAMPLE END
\keys_set:nn { b } { A = x, AA=y }
% \show \__keys_initialise:n
\ExplSyntaxOff

Hello
\end{document}





% ===> xetex primitive graphics
\documentclass{article}
\usepackage[scheme=plain]{ctex}
\makeatletter
\def\test{\XeTeXpicfile "logo.jpg"
  height \f@size pt}
\def\fSize{\f@size}
\makeatother
\setlength{\fboxsep}{0pt}

\begin{document}

\fbox{测}\fbox{试}:\raisebox{\dimexpr-\fontchardp\font`y}{\test}:
(depth: \hbox{\the\dimexpr\fontchardp\font`y})

\XeTeXuseglyphmetrics=0
\fbox{测}\fbox{试}:\raisebox{\XeTeXglyphbounds4\fontcharwd\font`你}{\fbox{\test}}:
\the\XeTeXglyphbounds4`节 (\fSize pt)\setbox0\hbox{测}\the\ht0 )
\end{document}




% affine transform for tikz
\InputIfFileExists{zlatex-cfg.tex}{}{}
\documentclass[
  layout=margin
]{../code/ztex}
\setlength{\fboxsep}{0pt}
\parindent0pt
\usepackage{lipsum}
\usepackage{graphicx}
\makeatletter
% \usepackage{newtxtext}


\begin{document}
\section{FIRST}
\lipsum[1][1-3]
\marginpar{
  \sffamily\small\lipsum[2][2-4]\vskip1em
  \ztoolboxaffine{\includegraphics[width=10em]{example-image-a}}{.5, 0, .25, .5}
}
\lipsum[1][1-3]


\lipsum[2]
\marginpar{
  \sffamily\small\lipsum[2][2-4]\vskip1em
  \centerline{
    \includegraphics[width=10em]{example-image-a}
  }
  \hb@xt@\marginparwidth{\hfill XXX \hfill}
  \label{fig:1}
}

\cref{fig:1}

\end{document}



% debug command: \ztoolboxaffine 
\InputIfFileExists{zlatex-cfg.tex}{}{}
\documentclass[
  % hyper, 
  % font={math=var-euler}
]{../code/ztex}
\setlength{\fboxsep}{0pt}

\begin{document}
% \fbox{XXX}\pdfsave\pdfsetmatrix{1 0 1 1}\rlap{XXX}\pdfrestore\par

% \fbox{XXX}\fbox{\ztoolboxaffine[debug]{XXX}{1, 0, 1, 1}}\par

% \ExplSyntaxOn\hbox_set:Nn \l_tmpa_box {XXX} \box_rotate:Nn \l_tmpa_box {45}\fbox{\box_use:N \l_tmpa_box}\ExplSyntaxOff


\fbox{XXX}\pdfsave\pdfsetmatrix{1 .5 .5 1}\rlap{\rule{1em}{1em}}\pdfrestore\par

XXX\ztoolboxaffine[]{\rule{1em}{1em}}{1, .5, .5, 1}XXX
\end{document}



% clean todo
\InputIfFileExists{zlatex-cfg.tex}{}{}
\documentclass[hyper, font={math=var-euler}]{../code/ztex}
\zthmstyle{obsidian}
\ztexloadlib{thm}
\zthmcnt{share}
\parindent0pt
% \zthmiconset{
%   theorem=$>$
% }
\makeatletter
\def\seeTargets{%
  \par\noindent\dotfill\\
  \string\@currentHref\ = \meaning\@currentHref\\
  \string\@currentHpage\ = \meaning\@currentHpage\\
  \string\Hy@currentbookmarklevel\ = \meaning\Hy@currentbookmarklevel\\
  \vskip3em
}
\makeatother

\begin{document}
\section{Test}
Hello world.
\zthmiconuse{theorem}
\zthmiconuse{lemma}
\seeTargets

\dotfill\par
\zthmtoc
\seeTargets

\vskip5em
\begin{remark}[Pythagoras]
  A simple theorem.
  \begin{align}
    \sum_{i=1}^{+\infty}{\int_{0}^{i}-\frac{1}{t}\mathrm{d}t} = \frac{\pi^2}{6}
  \end{align}
\end{remark}

\begin{lemma}[Pythagoras]
  A simple theorem.
\end{lemma}
\seeTargets

\newpage
\mbox{}\vskip10em
New added theorem
\zthmtocadd[subsection]{name=New:Added Thm ITEM}
\seeTargets

\newpage
XXX
\seeTargets
\end{document}




% ztex box shear transform
\documentclass{article}
\makeatletter\ExplSyntaxOn
% REF:
% 1. https://math.stackexchange.com/a/3521141/1235323
% 2. https://math.stackexchange.com/a/281087/1235323
\cs_new:Npn \__fp_to_rad:n #1
  { \fp_eval:n {#1/pi*180} }
\cs_new:Npn \__matrix_det:nnnn #1#2#3#4
  {
    \fp_eval:n { #1*#4 - #2*#3 }
  }
% (translation) + x-scale + y-scale + rotate
\coffin_new:N \l__affine_trans_coffin
\fp_new:N \g_affine_precision_fp
\fp_set:Nn \g_affine_precision_fp {0.0001}
\fp_new:N \l__affine_@@_a_fp
\fp_new:N \l__affine_@@_b_fp
\fp_new:N \l__affine_@@_c_fp
\fp_new:N \l__affine_@@_d_fp
\msg_set:nnn { module }{affine-det_zero}
  {
    current~determination~of~the~affine~transformation~
    matrix~equals~to~zero,~give~up~this~transformation
  }
\cs_new:Npn \__affine_transformation:nnnnn #1#2#3#4#5
  {% #1:box content; #2: a_11; #3: a_21; #4: a_12; #5: a_22
    \hcoffin_set:Nn \l__affine_trans_coffin {#1}
    %% check affine matrix determination 
    \fp_compare:nNnT 
      { abs(\__matrix_det:nnnn {#2}{#3}{#4}{#5}) } 
        < { \g_affine_precision_fp }
      { \prg_map_break:Nn \l__affine_matrix_det_zero 
        { \msg_warning:nn { module }{affine-det_zero} }}
    %% set matrix elements
    \fp_set:Nn \l__affine_@@_a_fp {#2}
    \fp_set:Nn \l__affine_@@_b_fp {#3}
    \fp_set:Nn \l__affine_@@_c_fp {#4}
    \fp_set:Nn \l__affine_@@_d_fp {#5}
    %% get factors
    \__box_affine_get_sx:     
    \__box_affine_get_theta:  
    \__box_affine_get_msy:    
    \__box_affine_get_sy:     
    \__box_affine_get_m:      
    \__box_affine_get_Sx:     
    \__box_affine_get_Sy:     
    \__box_affine_get_phi:    
    \__box_affine_get_omega:  
    %% box transformation
    \coffin_scale:Nnn \l__affine_trans_coffin 
      { \l__box_affine_sx_fp }
      { \l__box_affine_sy_fp }
    \coffin_rotate:Nn \l__affine_trans_coffin 
      { \__fp_to_rad:n {\l__box_affine_omega_fp} }
    \coffin_scale:Nnn \l__affine_trans_coffin
      { \l__box_affine_Sx_fp }
      { \l__box_affine_Sy_fp }
    \coffin_rotate:Nn \l__affine_trans_coffin
      { \__fp_to_rad:n {\l__box_affine_phi_fp} }
    \coffin_rotate:Nn \l__affine_trans_coffin 
      { \__fp_to_rad:n {\l__box_affine_theta_fp} }
    \prg_break_point:Nn \l__affine_matrix_det_zero { }
    \coffin_typeset:Nnnnn \l__affine_trans_coffin
      {l}{b}{0pt}{0pt}
  }
\NewDocumentCommand{\boxaffine}{m>{\SplitList{,}}m}
  {% #1:content; #2:matrix
    \__affine_transformation:nnnnn {#1}#2
  }
% internal calculating functions
\fp_new:N \l__box_affine_sx_fp
\cs_new:Nn \__box_affine_get_sx: 
  {
    \fp_set:Nn \l__box_affine_sx_fp 
      { \fp_eval:n {sqrt(\l__affine_@@_a_fp^2 + \l__affine_@@_b_fp^2)} }
  }
\fp_new:N \l__box_affine_theta_fp 
\cs_new:Nn \__box_affine_get_theta: 
  {
    \fp_set:Nn \l__box_affine_theta_fp 
      { \fp_eval:n {atan(\l__affine_@@_b_fp/\l__affine_@@_a_fp)} }
  }
\fp_new:N \l__box_affine_msy_fp 
\cs_new:Nn \__box_affine_get_msy: 
  {
    \fp_set:Nn \l__box_affine_msy_fp 
      { \fp_eval:n {
        \l__affine_@@_c_fp*cos(\l__box_affine_theta_fp) 
        + 
        \l__affine_@@_d_fp*sin(\l__box_affine_theta_fp)
      } }
  }
\fp_new:N \l__box_affine_sy_fp
\cs_new:Nn \__box_affine_get_sy: 
  {
    \bool_if:nTF
      {
        \fp_compare_p:nNn { abs(sin(\l__box_affine_theta_fp)) } 
          < {\c_zero_fp + \g_affine_precision_fp}
      }{
        \fp_set:Nn \l__box_affine_sy_fp 
          {
            ( \l__affine_@@_d_fp - \l__box_affine_msy_fp*sin(\l__box_affine_theta_fp) )
            / cos(\l__box_affine_theta_fp)
          }
      }{
        \fp_set:Nn \l__box_affine_sy_fp 
          {
            ( \l__box_affine_msy_fp*cos(\l__box_affine_theta_fp) - \l__affine_@@_c_fp )
            / sin(\l__box_affine_theta_fp)
          }
      }
  }
\fp_new:N \l__box_affine_m_fp
\cs_new:Nn \__box_affine_get_m: 
  {
    \fp_set:Nn \l__box_affine_m_fp 
      { \l__box_affine_msy_fp / \l__box_affine_sy_fp }
  }
\fp_new:N \l__box_affine_Sx_fp
\fp_new:N \l__box_affine_Sy_fp
\cs_new:Nn \__box_affine_get_Sx: 
  {
    \fp_set:Nn \l__box_affine_Sx_fp 
      { sqrt(\l__box_affine_m_fp^2/4 + 1) - \l__box_affine_m_fp/2 }
  }
\cs_new:Nn \__box_affine_get_Sy: 
  {
    \fp_set:Nn \l__box_affine_Sy_fp 
      { sqrt(\l__box_affine_m_fp^2/4 + 1) + \l__box_affine_m_fp/2 }
  }
\fp_new:N \l__box_affine_phi_fp
\fp_new:N \l__box_affine_omega_fp
\cs_new:Nn \__box_affine_get_phi: 
  {
    \fp_set:Nn \l__box_affine_phi_fp 
      { -pi/4 - 1/2*atan(\l__box_affine_m_fp/2) }
  }
\cs_new:Nn \__box_affine_get_omega: 
  {
    \fp_set:Nn \l__box_affine_omega_fp 
      { pi/4 - 1/2*atan(\l__box_affine_m_fp/2) }
  }
\ExplSyntaxOff\makeatother


\begin{document}
Original Text: XXX\par
$\det(A) = 0$: \boxaffine{XXX}{0, 0, 0, 2}\par  % det(A) = 0
Unit Matrix: \boxaffine{XXX}{1, 0, 0, 1}\par    % unit matrix
Scale Matrix: \boxaffine{XXX}{2, 0, 0, 2}\par   % scale
$x$-scale Matrix: \boxaffine{XXX}{2, 0, 0, 1}\par % x-scale
$y$-scale Matrix: \boxaffine{XXX}{1, 0, 0, 2}\par % y-scale
$x$-shear Matrix: \boxaffine{XXX}{1, 0, 1, 1}\par % x-sheae
$y$-shear Matrix: \boxaffine{XXX}{1, 1, 0, 1}\par % y-sheae
Graphics: \rule{2em}{2em}~\boxaffine{\rule{2em}{2em}}{1, 0, .5, 1}\par
pdfsetmatrix: \rule{2em}{2em}~\pdfsave\pdfsetmatrix{1 0 .5 1}\rlap{\rule{2em}{2em}}\pdfrestore 

% det(A) = 0 --> warning:
% Package module Warning: current determination of the affine transformation
% (module)                matrix equals to zero, give up this transformation
\end{document}




\InputIfFileExists{zlatex-cfg.tex}{}{}
\documentclass[]{../code/ztex}
\usepackage{graphicx}
\def\image{\includegraphics[width=10em]{example-image-a}}
\ExplSyntaxOn\makeatletter
% % \hshearbox{vertical_prescale_times_shearfactor}{one_divide_by_shearfactor}{content}
% % an initial vertical downscale is often necessary for a 3d projection
% \newcommand{\hshearbox}[3]{\scalebox{0.866025}[#2]{\rotatebox{210}%
%   {\scalebox{1.73205}[-0.57735]{\rotatebox{60}{\scalebox{-1.1547}[#1]{#3}}}}}}
% % \vshearbox{horizontal_prescale_times_shearfactor}{one_divide_by_shearfactor}{content}
% % an initial horizontal downscale is often necessary for a 3d projection
% \newcommand{\vshearbox}[3]{\scalebox{#2}[0.866025]{\rotatebox{210}%
%   {\scalebox{-0.57735}[1.73205]{\rotatebox{60}{\scalebox{#1}[-1.1547]{#3}}}}}}
\makeatother\ExplSyntaxOff
\parindent0pt


\begin{document}
Original Text: XXX
% \hshearbox{1}{.5}{HELLO}
% \ExplSyntaxOn
% \fp_eval:n { atan(1) }:% output radian 
% \fp_eval:n { sin(pi/2) }% input radian

% \fp_set:Nn \l_tmpa_fp { pi/4 }
% \fp_use:N \l_tmpa_fp
% \ExplSyntaxOff


%%%%%%%     BEGIN DEBUG    %%%%%%%
% \ztoolboxaffine{XXX}{0, 0, 0, 2}\par % det=0, works
% \ztoolboxaffine{XXX}{1, 0, 0, 1}\par % unit matrix --> works
% \ztoolboxaffine{XXX}{2, 0, 0, 2}\par % scale --> works
% \ztoolboxaffine{XXX}{2, 0, 0, 1}\par % x-scale --> works
% \ztoolboxaffine{XXX}{1, 0, 0, 2}\par % y-scale --> works
% \ztoolboxaffine{XXX}{1, 0, 1, 1}\par % x-shear --> works
% \ztoolboxaffine{XXX}{1, 1, 0, 1}\par % y-shear --> works
\image~\ztoolboxaffine{\image}{1, 0, .5, 1}
%%%%%%%     END DEBUG      %%%%%%%
\end{document}




% ztex box content align test
\documentclass{article}
\usepackage[showframe, textwidth=18cm]{geometry}
\usepackage{graphicx}
\usepackage{xcolor}
\usepackage{lipsum}
\ExplSyntaxOn\makeatletter
\cs_set_protected:Npn \__item_align:nnn #1#2#3
  {% #1:cmd, #2:width, #3:object
    \hb@xt@#2{
      \tl_map_inline:nn {#3} 
        {
          \seq_put_right:No \l_tmpa_seq {\exp_not:N #1{##1}}
        } 
      \edef\seq@count{\seq_count:N \l_tmpa_seq}
      \seq_map_inline:Nn \l_tmpa_seq
        {
          \edef\item@width{\dim_eval:n {#2/(\seq@count+1)}}
          \hskip\item@width\clap{##1}
        }\hskip\item@width\hss
    }
  }
\NewDocumentCommand\itemAlign{O{}mm}
	{
		\__item_align:nnn {#1}{#2}{#3}
	}
\makeatother\ExplSyntaxOff
\parindent0pt

\begin{document}
\itemAlign{\linewidth}{A}\par
\itemAlign{\linewidth}{AA}\par
\itemAlign{\linewidth}{{(A)}{(A)}{(A)}}\par

\vskip6em
\dotfill\par
\def\imageA{\includegraphics[width=5em]{example-image-a}}
\def\imageB{\includegraphics[width=5em]{example-image-b}}
\itemAlign{\linewidth}{\imageA}\par
\itemAlign{\linewidth}{\imageA\imageA}\par
\itemAlign{\linewidth}{\imageB\imageB\imageB}\par
\itemAlign{\linewidth}{\imageB\imageB\imageB\imageB}

\vskip6em
\dotfill\par
\def\parA{\parbox[c]{12em}{\lipsum[1][1-2]}}
\def\parB{\parbox[c]{12em}{\lipsum[1][3-5]}}
\itemAlign[\sffamily]{\linewidth}{\parA}\par
\itemAlign[\color{red}]{\linewidth}{\parA\parA}\par
\itemAlign[\fbox]{\linewidth}{\parB\parB\parB}
\end{document}




\InputIfFileExists{zlatex-cfg.tex}{}{}
\documentclass[
]{../code/ztex}
\parindent0pt
\makeatletter


\begin{document}
Hello world.\rlap{A}B


\zboxitemalign[type=tower]{\linewidth}{A}\par
\zboxitemalign[type=tower]{\linewidth}{AA}\par
\zboxitemalign[type=tower]{\linewidth}{AAA}\par
\zboxitemalign[type=tower]{\linewidth}{AAAA}\par
\zboxitemalign[type=tower]{\linewidth}{AAAAA}\par

\dotfill\par
\ExplSyntaxOn
\def\custom@type{
  \edef\seq@count{\seq_count:N \l__ztool_boxitem_seq}
  \seq_map_inline:Nn \l__ztool_boxitem_seq
    {
      \edef\item@width{\dim_eval:n {\total@width/(\seq@count+1)}}
      \hskip\item@width\clap{##1}
    }\hskip\item@width\hss
}
\ExplSyntaxOff
\def\Blue#1{\textcolor{blue}{\sffamily #1}}
\zboxitemalign[type=custom, cmd=\Blue, custom=\custom@type]{\linewidth}{AAAAAA}\par
\end{document}




% ztex check box and \item color in manual
\documentclass{article} 
\usepackage{xcolor}
\usepackage{ctex}


% REF: https://tex.stackexchange.com/q/247681/294585
% *: wont fix
% [<arg>]: done
% [<blank>]: undone
\usepackage{enumitem}
\newlist{todolist}{itemize}{2}
\setlist[todolist]{label=\checkmark}
\usepackage{pifont, amssymb}
\newcommand{\done}{\rlap{\raisebox{0.3ex}{\hspace{0.4ex}\tiny \ding{52}}}$\square$}
\newcommand{\undone}{$\square$}
\newcommand{\wontfix}{\rlap{\raisebox{0.3ex}{\hspace{0.4ex}\scriptsize \ding{56}}}$\square$}


\let\olditem\item
\RenewDocumentCommand{\item}{so}
  {
    \IfValueTF{#2}
      {\color{black}\def\checkmark{\IfBooleanTF{#1}{\wontfix}{\done}}% 
        \olditem\IfBooleanF{#1}{\IfValueT{#2}{#2-}已完成:}\color{gray}}
      {\color{black}\def\checkmark{\IfBooleanTF{#1}{\wontfix}{\undone}}%
        \olditem}
  }


\begin{document}        
\begin{todolist}
  \item undone text
  \item*[2025-05-10] wont fixed + done: I want this to be blue
  \item[2025-05-10] done: another text
  \item* done: one more
\end{todolist}

HELLO:  \undone{} -- 未完成; \done{} -- 已完成; \wontfix{} -- 不会完成
\end{document}




% ztex slide nav bug
\InputIfFileExists{zlatex-cfg.tex}{}{}
\documentclass[
  hyper,
  layout={slide, aspect=12|9}, 
]{../code/ztex}
\zslidethemeuse[
  UL = {text=},
  UR={text=\zslidenavsym}
]{AnnArborDefault}
\makeatletter
\def\seeTargets{%
  \par\noindent\dotfill\\
  \string\@currentHref\ = \meaning\@currentHref\\
  \string\@currentHpage\ = \meaning\@currentHpage\\
  \string\Hy@currentbookmarklevel\ = \meaning\Hy@currentbookmarklevel\\
  \vskip3em
}
% \makeatother

\title{Test hyper symbol}
\author{Eureka}
\date{\today}
\begin{document}
\maketitle
\section{FIRST}
% AAAA \zslidenavsym{}\par 
\dotfill\par 
% \hyper@link{link}{zslide@FIRST.1}{TEST HYPER-1}\par
% \hyper@link{link}{zslide@FIRST.2}{TEST HYPER-2}\par
% \hyper@link{link}{zslide@FIRST.3}{TEST HYPER-3}\par

\dotfill\par
% \seeTargets

\hyper@link{link}{zslide@FIRST.3}{TEST HYPER}

\newpage
BBBB 
\newpage
CCCC

% \hyper@anchor{zslide@\FirstMark{zslide-left}.3}

\section{SECOND}
AAAA-2
\newpage
BBBB-2
\newpage
CCCC-2

\end{document}



% ztex slide test
\InputIfFileExists{zlatex-cfg.tex}{}{}
\documentclass[
  % lang=cn,
  layout={slide, aspect=12|9, theme=AnnArborSpruce}, 
  classOption={12pt}
]{../code/ztex}
\usepackage{xcoffins}
\usepackage{lipsum}
% \usepackage{zhlipsum}
% \usepackage[hshift=0mm,vshift=0mm]{fgruler}
\def\isCJK#1{\ifnum`#1>19968 is CJK\else not CJK\fi}

\title{Test \ztex{} AOh}
\author{Eureka}
\date{\today}
\begin{document}
\maketitle

\section{FIRST}
If $Y$ is a Banach space, an equivalent definition is that the embedding operator
(the identity) $i:X\to Y$ is a compact operator.When applied to functional analysis, this
version of compact embedding is usually used with Banach spaces of functions. Several of
the Sobolev embedding theorems are compact embedding theorems. When an embedding is not 
compact, it may possess a related, but weaker, property of cocompactness.

\lipsum[1]


\section{International}%{国际形势}
The following term ``Compact embedding'' can be defined in Topology spaces or in
Normed spaces. In the first case: we can simply think that $X$ is compactly embedded in $Y$.

% \zhlipsum[1]

% \newpage
\zslideframetitle{New Frame TITLE}%{ 新的 Frame Title}
% \par\vspace*{3em}\par
% 派则指细流金义月无采列,走压看计和眼提
% 问接,作半极水红素支花。果都济素各半走,意红接器长标,等杏近乱共。层题提万
% 任号,信来查段格,农张雨。省着素科程建持色被什,所界走置派农难取眼,并细杆
% 至志本。
\lipsum[1]


\newpage
\lipsum[1]
% 派则指细流金义月无采列,走压看计和眼提
% 问接,作半极水红素支花。果都济素各半走,意红接器长标,等杏近乱共。层题提万
% 任号,信来查段格,农张雨。省着素科程建持色被什,所界走置派农难取眼,并细杆
% 至志本。

\section{TEXT HEIGHT}
\ExplSyntaxOn
\NewCoffin \encoff
\NewCoffin \cncoff
\NewCoffin \mathcoff
\NewCoffin \mixcoff
% \SetHorizontalCoffin \encoff {\Large Hello}
% \SetHorizontalCoffin \cncoff {\Large 国际形势}
% \SetHorizontalCoffin \mathcoff {$\sum\int$}
% \SetHorizontalCoffin \mixcoff {\Large 中文 Hello English}
\ExplSyntaxOff

\TypesetCoffin \encoff (0pt, 0pt)\hspace{4em}
% \TypesetCoffin \cncoff (0pt, 2pt)

% \isCJK{A} \isCJK{你}

\end{document}





\documentclass{article}


\begin{document}
\ExplSyntaxOn
\cs_new:Npn \__aaa_bbb:n #1 
  {HELLO~#1}

% \__aaa_bbb:n {WORLD}
\__aaa_bbb_:n {WORLD}
\ExplSyntaxOff
\end{document}




% test ztex after explcheck
\InputIfFileExists{zlatex-cfg.tex}{}{}
\documentclass[layout={slide}]{../code/ztex}
\zthmnew{Zaxiom, Ztheorem=Thm|{HTML}{a0d911}, Zproposition=Prop|blue}
\zthmnew[proof]{Zproof, Zexample=EXAMPLE|red, Zsolution=Solution|}
\zthmstyle{background}

\begin{document}
\begin{Zproof}[zthmnew-1]
  This is a Zproof zthmnew-1.
  \end{Zproof}
  \begin{Zexample}[zthmnew-2]
  This is a Zexample zthmnew-2.
  \end{Zexample}
  \begin{Ztheorem}[zthmnew-3]
  This is a Ztheorem zthmnew-3
  \end{Ztheorem}
\end{document}



% test thm module
\InputIfFileExists{zlatex-cfg.tex}{}{}
\documentclass[
  fancy,
  class=book,
  lang=cn, 
]{../code/ztex}
\makeatletter
\zthmstylenew{
  styleA={
    begin={
      \begin{tcolorbox}[
        enhanced jigsaw,
        breakable, sharp corners, 
        colframe=\thm@tmp@color, 
        top=2ex, boxrule=2pt, toprule=0pt, 
        fonttitle=\large\bfseries, colback=white, 
        attach boxed title to top left={xshift=3em,yshift=-\tcboxedtitleheight/2},
        coltitle=\thm@tmp@color, title={\zthmname~\zthmnumber},
        boxed title style={%
          empty, left=1pt, right=1pt, bottom=0pt,
          overlay={%
            \draw[color=\thm@tmp@color,line width=2pt,line cap=round]
            ([yshift=-1pt]frame.west)--
            ++(-2.9em,0) ([yshift=-1pt]frame.east)--
            ++(3em,0);
          }
        },
      ]
    },
    end = {\end{tcolorbox}},
  }
}
\makeatother
\zthmstyle{styleA}
\zthmlang{en}
\usepackage{zhlipsum}
\usepackage{lipsum}


\begin{document}
\mainmatter
\chapter{FIRST CHAP}
\section{FIRST}
\lipsum[1][1-2]
\begin{definition}[DEF-TEST]\label{def:A}
  \lipsum[1][1-2]
  \begin{align}
    \sum_{i=1}^{+\infty}{\int_{0}^{i}-\frac{1}{t}\mathrm{d}t} = \frac{\pi^2}{6}
  \end{align}
  \lipsum[1][3-5]
\end{definition}
\lipsum[1][3-5] -- \cref{thm:A}.


\vspace*{13em}
\section{SECOND}
\begin{theorem}[THM-TEST]\label{thm:A}
  \zhlipsum[1]
\end{theorem}
\lipsum[1][3-5] -- \cref{def:A}.
\end{document}






% ==> ztex font module test
\InputIfFileExists{zlatex-cfg.tex}{}{}
\documentclass[lang=cn]{../code/ztex}
% \documentclass{ctexart}
% \usepackage{lipsum}
% \usepackage{zhlipsum}
% \setmainfont{Times New Roman}
% \ztexset{mathSpec={font=mathpazo}}
% \zfontset{
%   sysfont,
%   % doc = lmm,
%   % text=times,
%   math = euler
% }
\zfontfamilynew[CJK]{
  cmd = YaHei, 
  name = msyh.ttc,
  path = ./Fonts/, 
  % feat = { bd=*bd } 
}
\zfontfamilynew{
  cmd = Arial,
  name = arial.ttf,
  path = ./Fonts/,
  feat = {sl=*i}
}
% % do NOT 'file name' and  'font name' in one decalre.
% \zfontfamilynew{
%   cmd = SourceCodePro,
%   name = Source Code Pro,
%   feat = { bd=Source Code Pro Bold }
% }
% \newfontfamily\TEST{Times New Roman}
%   [
%     BoldFont = Times New Roman,
%     ItalicFont = Times New Roman
%   ]
% \xeCJKsetup{EmboldenFactor=8}
% \setCJKfamilyfont{NewYH}{msyh.ttc}
%   [
%     Path = ./Fonts/,
%     AutoFakeBold = 12, % works
%     % AutoFakeBold = true,
%     % EmboldenFactor = 1, % do NOT make sense, when just use this.
%     % BoldFont = msyhbd.ttc
%   ]

% \setCJKfamilyfont{NewYHII}{msyh.ttc}
%   [
%     Path = ./Fonts/,
%     % AutoFakeBold = 8, % works.
%     % 
%     AutoFakeBold = true,
%     % EmboldenFactor = 8, 
%     % NOTE:
%     % 1. do NOT make sense, when just use this.
%     % 2. can NOT use this arg in local.
%   ]


\begin{document}
\ExplSyntaxOn
% \def\bdnum{\fp_use:N \g__xeCJK_embolden_factor_fp}
% \fp_gset:Nn \g__xeCJK_embolden_factor_fp {10}
\ExplSyntaxOff

% System: {你好世界,\bfseries 你好世界}.

% {\CJKfamily{NewYH}\bdnum{} 你好世界,\bfseries 你好世界.}

% {\CJKfamily{NewYHII}\bdnum{} 你好世界,\bfseries 你好世界.}

{\YaHei 你好世界,\bfseries 你好世界.}


{\Arial Hello world,\slshape Hello world.}


% {Hello world,\SourceCodePro Hello world,\bfseries Hello world.}
\end{document}





% ==> box scale
\InputIfFileExists{zlatex-cfg.tex}{}{}
\documentclass[lang=en, hyper]{../code/ztex}
\usepackage{minted}
% \usepackage[T4,T1]{fontenc}
% \newcommand{\USP}[1]{{\fontencoding{T4}\selectfont\char"20}}
\ExplSyntaxOn\makeatletter
\setlength\fboxsep{0pt}
% \NewDocumentCommand{\ztexscatter}{O{\use:n}mm}
%   {% #1:item cmd, #2:width; #3:object
%     \hb@xt@#2{
%       \tl_map_inline:nn {#3} 
%         {
%           \seq_put_right:No \l__ztex_scatter_seq {#1{##1}}
%         } 
%       \seq_use:Nn \l__ztex_scatter_seq { \hfill }
%     }
%     \seq_clear:N \l__ztex_scatter_seq
%   }
% \seq_new:N \l__ztex_scatter_seq
% \cs_generate_variant:Nn \tl_set:Nn {No}
% \gdef\ztoolWdScale#1{\textcolor{blue}{\ztool_onlyset_to_wd:nn {1em}{#1}}}
\gdef\ztoolWdScale#1{\textcolor{blue}{#1}}
\ExplSyntaxOff

\ExplSyntaxOn
\clist_new:N \l__ztex_doc_source_clist
\clist_clear:N \l__ztex_doc_source_clist
\cs_set:Npn \__ztex_doc_source:nn #1#2 
  {
    \clist_map_inline:nn {#2}
      {
        \clist_put_right:Nn \l__ztex_doc_source_clist 
          {
            \subsubsection{##1}
            \inputminted{latex}{../code/#1/ztex.#1.##1.tex}
          }
      }
  }
\newcommand{\inputZTeXSource}[2][module]
  {
    \__ztex_doc_source:nn {#1}{#2}
    \clist_use:Nn \l__ztex_doc_source_clist   
      { \newpage }
  }
\ExplSyntaxOff


\begin{document}
\section{\ztex{} and \ztex*}
\zTeX*{} introduction. Introduction to \zTeX{} and \ztex*.

Hello, this is \zlatex{} and \zlatex*{}. 


Space token:{\usefont{OT1}{cmtt}{m}{n}\asciispace}
% Space token: \USP{ }.
% \ztexscatter{5em}{mmm}


\underline{%
  \zboxitemalign[cmd=\ztoolWdScale, type=scatter]{9em}{{Tom}{Amy}{Jennery}}%
}

\underline{%
  \zboxitemalign[cmd=\ztoolWdScale, type=right]{9em}{{Tom}{Amy}{Jennery}}%
}


\underline{%
  \zboxitemalign[cmd=\ztoolWdScale, type=center]{9em}{Tom Amy\ Jenn ery}%
}

\underline{%
  \zboxitemalign[cmd=\ztoolWdScale, type=left]{9em}{{Tom}{Amy}{Jennery}}%
}


% \inputZTeXSource{box, font}
% \inputZTeXSource[library]{fancy, alias}

\ExplSyntaxOn
% \def\zTeX{
%   \texorpdfstring{
%     \ztool_set_to_wd:nn {7pt}{
%       \ztool_rotate:nn {89.25}{\(\aleph\)}
%     }\kern-1.5pt\hbox{}\TeX{}
%   }{zTeX}
% }


% \maketitle
% \ExplSyntaxOn
% \ztool_onlyset_to_wd:nn {2em}{A}\par
% \ztool_onlyset_to_wd:nn {2em}{AA}\par
% \ztool_onlyset_to_wd:nn {2em}{AAA}\par
% \ztool_onlyset_to_wd:nn {2em}{AAAA}\par
% \ztool_onlyset_to_wd:nn {2em}{AAAAAA}\par

% \ztool_onlyset_to_ht:nn {2.5em}{\fbox{\vbox{\hbox{A}}}}\quad
% \ztool_onlyset_to_ht:nn {2.5em}{\fbox{\vbox{\hbox{A}\hbox{A}}}}\quad
% \ztool_onlyset_to_ht:nn {2.5em}{\fbox{\vbox{\hbox{A}\hbox{A}\hbox{A}}}}\quad
% \ztool_onlyset_to_ht:nn {2.5em}{\fbox{\vbox{\hbox{A}\hbox{A}\hbox{A}\hbox{A}}}}


% box height test
% \hbox_set:Nn \l_tmpa_box {\huge\bfseries Hello}
% \dim_set:Nn \l_tmpa_dim { \box_ht:N \l_tmpa_box }
% \dim_show:N \l_tmpa_dim
% luatex:15.32524pt; xetex:15.32524pt % ==> no bug


% \hbox_set:Nn \l_tmpb_box {
%   \parbox[b][][r]{10em}{
%       {\huge\bfseries Hello}\\[0pt]
%       {\huge\bfseries world}
%     }
% }
% \dim_set:Nn \l_tmpb_dim { \box_ht:N \l_tmpb_box }
% \dim_show:N \l_tmpb_dim
% luatex:31.7667pt; xetex:31.7667pt % ==> no bug


% \long\def\formatTitle{{\huge\bfseries \zTeX{} 用户手册}}
% \long\def\formatAuthor{{\Large\bfseries World}}
% \newcommand\titleUpperBox[2][0pt]{
%   \parbox[b][#2][r]{\l_tmpa_dim}{
%     {\formatTitle}\\[#1]
%     {\formatAuthor}
%   }
% }
% \ztool_get_ht_plus_dp:Nn \l_tmpb_dim {\titleUpperBox{}}
% \dim_show:N \l_tmpb_dim % luatex:80.14294pt; xetex:\l_tmpb_dim=75.46144pt % bug


% \hbox_set:Nn \l_tmpa_box {\huge\bfseries 用户手册}
% \dim_set:Nn \l_tmpa_dim { \box_ht:N \l_tmpa_box }
% \dim_show:N \l_tmpa_dim
% luatex:19.43259pt; xetex:16.73853pt % bug !!!!
\ExplSyntaxOff
\end{document}


% \documentclass{article}
% \usepackage[T1]{fontenc}

% \begin{document}
% \ExplSyntaxOn
% \cs_set_eq:NN \cctabend \cctab_end:
% \cctab_const:Nn \g__ztex_keyval_cctab 
% {
%   \cctab_select:N \c_document_cctab
%   \char_set_catcode_active:n {124}
% }
% \cctab_begin:N \g__ztex_keyval_cctab
% \gdef|{XXX}

% ||\par
% \cctabend
% ||
% \ExplSyntaxOff
% \end{document}


% ==> format l3doc syntax description
\InputIfFileExists{zlatex-cfg.tex}{}{}
\documentclass[class=l3dox]{../code/ztex}
% \documentclass[twoside]{l3dox}
% \usepackage{xcolor}
% \usepackage{geometry}
\AtBeginDocument{
  % \DeleteShortVerb \"
  \DeleteShortVerb \|
}
\catcode`>=13\gdef>{\dotfill}
% \catcode`|=13
% \AddToHook{env/function/begin}{\def|{\textup{\string|}}}
% \AddToHook{env/function/after}{\catcode`|=12}
% key-value env 
\ExplSyntaxOn
\cs_new_protected:Npn \__codedoc_function_ztex:nnw #1#2
  {
    \__codedoc_function_typeset_start:
    \__codedoc_function_init:
    \tl_set:Nn \l__codedoc_macro_argument_tl {#2}
    \keys_set:nn { l3doc/function } {#1}
    \__codedoc_names_get_seq_ztex:nN {#2} \l__codedoc_names_seq
    \__codedoc_names_parse:
    \__codedoc_function_typeset:
    \__codedoc_function_reset:
    \__codedoc_function_descr_start:w
  }
\cs_new_protected:Npn \__codedoc_names_get_seq_ztex:nN #1#2
  {
    \bool_if:NTF \l__codedoc_names_verb_bool
      {
        \seq_clear:N #2
        \seq_put_right:No #2 { {#1} }
      }
      {
        \tl_set:Nn \l__codedoc_tmpa_tl {#1}
        \tl_remove_all:Ne \l__codedoc_tmpa_tl
          { \exp_not:N \obeyedline \c_percent_str }
        \tl_remove_all:Ne \l__codedoc_tmpa_tl
          { \exp_not:N \obeyedline }
        \__kernel_tl_set:Nx \l__codedoc_tmpa_tl { \l__codedoc_tmpa_tl }
        \tl_remove_all:Ne \l__codedoc_tmpa_tl
          { \iow_char:N \^^M \c_percent_str }
        \tl_remove_all:Ne \l__codedoc_tmpa_tl { \tl_to_str:n { ^ ^ A } }
        \tl_remove_all:Ne \l__codedoc_tmpa_tl { \iow_char:N \^^I }
        \tl_remove_all:Ne \l__codedoc_tmpa_tl { \iow_char:N \^^M }
        \__codedoc_detect_internals:N \l__codedoc_tmpa_tl
        \__codedoc_replace_at_at:N \l__codedoc_tmpa_tl
        \tl_set:Ne \l__codedoc_tmpa_tl 
          { 
            \clist_map_function:NN \l__codedoc_tmpa_tl 
              \__ztex_add_parent_key:n
          }
        \exp_args:NNe \seq_set_from_clist:Nn #2
          { \l__codedoc_tmpa_tl }
      }
  }
\cs_set:Npn \__ztex_add_parent_key:n #1 
  {
    \textcolor{gray}{\l__codedoc_parent_key_ztex_tl/}
    \tl_trim_spaces:n {#1},
  }
\cs_set_eq:NN \cctabend \cctab_end:
\cctab_const:Nn \g__ztex_keyval_cctab 
{
  \cctab_select:N \c_document_cctab
  \char_set_catcode_active:n {124}
}
\DeclareDocumentEnvironment { keyval } { O{} +m }
  { 
    \cctab_begin:N \g__ztex_keyval_cctab
    \__codedoc_function_ztex:nnw {#1} {#2} 
  }
  { 
    \__codedoc_function_end: 
    % \cctab_end: 
  }
\group_begin:
\catcode`|=\active
\gdef\ztexbar{\def|{\textup{YYY}}}
\group_end:
\DeclareDocumentEnvironment { syntax } { }
  { 
    \cctab_begin:N \g__ztex_keyval_cctab
    % \catcode`|=\active
    \ztexbar
    \__codedoc_syntax:w 
  }{
    \__codedoc_syntax_end:
    \cctab_end: 
    \ignorespacesafterend
  }
\ExplSyntaxOff


\begin{document}
\tableofcontents
\section{Hello}
Hello world.
\begin{function}[added=2024-04-27]{\testA}
  \begin{syntax}
    \cs{testA}\oarg{arg1}\meta{\upshape{AAA\string|BBB}}
    \cs{testA}\oarg{arg1}\meta{AAA|BBB}
  \end{syntax}
  A simple test for verb short hand, |.
\end{function}



Key value, |, env test:
\begin{keyval}[parent=lang]{en, cn, en cn}
\begin{syntax}
en > value forbidden 
en cn > value forbidden 
cn > initial value:\textcolor{red}{9999}
\end{syntax}
document class automatically processes and loads corresponding packages based
on user-specified options. Therefore, the packages and commands loaded by the z\TeX{}
document class vary depending on
\end{keyval}


\newpage
Hello world II.
\begin{function}[added=2024-04-27]{\testB}
  \begin{syntax}
    \cs{testB}\Arg{arg1}\meta{\upshape{AAA\string|BBB}}
    \cs{testB}\parg{arg1}\meta{AAA|BBB}
  \end{syntax}
  A simple test for verb short hand.
\end{function}
\end{document}