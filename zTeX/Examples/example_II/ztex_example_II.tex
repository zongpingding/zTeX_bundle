\documentclass[lang=cn]{ztex}
\ztexloadlib{alias}
\zthmhook*[theorem]{before=\vskip.25em}
\zthmproofhook*[proof]{before=\vskip.25em}


\begin{document}
\zaliasOn
\section{整数矩阵及其应用}
故对应于同一模之二方阵是左结合的. 反之, 二非奇异的左结合方阵对应于同一
模 .故若将所有的 $n$ 级非奇异方阵依左结合关系分类 ,则每一类代表一模 ,且不同
的类所代表的模也不同. 以后凡说到 ``模 $\K{M}$ 对应于方阵 $A$'', 此 $A$ 即表示
模 $\K{M}$ 所对应的一类方阵中的一个.


\begin{theorem}
  模 $\K{M}$ 包有模 $\K{N}$ 的充要条件是模 $\K{M}$ 所对应的方阵右除尽模 $\K{N}$ 所对应的方阵.
\end{theorem}

\begin{proof}
  命模 $\K{M}$ 及 $\K{N}$ 之底分别为 $y_1, y_2, \cdots, y_n$ 及 $z_1, z_2, \cdots, z_m$.
  所对应的方阵分别为 $A=(a_{ij})$ 及 $B=(b_{ij})$. 若 $\K{M}$ 包有模 $\K{N}$, 则显然此同余关系亦有反身,
  对称, 传递等三种性质, 故可将所有线性型依\hspace{-8pt} $\mod\K{M}$ 分类:属于同一类者互相同余,不同类者绝不同余.
  \[
    y = a_1x_1 + a_2x_2 + \cdots + a_nx_n
  \]
  如是所分成之类的数目名为 $\K{M}$ 之矩, 以 $N(\K{M})$ 记之 (其存在性还未证明). 显然 $\K{M}$ 本身即为其中
  之一类. 故不妨假定底已取标准形式 (4).任一线性型 ...
\end{proof}


\begin{theorem}
  若 $\K{M}\supseteq \K{N}$, $\K{M}, \K{N}$ 所对应的矩阵分别为 $A, B$, 则依\hspace{-8pt} $\mod\K{N}$ 将 $\K{M}$ 
  中的元素分类, 所得之类数为 $\frac{N(\K{N})}{N(\K{M})} = \frac{|B|}{|A|}$.
\end{theorem}
\begin{proof}
  由未定量 $x_1, x_2, \cdots, x_n$ 表出 $\K{D} = \{x_1, x_2, \cdots, x_n\}$ 也可由其他未定量表出. 如命
  \[\begin{pmatrix}
    x_1\\
    \vdots\\
    x_n
  \end{pmatrix}
  = \begin{pmatrix}
    u_{11} & u_{12} & \cdots & u_{1n}\\
    & \vdots & \cdots & \vdots\\
    u_{n1} & u_{n2} & \cdots & u_{nn}
  \end{pmatrix}
  = \begin{pmatrix}
    y_1\\
    \vdots\\
    y_n
  \end{pmatrix}
\]
故由定理 5.1 可知: 对固定的 $n$ 维模 $\K{M}$, 可经过模的换底及 $\K{D}$ 的换底, 
使其对应之方阵化为对角线方阵.
\end{proof}

两模 $\K{M}_1$ 及 $\K{M}_2$ 的所有公共元素成一模, 此模称为 $\K{M}_1$ 与 $\K{M}_2$ 的交, 以 $\K{M}_m$ 记之.
又 $\K{M}_1$ 及 $\K{M}_2$ 中所有元素的和、差所成的集合也是一模, 此模称为 $\K{M}_1$ 与 $\K{M}_2$ 的和, 
以 $\K{M}_d$ 记之.

\begin{theorem}
  设模 $\K{M}_1$, $\K{M}_2$, $\K{M}_m$, $\K{M}_d$ 分别对应于方阵 $M_1, M_2, M_m, M_d$, 则 $M_m$ 为 
  $M_1, M_2$ 之最小公倍, $M_d$ 为 $M_1, M_2$ 之最大公约.
\end{theorem}
\begin{proof}
  由 $\K{M}_1 \supseteq \K{M}_m$ 及 $\K{M}_2 \supseteq \K{M}_m$ 可知:
  \[
    M_m = A_1M_1 + A_2M_2.
  \]
  若 $M_3 = B_1M_1 = B_2M_2$ 为 $M_1, M_2$ 之任一公倍, $\K{M}_3$ 为 $M_3$ 对应之模, 则
  \[
    \K{M}_3 \subseteq \K{M}_1, \K{M}_3 \subseteq \K{M}_2,
  \] 
  因而
  \[
    \K{M}_3 \subseteq \K{M}_m,  M_3 = CM_m.
  \]
  即 $M_m$ 为 $M_1, M_2$ 之最小公倍 .同样可证 $M_d$ 为 $M_1, M_2$ 之最大公约.
\end{proof}
\zaliasOff
\end{document}