\ProvidesExplFile{ztikz.library.basic.tex}{2024/12/17}{1.0.0}{basic~library~for~ztikz}



% ----------------------------------------------------------------------
%                            basic packages
% ----------------------------------------------------------------------
\RequirePackage{tikz}
\RequirePackage{etoolbox}
\patchcmd{\pgfutil@InputIfFileExists}{\input #1}{%
  \@pushfilename
  \xdef\@currname{#1}
  \input #1
  \@popfilename
}{}{}
\usetikzlibrary{arrows.meta}
\usetikzlibrary{intersections}
\usetikzlibrary{patterns}
\usetikzlibrary{plotmarks}
\usetikzlibrary{positioning}
\usetikzlibrary{shapes.geometric}
\usetikzlibrary{decorations.markings}
\usetikzlibrary{fadings}



% ==> coordinate basic components
\ztikz_keys_define:nn { point }
  {
    type    .str_set:N  = \l__point_type_str,
    type    .initial:n = { * },
    radius  .dim_set:N = \l__point_radius_dim,
    radius  .initial:n = { 1pt },
    color   .tl_set:N  = \l__point_color_tl, 
    color   .initial:n = { black },
    opacity .tl_set:N  = \l__point_opacity_tl, 
    opacity .initial:n = { 1 },
    rotate  .fp_set:N  = \l__point_rotate_angle,
    rotate  .initial:n = { 0 },  
  }
\NewDocumentCommand\ShowPoint{ O{}mO{}O{} }
  {
    \group_begin:
    \exp_args:Nne \ztikz_keys_set:nn { point } { #1 }
    \seq_set_split:Nnn \l__point_list_seq { ; }{#2}
    \seq_set_split:Nnn \l__point_label_seq { ; }{#3}
    \int_step_inline:nnnn {1}{1}{\seq_count:N \l__point_list_seq}{        
      \draw plot [
        only~ marks,
        mark = \str_use:N \l__point_type_str, 
        mark~ size = \dim_use:N \l__point_radius_dim,
        mark~ options = {
          rotate  = \fp_use:N \l__point_rotate_angle, 
          opacity = \tl_use:N \l__point_opacity_tl, 
          color   = \tl_use:N \l__point_color_tl,
          ball~ color = \tl_use:N \l__point_color_tl,
        }
      ] coordinates{\seq_item:Nn \l__point_list_seq{##1}} 
        node[#4] {\seq_item:Nn \l__point_label_seq{##1}};
    }
    \group_end:
  }
\NewDocumentCommand\ShowGrid{ O{color=gray, very~ thin, step=1}m }
  {
    \seq_set_split:Nnn \l__grid_param_ii_seq { ; }{#2}
    \draw[#1] \seq_item:Nn \l__grid_param_ii_seq{1} grid \seq_item:Nn \l__grid_param_ii_seq{2};
  }
% intersection
\NewDocumentCommand\ShowIntersection{ omm }
  {
    \seq_set_split:Nnn \l__intersection_num_seq { ; }{#2}
    \path[name~ intersections={
      of=\seq_item:Nn \l__intersection_num_seq{1}~ 
      and~ \seq_item:Nn \l__intersection_num_seq{2}
    }]; 
    \int_step_inline:nnnn {1}{1}{#3}{
      \ShowPoint[#1]{(intersection-##1)}
    }
  }
% polygon plot
\ztikz_keys_define:nn { polygon }
  {
    radius       .fp_set:N  = \l__polygon_radius_fp,
    radius       .initial:n = { 1 },
    edgeColor    .tl_set:N  = \l__polygon_edge_color_tl,
    edgeColor    .initial:n = { black },
    fillColor    .tl_set:N  = \l__polygon_fill_color_tl,
    fillColor    .initial:n = { },
    fillOpacity  .fp_set:N  = \l__polygon_fill_opacity_fp,
    fillOpacity  .initial:n = { 0 },
    rotate       .fp_set:N  = \l__polygon_rotate_angle,
    rotate       .initial:n = { 0 },
    shift        .tl_set:N  = \l__polygon_shift_tl,
    shift        .initial:n = { (0,0) },
    marker       .tl_set:N  = \l__polygon_marker_option_tl,
    marker       .initial:n = { },
  }
\tl_new:N \l__ztikz_basic_poly_path_tl
\NewDocumentCommand\Polygon{ O{}m }
  {
    \group_begin:
    \ztikz_keys_set:nn { polygon } { #1 }
    % strip '(' and ')'
    \tl_replace_once:Nnn \l__polygon_shift_tl{(}{}
    \tl_replace_once:Nnn \l__polygon_shift_tl{)}{}
    \coordinate (mv) at (\tl_use:N \l__polygon_shift_tl);
    % create polygon
    \begin{scope}[shift=(mv), rotate=\fp_use:N \l__polygon_rotate_angle]
    % arg require: #2 $\ge$ 3 
    \int_step_inline:nnn {1}{#2}{
      % draw edges
      \fp_set:Nn \l_angle_fp {360/#2*##1*\c_one_degree_fp}
      \fp_set:Nn \l_angle_next_fp {360/#2*(##1+1)*\c_one_degree_fp}
      \draw [\tl_use:N \l__polygon_edge_color_tl] 
        ( \fp_eval:n {\l__polygon_radius_fp*cos(\l_angle_fp)}, 
          \fp_eval:n {\l__polygon_radius_fp*sin(\l_angle_fp)}
        ) -- ( 
          \fp_eval:n {\l__polygon_radius_fp*cos(\l_angle_next_fp)}, 
          \fp_eval:n {\l__polygon_radius_fp*sin(\l_angle_next_fp)}
        );
      % fill polygon path 
      \int_compare:nNnTF {##1}<{#2}
        {
          \tl_put_right:Nn \l__ztikz_basic_poly_path_tl {(p##1)--}
        }{
          \tl_put_right:Nn \l__ztikz_basic_poly_path_tl {(p##1)--cycle}
        }
      % mark coordinates
      \coordinate (p##1) at ( 
        \fp_eval:n {\l__polygon_radius_fp*cos(\l_angle_fp)}, 
        \fp_eval:n {\l__polygon_radius_fp*sin(\l_angle_fp)}
      );
    }
    % fill polygon (none-color -> opacity=1; or opacity=.75)
    \tl_if_empty:NTF \l__polygon_fill_color_tl {
      \fp_set:Nn \l__polygon_fill_opacity_fp {0}
    }{
      \fp_set:Nn \l__polygon_fill_opacity_fp {.75}
    }
    \fill [\tl_use:N \l__polygon_fill_color_tl, fill~opacity=\fp_use:N \l__polygon_fill_opacity_fp] \l__ztikz_basic_poly_path_tl;
    % show markers
    \int_step_inline:nnn {1}{#2}{
      \ShowPoint[\l__polygon_marker_option_tl]{(p##1)}
    }
    \end{scope} 
    \group_end:
  }


% ==> axis
\ztikz_keys_define:nn { axis }
  {
    % basic tick args
    tickStart       .fp_set:N   = \l__start_fp,
    tickStart       .initial:n  = { -5 },
    tickEnd         .fp_set:N   = \l__end_fp,
    tickEnd         .initial:n  = { 5 },
    axisRotate      .fp_set:N   = \l__axis_rotate_angle,
    axisRotate      .initial:n  = { 0 },
    % tick dimension spec
    mainStep        .fp_set:N   = \l__main_step_fp,
    mainStep        .initial:n  = { 1.0 },
    subStep         .fp_set:N   = \l__sub_step_fp,
    subStep         .initial:n  = { 0.1 },
    mainTickLabel   .tl_set:N   = \l__main_tick_label_tl,
    mainTickLabel   .initial:n  = { \fp_use:N {\CurrentFp} },
    tickLabelShift  .dim_set:N  = \l__tick_label_shift_dim,
    tickLabelShift  .initial:n  = { 0pt },
    mainTickLength  .dim_set:N  = \l__main_tick_length_dim,
    mainTickLength  .initial:n  = { 4pt },
    subTickLength   .dim_set:N  = \l__sub_tick_length_dim,
    subTickLength   .initial:n  = { 2pt },
    mainTickLabelPosition .tl_set:N  = \l__main_tick_label_position_tl,
    mainTickLabelPosition .initial:n = { below },
    % color spec
    axisColor       .tl_set:N   = \l__axis_color_tl,
    axisColor       .initial:n  = { black },
    mainTickColor   .tl_set:N   = \l__main_tick_color_tl,
    mainTickColor   .initial:n  = { black },
    subTickColor    .tl_set:N   = \l__sub_tick_color_tl,
    subTickColor    .initial:n  = { black },
    mainTickLabelColor .tl_set:N  = \l__main_tick_label_color_tl,
    mainTickLabelColor .initial:n = { black },
    % tick cross type spec
    tickStyle       .choice:,
    tickStyle/cross .code:n     = \tl_set:Nn \l__tick_spec_tl { cross },
    tickStyle/above .code:n     = \tl_set:Nn \l__tick_spec_tl { above },
    tickStyle/below .code:n     = \tl_set:Nn \l__tick_spec_tl { below },
  }
% ticks style
\tl_new:N  \l__tick_type_tl    % `main' or `sub'
\tl_new:N  \l__tick_spec_tl    % `cross', `above' or `below
\tl_new:N  \l__tick_color_tl 
\dim_new:N \l__tick_length_dim
\tl_new:N  \l__node_text_tl
% draw ticks (main or sub)
\cs_new_protected:Npn \ztikz_draw_axis_ticks_cs:n #1 
  {
    \str_case:NnT \l__tick_type_tl {
      {main}{
        \dim_set_eq:NN \l__tick_length_dim\l__main_tick_length_dim
        \tl_set:NV \l__tick_color_tl\l__main_tick_color_tl
        \tl_set:Nn \l__node_text_tl {\tl_use:N \l__main_tick_label_tl}
      }
      {sub}{
        \dim_set_eq:NN \l__tick_length_dim\l__sub_tick_length_dim
        \tl_set:NV \l__tick_color_tl \l__sub_tick_color_tl
        \tl_set:Nn \l__node_text_tl {}
      }
    }{}
    \str_case:VnT \l__tick_spec_tl {
      {cross}{
        \draw[\tl_use:N \l__tick_color_tl] 
          (#1, 0)++(0, \dim_eval:n {\l__tick_length_dim/2}) 
          -- ++(0, \dim_eval:n {-\l__tick_length_dim}) 
          node[\tl_use:N \l__main_tick_label_position_tl] 
          {
            \textcolor{\tl_use:N \l__main_tick_label_color_tl}
              {\tl_use:N \l__node_text_tl}
          };
      }
      {above}{
        \draw[\tl_use:N \l__tick_color_tl] (#1, 0) 
          -- ++(0, \dim_eval:n {\l__tick_length_dim/2}) 
          node[\tl_use:N \l__main_tick_label_position_tl] 
          {
            \textcolor{\tl_use:N \l__main_tick_label_color_tl}
            {\tl_use:N \l__node_text_tl}
          };
      }
      {below}{
        \draw[\tl_use:N \l__tick_color_tl] (#1, 0) 
          -- ++(0, \dim_eval:n {-\l__tick_length_dim/2}) 
          node[\tl_use:N \l__main_tick_label_position_tl=\dim_use:N \l__tick_label_shift_dim] 
          {
            \textcolor{\tl_use:N \l__main_tick_label_color_tl}
              {\tl_use:N \l__node_text_tl}
          };
      }
    }{}
  } 
% draw axis
\fp_new:N \CurrentFp
\int_new:N \l__substep_num_int
\NewDocumentCommand\ShowAxis{O{}m}
  {
    \group_begin:
    \ztikz_keys_set:nn { axis } { #1 }
    \seq_set_split:Nnn \l__points_seq { ; }{#2}
    \begin{scope}[rotate=\fp_use:N \l__axis_rotate_angle]
    \draw[->, \tl_use:N \l__axis_color_tl] \seq_item:Nn \l__points_seq{1} 
      -- \seq_item:Nn \l__points_seq{2};
    % draw ticks
    \fp_step_inline:nnnn 
      {\fp_eval:n {\l__start_fp}}
      {\fp_use:N \l__main_step_fp}
      {\fp_use:N \l__end_fp}
      {
        % main ticks
        \tl_set:Nn \l__tick_type_tl {main}
        \fp_gset:Nn \CurrentFp {##1}
        \ztikz_draw_axis_ticks_cs:n {##1}
        % sub ticks
        \tl_set:Nn \l__tick_type_tl {sub}
        \int_set:Nn \l__substep_num_int 
          {\fp_eval:n {floor(\l__main_step_fp/\l__sub_step_fp)}}
        \fp_compare:nNnTF {##1}<{\fp_eval:n {floor(\l__end_fp)}}{
          \fp_step_function:nnnN                 
            {\fp_eval:n {##1+\l__sub_step_fp}}
            {\fp_use:N \l__sub_step_fp}
            {\fp_eval:n {##1+\l__substep_num_int*\l__sub_step_fp}}
            \ztikz_draw_axis_ticks_cs:n
        }{}
      }
    \end{scope}
    \group_end:
  }
\NewDocumentCommand{\xAxis}{O{-2}O{8}}
  {
    \ShowAxis[
      tickStart=\fp_eval:n {#1+1}, 
      tickEnd=\fp_eval:n {#2-0.75},
      mainTickLabelPosition=below,
      mainStep=1,           subStep=.25, 
      axisRotate=0,         axisColor=black,
      mainTickColor=black,  subTickColor=black,
      mainTickLength=10pt,  subTickLength=5pt,
      tickLabelShift=0pt,   tickStyle=below, 
    ]{(#1, 0); (#2, 0)}
  }
\NewDocumentCommand{\yAxis}{O{-2}O{8}}
  {
    \ShowAxis[
      tickStart=\fp_eval:n {#1+1}, 
      tickEnd=\fp_eval:n {#2-0.75}, 
      mainStep=1,           subStep=.25, 
      axisRotate=90,        axisColor=black,
      mainTickColor=black,  subTickColor=black,
      mainTickLength=10pt,  subTickLength=5pt,
      tickLabelShift=0pt,   tickStyle=above, 
      mainTickLabelPosition=left
    ]{(#1, 0); (#2, 0)}
  }


% ==> statistic plot function                
\cs_new_protected:Npn \ztikz_statistic_plot_cs:nnnn #1#2#3#4 
  {% #1:starts option; #2:draw-keyval; #3:point-keyval; #4:filename
    \tl_if_empty:nTF {#3}{\draw[#2] plot[#1] file {#4};}
    {
      \group_begin:
      \keys_set:nn { ztikz / point } { #3 }
      \draw[#2] plot [
        % stairs options
        #1,
        % marker options
        mark = \str_use:N \l__point_type_str, 
        mark~ size = \dim_use:N \l__point_radius_dim,
        mark~ options = {
          rotate  = \fp_use:N \l__point_rotate_angle, 
          opacity = \tl_use:N \l__point_opacity_tl, 
          color   = \tl_use:N \l__point_color_tl,
          ball~ color = \tl_use:N \l__point_color_tl,
        }
      ] file {#4};
      \group_end:
    }
  }
\cs_generate_variant:Nn \ztikz_statistic_plot_cs:nnnn {ennn}

% stairs plot
\seq_new:N \l__statistic_option_tl
\NewDocumentCommand\StairsPlot{ O{plot-left;jump-left}O{color=black}O{}m }
  {
    \seq_set_split:Nnn \l__statistic_option_tl { ; }{#1}
    \str_case:enF {\seq_item:Nn \l__statistic_option_tl{1}}{
      {plot-left}{\tl_set:Nn \l__tmpa_tl {const~plot~mark~left}}
      {plot-right}{\tl_set:Nn \l__tmpa_tl {const~plot~mark~right}}
      {plot-mid}{\tl_set:Nn \l__tmpa_tl {const~plot~mark~mid}}
      {}{\tl_set:Nn \l__tmpa_tl {}}
    }{ 
      \msg_new:nnn {ztikz}{ztikz-stairs-plot}{current~stairs~plot~type~is:~'#1'~,~ invalide}
      \msg_error:nn {ztikz}{ztikz-stairs-plot}
    }
    \str_case:enF {\seq_item:Nn \l__statistic_option_tl{2}}{
      {jump-left}{\tl_set:Nn \l__tmpb_tl {jump~mark~left}}
      {jump-right}{\tl_set:Nn \l__tmpb_tl {jump~mark~right}}
      {jump-mid}{\tl_set:Nn \l__tmpb_tl {jump~mark~mid}}
      {}{\tl_set:Nn \l__tmpb_tl {}}
    }{ 
      \msg_new:nnn {ztikz}{ztikz-stairs-plot}{current~stairs~jump~type~is:~'#1'~,~ invalide}
      \msg_error:nn {ztikz}{ztikz-stairs-plot}
    }
    \ztikz_statistic_plot_cs:ennn {\l__tmpa_tl,\l__tmpb_tl}{#2}{#3}{#4}
  }
% stem plot
\NewDocumentCommand\StemPlot{ O{x}O{color=black}O{}m }
  {
    \str_case:enF {#1}{
      {x}{\tl_set:Nn \l__tmpa_tl {ycomb}}
      {y}{\tl_set:Nn \l__tmpa_tl {xcomb}}
      {o}{\tl_set:Nn \l__tmpa_tl {polar~ comb}}
      {}{\tl_set:Nn \l__tmpa_tl  {ycomb}}
    }{ 
      \msg_new:nnn {ztikz}{ztikz-stem-plot}{current~stem~plot~type~is:~'#1'~,~ invalide}
      \msg_error:nn {ztikz}{ztikz-stem-plot}
    }
    \ztikz_statistic_plot_cs:ennn {\l__tmpa_tl}{#2}{#3}{#4}
  }
% bar plot
\NewDocumentCommand\BarPlot{ O{ybar}O{color=black}O{}m }
  {
    \str_case:enF {#1}{
      {x}{\tl_set:Nn \l__tmpa_tl {ybar}}
      {y}{\tl_set:Nn \l__tmpa_tl {xbar}}
      {xc}{\tl_set:Nn \l__tmpa_tl {ybar~ interval}}
      {yc}{\tl_set:Nn \l__tmpa_tl {xbar~ interval}}
      {}{\tl_set:Nn \l__tmpa_tl  {ybar}}
    }{ 
      \msg_new:nnn {ztikz}{ztikz-bar-plot}{current~bar~plot~type~is:~'#1'~,~ invalide}
      \msg_error:nn {ztikz}{ztikz-bar-plot}
    }
    \ztikz_statistic_plot_cs:ennn {\l__tmpa_tl}{#2}{#3}{#4}
  }