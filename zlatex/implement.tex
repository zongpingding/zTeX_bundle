\chapter{z\LaTeX{} Implement}
\section{Class Option}
\subsection{Key-Value}
\renewcommand{\theFancyVerbLine}{
  \sffamily\textcolor{black}{\oldstylenums{\arabic{FancyVerbLine}}}
}
\setminted{numbersep=0pt}

\begin{minted}[linenos=true]{latex}
\RequirePackage {l3keys2e}
\cs_new_protected:Npn \zlatex_define_option:n
  { \keys_define:nn { zlatex / option } }
\cs_new_protected:Npn \zlatex_define:nn #1
  { \keys_define:nn { zlatex / #1} }
\cs_new_protected:Npn \zlatex_set:nn #1
  { \keys_set:nn { zlatex / #1 } }
\cs_new_protected:Npn \zlatex_pkg_load_last:n #1
  { \AddToHook{env/document/before}{#1} }
\cs_new_protected:Npn \zlatex_opt_set_last:n #1
  { \AddToHook{env/document/before}{#1} }
\end{minted}

\subsection{Load Options}
\begin{minted}[linenos=true]{latex}
% setup options for class
\zlatex_define_option:n {
  % fancy to load 'tikz', 'tcolorbox' etc,.
  fancy           .bool_gset:N  = \g__zlatex_fancy_bool,
  fancy           .initial:n    = { false },
  % zlatex language
  lang            .str_gset:N   = \g__zlatex_lang_str,
  lang            .initial:n    = { en },
  hyper           .bool_gset:N  = \g__zlatex_hyperref_bool,
  hyper           .initial:n    = { false },
  % class and options
  class           .str_gset:N   = \g__zlatex_subclass_type_str,
  class           .initial:n    = { article },
  classOption     .clist_gset:N = \g__zlatex_subclass_option_clist,
  classOption     .initial:n    = { oneside, 10pt },
  % zlatex options meta key 
  layout          .meta:nn      = { zlatex / layout }{#1},
  mathSpec        .meta:nn      = { zlatex / mathSpec }{#1},
  font            .meta:nn      = { zlatex / font }{#1},
  bib             .meta:nn      = { zlatex / bib }{#1},
  % toc setup
  toc             .multichoice:,
  toc / 2column   .code:n       =  { \zlatex_pkg_load_last:n {\RequirePackage[toc]{multitoc}} },
  toc / redef     .code:n       =  { 
    \str_case:VnF \g__zlatex_lang_str {
      {en}{ \zlatex_opt_set_last:n {\renewcommand{\contentsname}{\hfill\bfseries\huge Contents \hfill}} }
      {cn}{ \zlatex_opt_set_last:n {\renewcommand{\contentsname}{\hfill\bfseries\huge 目录     \hfill}} }
    }{\msg_error:nn {zlatex}{option-language}}
  },
  toc / unknown   .code:n       =  {
    \msg_new:nnn {zlatex}{option-toc}{Current~toc~option~is~:~'#1'~,default~toc~settings~substitute.}
    \msg_warning:nn {zlatex}{option-toc}
  },
}

% sub-key for each option
\zlatex_define:nn {layout}{
  margin              .bool_gset:N  = \g__zlatex_margin_bool,
  margin              .initial:n    = { false },
  slide               .bool_gset:N  = \g__zlatex_slide_bool,
  slide               .initial:n    = { false },
  aspect              .tl_gset:N    = \g__zlatex_aspectratio_tl,
  aspect              .initial:n    = { 12|9 },
}
\zlatex_define:nn {mathSpec}{
  alias               .bool_gset:N  = \g__zlatex_math_alias_bool,
  alias               .initial:n    = { false },
  envStyle            .tl_gset:N    = \g__zlatex_math_env_style_tl,
  envStyle            .initial:n    = { plain },
  font                .choice:,
  font / newtx        .code:n       = { \zlatex_pkg_load_last:n { \RequirePackage{newtxmath} } },
  font / mtpro2       .code:n       = { \zlatex_pkg_load_last:n { \RequirePackage[lite, subscriptcorrection, slantedGreek, nofontinfo]{mtpro2} } },
  font / euler        .code:n       = { \zlatex_pkg_load_last:n { \RequirePackage[OT1, euler-digits]{eulervm} } },
  font / mathpazo     .code:n       = { \let\rmbefore\rmdefault\zlatex_pkg_load_last:n { \RequirePackage{mathpazo} } \let\rmdefault\rmbefore},
  font / unknown      .code:n       = {
    \msg_new:nnn {zlatex}{option-mathfont}{Current~math~font~option~is~:~'#1'~,default~math-font~substitute.}
    \msg_warning:nn {zlatex}{option-mathfont}
  },
}
\zlatex_define:nn {font}{
  config              .bool_gset:N  = \g__zlatex_font_config_bool,
  config              .initial:n    = { false }, 
}
\zlatex_define:nn {bib}{
  source              .str_gset:N   = \g__zlatex_bib_source_str,
  source              .initial:n    = { ref.bib },
  backend             .str_gset:N   = \g__zlatex_bib_backend_str,
  backend             .initial:n    = { biber },
}

% option setup
\ProcessKeysOptions {zlatex / option}
\NewDocumentCommand{\zlatexSetup}{m}{
    \zlatex_set:nn {option}{#1}
}

% handle fancy option
\newif\ifloadtikz
\bool_if:NTF \g__zlatex_fancy_bool {
  \RequirePackage[many]{tcolorbox}
  \loadtikztrue
}{\loadtikzfalse}
\ExplSyntaxOff\ifloadtikz
  \RequirePackage{tikz}
  \usetikzlibrary{calc}
\fi\ExplSyntaxOn
\NewDocumentCommand{\zlatexSetup}{m}{
  \zlatex_set:nn {option}{#1}
}
\end{minted}

\section{class load}
\begin{minted}[linenos=true]{latex}
% pass clist options main subclass: 'article', 'book', 'ctexbook'
\str_case:VnF \g__zlatex_subclass_type_str {
    {article}{
        \PassOptionsToClass{\g__zlatex_subclass_option_clist}{ article }
        \LoadClass{article}
    }
    {book}{
        \PassOptionsToClass{\g__zlatex_subclass_option_clist}{ book }
        \LoadClass{book}   
    }
    {ctexbook}{
        \str_set:Nn \g__zlatex_lang_str {cn}
        \PassOptionsToClass{\g__zlatex_subclass_option_clist}{ ctexbook }
        \PassOptionsToPackage{quiet}{fontspec}
        \LoadClass{ctexbook}    
    }
}{\relax}
\end{minted}

\section{compile engine}
\begin{minted}[linenos=true]{latex}
% baisc document class and packages option
\msg_new:nnn {zlatex}{compile-engine-pdftex}{Current~compile~engine~is~XETEX,~use~PDFTEX~instead.}
\msg_new:nnn {zlatex}{compile-engine-xetex }{Current~compile~engine~is~PDFTEX,~use~XETEX~instead.}
\str_case:VnF \g__zlatex_lang_str {
  {en} { 
    \sys_if_engine_xetex:TF 
      {\bool_if:NF \g__zlatex_font_config_bool {\msg_warning:nn {zlatex}{compile-engine-pdftex}}}
      {\RequirePackage[utf8]{inputenc}}
    \RequirePackage[T1]{fontenc}
    \RequirePackage[english]{babel}
    \RequirePackage{microtype}
  }
  {cn} {
    \sys_if_engine_xetex:TF {}{\msg_error:nn {zlatex}{compile-engine-xetex}}
    \PassOptionsToPackage{quiet}{fontspec}
    \PassOptionsToPackage{no-math}{fontspec}
    \str_if_eq:VnF \g__zlatex_subclass_type_str {ctexbook}{
      \RequirePackage[UTF8, heading]{ctex}
      \linespread{1.3}
    }
  }
}{\msg_error:nn {zlatex}{option-language}}   
\end{minted}

\section{Layout}
\subsection{geometry}
\begin{minted}[linenos=true]{latex}
\RequirePackage{geometry}
% page layout 
\if@twoside 
    \geometry{a4paper, left=3cm, right=5.5cm, bottom=3.5cm, footskip=1.5cm, marginparsep=1em}
    \bool_if:NTF \g__zlatex_margin_bool {}{
        \msg_new:nnn {zlatex}{option-page-margin}{No~margin~option~is~only~accessible~in~oneside~layout,~margin~option~is~now~enabled~by~default.} 
        \msg_warning:nn {zlatex}{option-page-margin}
    }
\else 
    \bool_if:NTF \g__zlatex_margin_bool {
        \geometry{a4paper, left=3cm, right=5.5cm, bottom=3.5cm, footskip=1.5cm, marginparsep=1em}
        \dim_gset:Nn \marginparwidth{9.25em}
    }{
        \geometry{a4paper, left=3cm, right=3cm, bottom=3.5cm, footskip=1.5cm, marginparsep=1em}
        \renewcommand{\marginpar}[1]{\leftbar\noindent#1\endleftbar}
    }
\fi
\end{minted}

\subsection{Slide mode}
\begin{minted}[linenos=true]{latex}
% ==> slide mode
\exp_args:NNnx \seq_set_split:Nnn \l_tmpa_seq {|}{\g__zlatex_aspectratio_tl}
\bool_if:NTF \g__zlatex_slide_bool {
  % layout and mark
  \RequirePackage{titlesec}
  \geometry {
    papersize={\seq_item:Nn \l_tmpa_seq {1}cm, \seq_item:Nn \l_tmpa_seq {2}cm},
    hmargin=1cm, top=.4cm, headsep=0pt,
    includefoot, bottom=5.5pt,
    footskip=\dim_eval:n {1.25em + 5pt}
  }
  \IfClassLoadedTF{book}{
    \let\cleardoublepage\clearpage
    \renewcommand\chaptermark[1]{\markboth{#1}{}}
    \renewcommand\thesection{\arabic{section}}
    \AddToHook{shipout/firstpage}{\setcounter{page}{0}}
    \AddToHook{cmd/@schapter/after}{\thispagestyle{fancy}}
    \AddToHook{cmd/@chapter/after}{\thispagestyle{fancy}}
    \zlatex_opt_set_last:n {\renewcommand\mainmatter{}\renewcommand\frontmatter{}}
    \renewcommand\tableofcontents {
      \if@twocolumn\@restonecoltrue\onecolumn\else\@restonecolfalse\fi
      \section*{\contentsname\@mkboth{\MakeUppercase\contentsname}{\MakeUppercase\contentsname}}
      \@starttoc{toc}\if@restonecol\twocolumn\fi
      \setcounter{page}{1}
    }
    \zlatex_opt_set_last:n {
      \titleformat{\chapter}
        {\null\vfill\Huge}{}{10pt}{}[\vspace{\stretch{2}}\null]
      \titleformat{name=\chapter, numberless}
        {\null\vfill\Huge}{}{10pt}{}[\vspace{\stretch{2}}\null]
    }
    \zlatex_opt_set_last:n {\titlespacing*{\chapter}{0pt}{30pt}{0pt}}
  }{\relax}
  \renewcommand\sectionmark[1]{\markboth{#1}{}}
  \titleformat{\section}{\Large\color{zslideMain}}{}{0pt}{}[\bool_gset_true:N \g_new_sec_bool]
  % status bar
  \bool_new:N \g_new_sec_bool
  \definecolor{zslideMain}{HTML}{a30000}
  \definecolor{zslideII}{HTML}{e0e0e0}
  \definecolor{zslideIII}{HTML}{f0f0f0}
  \AddToHook{cmd/chapter/before}{\newpage}
  \AddToHook{cmd/section/before}{\newpage}
  \AddToHook{shipout/lastpage}{\label{zslide-last-page}}
  \AddToHook{cmd/tableofcontents/before}{\renewcommand{\contentsname}{Outline}}
  \AddToHook{shipout/after}{\bool_gset_false:N \g_new_sec_bool}
  \AddToHook{shipout/background}{\ifnum\thepage=0\else
    % header
    \_zslide_status_bar:nnnn {(0, -1.25em)}{zslideMain}{.5}{1.25em}
    \_zslide_status_bar:nnnn {(.5\paperwidth, -1.25em)}{zslideII}{.5}{1.25em}
    \bool_if:NT \g_new_sec_bool {\_zslide_status_bar:nnnn {(0, -3.25em)}{zslideIII}{1}{2em}}
    % foot
    \_zslide_status_bar:nnnn {(0, -\paperheight)}{zslideMain}{.33}{1.25em}
    \_zslide_status_bar:nnnn {(.33\paperwidth, -\paperheight)}{zslideIII}{.34}{1.25em}
    \_zslide_status_bar:nnnn {(.67\paperwidth, -\paperheight)}{zslideII}{.33}{1.25em}
  \fi}
  % metadata setup
  \AddToHook{cmd/maketitle/before}{
    \let\zslideTitle\@title
    \let\zslideAuthor\@author
    \let\zslideDate\@date
  }
  \zlatex_define:nn {slide-metadata}{
    UL    .tl_set:N  =  \l__zlatex_slide_UL_tl,
    UL    .initial:n =  { \textcolor{zslideII}{Section\ \thesection} }, 
    UR    .tl_set:N  =  \l__zlatex_slide_UR_tl,
    UR    .initial:n =  { \textcolor{zslideMain}{Subsection\ \thesubsection} },
    BL    .tl_set:N  =  \l__zlatex_slide_BL_tl,
    BL    .initial:n =  { \textcolor{zslideII}{\zslideAuthor} },
    BC    .tl_set:N  =  \l__zlatex_slide_BC_tl,
    BC    .initial:n =  { \textcolor{zslideMain}{\zslideTitle} },
    BR    .tl_set:N  =  \l__zlatex_slide_BR_tl,
    BR    .initial:n =  { \textcolor{zslideMain}{\zslideDate\quad \thepage/\pageref{zslide-last-page}} }
  }
  \cs_new_protected:Npn \_slide_metadate:n #1 {
    \tl_use:c {l__zlatex_slide_#1_tl}
  }
  \NewDocumentCommand{\zslideMetadataSetup}{m}{
    \zlatex_set:nn {slide-metadata}{#1}
  }
  \cs_new_protected:Npn \_zslide_status_bar:nnnn #1#2#3#4 {
    \put#1 {\textcolor{#2}{\rule{#3\paperwidth}{#4}}}
  }
  % init slide option
  \zlatex_opt_set_last:n {\pagestyle{fancy}}
  \renewcommand{\familydefault}{\sfdefault}
}{\newcommand\zslideMetadataSetup[1]{}}
\end{minted}

\subsection{fancyhdr}
\begin{minted}[linenos=true]{latex}
% fancy page settings
% fancy page settings
\cs_if_exist_use:NF \theauthor{
  \newcommand{\theauthor}{}
  \newcommand{\thetitle}{}
  \newcommand{\thedate}{}
}

\fancypagestyle{fancy}{
  \fancyhf{}  
  \dim_gset:Nn \headheight{15pt}
  \renewcommand{\headrule}{\hrule width\textwidth}
  % slide mode
  \bool_if:NTF \g__zlatex_slide_bool {
    \def\headrule{}
    \fancyhead[L]{\hbox to .42\paperwidth{\hss\sffamily\textcolor{white}{Section \ \thesection}\;}}
    \fancyhead[R]{\hbox to .42\paperwidth{\;\sffamily\textcolor{slideRed}{\leftmark}\hss}}
    \fancyfoot[L]{\sffamily\textcolor{white}{\theauthor}}
    \fancyfoot[C]{\sffamily\textcolor{slideRed}{\thetitle}}
    \fancyfoot[R]{\sffamily\textcolor{slideRed}{\hbox to .25\paperwidth{\hfill\thedate\kern2.5em\thepage/\pageref{LastPage}}}}
  }{
  % doc mode
    \if@twoside
      \fancyhead[EL]{\leftmark}
      \fancyhead[ER]{\thepage}
      \fancyhead[OL]{\thepage}
      \fancyhead[OR]{\rightmark}
    \else
      \IfClassLoadedTF{book}{
        \fancyhead[L]{\thepage}
        \fancyhead[R]{\rightmark}
      }{
        \fancyhead[L]{\thepage}
        \fancyhead[R]{\leftmark}
      }
    \fi
  }
}

% front and main matter cmds in book class
\IfClassLoadedTF{book}{
    \renewcommand\frontmatter{
        \cleardoublepage
        \pagestyle{plain}
        \@mainmatterfalse
        \pagenumbering{Roman}
    }
    \renewcommand\mainmatter{
        \cleardoublepage
        \pagestyle{fancy}
        \@mainmattertrue
        \pagenumbering{arabic}
    }
}{\relax}
\end{minted}


\section{math}
\subsection{math env theme}
\begin{minted}[linenos=true]{latex}
% color spec for zlatex
\zlatex_define:nn {color}{
    % structure color
    link            .tl_set:N     =  \l__zlatex_link_color_tl,
    link            .initial:n    =  { purple },
    cite            .tl_set:N     =  \l__zlatex_cite_color_tl,
    cite            .initial:n    =  { teal },
    url             .tl_set:N     =  \l__zlatex_url_color_tl,
    url             .initial:n    =  { RoyalRed  },
    chapter         .tl_set:N     =  \l__zlatex_chapter_color_tl,
    chapter         .initial:n    =  { RoyalRed },  
    chapter-rule    .tl_set:N     =  \l__zlatex_chapter_rule_color_tl,
    chapter-rule    .initial:n    =  { black },
    % math envs      color
    axiom           .tl_set:N     =  \l__zlatex_axiom_color_tl,
    axiom           .initial:n    =  { mathaxiomColor },
    definition      .tl_set:N     =  \l__zlatex_definition_color_tl,
    definition      .initial:n    =  { mathdefinitionColor },
    theorem         .tl_set:N     =  \l__zlatex_theorem_color_tl,
    theorem         .initial:n    =  { maththeoremColor },
    lemma           .tl_set:N     =  \l__zlatex_lemma_color_tl,
    lemma           .initial:n    =  { mathlemmaColor },
    corollary       .tl_set:N     =  \l__zlatex_corollary_color_tl,
    corollary       .initial:n    =  { mathcorollaryColor },
    proposition     .tl_set:N     =  \l__zlatex_proposition_color_tl,
    proposition     .initial:n    =  { mathpropositionColor },
    remark          .tl_set:N     =  \l__zlatex_remark_color_tl,
    remark          .initial:n    =  { mathremarkColor },
}
\NewDocumentCommand{\zlatexColorSetup}{m}{
    \zlatex_set:nn {color}{#1}
    \bool_if:NTF \g__zlatex_hyperref_bool {
      \hypersetup{
        colorlinks = true,
        urlcolor   = \tl_use:N \l__zlatex_url_color_tl,
        linkcolor  = \tl_use:N \l__zlatex_link_color_tl,
        citecolor  = \tl_use:N \l__zlatex_cite_color_tl,
      }
    }{\relax}
}
\zlatexColorSetup{link=purple, cite=teal, url=RoyalRed}
\end{minted}

\subsection{math environment}
\begin{minted}[linenos=true]{latex}
% framed env for user interface
\cs_new_protected:Npn \zlatexFramed:nn #1#2 {
    \DeclareDocumentEnvironment{#1}{O{#2}}{
        \def\FrameCommand{{\color{##1}\vrule width 3pt}\colorbox{##1!10}}
        \MakeFramed{\advance\hsize-\width\FrameRestore}\noindent   
    }{\endMakeFramed}
}
\NewDocumentCommand\zlatexFramed{mO{black}}{
    \zlatexFramed:nn {#1}{#2}
}

% theorem/proof-like envs list 
\clist_gset:Nn \g__zlatex_theoremlike_envs_clist  { 
    axiom, definition, theorem, 
    lemma, corollary,  proposition, remark 
}
\clist_gset:Nn \g__zlatex_prooflike_envs_clist  { 
    proof,    exercise, example, 
    solution, problem,  
}

% math envs' name accrodingt to language
\msg_new:nnn {zlatex}{mathenv-name}{Current~math~env~name~is~:~'#1'~,which~is-invalid.}
\str_case:VnTF \g__zlatex_lang_str { 
    {en}{
        \zlatex_define:nn {math-env}{
            math-env                .multichoice:,
            math-env / axiom        .code:n = { \tl_gset:cn {zlatex#1Name}{Axiom} },
            math-env / definition   .code:n = { \tl_gset:cn {zlatex#1Name}{Definition} },
            math-env / theorem      .code:n = { \tl_gset:cn {zlatex#1Name}{Theorem} },
            math-env / lemma        .code:n = { \tl_gset:cn {zlatex#1Name}{Lemma} },
            math-env / corollary    .code:n = { \tl_gset:cn {zlatex#1Name}{Corollary} },
            math-env / proposition  .code:n = { \tl_gset:cn {zlatex#1Name}{Proposition} },
            math-env / remark       .code:n = { \tl_gset:cn {zlatex#1Name}{Remark} },
            math-env / proof        .code:n = { \tl_gset:cn {zlatex#1Name}{Proof} },
            math-env / exercise     .code:n = { \tl_gset:cn {zlatex#1Name}{Exercise} },
            math-env / example      .code:n = { \tl_gset:cn {zlatex#1Name}{Example} },
            math-env / solution     .code:n = { \tl_gset:cn {zlatex#1Name}{Solution} },
            math-env / problem      .code:n = { \tl_gset:cn {zlatex#1Name}{Problem} },
            math-enc / unknown      .code:n = {
                \msg_error:nn {zlatex}{mathenv-name}
            },
        }
    }
    {cn}{
        \zlatex_define:nn {math-env}{
            math-env                .multichoice:,
            math-env / axiom        .code:n = { \tl_gset:cn {zlatex#1Name}{公理} },
            math-env / definition   .code:n = { \tl_gset:cn {zlatex#1Name}{定义} },
            math-env / theorem      .code:n = { \tl_gset:cn {zlatex#1Name}{定理} },
            math-env / lemma        .code:n = { \tl_gset:cn {zlatex#1Name}{引理} },
            math-env / corollary    .code:n = { \tl_gset:cn {zlatex#1Name}{推论} },
            math-env / proposition  .code:n = { \tl_gset:cn {zlatex#1Name}{命题} },
            math-env / remark       .code:n = { \tl_gset:cn {zlatex#1Name}{注记} },
            math-env / proof        .code:n = { \tl_gset:cn {zlatex#1Name}{证明} },
            math-env / exercise     .code:n = { \tl_gset:cn {zlatex#1Name}{练习} },
            math-env / example      .code:n = { \tl_gset:cn {zlatex#1Name}{示例} },
            math-env / solution     .code:n = { \tl_gset:cn {zlatex#1Name}{解} },
            math-env / problem      .code:n = { \tl_gset:cn {zlatex#1Name}{问题} },
            math-enc / unknown      .code:n = {
                \msg_error:nn {zlatex}{mathenv-name}
            },
        }
    }
}{\zlatex_set:nn {math-env}{math-env={axiom, definition, theorem, lemma, corollary, proposition, remark, proof, exercise, example, solution, problem}}}
{\msg_error:nn {zlatex}{mathenv-lang}}

% math env's style
\newtheoremstyle{zlatexMathEnv}
    {2pt}{2pt}{}
    {0pt}{\bfseries}{}
    {.25em}{\thmname{#1}~ \thmnumber{#2}~ \thmnote{(#3)}}
\theoremstyle{zlatexMathEnv}

% theorem-like env declaration
\msg_new:nnn {zlatex}{mathenv-type}{Current~math~env~is~:~'\str_use:N \g__zlatex_math_env_style_tl'~,only~'plain',~'leftbar',~'background',~'fancy',~'shadow',~'paris'~types~are-valid.}
\NewDocumentCommand{\zlatexMathEnvStyle}{m}{
  \tl_gset:Nn \g__zlatex_math_env_style_tl {#1}
}
\seq_new:N \l_tcb_flag_seq
\seq_set_from_clist:Nn \l_tcb_flag_seq {shadow, paris}
\seq_if_in:NVT \l_tcb_flag_seq \g__zlatex_math_env_style_tl {
  \@ifpackageloaded{tcolorbox}{\relax}{
    \RequirePackage[many]{tcolorbox}
  }
}
% Hook for Math Env Style
\NewDocumentCommand{\zlatexMathEnvStyleNew}{mm}{
  \AddToHook{zlatex/math/envstyle/begin}{#1}
  \AddToHook{zlatex/math/envstyle/end}{#2}
  \ActivateGenericHook{zlatex/math/envstyle/begin}
  \ActivateGenericHook{zlatex/math/envstyle/end}
  \tl_gset:Nn \g__zlatex_math_env_style_tl {HOOK}
}
\DeclareDocumentEnvironment{zlatexTheoremLikeFrame}{O{axiom}}{
  \def\TempColor{\tl_use:c {l__zlatex_#1_color_tl}}
  \str_case:VnF \g__zlatex_math_env_style_tl {
    {plain}{  
      \def\FrameCommand{}
      \MakeFramed {\advance\hsize-\width \FrameRestore}
    }
    {leftbar}{
      \def\FrameCommand{{\color{\TempColor}\vrule width 3pt}\hspace{5pt}}
      \MakeFramed {\advance\hsize-\width \FrameRestore}
    }
    {background}{
      \def\FrameCommand{\colorbox{\TempColor!10}}
      \MakeFramed {\advance\hsize-\width \FrameRestore}
    }
    {fancy}{
      \def\FrameCommand{{\color{\TempColor}\vrule width 3pt}\colorbox{\TempColor!10}}
      \MakeFramed{\advance\hsize-\width \FrameRestore}
    }
    {shadow}{
      \begin{tcolorbox}[
        enhanced~ jigsaw, breakable,
        top=1.5pt,  bottom=1.5pt,
        left=3pt,   right=3pt,
        boxrule=0pt, sharp~ corners,
        drop~ fuzzy~ shadow,
        colback={\TempColor!10}, 
        borderline~ west={3pt}{0pt}{\TempColor}
      ]
    }
    {paris}{
      \begin{tcolorbox}[
        enhanced,   breakable,
        top=1.5pt,  bottom=1.5pt,
        left=3pt,   right=3pt,
        boxrule=0pt,    sharp~ corners,
        colback=gray!5, drop~ fuzzy~ shadow,
        overlay={
          \draw[\TempColor, line~ width=0.2pt] (frame.north~ west)--(frame.north~ east);
          \draw[\TempColor, line~ width=3pt] ([yshift=1.5pt]frame.north~ west) -- +(2.5cm, 0);
        }
      ]  
    }
    {HOOK}{\UseHook{zlatex/math/envstyle/begin}}
  }{\msg_error:nn {zlatex}{mathenv-type}}
}{
  \str_case:VnF \g__zlatex_math_env_style_tl {
    {HOOK}{\UseHook{zlatex/math/envstyle/end}}
    {paris}{\end{tcolorbox}}
    {shadow}{\end{tcolorbox}}
    {plain}{\endMakeFramed}
    {leftbar}{\endMakeFramed}
    {background}{\endMakeFramed}
    {fancy}{\endMakeFramed}
  }{\msg_error:nn {zlatex}{mathenv-type}}
}

% loop to create math env, setup \cref
\clist_map_inline:Nn \g__zlatex_theoremlike_envs_clist {
    % theorem create
    \newtheorem{zlatex#1}{\tl_use:c {zlatex#1Name}}[section]

    % env create (3 types: 'leftbar', 'none' and 'backgroud')
    \NewDocumentEnvironment{#1}{O{}}{
        \begin{zlatexTheoremLikeFrame}[\tl_use:c {l__zlatex_#1_color_tl}]
        \begin{zlatex#1}[##1]
    }{\end{zlatex#1}\end{zlatexTheoremLikeFrame}}

    % cref settings
    \cs_generate_variant:Nn \exp_args:Nnnx {Nxxx}
    \str_case:VnF \g__zlatex_lang_str {
        {en}{
            \crefname{zlatex#1}{#1}{#1s}
            \creflabelformat{zlatex#1}{##2(##1)##3}
        }
        {cn}{
            \exp_args:Nxxx \crefname{zlatex#1}{\tl_use:c {zlatex#1Name}}{\tl_use:c {zlatex#1Name}}
            \creflabelformat{zlatex#1}{##2(##1)##3}
        }
    }{\msg_error:nn {zlatex}{option-lang}}
}

% proof-like env decalration
\NewDocumentEnvironment{zlatexProofLikeFrame}{O{}}{
    \def\FrameCommand{}
    \MakeFramed {\advance\hsize-\width \FrameRestore}
}{\endMakeFramed}
\renewcommand{\qedsymbol}{\ensuremath{\blacksquare}}
\clist_map_inline:Nn \g__zlatex_prooflike_envs_clist{
    \DeclareDocumentEnvironment{#1}{O{}}{
        \begin{zlatexProofLikeFrame}[]
        {\noindent{\bfseries\tl_use:c {zlatex#1Name}:}}
        \tl_set:Nn \l__arg_one_tl {#1}
    }{\str_if_eq:VnTF \l__arg_one_tl{proof}{\hfill\qedsymbol\par}{\par}\end{zlatexProofLikeFrame}}
}
\end{minted}

\subsection{Math alias}
\begin{minted}[linenos=true]{latex}
% math related cmds alias
\bool_if:NTF \g__zlatex_math_alias_bool{
    \RequirePackage{amssymb, mathtools}
    \RequirePackage{bm}          
    % Math Font 
    \newcommand{\dd}{\mathrm{d}}
    \newcommand{\C}[1]{\ensuremath{\mathcal{#1}}}
    \let\ss\S
    \renewcommand{\S}[1]{\ensuremath{\mathscr{#1}}}
    \newcommand{\B}[1]{\ensuremath{\mathbb{#1}}}
    \newcommand{\FF}[1]{\ensuremath{\mathbf{#1}}}
    \newcommand{\F}[1]{\ensuremath{\bm{#1}}}
    \newcommand{\R}[1]{\ensuremath{\mathrm{#1}}}
    \newcommand{\K}[1]{\ensuremath{\mathfrak{#1}}}
    % Math Arrow 
    \newcommand{\lr}{\ensuremath{\longrightarrow}}
    \let\LL\ll
    \renewcommand{\ll}{\ensuremath{\longleftarrow}}
    \newcommand{\equ}{\ensuremath{\Longleftrightarrow}\,}
    \newcommand{\sr}{\ensuremath{\longmapsto}}
    \newcommand{\lrr}[2][]{\ensuremath{\xRightarrow[#1]{#2}}}
    \renewcommand{\lll}[2][]{\ensuremath{\xLeftarrow[#1]{#2}}}
    \newcommand{\ns}{\ensuremath{\varnothing}}
    \newcommand{\A}{\ensuremath{\forall}}
    % Math Operator
    \newcommand{\alt}{\ensuremath{\mathrm{Alt}\;}}
    \newcommand{\sgn}{\ensuremath{\mathrm{sgn}\;}}
    \newcommand{\curl}{\ensuremath{\mathrm{curl}\;}}
    \newcommand{\grad}{\ensuremath{\mathrm{grad}\;}}
    \newcommand{\trace}{\ensuremath{\mathrm{trace}\;}}
    \renewcommand{\div}{\ensuremath{\mathrm{div}\;}}
}{}
\end{minted}

\section{structure Style}
\subsection{hyperref/index and bib}
\begin{minted}[linenos=true]{latex}
\str_case:VnF \g__zlatex_bib_backend_str {
    {bibtex}{\RequirePackage[backend=bibtex]{biblatex}}
    {biber}{\RequirePackage[backend=biber]{biblatex}}
}{\relax}
\exp_args:Nx \addbibresource{\str_use:N \g__zlatex_bib_source_str}
\RequirePackage{indextools}
\RequirePackage{hyperref}
\end{minted}

\subsection{cref name}
\begin{minted}[linenos=true]{latex}
\RequirePackage{indextools}
\RequirePackage{hyperref}
\RequirePackage[nameinlink]{cleveref}

% figure and table prefix for \cref 
\str_case:VnF \g__zlatex_lang_str {
    {en}{
        \IfClassLoadedTF{book}{
            \crefname{part}{part}{parts}
            \crefname{chapter}{chapter}{chapters}
        }{\relax}
        \crefname{section}{section}{sections}
        \crefname{subsection}{subsection}{subsections}
        \crefname{figure}{figure}{figures}
        \crefname{table}{table}{tables}
        \crefname{equation}{equation}{equations}
    }
    {cn}{
        \IfClassLoadedTF{book}{
            \crefname{part}{部分}{部分}
            \crefname{chapter}{章}{章}
        }{\relax}
        \crefname{section}{节}{节}
        \crefname{subsection}{小节}{小节}
        \crefname{figure}{图}{图}
        \crefname{table}{表}{表}
        \crefname{equation}{方程}{方程}
    }
}{\msg_error:nn {zlatex}{option-language}}    
\end{minted}

\subsection{contents}
\begin{minted}[linenos=true]{latex}
% partial ToC
\RequirePackage{titletoc}
\NewDocumentCommand\stopPartToc{mm}{
  \int_compare:nNnTF {#2}<{1}
    {\relax}{\stopcontents[#1]}
}
\NewDocumentCommand{\partialToc}{O{2}}{
  % depth setup
  \setcounter{tocdepth}{#1} 
  \titlecontents{psection}[2em]
    {} {\contentslabel{2em}} {} {\titlerule*[1pc]{.}\contentspage}
  \titlecontents{psubsection}[4.5em]
    {} {\contentslabel{2.5em}} {} {\titlerule*[1pc]{.}\contentspage}
  \titlecontents{psubsubsection}[5.5em]
    {} {\contentslabel{0em}} {} {\titlerule*[1pc]{.}\contentspage}
  % print pToc
  \IfClassLoadedTF{book}{ 
    \startcontents[chapters]
    \printcontents[chapters]{p}{1}{}
    \pretocmd{\chapter}{
      \stopPartToc{chapters}{\thechapter}
    }{}{}
  }{
    \startcontents[sections]
    \printcontents[sections]{p}{1}{}
    \pretocmd{\section}{
      \stopPartToc{sections}{\thesection}
    }{}{}
  }
}
\end{minted}

\subsection{Chapter and section}
\begin{minted}[linenos=true]{latex}
% chapter head style
\IfClassLoadedTF{book}{
    \RequirePackage{titlesec}
    \titleformat{\chapter}[display]
        {\bfseries\huge\color{black}}
        {\flushright\Large\color{\tl_use:N \l__zlatex_chapter_color_tl}
        \MakeUppercase{\chaptertitlename}\hspace{1ex}
        {\scalebox{4}{\thechapter}}}
        {10pt}
        {\color{\tl_use:N \l__zlatex_chapter_rule_color_tl}\titlerule\vspace{1ex}}
    % chapter space
    \titlespacing{\chapter}{0pt}{-30pt}{40pt}
    \titleformat{name=\chapter, numberless}
        {\normalfont\bfseries\Huge}
        {}{0pt}{}
}{\relax}
\end{minted}


\subsection{fancy chapter}
\begin{minted}[linenos=true]{latex}
% chapter head style
\newif\ifFancyChapter
\bool_if:NTF \g__zlatex_font_config_bool {
  \RequirePackage{fontspec}
  \newfontfamily{\Cinzel}{CinzelRegular.ttf}[
    BoldFont=CinzelBold.ttf,
    ItalicFont=SabonItalic.ttf
  ]
}{\def\Cinzel{\relax}}
\IfClassLoadedTF{book}{
  \newif\ifFancyChapter
  \bool_if:NTF \g__zlatex_fancy_bool {
    \RequirePackage[explicit]{titlesec}
    \FancyChaptertrue
    % numbered chapter format
    \makeatletter
    \titleformat{\chapter}[display]
    {\normalfont\huge\sffamily}{}
    {20pt}{
    \begin{tikzpicture}[overlay, remember~ picture]%
      % mark nodes (need 'calc' library)
      \coordinate (A) at ($(current~ page.north~ west)+(.125\paperwidth, 0pt)$);
      \coordinate (stripES) at ($(A)+(3em, -.25\paperheight)$);
      % chapter head
      \fill[zchapColor] (A) rectangle (stripES);
      \draw[draw=zchapColor] (stripES)++(.25em, 4em)   -- ++(.75\paperwidth-3.25em, 0pt);
      \draw[draw=zchapColor] (stripES)++(.25em, 1.5pt) -- ++(.75\paperwidth-3.25em, 0pt);
      \draw[draw=zchapColor] (stripES)++(.25em, 0em)   -- ++(.75\paperwidth-3.25em, 0pt);
      % chapter title and index
      \node[anchor=south, color=white] at ($(stripES)+(-1.5em, 0em)$)
        {\normalfont\normalsize\scalebox{4}{\thechapter}\Suffixnum{\thechapter}};
      \node[anchor=south~ west, inner~ sep=0pt, 
            yshift=4.25em, xshift=.25em, 
            font=\Large\bfseries, color=zchapColor
        ] at (stripES) {\z@subtitle};
      \node[anchor=south~ west, inner~ sep=0pt,
            yshift=1.5em, xshift=.25em,
            font=\Cinzel\Huge\bfseries, color=zchapColor
        ] at (stripES) {#1};
      % parbox insert
      \node[anchor=north~ west, inner~ sep=0pt] at ($(stripES)+(-3em, -1em)$){
        \parbox[t]{.3\paperwidth}{\normalfont\fontsize{10pt}{15pt}
          \selectfont\Cinzel\itshape\z@leftContent}
      };
      \node[anchor=north~ west, inner~ sep=0pt] at ($(stripES)+(-3em+.45em+.3\paperwidth, -1em)$){
        \parbox[t]{\dimeval{.45\paperwidth-.45em}}{
          \normalfont\fontsize{10pt}{15pt}\selectfont\z@rightContent}
      };
      % saying block
      \coordinate (sayingWN) at ($(current~ page.south~ west)+(0, .3\paperheight)$); 
      \shade[top~ color=white, bottom~ color=zchapColor!25] (sayingWN) 
        rectangle ++(1\paperwidth, 5pt);
      \shade[top~ color=zchapColor!25, bottom~ color=white] ($(sayingWN)+(0em, -.15\paperheight)$) 
        rectangle ++(1\paperwidth, -5pt);
      \node at ($(sayingWN)+(.5\paperwidth, -0.075\paperheight)$) {
        \parbox[t][][r]{.75\paperwidth}{\normalfont\fontsize{15pt}{22.5pt}\selectfont
          \MakeUppercase{\Cinzel\z@saying\\\hspace*{\fill}{\itshape\normalsize\z@sayauthor}}}
      };
    \end{tikzpicture}}[\thispagestyle{empty}\clearpage]
    \makeatother
    % unnumbered chapter format
    \titleformat{name=\chapter, numberless}
      {\normalfont\bfseries\Huge}
      {}{0pt}{#1}
  }{
    \RequirePackage{titlesec}
    \FancyChapterfalse
    % numbered chapter format
    \titleformat{\chapter}[display]
      {\bfseries\huge\color{black}}
      {\flushright\Large\color{\tl_use:N \l__zlatex_chapter_color_tl}
      \MakeUppercase{\chaptertitlename}\hspace{1ex}
      {\scalebox{4}{\thechapter}}}
      {10pt}
      {\color{\tl_use:N \l__zlatex_chapter_rule_color_tl}\titlerule\vspace{1ex}}
    % unnumbered chapter format
    \titleformat{name=\chapter, numberless}
      {\normalfont\bfseries\Huge}
      {}{0pt}{}
  }
  % chapter space
  \titlespacing{\chapter}{0pt}{-30pt}{40pt}
}{\relax}

% find number suffix: 1 -> st, 2 -> nd, ...
\prop_new:N \g_arabix_suffix_prop
\prop_set_from_keyval:Nn \g_arabix_suffix_prop {
  1=st, 2=nd, 3=rd, 11=th, 12=th, 13=th, 0=th, _=th
} 
\NewDocumentCommand\numSuffix{m}{
  \int_compare:nTF {11 <= #1 <= 13}
    {\prop_item:Ne \g_arabix_suffix_prop {#1}}
    {\int_compare:nTF {\int_mod:nn {#1}{10} > 3}
      {\prop_item:Ne \g_arabix_suffix_prop {_}}
      {\prop_item:Ne \g_arabix_suffix_prop {\int_mod:nn {#1}{10}}}
    }
}

% fancy chapter default material
\ExplSyntaxOff
\makeatletter
\ifFancyChapter
  % default settings
  \newcommand{\z@subtitle}{Subtitle}
  \newcommand{\z@saying}{SAYING}
  \newcommand{\z@sayauthor}{-- Author}
  \newcommand{\z@rightContent}{Right Content}
  \newcommand{\z@leftContent}{\includegraphics[width=1\linewidth]{example-image-duck}\\[.5em]Figure Description}
  % users' interface
  \NewDocumentCommand{\zsubtitle}{m}{\renewcommand\z@subtitle{#1}}
  \NewDocumentCommand{\zchapterSaying}{O{}m}{\renewcommand\z@saying{#2}\renewcommand\z@sayauthor{#1}}
  \NewDocumentCommand{\zchapterLContent}{m}{\renewcommand\z@leftContent{#1}}
  \NewDocumentCommand{\zchapterRContent}{m}{\renewcommand\z@rightContent{#1}}
\else 
  \NewDocumentCommand{\zsubtitle}{m}{\relax}
  \NewDocumentCommand{\zchapterSaying}{O{}m}{\relax}
  \NewDocumentCommand{\zchapterLContent}{m}{\relax}
  \NewDocumentCommand{\zchapterRContent}{m}{\relax}
\fi
\makeatother
\ExplSyntaxOn
\end{minted}

\subsection{maketitle}
\begin{minted}[linenos=true]{latex}
\makeatletter
\bool_if:NTF \g__zlatex_slide_bool {
  \renewcommand\maketitle{
    \null\vfill\begin{center}
      \begin{tabular}{c}{\huge\zslideTitle}\\[2em]
      \zslideAuthor\\[2em]\zslideDate\end{tabular}
      \vspace{\stretch{2}}\end{center}
    \thispagestyle{empty}\setcounter{page}{0}
    \newpage
  }
}{
  \renewcommand{\maketitle}{
    \bool_if:NT \g__zlatex_hyperref_bool {\hypersetup{pageanchor=false}}
    \newgeometry{margin=1.5in}
      \thispagestyle{empty}
      \begin{center}\vfill\vspace*{40pt}
      \rule{6pt}{77.5pt}
      \begin{tabular}[b]{l}{\huge\bfseries\@title}\\[5em]{\Large\bfseries\@author}\end{tabular}
      \par\vfill{\Large\textcolor{gray}{\@date}}
      \end{center}
    \restoregeometry
    \bool_if:NT \g__zlatex_hyperref_bool {\hypersetup{pageanchor=true}}
  }
}
\makeatother
\end{minted}