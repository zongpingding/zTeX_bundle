% ==> change zslide title box to rounded rectangle
\InputIfFileExists{zlatex-cfg.tex}{}{}
\documentclass[
  layout={slide, aspect=16|9},
]{../code/ztex}

\title{Rounded corner style Title Page}
\author{Eureka\quad and \quad \ztex{} Eureka}
\date{\today}
\begin{document}
\maketitle

\section{FIRST}
The FIRST section.
\end{document}









% ==> ztex partial toc test
\InputIfFileExists{zlatex-cfg.tex}{}{}
\documentclass[
  % class=l3doc,
  class=book,
]{../code/ztex}



\begin{document}
\ztexptoc[1]
% \startcontents[sections]
% \printcontents[sections]{}{1}{}

\dotfill\par
%%% book
\chapter{A-1}
\ztexptoc[2]

\section{A-1-1}
\subsection{A-1-1-1}
\subsection{A-1-1-2}
\section{A-1-2}

\chapter{A-2}
\section{A-2-1}
\section{A-2-2}
\chapter{A-3}


%%% article
% \section{A-1}
% \subsection{A-1-1}
% \subsection{A-1-2}
% \subsubsection{A-1-2-1}
% \subsubsection{A-1-2-2}
% \subsection{A-1-3}

% \section{A-2}
% \section{A-3}
\end{document}




% ==> add dashed line feature
\InputIfFileExists{zlatex-cfg.tex}{}{}
\documentclass[
  % layout=margin
]{../code/ztex}
\usepackage{lipsum}
\setlength{\fboxsep}{0pt}


\begin{document}
\setlength{\unitlength}{2mm}
\begin{picture}(27,16)(-1,-1)
\path(-1,-1)(-1,15)(26,15)(26,-1)(-1,-1)
% \linethickness{2pt}
\drawline[-10](0,0)(10,7)(5,11)
(0,7)(10,0)(10,7)(0,7)(0,0)(10,0)
% \linethickness{3pt} \roundcap \beveljoin
% \polyline(15.1,0.1)(25,7)(20,11)
% (15,7)(25,0)(25,7)(15,7)(15,0)(25,0)
\end{picture}
\end{document}




\InputIfFileExists{zlatex-cfg.tex}{}{}
\documentclass[
  % layout=margin
]{../code/ztex}
\usepackage{lipsum}
\setlength{\fboxsep}{0pt}



\begin{document}
\section{Internal commands}
\vskip1cm
\def\poly#1#2{\put(0, 0){\polygon #1#2}}
\makeatletter\ExplSyntaxOn
\begin{zpic}[unit=2em]
  % \poly {}{(1, 2)(2, 3)(4, 5)(4, 4)}
  % \poly {*}{(1, 2)(2, 3)(4, 5)(4, 4)}
  % \put (0, 0) {\polygon (0, 0)(1, 1)}
  % \put (2, 0) {\__@@_pic_polygon:nn {}{(0, 0)(1, 1)}} % works
  % \put (2, 0) {\__@@_pic_polyline:n {(0, 0)(1, 1)}}   % works
  % \put (2, 0) {\__@@_pic_polyvector:n {(0, 0)(1, 1)}}   % works
  % \zdraw[shift={2, 0}] (0, 0)(1, 1); % works
  % \zdraw[shift={2, 0}, vector, cycle] (0, 0)(1, 1)(2, 1); % works
  % \zdraw[shift={4, 0}, vector, fill] (0, 0)(1, 1)(2, 1); % works
  % \zdraw[shift={6, 0}, vector, cycle, fill] (0, 0)(1, 1)(2, 1); % works
  % \zdraw[shift={4, 0}, cycle, fill] (0, 0)(1, 1)(2, 1); % works
  % \zdraw[fill, cycle] (0, 0)(1, 0)(1, 1)(0, 1);
\end{zpic}
\ExplSyntaxOff\makeatother


\section{zpic env}
\lipsum[1]

XXX\rule{2cm}{2cm}%
\begin{zpic}[unit=2cm, xoffset=1]%[unit=.5cm, height=2, width=1, xoffset=1, yoffset=1]
  % 1. rectangle
  \zrectangle[fill=gray!20, arc=.1](0, 0)(2, 1)
  \zrectangle[draw=red, width=1pt](.5, .25)(1.5, .75)
  % 2. line / vecter
  \zline[width=3pt, draw=red](0, .5)(2, .5)
  \zvector[>=pst](0, 0)(1, 1)
  \zvector[draw=purple, width=2pt](1, 1)(2, 0)
  \put (0, 0){\vector(2, 1){2}}
  % 3. arc / circle
  \zarc[draw=blue, end=45](0, 0) % fill=<empty>
  \zarc[draw=blue, width=2pt, end=15, fill=, draw=red](0, 0)
  \zcircle[radius=.25, fill, draw=blue](1, .5)
  \zcircle[radius=.25, fill=orange, draw=none](1.5, 1)
  \zcircle[radius=.25, fill=red, draw=](2, .5)
  \put (2, 0){\circle{.5}}
  % 4. ovaL
  \put(1, 0){\oval(1, .5)[lb]}
\end{zpic}


\section{zdraw}
XXX{\color{teal}\rule{4em}{4em}}
\begin{zpic}[unit=2em]
  \zdraw (0, 0)(1, 0)(1, 1)(0, 1);
  \zdraw[] (0, 0)(1, 0)(1, 1)(0, 1);
  \zdraw[cycle, shift={2, 0}]  (0, 0)(1, 0)(1, 1)(0, 1);
  \zdraw[fill, shift={4, 0}]     (0, 0)(1, 0)(1, 1)(0, 1);
  \zdraw[draw=red, width=1pt, shift={6, 0}] (0, 0)(1, 0)(1, 1)(0, 1);
  \zdraw[vector, shift={8, 0}] (0, 0)(1, 0)(1, 1)(0, 1);
  \zdraw[vector, cycle, shift={10, 0}] (0, 0)    (1, 0)(1, 1)(0, 1);
  \zdraw[vector, fill, shift={12, 0}]    (0, 0)(1, 0)(1, 1)  (0, 1);
  \zdraw[vector, cycle, fill, shift={14, 0}] (0, 0)  (1, 0)(1, 1)(0, 1);
\end{zpic}


% \mbox{}\vskip6em
% XXX\rlap{\rule{1cm}{1cm}}\zrectangle[yoffset=1, xoffset=1, arc=.25]

% \mbox{}\vskip6em
% XXX\rlap{\zsquare[unit=.5cm, fill=black]}\zrectangle[unit=.5cm, yoffset=1, xoffset=1, arc=.125]

% \vskip1em
% XXX\zsquare[unit=1cm, fill=teal, width=.5, arc=.1]


% \section{vector}
% \setlength{\unitlength}{1cm}
% \dotfill\par
% \vskip4em
% XXX%
% \begin{picture}(0, 0)
%   \zvector(0, 0) (1, 1)
%   \zvector (1, 1) (2, 0)
% \end{picture}


% \def\P@radius{1}
% \def\P@width{10}
% \def\P@height{6}
% \begin{picture}(8,4)
%   \put(0,0){\vector(1,0){8}}  % x axis
%   \put(0,0){\vector(0,1){4}}  % y axis
%   \put(2,0){\line(0,1){3}}       % left side
%   \put(4,0){\line(0,1){3.5}}     % right side
%   \qbezier(2,3)(2.5,2.9)(3,3.25)
%     \qbezier(3,3.25)(3.5,3.6)(4,3.5)
%   \thicklines                 % below here, lines are twice as thick
%   \put(2,3){\line(4,1){2}}
%   \put(4.5,2.5){\framebox{Trapezoidal Rule}}
% \end{picture}
\end{document}







% ===> l3keys bug patch
\documentclass{article}

\begin{document}
\ExplSyntaxOn
\cs_set_protected:Npn \__keys_initialise:n #1
  {
    \exp_after:wN \__keys_find_key_module:wNN
      \l_keys_path_str \s__keys_stop
      \l_keys_key_tl \l_keys_key_str
    \tl_set_eq:NN \l_keys_key_tl \l_keys_key_str
    \tl_set:Nn \l_keys_value_tl {#1}
    \cs_if_exist:cTF { \c__keys_code_root_str \l_keys_path_str }
      {
        \str_clear:N \l__keys_inherit_str
        \__keys_execute:nn \l_keys_path_str {#1}
      }
      {
        \cs_if_exist:cT
          { \c__keys_inherit_root_str \__keys_parent:o \l_keys_path_str }
          { \__keys_execute_inherit: }
      }
  }

\keys_define:nn { a } {
  A .tl_set:N = \l__A_tl,
  A .code:n = {PARENT~`a'\par}
}
\keys_define:nn { aa } {
  AA .tl_set:N = \l__AA_tl,
  AA .code:n = {PARENT~`aa'\par}
}
\keys_define:nn { } {
  b .inherit:n = {a, aa},
  % b .inherit:n = a,  % ---> works
  % c .inherit:n = aa, % ---> works
  % b .inherit:n = aa, % ---> error
}
\keys_define:nn { b } {
  B .tl_set:N = \l__B_tl,
  B .default:n = {DEFAULT}, % --> works
  B .initial:n = {INITIAL}, % --> works
}
%% DOES NOT work EXAMPLE BEGIN
% \keys_define:nn { } {
%     b .inherit:n = aa,
% }
%% DOES NOT work EXAMPLE END
\keys_set:nn { b } { A = x, AA=y }
% \show \__keys_initialise:n
\ExplSyntaxOff

Hello
\end{document}





% ===> xetex primitive graphics
\documentclass{article}
\usepackage[scheme=plain]{ctex}
\makeatletter
\def\test{\XeTeXpicfile "logo.jpg"
  height \f@size pt}
\def\fSize{\f@size}
\makeatother
\setlength{\fboxsep}{0pt}

\begin{document}

\fbox{测}\fbox{试}:\raisebox{\dimexpr-\fontchardp\font`y}{\test}:
(depth: \hbox{\the\dimexpr\fontchardp\font`y})

\XeTeXuseglyphmetrics=0
\fbox{测}\fbox{试}:\raisebox{\XeTeXglyphbounds4\fontcharwd\font`你}{\fbox{\test}}:
\the\XeTeXglyphbounds4`节 (\fSize pt)\setbox0\hbox{测}\the\ht0 )
\end{document}




% affine transform for tikz
\InputIfFileExists{zlatex-cfg.tex}{}{}
\documentclass[
  layout=margin
]{../code/ztex}
\setlength{\fboxsep}{0pt}
\parindent0pt
\usepackage{lipsum}
\usepackage{graphicx}
\makeatletter
% \usepackage{newtxtext}


\begin{document}
\section{FIRST}
\lipsum[1][1-3]
\marginpar{
  \sffamily\small\lipsum[2][2-4]\vskip1em
  \ztoolboxaffine{\includegraphics[width=10em]{example-image-a}}{.5, 0, .25, .5}
}
\lipsum[1][1-3]


\lipsum[2]
\marginpar{
  \sffamily\small\lipsum[2][2-4]\vskip1em
  \centerline{
    \includegraphics[width=10em]{example-image-a}
  }
  \hb@xt@\marginparwidth{\hfill XXX \hfill}
  \label{fig:1}
}

\cref{fig:1}

\end{document}



% debug command: \ztoolboxaffine 
\InputIfFileExists{zlatex-cfg.tex}{}{}
\documentclass[
  % hyper, 
  % font={math=var-euler}
]{../code/ztex}
\setlength{\fboxsep}{0pt}

\begin{document}
% \fbox{XXX}\pdfsave\pdfsetmatrix{1 0 1 1}\rlap{XXX}\pdfrestore\par

% \fbox{XXX}\fbox{\ztoolboxaffine[debug]{XXX}{1, 0, 1, 1}}\par

% \ExplSyntaxOn\hbox_set:Nn \l_tmpa_box {XXX} \box_rotate:Nn \l_tmpa_box {45}\fbox{\box_use:N \l_tmpa_box}\ExplSyntaxOff


\fbox{XXX}\pdfsave\pdfsetmatrix{1 .5 .5 1}\rlap{\rule{1em}{1em}}\pdfrestore\par

XXX\ztoolboxaffine[]{\rule{1em}{1em}}{1, .5, .5, 1}XXX
\end{document}



% clean todo
\InputIfFileExists{zlatex-cfg.tex}{}{}
\documentclass[hyper, font={math=var-euler}]{../code/ztex}
\zthmstyle{obsidian}
\ztexloadlib{thm}
\zthmcnt{share}
\parindent0pt
% \zthmiconset{
%   theorem=$>$
% }
\makeatletter
\def\seeTargets{%
  \par\noindent\dotfill\\
  \string\@currentHref\ = \meaning\@currentHref\\
  \string\@currentHpage\ = \meaning\@currentHpage\\
  \string\Hy@currentbookmarklevel\ = \meaning\Hy@currentbookmarklevel\\
  \vskip3em
}
\makeatother

\begin{document}
\section{Test}
Hello world.
\zthmiconuse{theorem}
\zthmiconuse{lemma}
\seeTargets

\dotfill\par
\zthmtoc
\seeTargets

\vskip5em
\begin{remark}[Pythagoras]
  A simple theorem.
  \begin{align}
    \sum_{i=1}^{+\infty}{\int_{0}^{i}-\frac{1}{t}\mathrm{d}t} = \frac{\pi^2}{6}
  \end{align}
\end{remark}

\begin{lemma}[Pythagoras]
  A simple theorem.
\end{lemma}
\seeTargets

\newpage
\mbox{}\vskip10em
New added theorem
\zthmtocadd[subsection]{name=New:Added Thm ITEM}
\seeTargets

\newpage
XXX
\seeTargets
\end{document}




% ztex box shear transform
\documentclass{article}
\makeatletter\ExplSyntaxOn
% REF:
% 1. https://math.stackexchange.com/a/3521141/1235323
% 2. https://math.stackexchange.com/a/281087/1235323
\cs_new:Npn \__fp_to_rad:n #1
  { \fp_eval:n {#1/pi*180} }
\cs_new:Npn \__matrix_det:nnnn #1#2#3#4
  {
    \fp_eval:n { #1*#4 - #2*#3 }
  }
% (translation) + x-scale + y-scale + rotate
\coffin_new:N \l__affine_trans_coffin
\fp_new:N \g_affine_precision_fp
\fp_set:Nn \g_affine_precision_fp {0.0001}
\fp_new:N \l__affine_@@_a_fp
\fp_new:N \l__affine_@@_b_fp
\fp_new:N \l__affine_@@_c_fp
\fp_new:N \l__affine_@@_d_fp
\msg_set:nnn { module }{affine-det_zero}
  {
    current~determination~of~the~affine~transformation~
    matrix~equals~to~zero,~give~up~this~transformation
  }
\cs_new:Npn \__affine_transformation:nnnnn #1#2#3#4#5
  {% #1:box content; #2: a_11; #3: a_21; #4: a_12; #5: a_22
    \hcoffin_set:Nn \l__affine_trans_coffin {#1}
    %% check affine matrix determination 
    \fp_compare:nNnT 
      { abs(\__matrix_det:nnnn {#2}{#3}{#4}{#5}) } 
        < { \g_affine_precision_fp }
      { \prg_map_break:Nn \l__affine_matrix_det_zero 
        { \msg_warning:nn { module }{affine-det_zero} }}
    %% set matrix elements
    \fp_set:Nn \l__affine_@@_a_fp {#2}
    \fp_set:Nn \l__affine_@@_b_fp {#3}
    \fp_set:Nn \l__affine_@@_c_fp {#4}
    \fp_set:Nn \l__affine_@@_d_fp {#5}
    %% get factors
    \__box_affine_get_sx:     
    \__box_affine_get_theta:  
    \__box_affine_get_msy:    
    \__box_affine_get_sy:     
    \__box_affine_get_m:      
    \__box_affine_get_Sx:     
    \__box_affine_get_Sy:     
    \__box_affine_get_phi:    
    \__box_affine_get_omega:  
    %% box transformation
    \coffin_scale:Nnn \l__affine_trans_coffin 
      { \l__box_affine_sx_fp }
      { \l__box_affine_sy_fp }
    \coffin_rotate:Nn \l__affine_trans_coffin 
      { \__fp_to_rad:n {\l__box_affine_omega_fp} }
    \coffin_scale:Nnn \l__affine_trans_coffin
      { \l__box_affine_Sx_fp }
      { \l__box_affine_Sy_fp }
    \coffin_rotate:Nn \l__affine_trans_coffin
      { \__fp_to_rad:n {\l__box_affine_phi_fp} }
    \coffin_rotate:Nn \l__affine_trans_coffin 
      { \__fp_to_rad:n {\l__box_affine_theta_fp} }
    \prg_break_point:Nn \l__affine_matrix_det_zero { }
    \coffin_typeset:Nnnnn \l__affine_trans_coffin
      {l}{b}{0pt}{0pt}
  }
\NewDocumentCommand{\boxaffine}{m>{\SplitList{,}}m}
  {% #1:content; #2:matrix
    \__affine_transformation:nnnnn {#1}#2
  }
% internal calculating functions
\fp_new:N \l__box_affine_sx_fp
\cs_new:Nn \__box_affine_get_sx: 
  {
    \fp_set:Nn \l__box_affine_sx_fp 
      { \fp_eval:n {sqrt(\l__affine_@@_a_fp^2 + \l__affine_@@_b_fp^2)} }
  }
\fp_new:N \l__box_affine_theta_fp 
\cs_new:Nn \__box_affine_get_theta: 
  {
    \fp_set:Nn \l__box_affine_theta_fp 
      { \fp_eval:n {atan(\l__affine_@@_b_fp/\l__affine_@@_a_fp)} }
  }
\fp_new:N \l__box_affine_msy_fp 
\cs_new:Nn \__box_affine_get_msy: 
  {
    \fp_set:Nn \l__box_affine_msy_fp 
      { \fp_eval:n {
        \l__affine_@@_c_fp*cos(\l__box_affine_theta_fp) 
        + 
        \l__affine_@@_d_fp*sin(\l__box_affine_theta_fp)
      } }
  }
\fp_new:N \l__box_affine_sy_fp
\cs_new:Nn \__box_affine_get_sy: 
  {
    \bool_if:nTF
      {
        \fp_compare_p:nNn { abs(sin(\l__box_affine_theta_fp)) } 
          < {\c_zero_fp + \g_affine_precision_fp}
      }{
        \fp_set:Nn \l__box_affine_sy_fp 
          {
            ( \l__affine_@@_d_fp - \l__box_affine_msy_fp*sin(\l__box_affine_theta_fp) )
            / cos(\l__box_affine_theta_fp)
          }
      }{
        \fp_set:Nn \l__box_affine_sy_fp 
          {
            ( \l__box_affine_msy_fp*cos(\l__box_affine_theta_fp) - \l__affine_@@_c_fp )
            / sin(\l__box_affine_theta_fp)
          }
      }
  }
\fp_new:N \l__box_affine_m_fp
\cs_new:Nn \__box_affine_get_m: 
  {
    \fp_set:Nn \l__box_affine_m_fp 
      { \l__box_affine_msy_fp / \l__box_affine_sy_fp }
  }
\fp_new:N \l__box_affine_Sx_fp
\fp_new:N \l__box_affine_Sy_fp
\cs_new:Nn \__box_affine_get_Sx: 
  {
    \fp_set:Nn \l__box_affine_Sx_fp 
      { sqrt(\l__box_affine_m_fp^2/4 + 1) - \l__box_affine_m_fp/2 }
  }
\cs_new:Nn \__box_affine_get_Sy: 
  {
    \fp_set:Nn \l__box_affine_Sy_fp 
      { sqrt(\l__box_affine_m_fp^2/4 + 1) + \l__box_affine_m_fp/2 }
  }
\fp_new:N \l__box_affine_phi_fp
\fp_new:N \l__box_affine_omega_fp
\cs_new:Nn \__box_affine_get_phi: 
  {
    \fp_set:Nn \l__box_affine_phi_fp 
      { -pi/4 - 1/2*atan(\l__box_affine_m_fp/2) }
  }
\cs_new:Nn \__box_affine_get_omega: 
  {
    \fp_set:Nn \l__box_affine_omega_fp 
      { pi/4 - 1/2*atan(\l__box_affine_m_fp/2) }
  }
\ExplSyntaxOff\makeatother


\begin{document}
Original Text: XXX\par
$\det(A) = 0$: \boxaffine{XXX}{0, 0, 0, 2}\par  % det(A) = 0
Unit Matrix: \boxaffine{XXX}{1, 0, 0, 1}\par    % unit matrix
Scale Matrix: \boxaffine{XXX}{2, 0, 0, 2}\par   % scale
$x$-scale Matrix: \boxaffine{XXX}{2, 0, 0, 1}\par % x-scale
$y$-scale Matrix: \boxaffine{XXX}{1, 0, 0, 2}\par % y-scale
$x$-shear Matrix: \boxaffine{XXX}{1, 0, 1, 1}\par % x-sheae
$y$-shear Matrix: \boxaffine{XXX}{1, 1, 0, 1}\par % y-sheae
Graphics: \rule{2em}{2em}~\boxaffine{\rule{2em}{2em}}{1, 0, .5, 1}\par
pdfsetmatrix: \rule{2em}{2em}~\pdfsave\pdfsetmatrix{1 0 .5 1}\rlap{\rule{2em}{2em}}\pdfrestore 

% det(A) = 0 --> warning:
% Package module Warning: current determination of the affine transformation
% (module)                matrix equals to zero, give up this transformation
\end{document}




\InputIfFileExists{zlatex-cfg.tex}{}{}
\documentclass[]{../code/ztex}
\usepackage{graphicx}
\def\image{\includegraphics[width=10em]{example-image-a}}
\ExplSyntaxOn\makeatletter
% % \hshearbox{vertical_prescale_times_shearfactor}{one_divide_by_shearfactor}{content}
% % an initial vertical downscale is often necessary for a 3d projection
% \newcommand{\hshearbox}[3]{\scalebox{0.866025}[#2]{\rotatebox{210}%
%   {\scalebox{1.73205}[-0.57735]{\rotatebox{60}{\scalebox{-1.1547}[#1]{#3}}}}}}
% % \vshearbox{horizontal_prescale_times_shearfactor}{one_divide_by_shearfactor}{content}
% % an initial horizontal downscale is often necessary for a 3d projection
% \newcommand{\vshearbox}[3]{\scalebox{#2}[0.866025]{\rotatebox{210}%
%   {\scalebox{-0.57735}[1.73205]{\rotatebox{60}{\scalebox{#1}[-1.1547]{#3}}}}}}
\makeatother\ExplSyntaxOff
\parindent0pt


\begin{document}
Original Text: XXX
% \hshearbox{1}{.5}{HELLO}
% \ExplSyntaxOn
% \fp_eval:n { atan(1) }:% output radian 
% \fp_eval:n { sin(pi/2) }% input radian

% \fp_set:Nn \l_tmpa_fp { pi/4 }
% \fp_use:N \l_tmpa_fp
% \ExplSyntaxOff


%%%%%%%     BEGIN DEBUG    %%%%%%%
% \ztoolboxaffine{XXX}{0, 0, 0, 2}\par % det=0, works
% \ztoolboxaffine{XXX}{1, 0, 0, 1}\par % unit matrix --> works
% \ztoolboxaffine{XXX}{2, 0, 0, 2}\par % scale --> works
% \ztoolboxaffine{XXX}{2, 0, 0, 1}\par % x-scale --> works
% \ztoolboxaffine{XXX}{1, 0, 0, 2}\par % y-scale --> works
% \ztoolboxaffine{XXX}{1, 0, 1, 1}\par % x-shear --> works
% \ztoolboxaffine{XXX}{1, 1, 0, 1}\par % y-shear --> works
\image~\ztoolboxaffine{\image}{1, 0, .5, 1}
%%%%%%%     END DEBUG      %%%%%%%
\end{document}




% ztex box content align test
\documentclass{article}
\usepackage[showframe, textwidth=18cm]{geometry}
\usepackage{graphicx}
\usepackage{xcolor}
\usepackage{lipsum}
\ExplSyntaxOn\makeatletter
\cs_set_protected:Npn \__item_align:nnn #1#2#3
  {% #1:cmd, #2:width, #3:object
    \hb@xt@#2{
      \tl_map_inline:nn {#3} 
        {
          \seq_put_right:No \l_tmpa_seq {\exp_not:N #1{##1}}
        } 
      \edef\seq@count{\seq_count:N \l_tmpa_seq}
      \seq_map_inline:Nn \l_tmpa_seq
        {
          \edef\item@width{\dim_eval:n {#2/(\seq@count+1)}}
          \hskip\item@width\clap{##1}
        }\hskip\item@width\hss
    }
  }
\NewDocumentCommand\itemAlign{O{}mm}
	{
		\__item_align:nnn {#1}{#2}{#3}
	}
\makeatother\ExplSyntaxOff
\parindent0pt

\begin{document}
\itemAlign{\linewidth}{A}\par
\itemAlign{\linewidth}{AA}\par
\itemAlign{\linewidth}{{(A)}{(A)}{(A)}}\par

\vskip6em
\dotfill\par
\def\imageA{\includegraphics[width=5em]{example-image-a}}
\def\imageB{\includegraphics[width=5em]{example-image-b}}
\itemAlign{\linewidth}{\imageA}\par
\itemAlign{\linewidth}{\imageA\imageA}\par
\itemAlign{\linewidth}{\imageB\imageB\imageB}\par
\itemAlign{\linewidth}{\imageB\imageB\imageB\imageB}

\vskip6em
\dotfill\par
\def\parA{\parbox[c]{12em}{\lipsum[1][1-2]}}
\def\parB{\parbox[c]{12em}{\lipsum[1][3-5]}}
\itemAlign[\sffamily]{\linewidth}{\parA}\par
\itemAlign[\color{red}]{\linewidth}{\parA\parA}\par
\itemAlign[\fbox]{\linewidth}{\parB\parB\parB}
\end{document}




\InputIfFileExists{zlatex-cfg.tex}{}{}
\documentclass[
]{../code/ztex}
\parindent0pt
\makeatletter


\begin{document}
Hello world.\rlap{A}B


\zboxitemalign[type=tower]{\linewidth}{A}\par
\zboxitemalign[type=tower]{\linewidth}{AA}\par
\zboxitemalign[type=tower]{\linewidth}{AAA}\par
\zboxitemalign[type=tower]{\linewidth}{AAAA}\par
\zboxitemalign[type=tower]{\linewidth}{AAAAA}\par

\dotfill\par
\ExplSyntaxOn
\def\custom@type{
  \edef\seq@count{\seq_count:N \l__ztool_boxitem_seq}
  \seq_map_inline:Nn \l__ztool_boxitem_seq
    {
      \edef\item@width{\dim_eval:n {\total@width/(\seq@count+1)}}
      \hskip\item@width\clap{##1}
    }\hskip\item@width\hss
}
\ExplSyntaxOff
\def\Blue#1{\textcolor{blue}{\sffamily #1}}
\zboxitemalign[type=custom, cmd=\Blue, custom=\custom@type]{\linewidth}{AAAAAA}\par
\end{document}




% ztex check box and \item color in manual
\documentclass{article} 
\usepackage{xcolor}
\usepackage{ctex}


% REF: https://tex.stackexchange.com/q/247681/294585
% *: wont fix
% [<arg>]: done
% [<blank>]: undone
\usepackage{enumitem}
\newlist{todolist}{itemize}{2}
\setlist[todolist]{label=\checkmark}
\usepackage{pifont, amssymb}
\newcommand{\done}{\rlap{\raisebox{0.3ex}{\hspace{0.4ex}\tiny \ding{52}}}$\square$}
\newcommand{\undone}{$\square$}
\newcommand{\wontfix}{\rlap{\raisebox{0.3ex}{\hspace{0.4ex}\scriptsize \ding{56}}}$\square$}


\let\olditem\item
\RenewDocumentCommand{\item}{so}
  {
    \IfValueTF{#2}
      {\color{black}\def\checkmark{\IfBooleanTF{#1}{\wontfix}{\done}}% 
        \olditem\IfBooleanF{#1}{\IfValueT{#2}{#2-}已完成:}\color{gray}}
      {\color{black}\def\checkmark{\IfBooleanTF{#1}{\wontfix}{\undone}}%
        \olditem}
  }


\begin{document}        
\begin{todolist}
  \item undone text
  \item*[2025-05-10] wont fixed + done: I want this to be blue
  \item[2025-05-10] done: another text
  \item* done: one more
\end{todolist}

HELLO:  \undone{} -- 未完成; \done{} -- 已完成; \wontfix{} -- 不会完成
\end{document}




% ztex slide nav bug
\InputIfFileExists{zlatex-cfg.tex}{}{}
\documentclass[
  hyper,
  layout={slide, aspect=12|9}, 
]{../code/ztex}
\zslidethemeuse[
  UL = {text=},
  UR={text=\zslidenavsym}
]{AnnArborDefault}
\makeatletter
\def\seeTargets{%
  \par\noindent\dotfill\\
  \string\@currentHref\ = \meaning\@currentHref\\
  \string\@currentHpage\ = \meaning\@currentHpage\\
  \string\Hy@currentbookmarklevel\ = \meaning\Hy@currentbookmarklevel\\
  \vskip3em
}
% \makeatother

\title{Test hyper symbol}
\author{Eureka}
\date{\today}
\begin{document}
\maketitle
\section{FIRST}
% AAAA \zslidenavsym{}\par 
\dotfill\par 
% \hyper@link{link}{zslide@FIRST.1}{TEST HYPER-1}\par
% \hyper@link{link}{zslide@FIRST.2}{TEST HYPER-2}\par
% \hyper@link{link}{zslide@FIRST.3}{TEST HYPER-3}\par

\dotfill\par
% \seeTargets

\hyper@link{link}{zslide@FIRST.3}{TEST HYPER}

\newpage
BBBB 
\newpage
CCCC

% \hyper@anchor{zslide@\FirstMark{zslide-left}.3}

\section{SECOND}
AAAA-2
\newpage
BBBB-2
\newpage
CCCC-2

\end{document}



% ztex slide test
\InputIfFileExists{zlatex-cfg.tex}{}{}
\documentclass[
  % lang=cn,
  layout={slide, aspect=12|9, theme=AnnArborSpruce}, 
  classOption={12pt}
]{../code/ztex}
\usepackage{xcoffins}
\usepackage{lipsum}
% \usepackage{zhlipsum}
% \usepackage[hshift=0mm,vshift=0mm]{fgruler}
\def\isCJK#1{\ifnum`#1>19968 is CJK\else not CJK\fi}

\title{Test \ztex{} AOh}
\author{Eureka}
\date{\today}
\begin{document}
\maketitle

\section{FIRST}
If $Y$ is a Banach space, an equivalent definition is that the embedding operator
(the identity) $i:X\to Y$ is a compact operator.When applied to functional analysis, this
version of compact embedding is usually used with Banach spaces of functions. Several of
the Sobolev embedding theorems are compact embedding theorems. When an embedding is not 
compact, it may possess a related, but weaker, property of cocompactness.

\lipsum[1]


\section{International}%{国际形势}
The following term ``Compact embedding'' can be defined in Topology spaces or in
Normed spaces. In the first case: we can simply think that $X$ is compactly embedded in $Y$.

% \zhlipsum[1]

% \newpage
\zslideframetitle{New Frame TITLE}%{ 新的 Frame Title}
% \par\vspace*{3em}\par
% 派则指细流金义月无采列,走压看计和眼提
% 问接,作半极水红素支花。果都济素各半走,意红接器长标,等杏近乱共。层题提万
% 任号,信来查段格,农张雨。省着素科程建持色被什,所界走置派农难取眼,并细杆
% 至志本。
\lipsum[1]


\newpage
\lipsum[1]
% 派则指细流金义月无采列,走压看计和眼提
% 问接,作半极水红素支花。果都济素各半走,意红接器长标,等杏近乱共。层题提万
% 任号,信来查段格,农张雨。省着素科程建持色被什,所界走置派农难取眼,并细杆
% 至志本。

\section{TEXT HEIGHT}
\ExplSyntaxOn
\NewCoffin \encoff
\NewCoffin \cncoff
\NewCoffin \mathcoff
\NewCoffin \mixcoff
% \SetHorizontalCoffin \encoff {\Large Hello}
% \SetHorizontalCoffin \cncoff {\Large 国际形势}
% \SetHorizontalCoffin \mathcoff {$\sum\int$}
% \SetHorizontalCoffin \mixcoff {\Large 中文 Hello English}
\ExplSyntaxOff

\TypesetCoffin \encoff (0pt, 0pt)\hspace{4em}
% \TypesetCoffin \cncoff (0pt, 2pt)

% \isCJK{A} \isCJK{你}

\end{document}





\documentclass{article}


\begin{document}
\ExplSyntaxOn
\cs_new:Npn \__aaa_bbb:n #1 
  {HELLO~#1}

% \__aaa_bbb:n {WORLD}
\__aaa_bbb_:n {WORLD}
\ExplSyntaxOff
\end{document}




% test ztex after explcheck
\InputIfFileExists{zlatex-cfg.tex}{}{}
\documentclass[layout={slide}]{../code/ztex}
\zthmnew{Zaxiom, Ztheorem=Thm|{HTML}{a0d911}, Zproposition=Prop|blue}
\zthmnew[proof]{Zproof, Zexample=EXAMPLE|red, Zsolution=Solution|}
\zthmstyle{background}

\begin{document}
\begin{Zproof}[zthmnew-1]
  This is a Zproof zthmnew-1.
  \end{Zproof}
  \begin{Zexample}[zthmnew-2]
  This is a Zexample zthmnew-2.
  \end{Zexample}
  \begin{Ztheorem}[zthmnew-3]
  This is a Ztheorem zthmnew-3
  \end{Ztheorem}
\end{document}



% test thm module
\InputIfFileExists{zlatex-cfg.tex}{}{}
\documentclass[
  fancy,
  class=book,
  lang=cn, 
]{../code/ztex}
\makeatletter
\zthmstylenew{
  styleA={
    begin={
      \begin{tcolorbox}[
        enhanced jigsaw,
        breakable, sharp corners, 
        colframe=\thm@tmp@color, 
        top=2ex, boxrule=2pt, toprule=0pt, 
        fonttitle=\large\bfseries, colback=white, 
        attach boxed title to top left={xshift=3em,yshift=-\tcboxedtitleheight/2},
        coltitle=\thm@tmp@color, title={\zthmname~\zthmnumber},
        boxed title style={%
          empty, left=1pt, right=1pt, bottom=0pt,
          overlay={%
            \draw[color=\thm@tmp@color,line width=2pt,line cap=round]
            ([yshift=-1pt]frame.west)--
            ++(-2.9em,0) ([yshift=-1pt]frame.east)--
            ++(3em,0);
          }
        },
      ]
    },
    end = {\end{tcolorbox}},
  }
}
\makeatother
\zthmstyle{styleA}
\zthmlang{en}
\usepackage{zhlipsum}
\usepackage{lipsum}


\begin{document}
\mainmatter
\chapter{FIRST CHAP}
\section{FIRST}
\lipsum[1][1-2]
\begin{definition}[DEF-TEST]\label{def:A}
  \lipsum[1][1-2]
  \begin{align}
    \sum_{i=1}^{+\infty}{\int_{0}^{i}-\frac{1}{t}\mathrm{d}t} = \frac{\pi^2}{6}
  \end{align}
  \lipsum[1][3-5]
\end{definition}
\lipsum[1][3-5] -- \cref{thm:A}.


\vspace*{13em}
\section{SECOND}
\begin{theorem}[THM-TEST]\label{thm:A}
  \zhlipsum[1]
\end{theorem}
\lipsum[1][3-5] -- \cref{def:A}.
\end{document}






% ==> ztex font module test
\InputIfFileExists{zlatex-cfg.tex}{}{}
\documentclass[lang=cn]{../code/ztex}
% \documentclass{ctexart}
% \usepackage{lipsum}
% \usepackage{zhlipsum}
% \setmainfont{Times New Roman}
% \ztexset{mathSpec={font=mathpazo}}
% \zfontset{
%   sysfont,
%   % doc = lmm,
%   % text=times,
%   math = euler
% }
\zfontfamilynew[CJK]{
  cmd = YaHei, 
  name = msyh.ttc,
  path = ./Fonts/, 
  % feat = { bd=*bd } 
}
\zfontfamilynew{
  cmd = Arial,
  name = arial.ttf,
  path = ./Fonts/,
  feat = {sl=*i}
}
% % do NOT 'file name' and  'font name' in one decalre.
% \zfontfamilynew{
%   cmd = SourceCodePro,
%   name = Source Code Pro,
%   feat = { bd=Source Code Pro Bold }
% }
% \newfontfamily\TEST{Times New Roman}
%   [
%     BoldFont = Times New Roman,
%     ItalicFont = Times New Roman
%   ]
% \xeCJKsetup{EmboldenFactor=8}
% \setCJKfamilyfont{NewYH}{msyh.ttc}
%   [
%     Path = ./Fonts/,
%     AutoFakeBold = 12, % works
%     % AutoFakeBold = true,
%     % EmboldenFactor = 1, % do NOT make sense, when just use this.
%     % BoldFont = msyhbd.ttc
%   ]

% \setCJKfamilyfont{NewYHII}{msyh.ttc}
%   [
%     Path = ./Fonts/,
%     % AutoFakeBold = 8, % works.
%     % 
%     AutoFakeBold = true,
%     % EmboldenFactor = 8, 
%     % NOTE:
%     % 1. do NOT make sense, when just use this.
%     % 2. can NOT use this arg in local.
%   ]


\begin{document}
\ExplSyntaxOn
% \def\bdnum{\fp_use:N \g__xeCJK_embolden_factor_fp}
% \fp_gset:Nn \g__xeCJK_embolden_factor_fp {10}
\ExplSyntaxOff

% System: {你好世界,\bfseries 你好世界}.

% {\CJKfamily{NewYH}\bdnum{} 你好世界,\bfseries 你好世界.}

% {\CJKfamily{NewYHII}\bdnum{} 你好世界,\bfseries 你好世界.}

{\YaHei 你好世界,\bfseries 你好世界.}


{\Arial Hello world,\slshape Hello world.}


% {Hello world,\SourceCodePro Hello world,\bfseries Hello world.}
\end{document}





% ==> box scale
\InputIfFileExists{zlatex-cfg.tex}{}{}
\documentclass[lang=en, hyper]{../code/ztex}
\usepackage{minted}
% \usepackage[T4,T1]{fontenc}
% \newcommand{\USP}[1]{{\fontencoding{T4}\selectfont\char"20}}
\ExplSyntaxOn\makeatletter
\setlength\fboxsep{0pt}
% \NewDocumentCommand{\ztexscatter}{O{\use:n}mm}
%   {% #1:item cmd, #2:width; #3:object
%     \hb@xt@#2{
%       \tl_map_inline:nn {#3} 
%         {
%           \seq_put_right:No \l__ztex_scatter_seq {#1{##1}}
%         } 
%       \seq_use:Nn \l__ztex_scatter_seq { \hfill }
%     }
%     \seq_clear:N \l__ztex_scatter_seq
%   }
% \seq_new:N \l__ztex_scatter_seq
% \cs_generate_variant:Nn \tl_set:Nn {No}
% \gdef\ztoolWdScale#1{\textcolor{blue}{\ztool_onlyset_to_wd:nn {1em}{#1}}}
\gdef\ztoolWdScale#1{\textcolor{blue}{#1}}
\ExplSyntaxOff

\ExplSyntaxOn
\clist_new:N \l__ztex_doc_source_clist
\clist_clear:N \l__ztex_doc_source_clist
\cs_set:Npn \__ztex_doc_source:nn #1#2 
  {
    \clist_map_inline:nn {#2}
      {
        \clist_put_right:Nn \l__ztex_doc_source_clist 
          {
            \subsubsection{##1}
            \inputminted{latex}{../code/#1/ztex.#1.##1.tex}
          }
      }
  }
\newcommand{\inputZTeXSource}[2][module]
  {
    \__ztex_doc_source:nn {#1}{#2}
    \clist_use:Nn \l__ztex_doc_source_clist   
      { \newpage }
  }
\ExplSyntaxOff


\begin{document}
\section{\ztex{} and \ztex*}
\zTeX*{} introduction. Introduction to \zTeX{} and \ztex*.

Hello, this is \zlatex{} and \zlatex*{}. 


Space token:{\usefont{OT1}{cmtt}{m}{n}\asciispace}
% Space token: \USP{ }.
% \ztexscatter{5em}{mmm}


\underline{%
  \zboxitemalign[cmd=\ztoolWdScale, type=scatter]{9em}{{Tom}{Amy}{Jennery}}%
}

\underline{%
  \zboxitemalign[cmd=\ztoolWdScale, type=right]{9em}{{Tom}{Amy}{Jennery}}%
}


\underline{%
  \zboxitemalign[cmd=\ztoolWdScale, type=center]{9em}{Tom Amy\ Jenn ery}%
}

\underline{%
  \zboxitemalign[cmd=\ztoolWdScale, type=left]{9em}{{Tom}{Amy}{Jennery}}%
}


% \inputZTeXSource{box, font}
% \inputZTeXSource[library]{fancy, alias}

\ExplSyntaxOn
% \def\zTeX{
%   \texorpdfstring{
%     \ztool_set_to_wd:nn {7pt}{
%       \ztool_rotate:nn {89.25}{\(\aleph\)}
%     }\kern-1.5pt\hbox{}\TeX{}
%   }{zTeX}
% }


% \maketitle
% \ExplSyntaxOn
% \ztool_onlyset_to_wd:nn {2em}{A}\par
% \ztool_onlyset_to_wd:nn {2em}{AA}\par
% \ztool_onlyset_to_wd:nn {2em}{AAA}\par
% \ztool_onlyset_to_wd:nn {2em}{AAAA}\par
% \ztool_onlyset_to_wd:nn {2em}{AAAAAA}\par

% \ztool_onlyset_to_ht:nn {2.5em}{\fbox{\vbox{\hbox{A}}}}\quad
% \ztool_onlyset_to_ht:nn {2.5em}{\fbox{\vbox{\hbox{A}\hbox{A}}}}\quad
% \ztool_onlyset_to_ht:nn {2.5em}{\fbox{\vbox{\hbox{A}\hbox{A}\hbox{A}}}}\quad
% \ztool_onlyset_to_ht:nn {2.5em}{\fbox{\vbox{\hbox{A}\hbox{A}\hbox{A}\hbox{A}}}}


% box height test
% \hbox_set:Nn \l_tmpa_box {\huge\bfseries Hello}
% \dim_set:Nn \l_tmpa_dim { \box_ht:N \l_tmpa_box }
% \dim_show:N \l_tmpa_dim
% luatex:15.32524pt; xetex:15.32524pt % ==> no bug


% \hbox_set:Nn \l_tmpb_box {
%   \parbox[b][][r]{10em}{
%       {\huge\bfseries Hello}\\[0pt]
%       {\huge\bfseries world}
%     }
% }
% \dim_set:Nn \l_tmpb_dim { \box_ht:N \l_tmpb_box }
% \dim_show:N \l_tmpb_dim
% luatex:31.7667pt; xetex:31.7667pt % ==> no bug


% \long\def\formatTitle{{\huge\bfseries \zTeX{} 用户手册}}
% \long\def\formatAuthor{{\Large\bfseries World}}
% \newcommand\titleUpperBox[2][0pt]{
%   \parbox[b][#2][r]{\l_tmpa_dim}{
%     {\formatTitle}\\[#1]
%     {\formatAuthor}
%   }
% }
% \ztool_get_ht_plus_dp:Nn \l_tmpb_dim {\titleUpperBox{}}
% \dim_show:N \l_tmpb_dim % luatex:80.14294pt; xetex:\l_tmpb_dim=75.46144pt % bug


% \hbox_set:Nn \l_tmpa_box {\huge\bfseries 用户手册}
% \dim_set:Nn \l_tmpa_dim { \box_ht:N \l_tmpa_box }
% \dim_show:N \l_tmpa_dim
% luatex:19.43259pt; xetex:16.73853pt % bug !!!!
\ExplSyntaxOff
\end{document}


% \documentclass{article}
% \usepackage[T1]{fontenc}

% \begin{document}
% \ExplSyntaxOn
% \cs_set_eq:NN \cctabend \cctab_end:
% \cctab_const:Nn \g__ztex_keyval_cctab 
% {
%   \cctab_select:N \c_document_cctab
%   \char_set_catcode_active:n {124}
% }
% \cctab_begin:N \g__ztex_keyval_cctab
% \gdef|{XXX}

% ||\par
% \cctabend
% ||
% \ExplSyntaxOff
% \end{document}


% ==> format l3doc syntax description
\InputIfFileExists{zlatex-cfg.tex}{}{}
\documentclass[class=l3dox]{../code/ztex}
% \documentclass[twoside]{l3dox}
% \usepackage{xcolor}
% \usepackage{geometry}
\AtBeginDocument{
  % \DeleteShortVerb \"
  \DeleteShortVerb \|
}
\catcode`>=13\gdef>{\dotfill}
% \catcode`|=13
% \AddToHook{env/function/begin}{\def|{\textup{\string|}}}
% \AddToHook{env/function/after}{\catcode`|=12}
% key-value env 
\ExplSyntaxOn
\cs_new_protected:Npn \__codedoc_function_ztex:nnw #1#2
  {
    \__codedoc_function_typeset_start:
    \__codedoc_function_init:
    \tl_set:Nn \l__codedoc_macro_argument_tl {#2}
    \keys_set:nn { l3doc/function } {#1}
    \__codedoc_names_get_seq_ztex:nN {#2} \l__codedoc_names_seq
    \__codedoc_names_parse:
    \__codedoc_function_typeset:
    \__codedoc_function_reset:
    \__codedoc_function_descr_start:w
  }
\cs_new_protected:Npn \__codedoc_names_get_seq_ztex:nN #1#2
  {
    \bool_if:NTF \l__codedoc_names_verb_bool
      {
        \seq_clear:N #2
        \seq_put_right:No #2 { {#1} }
      }
      {
        \tl_set:Nn \l__codedoc_tmpa_tl {#1}
        \tl_remove_all:Ne \l__codedoc_tmpa_tl
          { \exp_not:N \obeyedline \c_percent_str }
        \tl_remove_all:Ne \l__codedoc_tmpa_tl
          { \exp_not:N \obeyedline }
        \__kernel_tl_set:Nx \l__codedoc_tmpa_tl { \l__codedoc_tmpa_tl }
        \tl_remove_all:Ne \l__codedoc_tmpa_tl
          { \iow_char:N \^^M \c_percent_str }
        \tl_remove_all:Ne \l__codedoc_tmpa_tl { \tl_to_str:n { ^ ^ A } }
        \tl_remove_all:Ne \l__codedoc_tmpa_tl { \iow_char:N \^^I }
        \tl_remove_all:Ne \l__codedoc_tmpa_tl { \iow_char:N \^^M }
        \__codedoc_detect_internals:N \l__codedoc_tmpa_tl
        \__codedoc_replace_at_at:N \l__codedoc_tmpa_tl
        \tl_set:Ne \l__codedoc_tmpa_tl 
          { 
            \clist_map_function:NN \l__codedoc_tmpa_tl 
              \__ztex_add_parent_key:n
          }
        \exp_args:NNe \seq_set_from_clist:Nn #2
          { \l__codedoc_tmpa_tl }
      }
  }
\cs_set:Npn \__ztex_add_parent_key:n #1 
  {
    \textcolor{gray}{\l__codedoc_parent_key_ztex_tl/}
    \tl_trim_spaces:n {#1},
  }
\cs_set_eq:NN \cctabend \cctab_end:
\cctab_const:Nn \g__ztex_keyval_cctab 
{
  \cctab_select:N \c_document_cctab
  \char_set_catcode_active:n {124}
}
\DeclareDocumentEnvironment { keyval } { O{} +m }
  { 
    \cctab_begin:N \g__ztex_keyval_cctab
    \__codedoc_function_ztex:nnw {#1} {#2} 
  }
  { 
    \__codedoc_function_end: 
    % \cctab_end: 
  }
\group_begin:
\catcode`|=\active
\gdef\ztexbar{\def|{\textup{YYY}}}
\group_end:
\DeclareDocumentEnvironment { syntax } { }
  { 
    \cctab_begin:N \g__ztex_keyval_cctab
    % \catcode`|=\active
    \ztexbar
    \__codedoc_syntax:w 
  }{
    \__codedoc_syntax_end:
    \cctab_end: 
    \ignorespacesafterend
  }
\ExplSyntaxOff


\begin{document}
\tableofcontents
\section{Hello}
Hello world.
\begin{function}[added=2024-04-27]{\testA}
  \begin{syntax}
    \cs{testA}\oarg{arg1}\meta{\upshape{AAA\string|BBB}}
    \cs{testA}\oarg{arg1}\meta{AAA|BBB}
  \end{syntax}
  A simple test for verb short hand, |.
\end{function}



Key value, |, env test:
\begin{keyval}[parent=lang]{en, cn, en cn}
\begin{syntax}
en > value forbidden 
en cn > value forbidden 
cn > initial value:\textcolor{red}{9999}
\end{syntax}
document class automatically processes and loads corresponding packages based
on user-specified options. Therefore, the packages and commands loaded by the z\TeX{}
document class vary depending on
\end{keyval}


\newpage
Hello world II.
\begin{function}[added=2024-04-27]{\testB}
  \begin{syntax}
    \cs{testB}\Arg{arg1}\meta{\upshape{AAA\string|BBB}}
    \cs{testB}\parg{arg1}\meta{AAA|BBB}
  \end{syntax}
  A simple test for verb short hand.
\end{function}
\end{document}