\ProvidesExplFile{ztikz.library.l3draw.tex}{2024/12/17}{1.0.0}{zdraw~library~for~ztikz}



% ==> init
\RequirePackage{l3draw}
\tl_new:N \g__ztikz_zdraw_unit_tl
\tl_gset:Nn \g__ztikz_zdraw_unit_tl {em}
\newcommand{\zdrawSetUnit}[1]
  {
    \tl_gset:Nn \g__ztikz_zdraw_unit_tl {#1}
  }
\newcommand{\zdrawSetPathWidth}[1]
  {
    \dim_set:Nn \l_draw_default_linewidth_dim {#1}
  }


% ==> define l3 colors                
% pre-defined:black, white, red, green, blue, cyan, magenta, yellow
% new-defined:orange, teal, purple, pink, aqua, tan, brown, gray
\color_set:nnn {orange}{HTML}{FFA500}
\color_set:nnn {teal}{HTML}{008080}
\color_set:nnn {purple}{HTML}{800080}
\color_set:nnn {pink}{HTML}{FFC0CB}
\color_set:nnn {aqua}{HTML}{00FFFF}
\color_set:nnn {tan}{HTML}{D2B48C}
\color_set:nnn {brown}{HTML}{A52A2A}
\color_set:nnn {gray}{HTML}{808080}
\NewDocumentCommand{\newcolor}{mmm}
  {
    \color_set:nnn {#1}{#2}{#3}
  }


% ==> arg split; Maybe Implement by '\ProcessedArgument' ??
\cs_generate_variant:Nn \int_step_inline:nnn {nen}
\cs_generate_variant:Nn \seq_set_split:Nnn {cnn}
\cs_set:Npn \args_split_cs:nn #1#2 
  {
    \seq_if_exist:cF {l_#2_seq}{
      \seq_new:c {l_#2_seq}
    }
    \seq_set_split:cnn {l_#2_seq}{,}{#1}
    \int_step_inline:nen {1}{\seq_count:c {l_#2_seq}}{
      \tl_set:Nn \l_tmpa_tl {l__arg_#2_\int_to_roman:n{##1}}
      \exp_args:Nfo \tl_if_exist:cF {\tl_use:N \l_tmpa_tl}{
        \tl_new:c {\tl_use:N \l_tmpa_tl}
      }
      \tl_set:ce {\tl_use:N \l_tmpa_tl}{\seq_item:cn {l_#2_seq}{##1}}
    }
  }


% ==> zdraw layer               
%% draw env 
\NewDocumentEnvironment{Zdraw}{O{}}
  {\ExplSyntaxOn\draw_begin:}
  {\draw_end:\ExplSyntaxOff}
\NewDocumentEnvironment{Zscope}{O{}}
  {\draw_path_scope_begin:}
  {\draw_path_scope_end:}
%% cmd alias
% point/line
\cs_new_eq:NN \zmoveto \draw_path_moveto:n
\cs_new_eq:NN \zlineto \draw_path_lineto:n
\cs_new_eq:NN \scolor \color_stroke:n
\cs_new_eq:NN \fcolor \color_fill:n
% base vector
\cs_new_eq:NN \zxvec   \draw_xvec:n
\cs_new_eq:NN \zyvec   \draw_yvec:n
% both follows ref (0, 0)
\cs_new_eq:NN \zpolar  \draw_point_polar:nn 
\cs_new_eq:NN \zcoor   \draw_point_vec:nn  
% shape and text
\cs_new_eq:NN \zrect \draw_path_rectangle_corners:nn
\cs_new_eq:NN \zcirc \draw_path_circle:nn
\cs_new_eq:NN \znewtext    \coffin_new:N
\cs_new_eq:NN \zsethtext   \hcoffin_set:Nn
\cs_new_eq:NN \zsetvtext   \vcoffin_set:Nnn
\cs_new_eq:NN \zscaletext  \coffin_scale:Nnn
\cs_new_eq:NN \zputtext    \draw_coffin_use:Nnnn
%% path operation
% scope
\cs_new_eq:NN \zbg \draw_path_scope_begin:
\cs_new_eq:NN \zeg \draw_path_scope_end:
% segement connect
\cs_new_eq:NN \zcapbutt   \draw_cap_butt:
\cs_new_eq:NN \zcaproun   \draw_cap_round:
\cs_new_eq:NN \zcaprect   \draw_cap_rectangle:
\cs_new_eq:NN \zclosepath \draw_path_close:
% transformatio
\cs_new_eq:NN \zshift     \draw_transform_shift:n
\cs_new_eq:NN \zxscale    \draw_transform_xscale:n
\cs_new_eq:NN \zyscale    \draw_transform_yscale:n
\cs_new_eq:NN \ztrans     \draw_transform_matrix:nnnn
\NewDocumentCommand{\usepath}{O{draw}}{
  \draw_path_use_clear:n {#1}
}


% ==> function draw                
% \function_plot:nnn {<function>}{<domain>}{<style>}
% functin = {<function>}
% domain  = {<x-start>, <x-step>, <x-end>, <y-min>, <y-max>}
% style   = {<unit>, <action>, <color-1>, <color-2>, <gradient-axis>}
\cs_generate_variant:Nn \fp_step_inline:nnnn {eeen}
\cs_set:Npn \function_plot:nnn #1#2#3 
  {
    % => split arg
    \args_split_cs:nn {#2}{domain}
    \args_split_cs:nn {#3}{style}
    % => draw function 
    \draw_begin:
      % normal part
      \str_case:VnT \l__arg_style_ii 
      {
        {stroke}{\exp_args:Ne \color_stroke:n {\tl_use:N \l__arg_style_iii}}
        {draw}{\exp_args:Ne \color_stroke:n {\tl_use:N \l__arg_style_iii}}
        {fill}{\exp_args:Ne \color_fill:n {\tl_use:N \l__arg_style_iii}}
        {clip}{\relax}
      }{
        % => start point
        \tl_set:Nn \l_tmpa_tl {#1}
        \tl_replace_all:Nne \l_tmpa_tl {x}{(\tl_use:N \l__arg_domain_i)}
        \draw_path_moveto:n {\l__arg_domain_i\l__arg_style_i, \l_tmpa_tl\l__arg_style_i}
        % loop to draw path
        \fp_step_inline:eeen {\l__arg_domain_i}{\l__arg_domain_ii}{\l__arg_domain_iii}{
            \tl_set:Nn \l_tmpa_tl {#1}
            \tl_replace_all:Nnn \l_tmpa_tl {x}{(##1)}
            \draw_path_lineto:n {##1 \l__arg_style_i, \l_tmpa_tl \l__arg_style_i}
        }
        \draw_path_use_clear:n {\l__arg_style_ii}
      }
      % shade plot part
      \str_if_eq:VnT \l__arg_style_ii {shade}
      {
        % start and end point for 'y-axis gradient'
        \tl_if_eq:VnT \l__arg_style_v {y}
        {
          \tl_if_exist:cF {l__start_tl}{
            \tl_new:N \l__start_tl
            \tl_new:N \l__end_tl
          } 
          \tl_set:Ne \l__start_tl {\l__arg_domain_iv}
          \tl_set:Ne \l__end_tl {\l__arg_domain_v}
        }
        % loop to plot segements
        \fp_step_inline:eeen {\l__arg_domain_i}{\l__arg_domain_ii}{\fp_eval:n {\l__arg_domain_iii-\l__arg_domain_ii}}
        {
          \tl_set:Nn \l_tmpa_tl {#1}
          \tl_set:Nn \l_tmpb_tl {#1}
          \tl_replace_all:Nnn \l_tmpa_tl {x}{(##1)}
          \tl_replace_all:Nnn \l_tmpb_tl {x}{(##1+\l__arg_domain_ii)}
          \str_case:VnF \l__arg_style_v {
            {x}{\color_gradient:eee {
                \fp_eval:n {(##1-\l__arg_domain_i)*(100/(\l__arg_domain_iii-\l__arg_domain_i))}
              }{\l__arg_style_iii}{\l__arg_style_iv}}
            {y}{\color_gradient:eee {
                \fp_eval:n {(\l_tmpa_tl-\l__start_tl)*(100/(\l__end_tl-\l__start_tl))}
              }{\l__arg_style_iii}{\l__arg_style_iv}}
          }{\relax}
          \draw_path_moveto:n {##1 \l__arg_style_i, \l_tmpa_tl \l__arg_style_i}
          \draw_path_lineto:n {(##1+\l__arg_domain_ii) \l__arg_style_i, \l_tmpb_tl \l__arg_style_i}
          % \draw_cap_rectangle:
          \draw_cap_round:
          \draw_path_use_clear:n {draw}
        }
      }
    \draw_end:
  }
% color gradient
\cs_set:Npn \color_gradient:nnn #1#2#3 
  {
    \fp_compare:nNnTF {#1}>{100}{
      \exp_args:Ne \color_select:n {#2!\fp_eval:n{abs(#1)/(\l__arg_domain_v-\l__arg_domain_iv)}!#3} 
    }{
      \fp_compare:nNnTF {#1}<{0}{
        \exp_args:Ne \color_select:n {#2!\fp_eval:n{abs(#1)/(\l__arg_domain_v-\l__arg_domain_iv)}!#3} 
      }{\color_select:n {#2!#1!#3}}
    }
  }
\cs_generate_variant:Nn \color_gradient:nnn {eee}
% plot functions interface
\ztikz_keys_define:nn { zdraw/zplot }
  {
    action        .tl_set:N   = \l__ztikz_zdraw_zplot_action_tl,
    action        .initial:n  = {draw},
    domain        .tl_set:N   = \l__ztikz_zdraw_zplot_domain_tl,
    domain        .initial:n  = {-5, 0.1, 5},
    range         .tl_set:N   = \l__ztikz_zdraw_zplot_range_tl,
    range         .initial:n  = {-5, 5},
    startColor    .tl_set:N   = \l__ztikz_zdraw_zplot_startColor_tl,
    startColor    .initial:n  = {black},
    endColor      .tl_set:N   = \l__ztikz_zdraw_zplot_endColor_tl,
    endColor      .initial:n  = {white},
    axis          .tl_set:N   = \l__ztikz_zdraw_zplot_axis_tl,
    axis          .initial:n  = {y}
  }
\cs_generate_variant:Nn \function_plot:nnn {nee}
\NewDocumentCommand\zplot{ O{}m }
  {
    \group_begin:
    \ztikz_keys_set:nn { zdraw/zplot } { #1 }
    \function_plot:nee
      {#2}
      {\l__ztikz_zdraw_zplot_domain_tl, \l__ztikz_zdraw_zplot_range_tl}
      {
        \g__ztikz_zdraw_unit_tl, \l__ztikz_zdraw_zplot_action_tl, 
        \l__ztikz_zdraw_zplot_startColor_tl, \l__ztikz_zdraw_zplot_endColor_tl,
        \l__ztikz_zdraw_zplot_axis_tl
      }
    \group_end:
  }
% gradient rule               
\cs_generate_variant:Nn \fp_step_inline:nnnn {nnen}
\cs_set_nopar:Npn \gradient_rule:nnn #1#2#3 
  {
    \args_split_cs:nn {#1}{dim}
    \args_split_cs:nn {#2}{color}
    \draw_begin:
    \fp_step_inline:nnen {0}{#3}{\l__arg_dim_i}{
      \draw_scope_begin:
        \exp_args:Ne \color_select:n {\l__arg_color_ii!\fp_eval:n {##1*100/\l__arg_dim_i}!\l__arg_color_i} 
        \exp_args:Nne \draw_path_rectangle_corners:nn 
          {(##1) \g__ztikz_zdraw_unit_tl, 0 \g__ztikz_zdraw_unit_tl}
          {(##1+#3) \g__ztikz_zdraw_unit_tl, \l__arg_dim_ii \g__ztikz_zdraw_unit_tl}
        \draw_path_use_clear:n {fill, draw}
      \draw_scope_end:
    }
    \draw_end:
  }
% gradient rule interface
\ztikz_keys_define:nn { zdraw/zrule }
  {
    width         .tl_set:N   = \l__ztikz_zdraw_zplot_width_tl,
    width         .initial:n  = {1},
    height        .tl_set:N   = \l__ztikz_zdraw_zplot_height_tl,
    height        .initial:n  = {1},
    startColor    .tl_set:N   = \l__ztikz_zdraw_zrule_startColor_tl,
    startColor    .initial:n  = {black},
    endColor      .tl_set:N   = \l__ztikz_zdraw_zrule_endColor_tl,
    endColor      .initial:n  = {white},
    step          .fp_set:N   = \l__ztikz_zdraw_zrule_step_fp,
    step          .initial:n  = {0.25}
  }
\cs_generate_variant:Nn \gradient_rule:nnn {eee}
\NewDocumentCommand\zrule{O{}}
  {
    \group_begin:
    \ztikz_keys_set:nn { zdraw/zrule } { #1 }
    \gradient_rule:eee 
      {\l__ztikz_zdraw_zplot_width_tl, \l__ztikz_zdraw_zplot_height_tl}
      {\l__ztikz_zdraw_zrule_startColor_tl, \l__ztikz_zdraw_zrule_endColor_tl}
      {\l__ztikz_zdraw_zrule_step_fp}
    \group_end:
  }