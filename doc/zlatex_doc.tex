\chapter{z\LaTeX{}文档类}\label{start-use-class}
\section{基本介绍}
本文档类z\LaTeX{}基于\cmd{book}类,主要用于满足和方便使用\LaTeX{}使用者进行书籍和笔记的排版需求。
z\LaTeX{}全部由\LaTeX3进行编写,采用\cmd{key-value}的方式进行选项配置,方便后续的模板拓展和维护.
如果使用者熟悉\LaTeX{},那么花费不到10min的时间,你便可以轻松使用本文档类用于日常的笔记撰写或者是正常的
书籍的排版. 

\section{Set Up z\LaTeX{}}
\subsection{兼容情况}
目前本文档类 z\LaTeX{} 还没有登陆CTAN,未来也没有这个打算。由于本文档类全部使用
\LaTeX3进行开发,所以如果你的\TeX{}Live过于老旧的话,则无法使用本宏包。目前已知
z\LaTeX{}文档类在各平台的兼容情况为:

\hspace*{10em}\parbox{8cm}{
\begin{itemize}
    \item[Windows]: \TeX{}Live最低版本2022
    \item[Linux]: \TeX{}Live最低版本2022
    \item[MacOS]: 兼容Mac{}\TeX{}2024(老版也应兼容) 
\end{itemize}}

\subsection{加载z\LaTeX{}}
由于z\LaTeX{}还没有传入CTAN(未来也不会),所以想要使用此文档类,可以有如下的两种方法:
\begin{itemize}
    \item 把此文档类放入你的项目文件夹下
    \item 在命令行运行命令: \cmd{kpsewhich -var-value=TEXMFHOME}, 然后把\cmd{zlatex.cls}放入此路径下的
        \cmd{tex/latex/}子目录下. 在Windows上一般是: \cmd{C:/Users/<name>/texmf/}, 在Linux下一般是
        \cmd{~/texmf/},具体路径以自己的实际情况为准.
\end{itemize}

\subsection{额外设置}
由于z\LaTeX{}文档类只加载了基本的宏包,所以想要实现其它的功能还请自行引入相关的宏包;
z\LaTeX{}引入的宏包机制请参见\cref{tab:basic-package}.

\subsection{最小工作示例}
z\LaTeX{}的最小工作示例如下\Footnote{可能需要根据自己的实际情况加以调整}.
首先是中文写作示例:

\begin{minted}{latex}
% compile engine: xelatex 
\documentclass[lang=cn, class=article]{zlatex}

\title{<title>}
\author{<author>}
\date{<date>}
\begin{document}
\maketitle
\frontmatter
% some preface
% \tableofcontents
% some claim etc.
\mainmatter

% wrting your document here ...
\end{document}
\end{minted}

其次是英文写作示例,你需要修改的地方只有两处; 首先就是把语言选项改为\cmd{lang=en},
其次便是把编译方式改为\cmd{pdflatex}.

\begin{minted}{latex}
% compile engine: pdflatex 
\documentclass[lang=en, class=article]{zlatex}

\title{<title>}
\author{<author>}
\date{<date>}
\begin{document}
\maketitle
\frontmatter
% some preface
% \tableofcontents
% some claim etc.
\mainmatter

% wrting your document here ...
\end{document}
\end{minted}

\subsection{在线体验}
为了让用户能够直接体验到本文档类,免去部分的环境配置阶段。我已将本模板部署在Overleaf上,
地址:\href{https://www.overleaf.com/project/661fd2772cafbff9df5e6fb4}{Overleaf ZLaTeX Project}.
用户直接打开此地址即可体验本文档类。Overleaf上体验项目对应的内容为本文档的Z\LaTeX{}部分的说明文档已经源码说明,
不含有zTikZ文档,由于部分的技术原因,zTikZ请在本地体验。

\section{宏包机制}
z\LaTeX{}文档类会根据用户指定的选项自动处理和加载对应的宏包,所以z\LaTeX{}文档类在不同的导言区选项声明下
加载的宏包和命令是不同的。本文档类内置导言区选项输出命令:\cmd{\zlatexOptions}\index{\cmd{\zlatexOptions}},
用于打印此时文档类z\LaTeX{}接收到的选项. 比如此时文档类接收到的选项为: 
\begin{center}
    \zlatexOptions
\end{center}

以下为详细的宏包加载信息:

\subsection{基本宏包}
基本宏包\index{basic packages},意味着不管你的导言区如何的配置,这些宏包都是会加载的. 宏包列表如下:

\begin{table}[H]
    \centering{\ttfamily
    \begin{tabular}{p{3cm}p{3cm}p{3cm}p{3cm}}
        \toprule
        expl3 & l3keys2e & framed & geometry \\
        fancyhdr & amsfonts & amsmath & amsthm  \\
        xcolor & biblatex & indextools & hyperref \\ 
        cleveref & graphicx & float & titletoc \\
        titlesec & esint & sidenotes & \\
        \bottomrule
    \end{tabular}}
    \caption{z\LaTeX{}文档类基本宏包}
    \label{tab:basic-package}
\end{table}

\subsection{语言类宏包}
根据不同的文档类语言,z\LaTeX{}会加载不同的和语言相关的宏包\index{language packages},在\cmd{lang=en(cn)}
下的宏包加载列表分别为:

\begin{table}[H]
    \centering{\ttfamily
    \begin{tabular}{p{2cm}p{4cm}p{2cm}p{2cm}p{2cm}}
        \toprule
        {\rmfamily lang=en} & inputenc(pdftex) & fontenc \\
        {\rmfamily lang=cn} & fontspec & ctex \\
        \bottomrule
    \end{tabular}}
    \caption{z\LaTeX{}文档类语言宏包}
    \label{tab:lang-package}
\end{table}

在\LaTeX{}排版中,正确使用{\bf 引号}的方法是分别使用\cmd{`}和\cmd{\'}排版单引号`和',
分别用\cmd{``}和\cmd{\'\'}排版双引号``和''。虽然\cmd{"}也能表示双引号,但却没有没有合适的单个符号用来表示前双引号,
所以在正文中并不常用(lshort-zh-cn.pdf, “2.3.5 标点符号”)。 另外,在排版要求中,当两层双引号嵌套使用时,
其外层需要使用{\bf 双引号},而内层则应该使用{\bf 单引号},并且不同的语言中,用于表示引号的字符也不完全一致。
为解决这些排版中的问题,可以使用\cmd{csquotes}宏包实现引号的灵活排版\Footnote{本内容摘自\href{https://wenda.latexstudio.net/article-5042.html}{latexstudio}}。

\subsection{数学字体类宏包}
为了方便新手较为方便的使用不同的数学字体,本模板设置了几个简单的数字切换命令.从前面的导言区数学字体
配置就可以看出,本模板会根据导言区设置不同的数学字体的功能了. 具体的加载宏包\index{math packages}规则如下:
\begin{itemize}
    \item \cmd{font=<none>}: 不加载任何的数学字体宏包,采用默认数学字体
    \item \cmd{font=newtx}: 加载宏包 \cmd{newtxmath}
    \item \cmd{font=euler}: 加载命令 \cmd{\RequirePackage[OT1, euler-digits]{eulervm}}
    \item \cmd{font=mtpro2}: 加载命令 \par
        \cmd{\RequirePackage[lite,subscriptcorrection,slantedGreek,nofontinfo]{mtpro2}}
    \item \cmd{font=mathpazo}: 加载命令 \cmd{\RequirePackage{mathpazo}},同时保留了文档类默认的正文字体样式.
\end{itemize}

如果使用者在导言区指定了选项\cmd{alias=true}, 那么z\LaTeX{}此时还会额外加载
宏包\index{optional packages}:\cmd{amssymb, mathtools, bm}.

z\LaTeX{}还默认加载了宏包\cmd{esint},用于提供更多的积分符号.下面仅举几例,更多的人信息请参见\cmd{esint}宏包:
\[
    \iiint\qquad 
    \idotsint\qquad
    \ointclockwise\qquad
    \fint\qquad
    \landdownint
\]

其实还有一个包和这个\cmd{esint}包类似,叫做\cmd{pdfMsym}. 关于后者的使用请参见官方文档\href{https://mirror-hk.koddos.net/CTAN/macros/generic/pdfmsym/pdfmsym-doc.pdf}{pdfMsym}.

\begin{leftbar}
如果你想要自己设置文档对应的数学字体,那么请参见"字体配置\cref{font-config}"中的数学字体配置部分.
\end{leftbar}

\section{文档类选项与命令}
本模板具有丰富的配置选项\index{配置选项}, 采用键值对\cmd{[<key 1>=<value 1>, <key 2>=<value 2>]}的形式对各个选项就行指定, 和具体的指定顺序无关, 
具体的可配置项和可用的配置值参见本节的后续内容以及\cref{fig:zlatex-options}. 现在先简要说明z\LaTeX{}文档类的主要配置内容:

\begin{itemize}
    \item 模板语言\index{模板语言}:目前本模板类仅支持中文和英文两种语言.
    \item 文档类选择\index{文档类选择}:可以在加载本文档类时指定\cmd{article, book, ctexbook}.
    \item 页面设置\index{页面设置}:页边距,版心,页眉页脚等的指定;目前本模板还未整理好对应的接口.
    \item 边注\index{边注}(marging):包含编注图片或文字注释,其中边注图片依赖宏包\cmd{sidenotes}.
    \item 数学字体\index{数学字体}:本模板预设了5种数学字体选相关:\cmd{}.
    \item 字体配置\index{字体大小}:用户可以根据自己的字体喜好配置自己喜欢的中英文字体.
\end{itemize} 

\subsection{导言区}
\newcommand{\zkey}[1]{\texttt{<#1>}}
目前z\LaTeX{}可以从文档类的加载选项中指定,也可以通过命令\cmd{\zlatexSetup}进行设置. 配置的键值对目前的层级为两级. 第一级
中主要的键为:\zkey{layout}, \zkey{mathSpec}, \zkey{font}, \zkey{bib}, \zkey{classOption}, \zkey{lang}, \zkey{toc}. 其中前
四个键(key)具有自己的独立的子键(sub-key),后面的两者具有不具有自己的子键,直接指定即可. 关于各层key-value的关系请见\cref{fig:zlatex-options}.

\begin{figure}[!htb]
    \centering
    \includegraphics[width=.9\linewidth]{./pics/zlatex_options.pdf}
    \caption{zlatex options map}
    \label{fig:zlatex-options}
\end{figure}

从\cref{fig:zlatex-options}可以看出,z\LaTeX{} 的配置是比较复杂的,但是新手也不必担心,默认的选项配置下已经能够得到一个效果比较好的文档了.
下面详细说明各个键的指定方式和作用. 

\subsubsection{classOption}
这里的\zkey{classOption}表示你加载的对应文档类对于的合法options. 比如对应\cmd{article}文档类,合法的一个\zkey{classOption}
=\texttt{\{10pt, oneside, fleqn\}}. 如果你在加载文档类时指定的\zkey{class}为\cmd{ctexbook},那么此时你就可以在这个\zkey{classOption}
中填入如下的内容:\cmd{zihao=5, 12pt, heading=false, linespread=1.3}等\cmd{ctex}手册中的合法选项. 如果对于默认的\cmd{article,book}
文档类不熟悉,可以参见网站:\href{https://texblog.org/2013/02/13/latex-documentclass-options-illustrated/}{latex-documentclass-options}.

\subsubsection{lang}
键\zkey{lang}的合法值为\cmd{en, cn};第二个关于文档语言的指定是比较容易理解的,但是还是得说明对应的一些注意事项.\cmd{lang=cn}或者\cmd{class=ctexbook}时仅支持为\cmd{xelatex}
编译;在指定\cmd{lang=en}时,\cmd{pdflatex, xelatex}二者都是可以接受的,但是建议采用\cmd{pdflatex}, 因为在指定为\cmd{en}时部分的西文宏包
可能会有冲突的危险.当\cmd{lang=en}, 并且采用\cmd{pdflatex}进行编译时,z\LaTeX{}会引入宏包\cmd{inputenc}, 然而此宏包对\cmd{xelatex}是
没有适配的.

\subsubsection{toc}
键\zkey{toc}的合法值为\cmd{redef, 2column},这一选项主要用于文档的目录格式设置,可以设置目录页对应的标题,目录的栏数.
\zkey{toc}\index{\cmd{toc}}选项可以同时指定\cmd{redef, 2column},分别代表重定义目录名(居中加粗),以及使用双栏目录排版.
后续可能会把这个键\zkey{toc}放入\zkey{layout}中,所以这个键目前接口还不稳定, 请谨慎使用.

\subsubsection{bib}
本模板采用的文献后端默认是\cmd{biber},可以在导言区通过\cmd{backend=bibtex}来更改后端为\cmd{bibtex}.这说明,在默认配置下,
你编译文档时应采用\cmd{biber}命令, 而非\cmd{bibtex}命令. 同样的,你可以通过\zkey{bib}的子键\zkey{source}来指定存放参考文献的
文件的名称,默认为\cmd{ref.bib}. 

\subsubsection{layout}
\zkey{layout}键可以用于指定整个文档的布局,包括版心,页边距,边注等.目前模板提供了其对的一个子键\zkey{margin}.
此键的合法值均为bool值。指定方式如下:
\begin{minted}{latex}
% {margin} <=> {margin=true}
\documentclass[
    layout={margin}
]{zlatex}
\end{minted}

默认情况下\zkey{margin}的值为\cmd{false}. 如果使用了上述的示例代码,那么此时的文档就会有边注. 一般情况下,使用者是不需要加载这个选项的. 
一个需要注意的点就是:\cmd{margin=false} 只有在\cmd{oneside}时才有效,否则会抛出警告. 同时需要注意,如果原来含有\cmd{\marginpar}命令的
文档,在指定\cmd{margin=false}后,对应的\cmd{\marginpar}环境,会被替换为\cmd{framed}宏包提供的\cmd{leftbar}环境.

\subsubsection{mathSpec}
此选项可以用于设置文档类的如下内容:
\begin{itemize}
    \item 数学字体(\zkey{font}=\cmd{<name>}):可用\cmd{<name>}:\cmd{newtx, mtpro2, euler, mathpazo}
    \item 定理类环境样式(\zkey{envStyle}=\cmd{<name>}):可用\cmd{<name>}有:\cmd{plain, leftbar, background, fancy}
    \item 字体命令别称加载(\zkey{alias}):可以直接输入\cmd{alias, alias=false}或者是输入\cmd{alias=true, alias=false}
\end{itemize}

比如你想要设置键(\zkey{mathSpec})对应子键:\cmd{font, envStyle, alias}的值(value), 其中最后一个key为bool值.于是你
可以在导言区中这样设置:

\begin{minted}{latex}
\documentclass[
    mathSpec={
        alias,
        envStyle=background,  
        font=newtx
    },
]{zlatex}
\end{minted}

上述的声明表示,文档类的数学环境样式为\cmd{background},数学字体为\cmd{newtxmath}. 同时加载了数学命令别称.
于是此时常用的如下命令:
\begin{minted}{latex}
\mathbb{X}    \mathcal{X}    \mathrm{X}   ...
\end{minted}

可以替换为如下更加简洁的形式:
\begin{minted}{latex}
\B{X}    \C{X}    \R{X}    ...
\end{minted}

关于\cmd{alias}的更加详细的使用,请参见如下的章节"快捷命令\cref{快捷命令}". \cmd{alias}选项可以根据个人习惯进行选择,默认
情况下并不会加载。但是在加载此选项后,默认的两个\LaTeX{}指令\cmd{\S, \ll} 会被覆盖,分别被更名为 \cmd{\ss, \LL}:(\ss, $\LL$). 

同时需要注意的是,本模板提供给的部分数学字体设置并不一定符合每一个人,且由于本模板的开发环境为\cmd{WSL+Archlinux},所以在平台迁移后部分
字体配置可能会出现问题. 其中的\cmd{mtpro2}字体并非免费字体,请注意. 如果想要自己配置数学字体可以参见后文``Unicode-math:\cref{数学字体}''

\begin{leftbar}
关于\cmd{envStyle}的详细信息请参见"常用数学环境\cref{常用数学环境}"
\end{leftbar}

\subsection{快捷命令}\label{快捷命令}
关于文档类选项\cmd{alias}\index{\cmd{alias}}的进一步说明,默认的自定义命令可能并不一定符合每一个人的习惯,所以请谨慎加载此选项.
在\cmd{alias=true}后,z\LaTeX{}会进行如下命令的声明/重定义,以及宏包的加载:
\begin{minted}{latex}
\RequirePackage{amssymb, mathtools}
\RequirePackage{bm}          
% Math Font 
\newcommand{\dd}{\mathrm{d}}
\newcommand{\C}[1]{\ensuremath{\mathcal{#1}}}
\let\ss\S
\renewcommand{\S}[1]{\ensuremath{\mathscr{#1}}}
\newcommand{\B}[1]{\ensuremath{\mathbb{#1}}}
\newcommand{\FF}[1]{\ensuremath{\mathbf{#1}}}
\newcommand{\F}[1]{\ensuremath{\bm{#1}}}
\newcommand{\R}[1]{\ensuremath{\mathrm{#1}}}
\newcommand{\K}[1]{\ensuremath{\mathfrak{#1}}}
% Math Arrow 
\newcommand{\lr}{\ensuremath{\longrightarrow}}
\let\LL\ll
\renewcommand{\ll}{\ensuremath{\longleftarrow}}
\newcommand{\equ}{\ensuremath{\Longleftrightarrow}\,}
\newcommand{\sr}{\ensuremath{\longmapsto}}
\newcommand{\lrr}[2][]{\ensuremath{\xRightarrow[#1]{#2}}}
\renewcommand{\lll}[2][]{\ensuremath{\xLeftarrow[#1]{#2}}}
\newcommand{\ns}{\ensuremath{\varnothing}}
\newcommand{\A}{\ensuremath{\forall}}
% Math Operator
\newcommand{\alt}{\ensuremath{\mathrm{Alt}\;}}
\newcommand{\sgn}{\ensuremath{\mathrm{sgn}\;}}
\newcommand{\curl}{\ensuremath{\mathrm{curl}\;}}
\newcommand{\grad}{\ensuremath{\mathrm{grad}\;}}
\newcommand{\trace}{\ensuremath{\mathrm{trace}\;}}
\renewcommand{\div}{\ensuremath{\mathrm{div}\;}}
\end{minted}

上述的几个常见的数学运算符声明可能并不是那么的规范,如果有必要,可以自行进行修改.
可以参见命令\cmd{\DeclareMathOperator}.

\subsection{zlatexSetup}
本模板同样也提供了命令\cmd{\zlatexSetup}用于在导言区对整个文档进行设置,其使用方法和前面的几乎是一致的. 如下的
两段声明是等价的:
\begin{minted}{latex}
\documentclass[
    mathSpec={envStyle=background, alias, font=mathpazo},
    toc={redef, 2column},
    bib={backend=bibtex}
]{zlatex}
\end{minted}

\begin{minted}{latex}
\zlatexSetup{
    mathSpec={
        envStyle=background,
        font=mathpazo, 
        alias
    },
    toc={redef, 2column},
    bib={backend=bibtex}
}
\end{minted}

但是命令\cmd{\zlatexSetup}并不能进行所有的选项指定,如\zkey{class}, \zkey{lang}这两者就只能在加载文档类时声明,不能再后续
使用此命令声明. 同时,本命令不包含本模板的颜色主题设置,本模板的颜色主题设置请使用命令:\cmd{\zlatexColorSetup},关于模板的
色彩设置请参见``模板配色\cref{模板配色}''. 

如果你加载了宏包\cmd{ctex},那么此时你便可以使用此宏包提供的命令\cmd{\ctexset}进行对应项的设置.

\subsection{zlatexColorSetup}\label{模板配色}
z\LaTeX{}提供了\cmd{\zlatexColorSetup}\index{\cmd{\zlatexColorSetup}}用于设置整个模板的配色。
可供用户配置的选项有:
\begin{itemize}
    \item Hyperref宏包对应的颜色,对应的键为\cmd{link, url, cite}, 三者颜色默认不同.
    \item Chapter章节计数器颜色,对应的键为\cmd{chapter}.
    \item Chapter章节ruler颜色,对应的键为\cmd{chapter-rule}.
    \item 所有数学环境对应的颜色,对应的键为\cmd{<math-env-name>},如\cmd{axiom, definition, theorem, remark}等.
\end{itemize}

下面给出设置具体色彩的示例代码以及模板的默认配色:
\begin{minted}{latex}
\zlatexColorSetup{
    link            = purple,
    chapter-rule    = black,
    axiom           = purple,
    definition      = blue
}
\end{minted}

\newcommand{\block}[1]{{\color{#1}\rule{1em}{1em}}}
\begin{table}[H]
    \centering
    \begin{tabular}{ccccccccc}
        \toprule
        结构元素 & chapter & chapter-rule & link & url & cite \\
        \midrule 
        颜色 & \block{RoyalRed} & \block{black} & \block{purple}& \block{RoyalRed} & \block{blue}\\
        \midrule
        数学环境 & axiom & definition & theorem & lemma & corollary & proposition & remark & \\  
        \midrule 
        颜色 & \block{mathaxiomColor} & \block{mathdefinitionColor} & \block{maththeoremColor} & \block{mathlemmaColor}& \block{mathcorollaryColor}& \block{mathpropositionColor}& \block{mathremarkColor}\\
        \bottomrule
    \end{tabular}
    \caption{z\LaTeX{}文档类默认配色}
    \label{tab:zlatex-default-color}
\end{table}


\subsection{zlatexFramed}
目前本系列提供命令\cmd{\zlatexFramed{<name>}[<color>]}\index{\cmd{\zlatexFramed}}用于创建类似MarkDown
的彩色引用环境. 参数中的\cmd{<name>}表示声明环境的名称\footnote{如果此环境已存在,那么该环境会被Override},
\cmd{<color>}表示此环境的背景颜色.一个简单的使用样例如下:

\begin{minted}{latex}
% 环境 'refer' 声明
\zlatexFramed{refer}[orange]
% 使用环境 'refer'
\begin{refer}%
% something wrting here
\end{refer}
\end{minted}

\zlatexFramed{refer}[orange]
\begin{refer}%
As any dedicated reader can clearly see, the Ideal of practical
reason is a representation of, as far as I know, the things in themselves;

劳仑衣普桑,认至将指点效则机,最你更枝。想极整月正进好志次回总般,段然取向
使张规军证回,世市总李率英茄持伴。
\end{refer}

\begin{leftbar}
    在上面的refer环境开始时插入一个\%可以用于消除多余的空格
\end{leftbar}

\subsection{接口完善}
目前的z\LaTeX{}接口还不够丰富,没有进行相关的\cmd{Hook}(钩子)的声明,所以用户可以配置的选项是比较少的,
只要能够把导言区设置规范,那么剩下的内容你几乎是不用在设置了.

\section{z\LaTeX{}接口}
z\LaTeX{}的接口正在不断的完善中,所以目前的接口可能并不是那么稳定。(我已经尽力让接口规范和稳定了)

\subsection{命令声明}
后面我可能会考虑建立一个用于自定义命令的接口,采用键值对的方式进行配置,而不是默认的位置参数或者是xparse提供的可选,
默认,强制参数等。依托于\LaTeX3的\cmd{key-value}模块,这样的接口会更加的灵活,方便用户进行配置.

\subsection{盒子接口}
目前我打算通过\LaTeX3的盒子接口进行接口的定制,主要基于宏包\cmd{xcoffins},目前此宏包的接口已经稳定,详细信息请参见
\begin{itemize}
    \item \href{https://ctan.org/pkg/l3experimental}{l3experimental}
    \item \href{https://tex.stackexchange.com/a/397835/294585}{interface of xcoffins}
\end{itemize}

但是目前z\LaTeX{}的盒子接口还没有进行正式的配置. 如果用户能够学会使用盒子,(join box)拼接盒子.更进一步,用户能够对盒子进行
一些常规的操作:包括盒子的resize, scale, rotate. 那么你其实已经能够完成大部分的复杂操作了(除去少部分必须使用tikz). 而且一般
盒子操作对于\LaTeX{}的普通使用者来说可能是极为陌生的,于是要提供一个便于让用户使用,同时也要保证接口的稳定性的系列命令是不容易
的.

\subsection{TikZ接口}
本人并不打算在z\LaTeX{}中引进 \cmd{tikz} 宏包,因为我觉得这个宏包太过于庞大,很多的功能都不是一个文档类所必须的.
可能我会在后续引入\cmd{l3draw}\index{\cmd{l3draw}}模块用于TikZ操作. 

由于本模板可能会抛弃\cmd{tikz, tcolorbox}等宏包,所以我会在后续的版本中我会加入\cmd{xcoffins,l3draw,l3opacity}
等l3experimental宏包.然后基于这些宏包定制属于z\LaTeX{}文档类的命令,提供这些命令的用户接口,完善图形绘制功能.
这些宏包大部分还处于experiment阶段,所以目前z\LaTeX{}提供的接口也不会过于的稳定,请见谅. 同样的,\cmd{l3draw}其实还没有
完全实现PGF对应的全部功能,比如shade操作。

一般情况下有了\cmd{xcoffins}宏包后便可以进行大部分的操作了,如果使用者还需要绘制部分复杂的图形只能够借用\cmd{tikz}
宏包,不妨先在另一个文档类中生成对应的pdf文件,然后在正文中直接插入pdf即可.

\begin{leftbar}
本文档类配套的\cmd{ztikz}库提供了丰富的和TikZ绘图,数值计算,以及部分的图像处理功能.具体使用请参见下一个单元.
\end{leftbar}

\subsection{常用数据结构}
在\LaTeX{}3中其实已经提供了很多的数据结构,我们可以据此构造:\textbf{列表, 字典, 整形数组, 浮点数组}
等我们常用的数据结构.

\subsection{计数器}
目前的计数器部分继承自本文档类加载的子文档类(article, book),部分是使用\cmd{amsthm}\index{\cmd{amsthm}}宏包定义的
数学环境计数器,如 theorem, definition, corollary, example, axiom, remark等. 用户在使用过程中可以按照自己的需要使用
这部分已经声明的计数器.

目前z\LaTeX{}提供了一个命令\cmd{\zlatexUpdateCounterAfter}\index{\cmd{\zlatexUpdateCounterAfter}}用于设置
计数器的更新,使用格式为:
\begin{minted}{latex}
\zlatexUpdateCounterAfter{<child>}{<father>}
\end{minted}

也就是让上述的\cmd{<child>}计数器随着\cmd{<father>}父计数器的更新而更新,本命令的实现原型为:
\begin{minted}{latex}
\NewDocumentCommand{\zlatexUpdateCounterAfter}{mm}{
    \@addtoreset{#1}{#2}
}
\end{minted}

关于本文档类的公式计数器的说明,本文档类公式计数器默认跟随\cmd{section}计数器更新,在z\LaTeX{}的源码中的声明为:
\begin{minted}{latex}
\counterwithin{equation}{section}
\end{minted}

\section{自定义}
\subsection{封面}
本文档类并没有内建复杂的封面格式,只是简单的重定义了\cmd{\maketitle}命令用于生成封面. 
声明如下:
\begin{minted}{latex}
\renewcommand{\maketitle}{
    \begin{titlepage}
        \vfill\vspace*{40pt}
        \noindent\hspace*{134pt}\rule[-75pt]{6pt}{95pt}{\hspace*{10pt}\leavevmode\parbox[t]{25em}{\fontsize{25}{25}\selectfont\bfseries\@title}}\par
        \vspace*{-15pt}
        \noindent\hspace*{150pt}{\leavevmode\parbox[t]{20em}{\Large\bfseries\@author}}\par
        \vfill
        \noindent\hspace*{150pt}{\Large\textcolor{gray}{\@date}}
    \end{titlepage}} 
\end{minted}

如果使用者想要使用更加美观的封面,请手动加载\cmd{tikz}宏包,自己定义.

\subsection{目录}
尽管在z\LaTeX{}的加载选项一节便已经说明了z\LaTeX{}文档类默认加载了\cmd{titletoc}宏包用于目录的格式定制,
并且提供了对应的加载选项 \cmd{toc}已经对应的可选值\cmd{redef, 2column}, 但是在本文档类中并没有对目录的格式进行
更加深度的定制,可能后续会开发对应的接口. 

如果用户想要定制目录的格式可以使用本文默认加载的\cmd{titlesec}宏包或者是自行加载\cmd{tikz}宏包对目录进行定制.
这里给出第一种方案下的一个定制样例,效果可见\cref{fig:toc-example}:
\begin{minted}{latex}
% \RequirePackage{zhnumber}
\titlecontents*{chapter}[0pc]{\large\bfseries}
    {\textcolor{\tl_use:N \l__zlatex_link_color_tl}{\thecontentslabel}}{}
    {\normalfont\titlerule*[1pc]{.}\contentspage}[]
\titlecontents{section}[1.8pc]{\addvspace{3pt}\tl_use:N\g__sec_symbol_tl\,}
    {\textcolor{\tl_use:N \l__zlatex_link_color_tl}{\S\;\thecontentslabel}}{}
    {\titlerule*[1pc]{.}\contentspage}%[{\tocsym[sec]{}}]
\titlecontents{subsection}[3pc]{\addvspace{3pt}\tl_use:N\g__subsec_symbol_tl\,}
    {\textcolor{\tl_use:N \l__zlatex_link_color_tl}{\thecontentslabel}}{}
    {\titlerule*[1pc]{.}\contentspage}%[{\tocsym[subsec]{}}]
\titlecontents*{subsubsection}[5.8pc]{\small}
    {\textcolor{\tl_use:N \l__zlatex_link_color_tl}{\thecontentslabel}}{}
    {(\thecontentspage)}[\qquad]

% append of prepend item to toc
% switch left indent for subsection (none->3pc)
\tl_new:N \g__sec_symbol_tl
\tl_new:N \g__subsec_symbol_tl
\tl_set:Nn \g__sec_symbol_tl {}
\tl_set:Nn \g__subsec_symbol_tl {}
\NewDocumentCommand\tocsym{O{sec}O{}}{
    \str_case:nnF {#2}{
        {star}{ \tl_set:cn {g__#1_symbol_tl}{\ding{73}} }
        {=}{ \tl_set:cn {g__#1_symbol_tl}{\ding{104}} } %※
    }{\tl_set:cn {g__#1_symbol_tl}{#2}}
} 
\end{minted}

但是单独设置ToC中的样式往往是不够的,可能还需要设置不同计数器(章节)的格式,参见\cref{conter-settings}.

\subsection{页眉页脚}
本文档类采用\cmd{fancyhdr}进行页眉页脚的定制,目前已经写死在文档类中,如果使用者想要自定义页眉页脚,可以直接
重定义\cmd{\fancyhead, \fancyfoot}命令.或者是页面样式(pagestyle)对应的\cmd{fancy}样式. 后续会考虑添加对应的接口.

\subsection{章节格式}\label{conter-settings}
目前还不支持指定章节格式,等后续在添加. 使用者可以加载\cmd{titlesec, tocloft}等宏包进行自定义. z\LaTeX{}
文档类默认加载了\cmd{titlesec, titletoc}宏包用于章节格式和目录的格式定制. 如果使用
者想要自定义章节格式,直接使用\cmd{titlesec}宏包的\cmd{\titleformat}命令覆盖本模板的原始定义即可,
或者是其他的命令. 

在前面我们已经提到可以使用\cmd{titletoc}宏包对目录进行定制,这里给出与之相配的计数器格式设置。
\begin{minted}{latex}
% title format
\titleformat{\section}
    {\bfseries\centering\Large}
    {\thesection}
    {0ex}{}{}
\titleformat{\subsection}
    {\bfseries\large}
    {\thesubsection}
    {0ex}{}{}
\titleformat{\subsubsection}
    {\bfseries}
    {\thesubsubsection}
    {0ex}{}{}

% (sub)subsection number styel
\setcounter{secnumdepth}{5}
\renewcommand{\thesection}{{\thechapter.\arabic{section}\;}}
\renewcommand{\thesubsection}{{\zhnum{subsection}、}\hspace*{-1ex}}
\renewcommand{\thesubsubsection}{{\alph{subsubsection}.}\;}
\end{minted}

最终就可以得到如下样式的目录格式:
\begin{figure}[!htb]
    \centering
    \includegraphics[width=.32\linewidth]{./pics/ToC_1.pdf}
    \includegraphics[width=.32\linewidth]{./pics/ToC_2.pdf}
    \includegraphics[width=.32\linewidth]{./pics/ToC_3.pdf} 
    \caption{ToC Example}
    \label{fig:toc-example}
\end{figure}


\begin{leftbar}
但是本文档类默认不加载\cmd{tikz, pgf}宏包,想要使用这两个宏包定义更加复杂章节样式,请手动加载,并
设置自己喜欢的章节格式. 也许后续我会在z\LaTeX{}的加载选项中添加一个tikz选项,从而可以让用户自定义章节格式.
\end{leftbar}

\section{字体配置}\index{font-config}\label{font-config}
\subsection{一些术语}
在字体上,国内环境可以说是一团浆糊,再加之M\$ word对用户的不断误导与溺爱,导致无数的用户不知道字体,字体族,
字体系列,字形的区别. 下面给出一些常见注意事项:
\begin{itemize}
    \item 可以不正确的认为一个字体就是一个字体族,其中包含了一个字体系列
    \item 一般这个字形是对于西文字体而言的,对于中文无效. 中文所谓的斜体概念都是 word 这个毒瘤产生的
    \item 本来\TeX{}这玩意儿就是为了西文发明的,也就能理解对亚洲文字支持差
    \item 部分字体是给西文设计的,所以对中文才没有起作用.
    \item 做字体是一个吃力不讨好的活,没多少人愿意做
    \item 其实我们常说的字体,是指的Font Face.
    \item 深受M\$ Office迫害的用户很多喜欢Times New Roman字体,但是它并没有对应的数学字体(\cmd{newtxmath}不是).
    \item \textbf{注意字体版权:} Windows默认的字体全部都是有商业版权的。原则上,只有正版Windows用户方才可以用于非商业用途。如需用于商业,必须获得授权
\end{itemize} 

\subsection{模板预设}\index{模板字体预设}
字体配置一直以来都是一个比较头疼的问题,尤其是对于中文文档. z\LaTeX{}目前具有两套字体设置:英文和中文.
英文(\cmd{lang=en})下的字体配置为:
\begin{minted}{latex}
\RequirePackage[utf8]{inputenc}
\RequirePackage[T1]{fontenc}
\RequirePackage[english]{babel} 
\end{minted}

在中文(\cmd{lang=cn})环境下的字体配置为:
\begin{minted}{latex}
\RequirePackage[UTF8, heading]{ctex}
% 如果加载了ctexbook则为:
\LoadClass[oneside, 11pt]{book}
\end{minted}

除了上述的设置外,本模板目前没有进行任意字体设置. 但是为了方便用户,后续,本模板可能会提供一个
\cmd{\zlatexFontSetup}\index{\cmd{\zlatexFontSetup}}命令进行对应的模板字体设置,包含正文字体,数学字体等. 
在介绍此命令之前,首先介绍\LaTeX{}中调用外部字体或者是系统字体的方法.

\subsection{查看系统字体}\index{查看系统字体}
首先介绍正文字体,正文字体的设置基于\cmd{fontspec}宏包, 可以解决在\XeLaTeX{}, Lua\LaTeX{}下的字体配置问题.此宏包提供了如下
命令用于正文中各种字体族的设置\index{\cmd{\setmainfont}}\index{\cmd{\setsansfont}}\index{\cmd{\setmonofont}}:
\begin{itemize}
    \item \cmd{\setmainfont}: 设置正文的主要字体(罗马字体), 对应命令\cmd{\textrm}
    \item \cmd{\setsansfont}: 设置正文的无衬线字体, 对应命令\cmd{\textsf}
    \item \cmd{\setmonofont}: 设置正文的等宽字体(打字机字体), 对应命令\cmd{\texttt}
\end{itemize}

\begin{leftbar}
如果想要在pdf\TeX{}下使用别的字体(你自己下载的字体,系统中别的字体),那么最好的方法是使用别人写好的宏包。因为
在pdf\TeX{}下使用别的字体,可能会需要一些\cmd{.tfm, .fd, .map, .sty}等文件,这个对于一般的\LaTeX{}使用者来说过于复杂了.
可以参见网站:\href{https://tug.org/FontCatalogue/allfonts.html}{FontCatalogue}, 在这个网站上你可以看到很多的为pdf\TeX{}
设计的字体宏包,然后你根据自己的喜好进行对应宏包的设置即可.
\end{leftbar}

\begin{figure}[!htb]
    \centering
    \includegraphics[width=.75\linewidth]{./pics/pdftex_font_config.png}
    \caption{Font Catalogue}
    \label{fig:FontCatalogue}
\end{figure}


\begin{leftbar}
在设置全文的中(日韩)字体时也请使用命令\cmd{\setCJKmainfont}命令,格式和上述的命令一样. 也许在设置文档的中文字体时你还需要在导言区
加载如下命令:
\end{leftbar}
\begin{minted}{latex}
\renewcommand\CJKrmdefault{}
\end{minted}

这几个命令可以用于设置整个文档的字体,但是同样的是,我们可以在一个文档中使用多个字体. 在讨论这个的时候,
我们遇到了一个问题? 怎么查看自己电脑上有哪些可用的字体(包括数学字体)? 在Windows和Linux均可以通过如下命令进行
查看:
\begin{minted}{shell}
fc-list 
\end{minted}

然后你就会看到如下的类似输出, 在Linux下可能是这样的:
\begin{minted}{text}
/usr/share/fonts/100dpi/lubI24.pcf.gz: B&H LucidaBright:style=Italic
/usr/share/fonts/100dpi/UTBI__12-ISO8859-1.pcf.gz: Adobe Utopia:style=Bold Italic
/usr/share/fonts/75dpi/luIS08-ISO8859-1.pcf.gz: B&H Lucida:style=Sans Italic
/usr/share/fonts/75dpi/luIS10-ISO8859-1.pcf.gz: B&H Lucida:style=Sans Italic
/usr/share/fonts/75dpi/luIS14-ISO8859-1.pcf.gz: B&H Lucida:style=Sans Italic
/usr/share/fonts/100dpi/luRS18-ISO8859-1.pcf.gz: B&H Lucida:style=Sans
/usr/share/fonts/75dpi/courB18-ISO8859-1.pcf.gz: Adobe Courier:style=Bold
/usr/share/fonts/100dpi/lubBI24.pcf.gz: B&H LucidaBright:style=Italic
/usr/share/fonts/100dpi/courO12.pcf.gz: Adobe Courier:style=Oblique
\end{minted}

在windows上可能是这样的(如果安装了\TeX{}Live):
\begin{minted}{text}
C:/WINDOWS/fonts/times.ttf: Times New Roman:style=Regular,Normal, ...
C:/WINDOWS/fonts/GeoSlab703 Md BT Bold.ttf: GeoSlab703 Md BT:style=Bold
C:/WINDOWS/fonts/vgasysr.fon: System:style=Regular
C:/WINDOWS/fonts/seguibl.ttf: Segoe UI,Segoe UI Black:style=Black,Regular
C:/texlive/2024/texmf-dist/fonts/opentype/public/drm/drmsl11.otf: drmsl11:style=Regular
C:/texlive/2024/texmf-dist/fonts/opentype/public/fonts-tlwg/Garuda-Oblique.otf: Garuda:style=Oblique
C:/texlive/2024/texmf-dist/fonts/truetype/public/junicodevf/JunicodeVF-Roman.ttf: Junicode VF:style=Exp Bold
C:/texlive/2024/texmf-dist/fonts/opentype/public/qualitype/QTTechtone-BoldItalic.otf: QTTechtone:style=BoldItalic
C:/texlive/2024/texmf-dist/fonts/opentype/public/tempora/Tempora-Italic.otf: Tempora:style=Italic
C:/texlive/2024/texmf-dist/fonts/opentype/public/drm/drmsym11.otf: drmsym11:style=Regular
\end{minted}

我们可以通过 ``font name'' 和  ``file name'' 两种形式来调用字体. 比如我们想要调用 ``Times New Roman'' 字体,
在第一个输出示例中,可以看到 ``Times New Roman'' 对应的文件名为 ``times.ttf'', 那么我们可以通过如下命令进行调用:
\begin{minted}{latex}
% by file name 
\setmainfont{times.ttf}

% by font name
\setmainfont{Times New Roman}
\end{minted}

\begin{leftbar}
如果你不想在你的本地安装过多的字体,我常常会建议使用``file name''的格式.只需要把对应的字体文件放到你的项目文件夹下就行了,
免去了字体安装这一个步骤.这个时候你就需要在可选参数中指定键\cmd{Path}对应的值了,对应的是项目下的字体路径. 这里给出一个示例:
假如你把字体放到了项目根路径下的\cmd{Fonts}文件夹,那么声明格式为:
\end{leftbar}
\begin{minted}{latex}
\setmainfont[Path = Fonts/]{<font name or file name>}
\end{minted}

在安装字体后务必使用如下命令刷新系统字体缓存,否则\cmd{fontspec}或\cmd{xeCJK}宏包无法找到系统中对应的字体:
\begin{minted}{shell}
fc-cache -f -v
\end{minted}

通过上面的两条命令我们便可以轻松的设置文档中\cmd{\textrm{<content>}}对应的字体\Footnote{默认情况下,文档中的字体族即为roman family}
为``Times New Roman''.

\subsection{字体系列不全}\index{字体系列}
但是你可能也会遇到如下的编译警告:
\begin{minted}{text}
LaTeX Font: Font shape `TU/AlegreyaSans-ExtraBoldItalic.otf(0)/b/n' undefined
(Font)	using `TU/AlegreyaSans-ExtraBoldItalic.otf(0)/m/n' instead.
\end{minted}

上述的字体是我在导言区设置了如下字体命令导致的:
\begin{minted}{latex}
\setmainfont{AlegreyaSans-ExtraBoldItalic.otf}
\end{minted}

为什么会有这个警告? 其实就是出在一个字体加粗命令\cmd{\textbf}上,我并没有指定这个新的文章中字体族(rmfamily)所对应的
粗体字体. 所以\TeX{}采用了默认的 ``m(edium)'', 而不是 ``b(old)''. 解决这个问题有两个方法:
\begin{itemize}
    \item 通过``font name''进行设置的情况下,如果你的本地字库够全,那么\TeX{}会自动处理这个问题.
    \item 通过``file name''进行设置的情况下,你可以通过如下命令进行设置,这里以\cmd{comic.ttf}为例:
\end{itemize}

\begin{minted}{latex}
\setmainfont{comic.ttf}[
    BoldFont=comicbd.ttf,
    ItalicFont=comici.ttf,
    BoldItalicFont=comicz.ttf
]
\end{minted}

如果你在Windows命令行使用如下命令:
\begin{minted}{shell}
fc-list | rg comic | rg WINDOWS
\end{minted}

那么可以得到如下的输出:
\begin{minted}{shell}
C:/WINDOWS/fonts/comicbd.ttf: Comic Sans MS:style=Bold
C:/WINDOWS/fonts/comici.ttf: Comic Sans MS:style=Italic
C:/WINDOWS/fonts/comicz.ttf: Comic Sans MS:style=Bold Italic
\end{minted}

说明在本地的名为``Comic Sans MS''的字体族具有``Bold, Italic, Bold Italic''三种字形. 通过下面的命令我们便可以
简单的通过第一种方式解决这个字体加粗问题:
\begin{minted}{latex}
\setmainfont{Comic Sans MS}
\end{minted}

\begin{leftbar}%
就像上述备注的第二种``file name''方式的好处外,By ``font name'' 也有上述的优点. 尽管对于字体的加粗和斜体你可以偷懒,
在字体族声明的可选参数中加入\cmd{AutoFakeBold, AutoFakeSlant}参数,它会自动同时实现中英文伪粗体和伪斜体,不用你自己单独去指定。
但是我并不推荐. 还是给出一个示例:
\end{leftbar}

\begin{minted}{latex}
\setCJKfamilyfont{AutoFakeBoldSlant}{<font-name>.ttf}[AutoFakeBold , AutoFakeSlant]
\end{minted}

在通过 ``font name''进行字体的声明时,可以考虑把字体放到如下的位置,这样可以避免书写字体的路径:
\begin{itemize}
    \item MacOS:\cmd{~/Library/Fonts}
    \item Windows:\cmd{C:\Windows\Fonts}
    \item Linux 或 Windows: 放在TEXMF tree对应的路径, 见前文.
\end{itemize}

\subsection{字体族声明}\index{字体族声明}
上述的三个命令都是设置全文对应的字体族,如果你想要改变局部的字体族,那么可以使用\cmd{fontspec}宏包提供的
命令\cmd{\newfontfamily}和\cmd{XeCJK}宏包提供的\cmd{\setCJKfamilyfont}分别进行英文与中日韩文字的字体族声明.
二者的声明方式为:
\begin{minted}{latex}
\newfontfamily{\FamilyA}{FontA.ttf}
\setCJKfamilyfont{FamilyB}{FontB.ttf}
\end{minted}

上述两条命令声明了\cmd{FamilyA, FamilyB}两个字体族,分别对应了\cmd{FontA.ttf, FontB.ttf}两个字体文件.想要使用
这两个字体族,在文档中按照如下方式使用:
\begin{minted}{latex}
{\FamilyA <Your Content>}
{\CJKfamily{FamilyB} <你的内容>}
\end{minted}

比如你需要在文档中输入俄语,由于\cmd{article}文档类默认的字体Computer modern并没有包含这些俄语字母字形(Glyph).
所以你可能就需要使用一个含有这类字形的字体了,推荐\cmd{CMU Serif}。可以安装进系统也可以放在项目文件夹下,这里放在
项目文件夹\cmd{./Fonts/}下为例:
\begin{minted}{latex}
\newfontfamily{\russia}[Path=./Fonts/]{cmunrm.ttf}[
    BoldFont=cmunbx.ttf,
    ItalicFont=cmunbi.ttf
]
\end{minted}

然后使用如下格式调用:
\begin{minted}{latex}
{\russia <俄语内容>}
\end{minted}

\begin{leftbar}
{\russia Жизнь, как прекрасная мелодия, только текст перепутались.}\par 
\noindent{\kaishu 生命是首美丽的曲子,只是歌词有些纠结}
\end{leftbar}


当然了,你也可以通过\cmd{\setmainfont}命令,从而进行全局字体设置. 最后还是给出声明中文字体族的命令:
\begin{minted}{latex}
\setCJKfamilyfont{fangsong}{simfang.ttf}[
    Path=./Fonts/,
    BoldFont=simfangbold.ttf
]
\end{minted}

然后使用如下格式调用:
\begin{minted}{latex}
{\CJKfamily{fangsong} <你的中文内容>}
\end{minted}

\subsection{Emoji}\index{Emoji}
如果用户想要使用Emoji,在阅读完上述的字体配置后,相信读者已经能够独自解决此问题了. 这里给出大致的解决思路:
\begin{itemize}
    \item 把对应的Emoji对应的unicode放到一个支持显示表情包的字体组环境中.
    \item 加载\href{http://mirrors.ctan.org/fonts/fontawesome5/doc/fontawesome5.pdf}{fontawesome5}宏包.
    \item 更换编译引擎为Lua\TeX.
\end{itemize}


\subsection{数学字体}\label{数学字体}\index{数学字体}
目前z\LaTeX{}文档类提供了几种常见的数学字体宏包接口,分别是\cmd{computer moder math, newtxmath, eulervm, mtpro2}. 
其中\cmd{computer modern math}即为默认的数学字体,推荐新手使用. 数学字体并不仅仅包含我们普通群众认为的Glyph本身,
一套数学字体往往还需要对应的``距离表'',用于指定各个数学符号之间的距离. 数学字体还包含很多复杂的东西,并没有你认为的那么简单.

不想要折腾的用户可以直接看 \href{http://mirrors.ctan.org/info/Free_Math_Font_Survey/en/survey.pdf}{Free-Math-Font}这个 
文档,你配置不来,但是你至少得做一个合格得调包侠吧.但是这里不回去介绍上述接口对应宏包的使用方法,下面重点介绍
宏包\href{http://mirrors.ctan.org/macros/unicodetex/latex/unicode-math/unicode-math.pdf}{unicode-math}. 

\cmd{unicode-math}\index{\cmd{unicode-math}}宏包的基本使用格式为:
\begin{minted}{latex}
% after \usepackage{fontspec}
\usepackage{unicode-math}
% set main math font (necessary)
\setmathfont{Latin Modern Math}
\end{minted}

上述的命令\cmd{\setmathfont}\index{\cmd{\setmathfont}}由宏包\cmd{unicode-math}提供,用于设置文档中数学字体. 在指定文档的数学字体后,你可以
再声明一系列的局部数学字体命令. 但是在这之前,请允许我先介绍一下怎么查看你的系统中有哪些数学字体. 

以Windows为例,在命令行中运行命令:
\begin{minted}{shell}
fc-list | rg Math
\end{minted}

然后你可以得到大致如下的结果:
\begin{minted}{text}
C:/WINDOWS/fonts/eumat2.ttf: Euclid Math Two:style=Regular
C:/WINDOWS/fonts/eumat2b.ttf: Euclid Math Two:style=Bold
C:/WINDOWS/fonts/cambria.ttc: Cambria Math:style=Regular
C:/texlive/2024/texmf-dist/fonts/opentype/public/xits/XITSMath-Bold.otf: XITS Math:style=Bold
C:/texlive/2024/texmf-dist/fonts/opentype/public/xits/XITSMath-Regular.otf: XITS Math:style=Regular
C:/texlive/2024/texmf-dist/fonts/opentype/public/tex-gyre-math/texgyretermes-math.otf: TeX Gyre Termes Math:style=Regular
\end{minted}

和前文的设置正文字体一样,可以通过``font name'' 或者是 ``file name''进行数学字体的设置, 所以在这里不再赘述. 
这里介绍一下怎么设置局部数学字体. 通过如下命令:
\begin{minted}{latex}
\setmathfont[range={\mathscr,\mathbfscr}]{TeX Gyre Termes Math}
\end{minted}

在此命令之后的所有\cmd{\mathscr, \mathbfscr}命令均会使用``TeX Gyre Termes Math''字体. 其他不变, 比如\cmd{\mathrm, \mathbf}等.
后续如果还想使用其他数学字体,只需要再次声明不同的\cmd{\setmathfont}命令即可.

\begin{leftbar}
值得注意的是:在加载宏包\cmd{unicode-math}后,数学字符的加粗请采用\cmd{\boldsymbol}或\cmd{\symbf, \symbfit, \symbfup}等命令.
\end{leftbar}

关于\cmd{unicode-math}宏包的更多功能,请参见对应的宏包手册. 


\subsection{常见字体问题}\index{常见字体问题}
有了上述的字体配置基础知识的介绍,你应该能看懂部分的模板中的字体配置命令了. 在看懂这些命令后,也就能够知道一些常见的字体问题的背后原因了:
\begin{itemize}
    \item 为什么Windows下中文加粗变成了黑体?
    \item 为什么\TeX{}会报字体缺失的警告或这是错误?
    \item 怎么在Windows上使用Linux下的字体?
    \item 怎么使用Founder字体 ?
    \item ...
\end{itemize}

就比如Windows下加粗便黑体的问题\Footnote{具体参考请见:\href{https://zhuanlan.zhihu.com/p/538459335}{LaTeX 中文字体配置基础指南}},
如果你去看\cmd{ctex-fontset-windows.def}这个文件,由如下的声明:
\begin{minted}{latex}
% 中文默认字体:宋体,粗体以黑体代替,斜体以楷书代替
\setCJKmainfont   { SimSun } [ BoldFont = SimHei , ItalicFont = KaiTi ]
% 中文无衬线字体:微软雅黑,粗体为对应的微软雅黑粗体
\setCJKsansfont   { Microsoft~YaHei } [ BoldFont = *~Bold ]
% 中文等宽字体:仿宋
\setCJKmonofont   { FangSong }
% 设置中文字族
\setCJKfamilyfont { zhsong  } { SimSun          }
\setCJKfamilyfont { zhhei   } { SimHei          }
\setCJKfamilyfont { zhfs    } { FangSong        }
\setCJKfamilyfont { zhkai   } { KaiTi           }
\setCJKfamilyfont { zhyahei } { Microsoft~YaHei } [ BoldFont = *~Bold ]
\setCJKfamilyfont { zhli    } { LiSu            }
\setCJKfamilyfont { zhyou   } { YouYuan         }
% 字体命令,可用于文档中自由设置字体
\NewDocumentCommand \songti   { } { \CJKfamily { zhsong  } }
\NewDocumentCommand \heiti    { } { \CJKfamily { zhhei   } }
\NewDocumentCommand \fangsong { } { \CJKfamily { zhfs    } }
\NewDocumentCommand \kaishu   { } { \CJKfamily { zhkai   } }
\NewDocumentCommand \lishu    { } { \CJKfamily { zhli    } }
\NewDocumentCommand \youyuan  { } { \CJKfamily { zhyou   } }
\NewDocumentCommand \yahei    { } { \CJKfamily { zhyahei } }
\end{minted}

上面有一句\cmd{BoldFont = SimHei}, 就表示\cmd{\textbf{<文字>}}中的内容是``黑体''。但是在Linux下对应的设置为:
\texttt{BoldFont = FandolSong-Bold},这就是正真的粗体了.

比如Elegant\LaTeX{}系列中的Book文档类,你能在其中看到如下的字体声明命令:
\begin{minted}{latex}
% 英文部分
\setmainfont{TeXGyreTermesX}[
    UprightFont = *-Regular ,
    BoldFont = *-Bold ,
    ItalicFont = *-Italic ,
    BoldItalicFont = *-BoldItalic ,
    Extension = .otf ,
    Scale = 1.0
]
\setsansfont{texgyreheros}[
    UprightFont = *-regular ,
    BoldFont = *-bold ,
    ItalicFont = *-italic ,
    BoldItalicFont = *-bolditalic ,
    Extension = .otf ,
    Scale = 0.9
]

% 中文部分
\RequirePackage[UTF8, scheme=plain, fontset=none]{ctex}
\setCJKmainfont[BoldFont={FZHei-B01},ItalicFont={FZKai-Z03}]{FZShuSong-Z01}
\setCJKsansfont[BoldFont={FZHei-B01}]{FZKai-Z03}
\setCJKmonofont[BoldFont={FZHei-B01}]{FZFangSong-Z02}
\setCJKfamilyfont{zhsong}{FZShuSong-Z01}
\setCJKfamilyfont{zhhei}{FZHei-B01}
\setCJKfamilyfont{zhkai}[BoldFont={FZHei-B01}]{FZKai-Z03}
\setCJKfamilyfont{zhfs}[BoldFont={FZHei-B01}]{FZFangSong-Z02}
\newcommand*{\songti}{\CJKfamily{zhsong}}
\newcommand*{\heiti}{\CJKfamily{zhhei}}
\newcommand*{\kaishu}{\CJKfamily{zhkai}}
\newcommand*{\fangsong}{\CJKfamily{zhfs}}
\end{minted}

所以C\TeX{}还是给我们解决了很多的字体问题的,如果你的文档中加载了\cmd{ctex}宏包,不妨使用其提供的\cmd{<fontset>}
选项进行字体的设置. 实在不行,在考虑自行配置字体.

\subsection{About Future}
本节并没有介绍在pdf\TeX{}下的字体配置方法,因为如果你想在此引擎下使用字体,那么你可能需要自己去做一些复杂的字体工作,
自己去做对应的\texttt{tfm, map, enc, vf}文件,这并不简单. 

将来我可能会介绍一点关于在PDF中抽取现成的字体,或者是虚拟字体相关的工作. 但是这些都是后话了.所以如果你想要了解更多字体相关的知识,
可以多看看文档\cmd{fontspec, ctex}或者是{The \LaTeX{} Companion}这本书中的内容.

还有一些FallBack字体(主字体文件中不存在字符的备用字体)的问题? 比如在Windows下可以使用其自带的一个超大字符集\cmd{SimSun-ExtB},设置命令
如下:
\begin{minted}{latex}
\xeCJKsetup{AutoFallBack=true}
\setCJKmainfont{SimSun}[FallBack=SimSun-ExtB]
\end{minted}

后续可能就是一些OpenTrueType字体或者是Unicode的坑了,即使是换了Lua\TeX{}这些坑你还是得踩的. 看看有生之年能不能
用MetaFont或者是FontForge, Glyphs之类的工具设计一个自己的字体.

\section{数学环境}
\subsection{常用数学环境}\label{常用数学环境}
本文档类使用宏包\cmd{amsthm}定义了如下数学环境;大致分为两类: 定理类环境和证明类环境;默认情况下定理类环境和证明类环境
相同.具体的环境名称见下方:

\begin{multicols}{2}
\begin{itemize}
    \item 定理类环境
        \begin{itemize}
        \item axiom
        \item definition
        \item theorem 
        \item lemma
        \item corollary 
        \item proposition
        \item remark 
        \end{itemize}
    \item 证明类环境
    \begin{itemize}
        \item proof
        \item exercise
        \item example
        \item solution
        \item problem
    \end{itemize}
\end{itemize}    
\end{multicols}

常用的数学环境本模板基本已经覆盖,如果你有其他的数学环境需求,可以自行添加. 但是在添加之前请先
仔细阅读本文档,确保没有冲突之处。后续可能会开发一个对应的接口用于这类环境的统一声明和管理.

\subsection{使用方法}
现在介绍怎么使用这些具体的内置数学环境,上述的每一个环境的基本调用格式如下:
\begin{minted}{latex}
\begin{<theorem like env>}[<theorem name>]
你的定理内容就写在这个环境的内部.

your theorem writing here. 
\end{<theorem like env>}
\end{minted}

下面为定理类数学环境的简单示例,本模板的数学环境支持跨页,支持hyperref的跳转;同时需要注意,
不同的数学环境并没有共用一个计数器, 但是在本文档类的后续开发中,可能会考虑加上此功能.

想要对定理类环境添加\cmd{label}的语法如下:
\begin{minted}{latex}
\begin{<theorem like env>}[<theorem name>]\label{thm:test}
你的定理内容就写在这个环境的内部.
    
your theorem writing here. 
\end{<theorem like env>}
\end{minted}

后续引用直接使用命令\cmd{\cref{thm:test}}, 比如引用刚才标记的 \cref{thm:test},
可以看到,这个是可以精确跳转到对应的定理处的. 同时本模板中的\cmd{\cref}\index{\cmd{\cref}} 命令会自动根据计数器的类别
和文档的语言选项决定具体的引用格式. 针对于图表的引用也是同理的,你只需要把这一切都交给\cmd{\cref}即可. 相关的详细信息还请参见
本文档后面部分的\cmd{标签与引用}.


\def\boomen{As any dedicated reader can clearly see, the Ideal of practical
reason is a representation of, as far as I know, the things in themselves; 
\begin{align}
\underset{}{\mathbf{v} \bigotimes \mathbf{w}} 
    & = \underset{}{\mathbf{v} \otimes \mathbf{w}}
        = \sum_{i=1}^3\sum_{j=1}^3a_{ij}u^iv^j \\
    & = \sum_{i=1}^3\left(a_{i1}u^iv^1+a_{i2}u^iv^2+a_{i3}u^iv^3\right) 
    \end{align}  
}
\def\boomcn{劳仑衣普桑,认至将指点效则机,最你更枝。想极整月正进好志次回总般,段然取向
使张规军证回,世市总李率英茄持伴。}

\subsection{定理类环境样式}
z\LaTeX{}中的数学环境有3套样式\index{数学环境样式},分别为\cmd{plain,leftbar,background,fancy}. 其中\cmd{plain}表示数学环境不加载任何的修饰,
\cmd{leftbar}表示数学环境的左侧使用\cmd{framed}宏包提供的\cmd{leftbar}命令进行修饰,\cmd{background}表示给对应的段落加上色彩背景,\cmd{fancy}表示
数学环境加载\cmd{leftbar}的同时设置其背景颜色为对应颜色的10\%. 

只需要用户在加载本文档类时指定\cmd{env-style}选项即可,比如本示例文档的数学环境主题为\cmd{leftbar}(默认样式为\cmd{plain}):
\begin{minted}{latex}
\documentclass[
    mathSpec={envStyle=leftbar}
]{zlatex}
\end{minted}

本模板没有加载Tcolorbox宏包用于这部分盒子的设计, 仅使用了一个比较简单的framed宏包。但是对于一篇普通的笔记或者是自己的文章
来说,你并不需要这些多余的色彩高亮盒子.

\def\mstyle#1{\noindent\lower.25ex\hbox{\ding{224}}\;\textbf{#1}\par}
\mstyle{plain样式}
\ExplSyntaxOn
\DeclareDocumentEnvironment{zlatexTheoremLikeFrame}{O{}}{\vspace*{5pt}}{\vspace*{5pt}}
\ExplSyntaxOff
\begin{theorem}[prime theorem]\label{thm:test}
    \boomen \par 
    \boomcn
\end{theorem}

\begin{definition}[prime definition]
    \boomen \par 
    \boomcn
\end{definition}

\mstyle{leftbar样式}
\ExplSyntaxOn
\DeclareDocumentEnvironment{zlatexTheoremLikeFrame}{O{black}}{
    \def\FrameCommand{{\color{#1}\vrule width 3pt}\hspace{5pt}}
    \MakeFramed {\advance\hsize-\width \FrameRestore}
}{\endMakeFramed}
\ExplSyntaxOff
\begin{lemma}[prime lemma]
    \boomen \par 
    \boomcn
\end{lemma}

\begin{remark}[prime remark]
    \boomen \par 
    \boomcn
\end{remark}


\mstyle{background样式}
\ExplSyntaxOn
\DeclareDocumentEnvironment{zlatexTheoremLikeFrame}{O{black}}{
    \def\FrameCommand{\colorbox{#1!10}}
    \MakeFramed {\advance\hsize-\width \FrameRestore}
}{\endMakeFramed}
\ExplSyntaxOff
\begin{lemma}[prime lemma]
    \boomen \par 
    \boomcn
\end{lemma}

\begin{remark}[prime remark]
    \boomen \par 
    \boomcn
\end{remark}


\mstyle{fancy样式}
\ExplSyntaxOn
\DeclareDocumentEnvironment{zlatexTheoremLikeFrame}{O{black}}{
    \def\FrameCommand{{\color{#1}\vrule width 3pt}\colorbox{#1!10}}
    \MakeFramed{\advance\hsize-\width \FrameRestore}
}{\endMakeFramed}
\ExplSyntaxOff

\begin{axiom}[prime axiom]
    \boomen \par 
    \boomcn
\end{axiom}

\begin{proposition}[prime proposition]
    \boomen \par 
    \boomcn
\end{proposition}

\ExplSyntaxOn
\DeclareDocumentEnvironment{zlatexTheoremLikeFrame}{O{black}}{
    \def\FrameCommand{{\color{#1}\vrule width 3pt}\hspace{5pt}}
    \MakeFramed {\advance\hsize-\width \FrameRestore}
}{\endMakeFramed}
\ExplSyntaxOff

\subsection{证明类环境}
证明类环境比较朴素,没有可选的默认参数,使用方法见下:
\begin{minted}{latex}
\begin{<proof like env>}
    定理内容.
\end{<proof like env>}
\end{minted}
\vspace*{-2em}

\begin{proof}
    \boomen \par 
    \boomcn
\end{proof}

\begin{example}
    \boomen \par 
    \boomcn
\end{example}

你可以自行定制Proof环境的结束标志,但是需要注意的一点是:你的标志必须放入公式环境,如果你的结束标志
只能用于公式环境时. 例如,把证明结束符从 \(\blacksquare\) 替换为 $\square$:
\begin{minted}{latex}
\renewcommand{\qedsymbol}{\ensuremath{\square}}
\end{minted}


\subsection{注意事项}
默认的数学类环境均采用正体\cmd{\upshape},如果使用者不喜欢前者默认的``正体''字体样式,
可以直接在数学类环境开始时使用字体命令\cmd{\itshape}进行原有字体样式的覆盖,示例如下:

\begin{minted}{latex}
\begin{theorem}[test theorem]\itshape
    你好, Hello world !
\end{theorem}
\end{minted}

\begin{remark}\itshape
    \boomen \par 
    \boomcn
\end{remark}

同时,本文档类中数学类环境和前文的自定义高亮环境\cmd{\zlatexFramed}均默认首行不缩进,需手动添加缩进.

\subsection{自定义数学环境}
目前还没有开发对应的接口,主要是目前的格式基本已经够用了.


\section{标签与引用}
\subsection{Footnote}
可能有人不喜欢默认的脚注没有在页脚的位置,而是在页脚偏上的位置,用户可以独立加载宏包\cmd{footmisc}用于
强制脚注位于页面底部,本文档类不打算添加此宏包,用户可以自行在导言区添加如下命令:
\begin{minted}{latex}
\usepackage[bottom]{footmisc}
\end{minted}

\subsection{Cleveref}
z\LaTeX{}文档类加载了\cmd{cleveref}宏包来构建标签-引用系统。常规的\cmd{\label{}}操作并没有什么变化,
区别主要在引用标签功能上。对于普通的模板你可能会看到如下的说明: 使用\cmd{\eqref}进行公式标签的索引,
使用\cmd{\figref}进行图片的索引,使用\cmd{\tabref}进行表格的索引... 使用此命令可以避免书写如下
格式的引用代码:

\begin{minted}{latex}
定理:\ref{thm:test}
% or 
\newcommand\thmref{定理:\ref{#1}}
\end{minted}

在z\LaTeX{}中,引用格式预设值如下(至于多个标签引用时,只有\cmd{lang=en}时采用部分变化,对应的前缀变为复数):
\begin{table}[H]
    \centering
    \begin{tabular}{cccccccc}
        \toprule
        语言 & 公式 & 图片 & 表格 & part & chapter & section & subsection\\
        \midrule
        \cmd{lang=en} & equation & figure & table & part & chapter & section & subsection \\
        \cmd{lang=cn} & 方程 & 图 & 表 & 部分 & 章 & 节 & 小节\\
        \bottomrule
    \end{tabular}
    \caption{cref引用格式}
    \label{tab:sys-cref}
\end{table}

对于\cmd{cleveref}中的其它命令,如\cmd{\Cref}, \cmd{\crefrange}, \cmd{\Crefrange}等等,
本文档类未对其进行修改,所以以上命令均是兼容的,详细的使用说明请参见\cmd{cleveref}宏包的官方文档.

\subsection{图片与(列)表}
z\LaTeX{}采用\cmd{cleveref}提供的引用命令,本文档类内置的\cmd{\cref}命令的用法和
原始宏包中的\cmd{\cref}\index{\cmd{\cref}}的用法是一样的,只是在引用的时候会根据文档的语言选项进行
对应的prefix更改.比如在\cmd{lang=cn}时把默认的\cmd{fig 1.1}改为中文环境下的 \cmd{图 1.1}.

这其实也就意味着,本文档类中还可以使用\cmd{cleveref}提供的所有的引用命令,
比如\cmd{\Cref, \crefrange, \Crefrange}等等.更多的详细信息可以参见\cmd{cleveref}的
官方文档.

\section{列表环境}
z\LaTeX{}对\cmd{book}文档类的无编号计数器进行了定制,有序列表和无序列表现在的具体样式如下:

\begin{multicols}{2}
    \begin{itemize}
        \item 一级项目
        \begin{itemize}
            \item 二级项目
            \begin{itemize}
                \item 三级项目
            \end{itemize}
        \end{itemize}
    \end{itemize}
    
    \begin{enumerate}
        \item 一级项目
        \begin{enumerate}
            \item 二级项目
            \begin{enumerate}
                \item 三级项目
            \end{enumerate}
        \end{enumerate}
    \end{enumerate}
\end{multicols}


\section{文献引用}
\subsection{基本设置}
这里说明部分关于文献引用的知识.如果你想要把``参考文献''栏目加入目录,可以使用
命令\cite{ahlfors1953complex}:

\begin{minted}{latex}
\addcontentsline{toc}{chapter}{参考文献} % or
\addcontentsline{toc}{chapter}{Bibliography}
\end{minted}

或者是可以这样设置:
\begin{minted}{latex}
\printbibliography[heading=bibnumbered, title={参考文献}]
\end{minted}

其中\cmd{heading=bibnumbered}表示参考文献页的标题是编号的,\cmd{title}表示标题的名称.
其实还有一种方法可用,不过没有前面的两种那么的便捷,而且也并不是那么的完美:
\begin{minted}{latex}
% \usepackage{zhnumber}
\addtocounter{chapter}{1}
\addcontentsline{toc}{chapter}{第\zhnum{chapter}章~~参考文献}
\end{minted}

这样做的缺点就是,在pdf书签和目录中均会会显示为\texttt{第五章  参考文献},但是我们是不需要在
前者中显示``第五章''的,所以这种方法并不是那么的完美.

\subsection{使用样例}
使用\cmd{\cite{<ref>}}进行参考文献的引用, 然后使用命令\cmd{\printbibliography}输出参考文献,
或者是使用\cmd{\nocite{*}}输出\cmd{.bib}中的所有条目.下面是一个简单的例子:

\begin{minted}{latex}
% 参考文献: ref.bib
@book{ahlfors1953complex,
    title={Complex Analysis},
    author={Ahlfors},
    year={1953},
    publisher={McGraw-Hill},
    address={New York}
}

% 正文引用
\cite{ahlfors1953complex}
\end{minted}

默认参考文献页的名称为:Bibliography,若想要自定义名称,可以在输出参考文献前使用如下命令:
\begin{minted}{latex}
\renewcommand{\bibname}{参考文献}
\printbibliography
\end{minted}


\section{索引}
\subsection{使用方法}
z\LaTeX{}文档类采用\cmd{indextools}宏包进行索引的生成,并不没有采用传统的\cmd{makeidx}宏包.
具体的用法和\cmd{indextools}宏包的一致,这里给一个简单的示例:

\begin{minted}{latex}
% 导言区
\makeindex[title=Concept index]
% 添加索引到目录,生成索引
\addcontentsline{toc}{part}{Index}
\printindex
\end{minted}

或者是你可以在你文档的导言区声明某种\cmd{index}的类型,比如\cmd{person},然后就可以在文中使用
\cmd{\index[person]{<the person>}} 来进行索引,最后使用如下命令进行索引的打印和索引的导言区
定制:

\begin{minted}{latex}
% 导言区
\makeindex[name=person, title=Index of names, columns=3]
% 文档末尾
\indexprologue{In this index you’ll find only famous people’s names}
\printindex[person]
\end{minted}

使用\cmd{\index}命令时在此命令中的名词是不会显示在PDF文档中的,所以如果你要添加一个``函数''
的index项目时,在你的\TeX{}文档中应该这样写:

\begin{minted}{latex}
函数\index{函数}是从集合到 ...
\end{minted}

若要定制\cmd{\printindex}输出索引条目格式,可以在导言区进行一定的设置,\cmd{indextools}
宏包提供了命令\cmd{\indexsetup{}}用于格式设定, 采用键值对的格式进行指定. 可用的键有:
\cmd{level, toclevel, ...},详情请参见文档. 默认的\cmd{level}为\cmd{\chapter*(\section* 在article中)},
表示此章节为无编号章节. 我们这里设置为有编号章节,level为chapter:
\begin{minted}{latex}
\indexsetup{level=\chapter}
\end{minted}

或者是和前面的\cmd{\printbibliography}类似,采用如下的不那么完美的做法:
\begin{minted}{latex}
% \usepackage{zhnumber}
\addtocounter{chapter}{1}
\addcontentsline{toc}{chapter}{第\zhnum{chapter}章~~部分命令/名词索引}
\end{minted}

\subsection{Bug}
目前的index生成工具\cmd{indextools}宏包和tikz的\cmd{external}库有冲突,具体表现为:
当\cmd{indextools}和\cmd{external}库同时使用时,在第一次编译此文档时会抛出如下错误信息:

\begin{minted}{latex}
===== 'mode=convert with system call': Invoking 'pdflatex -halt-on-error -inter
action=batchmode -jobname "tikzdatamain-figure0" "\def\tikzexternalrealjob{release
}\input{release}"' ========

! Package tikz Error: Sorry, the system call 'pdflatex -halt-on-error -interact
ion=batchmode -jobname "tikzdata/release-figure0" "\def\tikzexternalrealjob{release}\i
nput{release}"' did NOT result in a usable output file 'tikzdatamain-figure0' (exp
ected one of .pdf:.jpg:.jpeg:.png:). Please verify that you have enabled system
    calls. For pdflatex, this is 'pdflatex -shell-escape'. Sometimes it is also na
med 'write 18' or something like that. Or maybe the command simply failed? Erro
r messages can be found in 'tikzdata/release-figure0.log'. If you continue now, I'l
l try to typeset the picture.
\end{minted}

关于此问题我已经在Github上给作者提了\href{https://github.com/maieul/indextools/issues/17}{Issue},
同时也在\TeX-SE上发出了\href{https://tex.stackexchange.com/questions/712716/indextools-confilict-with-tikz-library-external}{提问}.
可以关注上述的问题找到解决方法.

目前的解决方法有两个:
\begin{itemize}
    \item 取消加载indextools宏包,改用传统的\cmd{makeidx}宏包.(需自行去修改\cmd{zlatex.cls}中的加载项)
    \item 仍然使用此宏包,但是在第一遍(tikz图片还没有缓存时)取消导言区以及文档末尾的如下命令:
\begin{minted}{latex}
% 导言区
\makeindex[title=Test Title, columns=3]
% 文末
\addcontentsline{toc}{chapter}{部分名词索引}
\printindex
\end{minted}
        然后在文档的第二次编译时取消两处命令的注释,以此达到正常编译的目的.
\end{itemize}

\begin{leftbar}
\noindent 为何我一再坚持使用\cmd{indextools}宏包? 相较于传统的\cmd{makaidx}宏包需要在命令行中
先使用\LaTeX{}引擎编译,然后使用\cmd{makeindex}命令编译,最后再使用\LaTeX{}引擎编译两遍。\cmd{indextools}
宏包可以在不超过两次的\LaTeX{}引擎编译下直接生成对应的index,方便了许多.
\end{leftbar}