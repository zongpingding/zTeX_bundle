\chapter{z\LaTeX{}系列}
\section{简介}
\subsection{为何叫z?}
也不知道为什么这个系列名称要加以`z'的前缀,可能是因为个人爱好,或是因为觉得这个字母对自己而言有着一些别的意味。
最开始此系列中此包含一个基本的文档类,叫做 $\pi$\LaTeX{}, 但是后面自己想开发一个用于绘图的宏包,主要基于TiKZ.
用于常见平面图形的绘制以及外部程序的交互. 也许是看到了\cmd{tikz}库名称中的``z'',于是便以`z'为前缀,产生了
z\LaTeX{}\index{z\LaTeX{}}系列。


\subsection{项目地址}
目前本项目已经在GitHub, Gitlab, Gitee上开源,地址如下:
\begin{itemize}
    \item GitHub: \href{https://github.com/zongpingding/ZLaTeX_ZTikZ}{https://github.com/zongpingding/ZLaTeX\_ZTikZ}
    \item Gitlab: \href{https://gitlab.com/zongpingding/ZLaTeX_ZTikZ}{https://gitlab.com/zongpingding/ZLaTeX\_ZTikZ}
    \item Gitee:  \href{https://gitee.com/zongpingding/ZLaTeX_ZTikZ}{https://gitee.com/zongpingding/ZLaTeX\_ZTikZ}
\end{itemize}

项目中包含z\LaTeX{}文档类源码\cmd{zlatex.cls},zTikZ宏包源码\cmd{zTikZ.sty},以及二者的说明文档. 后续在开发过程中,
可能会保证Github的同步更新,至于Gitlab与Gitee则不一定会同步本系列的最新版.

\subsection{基本组成}
本系列目前包含以下的两个组成部分,一个文档类和一个绘图库:
\begin{itemize}
    \item z\LaTeX{}文档类
    \item zTikZ\index{zTikZ}宏包
\end{itemize}

其中前者主要用于指定排版文档的基本属性,后者主要用于绘图\Footnote{众所周知的,在\LaTeX{}中绘图是一件十分痛苦的事情,
于是乎你会看到很多书籍或笔记中的图形都是手绘或者是截图,并非矢量图}。其实从这个介绍文档就可以看出,本模板是十分的朴素的,
没有十分华丽的色彩和精美的页面布局,但是在折腾了许久的\LaTeX{}之后,现在这个模板才是最对我胃口的;至于,是否适合你,
那就不得而知了。你可以去使用更加精美的模板,比如 \href{https://github.com/ElegantLaTeX}{Elegant\LaTeX{}}, 
\href{https://github.com/BeautyLaTeX/Beautybook}{Beauty\LaTeX{}} 等优秀的模板. 

\section{模板设计}
\subsection{设计历程}
本模板的设计经历了相当长的一个周期,从最开始的初始\LaTeX{},我把自己常用的宏扔到了一个\cmd{.sty}文件中,以为这就是
一个宏包了;之后了解到了\href{https://github.com/ElegantLaTeX}{Elegant\LaTeX{}}系列模板,也使用这个系列中的book文档类写了一点
自己的笔记,但是用了一段时间之后总归是不满意,很多地方都想要自己定制,不喜欢模板默认的样式;奈何自己当时的水平不够,打开模板,看到的就是
一堆的乱码。但是,后来也知道了有知乎上的优秀文章,所以就去看这些文章,慢慢的积累,渐渐的对\LaTeX{}熟悉了一些,于是就着手设计属于
自己的模板。

第一版的z\LaTeX{}其实是完全仿照Elegant\LaTeX{}的book文档类,然后一步一步的慢慢加东西,进行一些简单的修改,比如字体,颜色等等。
但是写到后面,发现这个代码的的结构太不好控制了\Footnote{其实最开始这个zTikZ宏包和z\LaTeX{}是一体的,当时的代码是极其混乱的}.
尤其是其中的模板语言切换,那个\cmd{\ifdefstring}语句写起来是极其痛苦的。下面就是当初写的代码片段:

\begin{codeprint}
\DeclareVoidOption{cn}{\kvs{lang=cn}}
\DeclareVoidOption{en}{\kvs{lang=en}}
\DeclareStringOption[cn]{lang}
\end{codeprint}

再加上当时的基本文档类是\cmd{article},很多\cmd{book}文档类的内部计数器和章节命令都没有,需要自己去声明;但是结果往往是自己设计的命令
和别的宏包还不协调,冲突. 其中最重要的就是\cmd{hyperref}宏包了,初代模板中它的跳转功能是不正常的,由于自己定义的计数器不正确,
在使用\cmd{\label}命令时,激活的章节元素(跳转位置)根本不对。当初的目录结构也是自己设计,但是也有着同样的跳转为题.
初代z\LaTeX{}文档类全部采用\LaTeX{}2$\varepsilon$进行构建,很多的宏展开的地方都写的很繁琐,而且大部分的实现方案都是在
\TeX-StackExchange上找到的,很多时候都是处于一种能跑就行的状态,并不知道其背后的原理. 

后来自己便把zTiKZ从中z\LaTeX{}文档类中剥离出来,同时使用\LaTeX{}3对原始文档类和zTikZ进行重构.其中z\LaTeX{}文档类继承自\cmd{book}
文档类,之后几乎所有命令几乎都自己书写,知道它们的具体作用,对其他的宏包的影响。于是z\LaTeX{}系列就诞生了,
果然,在使用\LaTeX3对原始项目进行重构之后,整个项目的代码清爽了许多,比如下面的z\LaTeX{}文档类选项声明:

\begin{codeprint}
\zlatex_define_option:n {
    % language
    lang                  .str_gset:N   =  \g__zlatex_lang_str,
    lang                  .initial:n    =  { en },
    % page layout
    layout                .str_gset:N   =  \g__zlatex_layout_str,
    layout                .initial:n    =  { twoside },
    % margin option
    margin                .bool_gset:N  =  \g__zlatex_margin_bool,
    margin                .initial:n    =  { true },
}
\ProcessKeysOptions {zlatex / option}
\end{codeprint}

\subsection{设计参考}
这个模板自然不可能是我一个人独立开发,在开发过程中参考了诸多优秀文档类/模板,参考最多的{C\TeX{}art}文档类,几乎是
本项目的大部分代码思路来源。此文档类完全采用\LaTeX3语法写成,本文档类中的\textbf{选项配置}模块主要参见\TeX-StackExchange上
的讨论,采用了\LaTeX3的\cmd{key-value}模块;这样的好处有:选项配置简洁,符合人们习惯,模板维护方便.


\subsection{设计原则}
其实这个标题有一点太大了,什么是设计原则,我也不知道,但是我就只是想让我的模板看着舒服。怎么才能让自己的模板看着舒服呢?
我也不知道,但是我觉得肯定和页边距,字体大小,字体样式等的有关。并且这三者一定是相互影响的. 

比如你的页边距变大之后,压缩了你的版心大小,那么此时你的正文字体一定得做相应的改变. 那么一行多少个字合适呢?
去查了一下\TeX.SE, 针对于英文,一行的字母个数在65-90是比较合适的,并且字体尺寸一般为\cmd{10pt,11pt,12pt};
页边距到底设置多少呢? 自己去比对了Elegan\LaTeX{}和其它模板的页边距(就差用尺子量了); 好歹后面发现了一个宏包,
可以在生成的PDF中查看页面布局尺寸等信息, 这个宏包就是\cmd{fgruler}, 使用语法也是很简单的,如下:

\begin{codeprint}
\usepackage[hshift=0mm,vshift=0mm]{fgruler}
\end{codeprint}

当你在导言区引入之后,便可以在你的每一个页面的看到如\cref{fig:fgruler-example}的效果, 这样就 
不用打印出来用尺子量了.

\begin{figure}[!htb]
    \centering
    \includegraphics[width=.75\linewidth]{./pics/fgruler_example.pdf}
    \caption{页面布局示意图}
    \label{fig:fgruler-example}
\end{figure}

在设计本模板的时,我也一直在纠结字体的问题,我应该把字体打包进入模板吗? 或者是我应该在模板中给用户进行默认的字体设置吗?
在这个系列的上一版中我就去找了一些免费的中文字体和西文字体,直接放在模板的文件夹下,但是这样产生的问题就很多了:

\begin{itemize}
    \item 用户需要这个字体吗, 增加的字体会变成这个模板的负担吗 ?
    \item 这个字体真的免费吗 ?
    \item 中文字体的字形往往是不全的,怎么解决 ? 
\end{itemize}

于是最终的办法就是,我的模板不负责字体的设置,不添加任何和字体相关的配置,所有的字体由用户指定. 


最后参考这这些标准,一步一步的调整,使得整体的页面布局稍微的合理些. 在设计这个模板时,还要考虑行距等各种元素。
但是设计一个模板,你考虑的还不只这些,反正就是,如果你不会的话,那么就一切保持默认; 

\vspace*{10em}
\begin{center}
{\huge Be simple, Be fool}
\end{center}
\vspace*{10em}

\section{兼容性}
目前本系列已经实现Windows和Linux下的兼容; 但是MacOS下:目前仅支持z\LaTeX{}文档类.
zTikZ还未进行适配(参见下文了解具体原因),所以不保证本系列中的zTikZ文档类可以在MacOS下正常运行.
具体的兼容情况请参见后续的兼容性章节.

\chapter{z\LaTeX{}文档类}
\section{基本介绍}
本文档类z\LaTeX{}基于\cmd{book}类,主要用于满足和方便使用\LaTeX{}使用者进行书籍和笔记的排版需求。
z\LaTeX{}全部由\LaTeX3进行编写,采用\cmd{key-value}的方式进行选项配置,方便后续的模板拓展和维护.
如果使用者熟悉\LaTeX{},那么花费不到10min的时间,你便可以轻松使用本文档类用于日常的笔记撰写或者是正常的
书籍的排版. 

z\LaTeX{}的源代码完全开放,欢迎各位对源代码的修改以及二次分发.


\section{Set Up z\LaTeX{}}
\subsection{兼容情况}
目前本文档类 z\LaTeX{} 还没有登陆CTAN,未来也没有这个打算。由于本文档类全部使用
\LaTeX3进行开发,所以如果你的\TeX{}Live过于老旧的话,则无法使用本宏包。目前已知
z\LaTeX{}文档类在各平台的兼容情况为:

\hspace*{10em}\parbox{8cm}{
\begin{itemize}
    \item[Windows]: \TeX{}Live最低版本2022
    \item[Linux]: \TeX{}Live最低版本2022
    \item[MacOS]: 兼容Mac{}\TeX{}2024(老版也应兼容) 
\end{itemize}}

\subsection{加载z\LaTeX{}}
由于z\LaTeX{}还没有传入CTAN(未来也不会),所以想要使用此文档类,可以有如下的两种方法:
\begin{itemize}
    \item 把此文档类放入你的项目文件夹下
    \item 在命令行运行命令: \cmd{kpsewhich -var-value=TEXMFHOME}, 然后把\cmd{zlatex.cls}放入此路径下的
        \cmd{tex/latex/}子目录下. 在Windows上一般是: \cmd{C:/Users/<name>/texmf/}, 在Linux下一般是
        \cmd{~/texmf/},具体路径以自己的实际情况为准.
\end{itemize}

\subsection{额外设置}
由于z\LaTeX{}文档类只加载了基本的宏包,所以想要实现其它的功能还请自行引入相关的宏包;
z\LaTeX{}引入的宏包机制请参见\cref{tab:basic-package}.

\subsection{最小工作示例}
z\LaTeX{}的最小工作示例如下\Footnote{可能需要根据自己的实际情况加以调整}.
首先是中文写作示例:

\begin{codeprint}
% compile engine: xelatex 
\documentclass[lang=cn]{zlatex}

\title{<title>}
\author{<author>}
\date{<date>}
\begin{document}
\maketitle
\frontmatter
% some preface
% \tableofcontents
% some claim etc.
\mainmatter

% wrting your document here ...
\end{document}
\end{codeprint}

其次是英文写作示例,你需要修改的地方只有两处; 首先就是把语言选项改为\cmd{lang=en},
其次便是把编译方式改为\cmd{pdflatex}.

\begin{codeprint}
% compile engine: pdflatex 
\documentclass[lang=en]{zlatex}

\title{<title>}
\author{<author>}
\date{<date>}
\begin{document}
\maketitle
\frontmatter
% some preface
% \tableofcontents
% some claim etc.
\mainmatter

% wrting your document here ...
\end{document}
\end{codeprint}

\section{宏包机制}
z\LaTeX{}文档类会根据用户指定的选项自动处理和加载对应的宏包,所以z\LaTeX{}文档类在不同的导言区选项声明下
加载的宏包和命令是不同的。本文档类内置导言区选项输出命令:\cmd{\zlatexOptions}\index{\cmd{\zlatexOptions}},
用于打印此时文档类z\LaTeX{}接收到的选项. 比如此时文档类接收到的选项为: 
\begin{center}
    \zlatexOptions
\end{center}

以下为详细的宏包加载信息:

\subsection{基本宏包}
基本宏包\index{basic packages},意味着不管你的导言区如何的配置,这些宏包都是会加载的. 宏包列表如下:

\begin{table}[H]
    \centering{\ttfamily
    \begin{tabular}{p{3cm}p{3cm}p{3cm}p{3cm}}
        \toprule
        expl3 & l3keys2e & framed & geometry \\
        fancyhdr & amsfonts & amsmath & amsthm  \\
        xcolor & biblatex & indextools & hyperref \\ 
        cleveref & graphicx & float & titletoc \\
        titlesec & tocloft & &\\
        \bottomrule
    \end{tabular}}
    \caption{z\LaTeX{}文档类基本宏包}
    \label{tab:basic-package}
\end{table}

\subsection{语言类宏包}
根据不同的文档类语言,z\LaTeX{}会加载不同的和语言相关的宏包\index{language packages},在\cmd{lang=en(cn)}
下的宏包加载列表分别为:

\begin{table}[H]
    \centering{\ttfamily
    \begin{tabular}{p{2cm}p{4cm}p{2cm}p{2cm}p{2cm}}
        \toprule
        {\rmfamily lang=en} & inputenc(pdftex) & fontenc & csquotes & babel \\
        {\rmfamily lang=cn} & fontspec & ctex \\
        \bottomrule
    \end{tabular}}
    \caption{z\LaTeX{}文档类语言宏包}
    \label{tab:lang-package}
\end{table}

\subsection{数学类宏包}
从前面的导言区数学字体配置就可以看出,本模板会根据导言区设置不同的数学字体的功能了. 具体的加载
宏包\index{math packages}规则如下:
\begin{itemize}
    \item \cmd{math-font=<none>}: 不加载任何的数学字体宏包,采用默认数学字体
    \item \cmd{math-font=newtx}: 加载宏包 \cmd{newtxmath}
    \item \cmd{math-font=euler}: 加载命令 \cmd{\RequirePackage[OT1, euler-digits]{eulervm}}
    \item \cmd{math-font=mtpro2}: 加载命令 \par
        \cmd{\RequirePackage[lite,subscriptcorrection,slantedGreek,nofontinfo]{mtpro2}}
\end{itemize}

如果使用者在导言区指定了选项\cmd{math-alias=true}, 那么z\LaTeX{}此时还会额外加载
宏包\index{optional packages}:\cmd{amssymb, mathtools, bm}.


\section{文档类选项}
本模板具有丰富的配置选项\index{配置选项},包含页面设置\index{页面设置},页边距,边注\index{边注},数学字体\index{数学字体},
字体大小\index{字体大小},模板语言\index{模板语言}; 采用键值对\cmd{[<key 1>=<value 1>, <key 2>=<value 2>]}的形式
对各个选项就行指定, 和具体的指定顺序无关, 具体的可配置项和可用的配置值参见\cref{table:zlatex-option}:

\subsection{配置方法}
\begin{table}[H]
    \centering
    \begin{tabular}{p{4cm}p{5cm}p{3cm}}
    \toprule
    选项\cmd{<key>} & 可选值\cmd{<value>} & 默认值 \\[.25em]
    \hline
    \cmd{lang} & \cmd{en, cn} & \cmd{en} \\
    \cmd{layout} & \cmd{oneside, twoside} & \cmd{twoside} \\
    \cmd{margin} & \cmd{false, true} & \cmd{true} \\
    \cmd{fontsize} & \cmd{10pt, 11pt, 12pt} & \cmd{11pt} \\
    \cmd{math-alias} & \cmd{false, true} & \cmd{false} \\
    \cmd{math-font} & \cmd{newtx, mtpro2, euler} & \cmd{<none>} \\
    \cmd{bib-source} & \cmd{<自定义>} & \cmd{ref.bib} \\
    \cmd{toc} & \cmd{rename, 2column} &  \cmd{<none>}\\
    \bottomrule
    \end{tabular}
    \caption{z\LaTeX{}配置选项}
    \label{table:zlatex-option}
\end{table}

目前的z\LaTeX{}接口还不够丰富,没有进行相关的\cmd{Hook}(钩子)的声明,所以用户可以配置的选项是比较少的,
只要能够把导言区设置规范,那么剩下的内容你几乎是不用在设置了.

\subsection{注意事项}
下面是一些你在指定文档类选项时应该注意到的问题:
\begin{itemize}
    \item \cmd{margin=false} 只有在指定 \cmd{layout=oneside}时才会启用,否则会抛出警告. 同时需要注意,如果原来
        含有\cmd{\marginpar}命令的文档,在指定\cmd{margin=false}后,对应的\cmd{\marginpar}环境,会被替换为\cmd{framed}宏包
        提供的\cmd{leftbar}环境.
    \item \cmd{lang=cn} 时仅支持编译方式为 \cmd{xelatex}, 在指定 \cmd{lang=en}时,\cmd{pdflatex, xelatex}
        二者都是可以接受的,但是建议采用 \cmd{pdflatex}, 因为在指定为 \cmd{en}时部分的西文宏包可能会有冲突的危险,
        因为当\cmd{lang=en}, 并且采用\cmd{pdflatex}进行编译时,z\LaTeX{}会引入宏包\cmd{inputenc}, 然而此宏包
        对\cmd{xelatex}是没有适配的.
    \item 数学字体选项不一定符合每一个人,本模板的开发环境为 \cmd{WSL+Archlinux}. 同时其中的
        \cmd{mtpro2}字体并非免费字体,请注意.
    \item \cmd{math-alias}选项可以根据个人习惯进行选择,默认情况下并不会加载。但是在加载此选项后,默认的两个\LaTeX{}
        指令\cmd{\S, \ll} 会被覆盖,分别被更名为 \cmd{\ss, \LL}:(\ss, $\LL$). 
    \item \cmd{toc}\index{\cmd{toc}}选项可以同时指定\cmd{rename, 2column},分别代表重定义目录名(居中加粗),以及使用双栏目录排版.
\end{itemize}

\subsection{数学指令}
关于文档类选项\cmd{math-alias}\index{\cmd{math-alias}}的进一步说明,默认的自定义命令可能并不一定符合每一个人的习惯,所以请谨慎加载此选项.
在\cmd{math-alias=true}后,z\LaTeX{}会进行如下命令的声明/重定义,以及宏包的加载:
\begin{codeprint}
\RequirePackage{amssymb, mathtools}
\RequirePackage{bm}          
% Math Font 
\newcommand{\dd}{\mathrm{d}}
\newcommand{\C}[1]{\ensuremath{\mathcal{#1}}}
\let\ss\S
\renewcommand{\S}[1]{\ensuremath{\mathscr{#1}}}
\newcommand{\B}[1]{\ensuremath{\mathbb{#1}}}
\newcommand{\FF}[1]{\ensuremath{\mathbf{#1}}}
\newcommand{\F}[1]{\ensuremath{\bm{#1}}}
\newcommand{\R}[1]{\ensuremath{\mathrm{#1}}}
\newcommand{\K}[1]{\ensuremath{\mathfrak{#1}}}
% Math Arrow 
\newcommand{\lr}{\ensuremath{\longrightarrow}}
\let\LL\ll
\renewcommand{\ll}{\ensuremath{\longleftarrow}}
\newcommand{\equ}{\ensuremath{\Longleftrightarrow}\,}
\newcommand{\sr}{\ensuremath{\longmapsto}}
\newcommand{\lrr}[2][]{\ensuremath{\xRightarrow[#1]{#2}}}
\renewcommand{\lll}[2][]{\ensuremath{\xLeftarrow[#1]{#2}}}
\newcommand{\ns}{\ensuremath{\varnothing}}
\newcommand{\A}{\ensuremath{\forall}}
% Math Operator
\newcommand{\alt}{\ensuremath{\mathrm{Alt}\;}}
\newcommand{\sgn}{\ensuremath{\mathrm{sgn}\;}}
\newcommand{\curl}{\ensuremath{\mathrm{curl}\;}}
\newcommand{\grad}{\ensuremath{\mathrm{grad}\;}}
\newcommand{\trace}{\ensuremath{\mathrm{trace}\;}}
\renewcommand{\div}{\ensuremath{\mathrm{div}\;}}
\end{codeprint}

\section{z\LaTeX{}接口}
z\LaTeX{}的接口正在不断的完善中,所以目前的接口可能并不是那么稳定。(我已经尽力让接口规范和稳定了)

\subsection{命令声明}
后面我可能会考虑建立一个用于自定义命令的接口,采用键值对的方式进行配置,而不是默认的位置参数或者是xparse提供的可选,
默认,强制参数等。依托于\LaTeX3的\cmd{key-value}模块,这样的接口会更加的灵活,方便用户进行配置.

\subsection{盒子接口}
由于目前我还没有弄清楚\LaTeX3的盒子操作,所以z\LaTeX{}的盒子接口还没有进行完善,但是我会尽快的进行完善.

\subsection{TikZ接口}
本人并不打算在z\LaTeX{}中使用 \cmd{tikz} 宏包,因为我觉得这个宏包太过于庞大,很多的功能都不是一个文档类必须的.
可能我会在后续引入\cmd{l3draw}\index{\cmd{l3draw}}模块用于TikZ操作. 

\begin{leftbar}
本文档类配套的\cmd{ztikz}库提供了丰富的和TikZ绘图,数值计算,以及部分的图像处理功能.具体使用请参见下一个单元.
\end{leftbar}

\subsection{计数器}
目前的计数器部分继承自 \cmd{book}文档类和使用\cmd{amsthm}\index{\cmd{amsthm}}宏包定义的数学环境计数器 
theorem, definition, corollary, example, axiom, remark.

目前z\LaTeX{}提供了一个命令\cmd{\zlatexUpdateCounterAfter}\index{\cmd{\zlatexUpdateCounterAfter}}用于设置
计数器的更新,使用格式为:
\begin{codeprint}
\zlatexUpdateCounterAfter{<child>}{<father>}
\end{codeprint}

也就是让上述的\cmd{<child>}计数器随着\cmd{<father>}父计数器的更新而更新,本命令的实现原型为:
\begin{codeprint}
\NewDocumentCommand{\zlatexUpdateCounterAfter}{mm}{
    \@addtoreset{#1}{#2}
}
\end{codeprint}

关于本文档类的公式计数器的说明,本文档类公式计数器默认跟随\cmd{section}计数器更新,在z\LaTeX{}的源码中的声明为:
\begin{codeprint}
\counterwithin{equation}{section}
\end{codeprint}

\subsection{模板配色}
z\LaTeX{}提供了\cmd{\zlatexColorSetup}\index{\cmd{\zlatexColorSetup}}用于设置整个模板的配色。
可供用户配置的选项有:
\begin{itemize}
    \item Hyperref宏包对应的颜色,对应的键为\cmd{link, url, cite}, 三者颜色默认不同.
    \item Chapter章节计数器颜色,对应的键为\cmd{chapter}.
    \item Chapter章节ruler颜色,对应的键为\cmd{chapter-rule}.
    \item 所有数学环境对应的颜色,对应的键为\cmd{<math-env-name>},如\cmd{axiom, definition, theorem, remark}等.
\end{itemize}

下面给出设置具体色彩的示例代码以及模板的默认配色:
\begin{codeprint}
\zlatexColorSetup{
    link            = purple,
    chapter-rule    = black,
    axiom           = purple,
    definition      = blue
}
\end{codeprint}

\newcommand{\block}[1]{{\color{#1}\rule{1em}{1em}}}
\begin{table}[H]
    \centering
    \begin{tabular}{ccccccccc}
        \toprule
        结构元素 & chapter & chapter-rule & link & url & cite \\
        \midrule 
        颜色 & \block{RoyalRed} & \block{black} & \block{purple}& \block{RoyalRed} & \block{teal}\\
        \midrule
        数学环境 & axiom & definition & theorem & lemma & corollary & proposition & remark & \\  
        \midrule 
        颜色 & \block{mathaxiomColor} & \block{mathdefinitionColor} & \block{maththeoremColor} & \block{mathlemmaColor}& \block{mathcorollaryColor}& \block{mathpropositionColor}& \block{mathremarkColor}\\
        \bottomrule
    \end{tabular}
    \caption{z\LaTeX{}文档类默认配色}
    \label{tab:zlatex-default-color}
\end{table}


\subsection{引用环境}
目前本系列提供命令\cmd{\zlatexFramed{<name>}[<color>]}\index{\cmd{\zlatexFramed}}用于创建类似MarkDown
的彩色引用环境. 参数中的\cmd{<name>}表示声明环境的名称\footnote{如果此环境已存在,那么该环境会被Override},
\cmd{<color>}表示此环境的背景颜色.一个简单的使用样例如下:

\begin{codeprint}
% 环境 'refer' 声明
\zlatexFramed{refer}[orange]
% 使用环境 'refer'
\begin{refer}%
% something wrting here
\end{refer}
\end{codeprint}

\zlatexFramed{refer}[orange]
\begin{refer}%
As any dedicated reader can clearly see, the Ideal of practical
reason is a representation of, as far as I know, the things in themselves;

劳仑衣普桑,认至将指点效则机,最你更枝。想极整月正进好志次回总般,段然取向
使张规军证回,世市总李率英茄持伴。
\end{refer}

\begin{leftbar}
    在上面的refer环境开始时插入一个\%可以用于消除多余的空格
\end{leftbar}

\section{自定义}
\subsection{封面}
本文档类并没有内建复杂的封面格式,只是简单的重定义了\cmd{\maketitle}命令用于生成封面. 
声明如下:
\begin{codeprint}
\renewcommand{\maketitle}{
    \begin{titlepage}
        \vfill\vspace*{40pt}
        \noindent\hspace*{134pt}\rule[-75pt]{6pt}{95pt}{\hspace*{10pt}\fontsize{25}{25}\selectfont\bfseries\@title}\par
        \vspace*{-15pt}
        \noindent\hspace*{150pt}{\Large\bfseries\@author}\par
        \vspace*{480pt}
        \noindent\hspace*{150pt}{\Large\textcolor{gray}{\@date}}
        \vfill
    \end{titlepage}
} 
\end{codeprint}

如果使用者想要使用更加美观的封面,请手动加载\cmd{tikz}宏包,自己定义.

\subsection{目录}
尽管在z\LaTeX{}的加载选项一节便已经说明了z\LaTeX{}文档类默认加载了\cmd{titletoc, tocloft}宏包用于目录的格式定制,
并且提供了对应的加载选项 \cmd{toc}已经对应的可选值\cmd{rename, 2column}, 但是在本文档类中并没有对目录的格式进行
更加深度的定制,可能后续会开发对应的接口.

\subsection{页眉页脚}
本文档类采用\cmd{fancyhdr}进行页眉页脚的定制,目前已经写死在文档类中,如果使用者想要自定义页眉页脚,可以直接
重定义\cmd{\fancyhead, \fancyfoot}命令.或者是页面样式(pagestyle)对应的\cmd{fancy}样式. 后续会考虑添加对应的接口.

\subsection{章节格式}
目前还不支持指定章节格式,等后续在添加. 使用者可以加载\cmd{titlesec}等宏包进行自定义. z\LaTeX{}
文档类默认加载了\cmd{titlesec, titletoc, tocloft}宏包用于章节格式和目录的格式定制. 如果使用
者想要自定义章节格式,直接使用\cmd{titlesec}宏包的\cmd{\titleformat}命令覆盖本模板的原始定义即可,
或者是其他的命令. 

\begin{leftbar}
但是本文档类默认不加载\cmd{tikz, pgf}宏包,想要使用这两个宏包定义更加复杂章节样式,请手动加载,并
设置自己喜欢的章节格式. 也许后续我会在z\LaTeX{}的加载选项中添加一个tikz选项,从而可以让用户自定义章节格式.
\end{leftbar}



\section{数学环境}
\subsection{常用数学环境}
本文档类使用宏包\cmd{amsthm}定义了如下数学环境;大致分为两类: 定理类环境和证明类环境;其中 
的定理类环境相较于证明类环境多一个带有颜色的\cmd{leftbar}\index{\cmd{leftbar}}. 具体的环境名称见下方:

\begin{multicols}{2}
\begin{itemize}
    \item 定理类环境
        \begin{itemize}
        \item axiom
        \item definition
        \item theorem 
        \item lemma
        \item corollary 
        \item proposition
        \item remark 
        \end{itemize}
    \item 证明类环境
    \begin{itemize}
        \item proof
        \item exercise
        \item example
        \item solution
        \item problem
    \end{itemize}
\end{itemize}    
\end{multicols}

z\LaTeX{}中的数学环境有3套主题\index{数学环境主题},分别为\cmd{none,leftbar,all}. 其中\cmd{none}表示数学环境不加载任何的修饰,
\cmd{leftbar}表示数学环境的左侧使用\cmd{framed}宏包提供的\cmd{leftbar}命令进行修饰,\cmd{all}表示数学环境加载
\cmd{leftbar}的同时设置其背景颜色为对应颜色的10\%. 

只需要用户在加载本文档类时指定\cmd{math-env-theme}选项即可,比如本示例文档的数学环境主题为\cmd{leftbar}(默认主题):
\begin{codeprint}
\documentclass[math-env-theme=leftbar]{zlatex}
\end{codeprint}

\subsection{定理类环境}
现在介绍怎么使用这些具体的内置数学环境,上述的每一个环境的基本调用格式如下:
\begin{codeprint}
\begin{<theorem like env>}[<theorem name>]
你的定理内容就写在这个环境的内部.

your theorem writing here. 
\end{<theorem like env>}
\end{codeprint}

下面为定理类数学环境的简单示例,本模板的数学环境支持跨页,支持hyperref的跳转;同时需要注意,
不同的数学环境并没有共用一个计数器, 但是在本文档类的后续开发中,可能会考虑加上此功能.

想要对定理类环境添加\cmd{label}的语法如下:
\begin{codeprint}
\begin{<theorem like env>}[<theorem name>]\label{thm:test}
你的定理内容就写在这个环境的内部.
    
your theorem writing here. 
\end{<theorem like env>}
\end{codeprint}

后续引用直接使用命令\cmd{\cref{thm:test}}, 比如引用刚才标记的 \cref{thm:test},
可以看到,这个是可以精确跳转到对应的定理处的. 同时本模板中的\cmd{\cref}\index{\cmd{\cref}} 命令会自动根据计数器的类别
和文档的语言选项决定具体的引用格式. 针对于图表的引用也是同理的,你只需要把这一切都交给\cmd{\cref}即可. 相关的详细信息还请参见
本文档后面部分的\cmd{标签与引用}.


\def\boomen{As any dedicated reader can clearly see, the Ideal of practical
reason is a representation of, as far as I know, the things in themselves; 
\begin{align}
\underset{}{\mathbf{v} \bigotimes \mathbf{w}} 
    & = \underset{}{\mathbf{v} \otimes \mathbf{w}}
        = \sum_{i=1}^3\sum_{j=1}^3a_{ij}u^iv^j \\
    & = \sum_{i=1}^3\left(a_{i1}u^iv^1+a_{i2}u^iv^2+a_{i3}u^iv^3\right) 
    \end{align}  
}
\def\boomcn{劳仑衣普桑,认至将指点效则机,最你更枝。想极整月正进好志次回总般,段然取向
使张规军证回,世市总李率英茄持伴。}

\subsubsection{none主题}
\ExplSyntaxOn
\DeclareDocumentEnvironment{zlatexTheoremLikeFrame}{O{}}{\vspace*{5pt}}{\vspace*{5pt}}
\ExplSyntaxOff
\begin{theorem}[prime theorem]\label{thm:test}
    \boomen \par 
    \boomcn
\end{theorem}

\begin{definition}[prime definition]
    \boomen \par 
    \boomcn
\end{definition}

\subsubsection{leftbar主题}
\ExplSyntaxOn
\DeclareDocumentEnvironment{zlatexTheoremLikeFrame}{O{black}}{
    \def\FrameCommand{{\color{#1}\vrule width 3pt}\hspace{5pt}}
    \MakeFramed {\advance\hsize-\width \FrameRestore}
}{\endMakeFramed}
\ExplSyntaxOff
\begin{lemma}[prime lemma]
    \boomen \par 
    \boomcn
\end{lemma}

\begin{remark}[prime remark]
    \boomen \par 
    \boomcn
\end{remark}


\subsubsection{all主题}
\ExplSyntaxOn
\DeclareDocumentEnvironment{zlatexTheoremLikeFrame}{O{black}}{
    \def\FrameCommand{{\color{#1}\vrule width 3pt}\colorbox{#1!10}}
    \MakeFramed{\advance\hsize-\width \FrameRestore}
}{\endMakeFramed}
\ExplSyntaxOff

\begin{axiom}[prime axiom]
    \boomen \par 
    \boomcn
\end{axiom}

\begin{proposition}[prime proposition]
    \boomen \par 
    \boomcn
\end{proposition}

\ExplSyntaxOn
\DeclareDocumentEnvironment{zlatexTheoremLikeFrame}{O{black}}{
    \def\FrameCommand{{\color{#1}\vrule width 3pt}\hspace{5pt}}
    \MakeFramed {\advance\hsize-\width \FrameRestore}
}{\endMakeFramed}
\ExplSyntaxOff

\subsection{证明类环境}
证明类环境的使用方法和前者几乎差不多,比较朴素,没有彩色的左边界竖线, 也没有可选的默认参数; 
一般建议空一行再开始此类环境,下面给出两个个示例,剩下的环境便不一一例举了;

\begin{codeprint}
\begin{<proof like env>}
    你 的 定 理 内 容 就 写 在 这 个 环 境 的 内 部 .
    your proof writing here.
\end{<proof like env>}
\end{codeprint}

\vspace*{4em}
\begin{proof}
    \boomen \par 
    \boomcn
\end{proof}

\begin{example}
    \boomen \par 
    \boomcn
\end{example}

你可以自行定制Proof环境的结束标志,但是需要注意的一点是:你的标志必须放入公式环境,如果你的结束标志
只能用于公式环境时. 例如,把证明结束符从 \(\blacksquare\) 替换为 $\square$:
\begin{codeprint}
\renewcommand{\qedsymbol}{\ensuremath{\square}}
\end{codeprint}


\subsection{注意事项}
默认的数学类环境均采用正体\cmd{\upshape},如果使用者不喜欢前者默认的``正体''字体样式,
可以直接在数学类环境开始时使用字体命令\cmd{\itshape}进行原有字体样式的覆盖,示例如下:

\begin{codeprint}
\begin{theorem}[test theorem]\itshape
    你好, Hello world !
\end{theorem}
\end{codeprint}

\begin{remark}\itshape
    \boomen \par 
    \boomcn
\end{remark}

同时,本文档类中数学类环境和前文的自定义高亮环境\cmd{\zlatexFramed}均默认首行不缩进,需手动添加缩进.

\subsection{自定义数学环境}
目前还没有开发对应的接口,主要是目前的格式基本已经够用了.


\section{标签与引用}
\subsection{footnote}
可能有人不喜欢默认的脚注没有在页脚的位置,而是在页脚偏上的位置,用户可以独立加载宏包\cmd{footmisc}用于
强制脚注位于页面底部,本文档类不打算添加此宏包,用户可以自行在导言区添加如下命令:
\begin{codeprint}
\usepackage[bottom]{footmisc}
\end{codeprint}

\subsection{Cleveref}
z\LaTeX{}文档类加载了\cmd{cleveref}宏包来构建标签-引用系统。常规的\cmd{\label{}}操作并没有什么变化,
区别主要在引用标签功能上。对于普通的模板你可能会看到如下的说明: 使用\cmd{\eqref}进行公式标签的索引,
使用\cmd{\figref}进行图片的索引,使用\cmd{\tabref}进行表格的索引... 使用此命令可以避免书写如下
格式的引用代码:

\begin{codeprint}
定理:\ref{thm:test}
% or 
\newcommand\thmref{定理:\ref{#1}}
\end{codeprint}

在z\LaTeX{}中,引用格式预设值如下(至于多个标签引用时,只有\cmd{lang=en}时采用部分变化,对应的前缀变为复数):
\begin{table}[H]
    \centering
    \begin{tabular}{p{3cm}p{3cm}p{3cm}p{3cm}}
        \toprule
        语言 & 公式 & 图片 & 表格 \\
        \midrule
        \cmd{lang=en} & equation & figure & table \\
        \cmd{lang=cn} & 方程 & 图 & 表 \\
        \bottomrule
    \end{tabular}
    \caption{cref引用格式}
    \label{tab:sys-cref}
\end{table}

对于\cmd{cleveref}中的其它命令,如\cmd{\Cref}, \cmd{\crefrange}, \cmd{\Crefrange}等等,
本文档类未对其进行修改,所以以上命令均是兼容的,详细的使用说明请参见\cmd{cleveref}宏包的官方文档.

\subsection{图片与(列)表}
z\LaTeX{}采用\cmd{cleveref}提供的引用命令,本文档类内置的\cmd{\cref}命令的用法和
原始宏包中的\cmd{\cref}\index{\cmd{\cref}}的用法是一样的,只是在引用的时候会根据文档的语言选项进行
对应的prefix更改.比如在\cmd{lang=cn}时把默认的\cmd{fig 1.1}改为中文环境下的 \cmd{图 1.1}.

这其实也就意味着,本文档类中还可以使用\cmd{cleveref}提供的所有的引用命令,
比如\cmd{\Cref, \crefrange, \Crefrange}等等.更多的详细信息可以参见\cmd{cleveref}的
官方文档.

\subsection{列表环境}
z\LaTeX{}对\cmd{book}文档类的无编号计数器进行了定制,有序列表和无序列表现在的具体样式如下:

\begin{multicols}{2}
    \begin{itemize}
        \item 一级项目
        \begin{itemize}
            \item 二级项目
            \begin{itemize}
                \item 三级项目
            \end{itemize}
        \end{itemize}
    \end{itemize}
    
    \begin{enumerate}
        \item 一级项目
        \begin{enumerate}
            \item 二级项目
            \begin{enumerate}
                \item 三级项目
            \end{enumerate}
        \end{enumerate}
    \end{enumerate}
\end{multicols}


\section{文献引用}
\subsection{基本设置}
本模板采用的文献引擎是\cmd{biber}, 这说明,你在编译你的文档时应该采用\cmd{biber}, 而非
\cmd{bibtex}. 如果你想要把``参考文献''栏目加入目录,可以使用命令:

\begin{codeprint}
\addcontentsline{toc}{chapter}{参考文献} % or
\addcontentsline{toc}{chapter}{Bibliography}
\end{codeprint}


\subsection{使用样例}
使用\cmd{\cite{<ref>}}进行参考文献的引用, 然后使用命令\cmd{\printbibliography}输出参考文献.
下面举一个简单的例子:

\begin{codeprint}
% 参考文献: ref.bib
@book{ahlfors1953complex,
    title={Complex Analysis},
    author={Ahlfors},
    year={1953},
    publisher={McGraw-Hill},
    address={New York}
}

% 正文引用
\cite{ahlfors1953complex}
\end{codeprint}


\section{索引}
\subsection{使用方法}
z\LaTeX{}文档类采用\cmd{indextools}宏包进行索引的生成,并不没有采用传统的\cmd{makeidx}宏包.
具体的用法和\cmd{indextools}宏包的一致,这里给一个简单的示例:

\begin{codeprint}
% 导言区
\makeindex[title=Concept index]
% 添加索引到目录,生成索引
\addcontentsline{toc}{part}{Index}
\printindex
\end{codeprint}

或者是你可以在你文档的导言区声明某种\cmd{index}的类型,比如\cmd{person},然后就可以在文中使用
\cmd{\index[person]{<the person>}} 来进行索引,最后使用如下命令进行索引的打印和索引的导言区
定制:

\begin{codeprint}
% 导言区
\makeindex[name=person, title=Index of names, columns=3]
% 文档末尾
\indexprologue{In this index you’ll find only famous people’s names}
\printindex[person]
\end{codeprint}

使用\cmd{\index}命令时在此命令中的名词是不会显示在PDF文档中的,所以如果你要添加一个``函数''
的index项目时,在你的\TeX{}文档中应该这样写:

\begin{codeprint}
函数\index{函数}是从集合到 ...
\end{codeprint}

\subsection{Bug}
目前的index生成工具\cmd{indextools}宏包和tikz的\cmd{external}库有冲突,具体表现为:
当\cmd{indextools}和\cmd{external}库同时使用时,在第一次编译此文档时会抛出如下错误信息:

\begin{codeprint}
===== 'mode=convert with system call': Invoking 'pdflatex -halt-on-error -inter
action=batchmode -jobname "tikzdatamain-figure0" "\def\tikzexternalrealjob{release
}\input{release}"' ========

! Package tikz Error: Sorry, the system call 'pdflatex -halt-on-error -interact
ion=batchmode -jobname "tikzdata/release-figure0" "\def\tikzexternalrealjob{release}\i
nput{release}"' did NOT result in a usable output file 'tikzdatamain-figure0' (exp
ected one of .pdf:.jpg:.jpeg:.png:). Please verify that you have enabled system
    calls. For pdflatex, this is 'pdflatex -shell-escape'. Sometimes it is also na
med 'write 18' or something like that. Or maybe the command simply failed? Erro
r messages can be found in 'tikzdata/release-figure0.log'. If you continue now, I'l
l try to typeset the picture.
\end{codeprint}

关于此问题我已经在Github上给作者提了\href{https://github.com/maieul/indextools/issues/17}{Issue},
同时也在\TeX-SE上发出了\href{https://tex.stackexchange.com/questions/712716/indextools-confilict-with-tikz-library-external}{提问}.
可以关注上述的问题找到解决方法.

目前的解决方法有两个:
\begin{itemize}
    \item 取消加载indextools宏包,改用传统的\cmd{makeidx}宏包.(需自行去修改\cmd{zlatex.cls}中的加载项)
    \item 仍然使用此宏包,但是在第一遍(tikz图片还没有缓存时)取消导言区以及文档末尾的如下命令:
\begin{codeprint}
% 导言区
\makeindex[title=Test Title, columns=3]
% 文末
\addcontentsline{toc}{chapter}{部分名词索引}
\printindex
\end{codeprint}
        然后在文档的第二次编译时取消两处命令的注释,以此达到正常编译的目的.
\end{itemize}

\begin{leftbar}
\noindent 为何我一再坚持使用\cmd{indextools}宏包? 相较于传统的\cmd{makaidx}宏包需要在命令行中
先使用\LaTeX{}引擎编译,然后使用\cmd{makeindex}命令编译,最后再使用\LaTeX{}引擎编译两遍。\cmd{indextools}
宏包可以在不超过两次的\LaTeX{}引擎编译下直接生成对应的index,方便了许多.
\end{leftbar}